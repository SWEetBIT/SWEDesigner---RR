\begin{figure}[h!]
	\centering
	\includegraphics[scale=0.8]{res/sections/SpecificaFrontEnd/Components/Disegnetti/class-add-method.png}
	\caption{Diagramma della classe Class-add-method}
\end{figure}

\begin{itemize}
	\item \textbf{Descrizione:}\\
	Gestisce l'aggiunta di un metodo a una classe.
	\item \textbf{Utilizzo:}\\
	Viene utilizzata dal EditClassMenuComponent per aggiungere metodi alle classi.
	\item \textbf{Metodi:}
		\begin{itemize}
			\item \emph{-constructor(private classMenuService: ClassMenuService)}\\
    		Costruttore della classe\\
    		\textbf{Parametri:}
    		\begin{itemize}
    			\item \emph{classMenuService: ClassMenuService}\\
    			Crea un istanziazione di ClassMenuService
    		\end{itemize}
    		\item \emph{+addParam(type: string, name: string)}\\
    		Aggiunge un parametro alla lista dei parametri del metodo\\
    		\textbf{Parametri:}
    		\begin{itemize}
    			\item \emph{type: string}\\
    			Tipo del parametro
    			\item \emph{name: string}\\
    			Nome del parametro
    		\end{itemize}
    		\item \emph{+removeParam(type: string, name: string)}\\
    		Rimuove un parametro dalla lista dei parametri del metodo\\
    		\textbf{Parametri:}
    		\begin{itemize}
    			\item \emph{type: string}\\
    			Tipo del parametro
    			\item \emph{name: string}\\
    			Nome del parametro
    		\end{itemize}
    		\item \emph{+addMetodo(nome: string, staticMet: boolean, constructor: boolean, tipo: string, acc: string)}\\
    		Aggiunge un nuovo metodo\\
    		\textbf{Parametri:}
    		\begin{itemize}
    			\item \emph{nome: string}\\
    			Nome del metodo
    			\item \emph{staticMet: boolean}\\
    			True se è marcato static
    			\item \emph{constructor: boolean}\\
    			True se è un costruttore
    			\item \emph{tipo: string}\\
    			Tipo di ritorno del metodo
    			\item \emph{acc: string}\\
    			Visibilità del metodo
    		\end{itemize}
    		\item \emph{+justOneCheckbox(event: any)}\\
    		Controlla che ci sia solo un elemento sulla checkbox\\
    		\textbf{Parametri:}
    		\begin{itemize}
    			\item \emph{event: any}\\
    			Nome dell'elemento
    		\end{itemize}
		\end{itemize}
\end{itemize}