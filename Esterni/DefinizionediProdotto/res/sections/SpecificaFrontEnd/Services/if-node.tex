\begin{figure}[h!]
	\centering
	\includegraphics[scale=0.8]{res/sections/SpecificaFrontEnd/Services/Disegnetti/if-node.png}
	\caption{Diagramma della classe If-node}
\end{figure}

\begin{itemize}
	\item \textbf{Descrizione:}\\
	Modello che rappresenta il nodo if nel diagramma delle attività
	\item \textbf{Utilizzo:}\\
	Utilizzato per la gestione e la generazione del codice del nodo if nel diagramma della attività
	\item \textbf{Attributi:}
		\begin{itemize}
			\item \emph{-succElse: string}\\
			Si riferisce all'else statement
		\end{itemize}
	\item \textbf{Metodi:}
		\begin{itemize}
			\item \emph{-constructor(id: string)}\\
    		Costruttore della classe\\
    		\textbf{Parametri:}
    		\begin{itemize}
    			\item \emph{id: string}\\
    			Id della shape
    		\end{itemize}
    		\item \emph{+getSuccElse()}\\
    		Ritorna il valore di succElse
    		\item \emph{+setSuccElse(succElse: string)}\\
    		Assegna un valore a succElse\\
    		\begin{itemize}
    			\item \emph{succElse: string}\\
    			Valoreda assegnare
    		\end{itemize}
    		\item \emph{+getType()}\\
    		Ritorna il tipo della classe
    		\item \emph{+toCode(sh: AllShape, code: string)}\\
    		Converte la shape in codice\\
    		\textbf{Parametri:}
    		\begin{itemize}
    			\item \emph{sh: AllShape}\\
    			Shape da convertire
    			\item \emph{code: string}\\
    			Stringa di codice
    		\end{itemize}
    	\end{itemize}
\end{itemize}