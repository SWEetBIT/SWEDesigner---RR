\begin{figure}[h!]
	\centering
	\includegraphics[scale=0.8]{res/sections/SpecificaFrontEnd/Services/Disegnetti/variabile.png}
	\caption{Diagramma della classe Variabile}
\end{figure}

\begin{itemize}
	\item \textbf{Descrizione:}\\
	Gestisce le variabili.
	\item \textbf{Utilizzo:}\\
	Viene utilizzato dal component padre per la gestione della varibili.
	\item \textbf{Metodi:}
		\begin{itemize}
			\item \emph{-type: string}\\
    		Rappresenta il tipo della variabile
    		\item \emph{-name: string}\\
    		Rappresenta il nome della variabile
    		\item \emph{-value: string}\\
    		Rappresenta il valore della variabile
		\end{itemize}
	\item \textbf{Metodi:}
		\begin{itemize}
			\item \emph{-constructor(type: string, name: string, value: string)}\\
    		Costruttore della classe\\
    		\textbf{Parametri:}
    		\begin{itemize}
    			\item \emph{type: string}\\
    			Inizializza il tipo
    			\item \emph{name: string}\\
    			Inizializza il nome
    			\item \emph{value: string}\\
    			Inizializza il valore
    		\end{itemize}
    		\item \emph{+getType()}\\
    		Ritorna il tipo della variabile
    		\item \emph{+getName()}\\
    		Ritorna il nome della variabile
    		\item \emph{+getValue()}\\
    		Ritorna il valore della variabile
    		\item \emph{+setType(type: string)}\\
    		Setta il tipo della variabile\\
    		\textbf{Parametri:}
    		\begin{itemize}
    			\item \emph{type: string}\\
    			Tipo della variabile
    		\end{itemize}
    		\item \emph{+setName(name: string)}\\
    		Setta il nome della variabile\\
    		\textbf{Parametri:}
    		\begin{itemize}
    			\item \emph{name: string}\\
    			Nome della variabile
    		\end{itemize}
    		\item \emph{+setValue(value: string)}\\
    		Setta il valore della variabile\\
    		\textbf{Parametri:}
    		\begin{itemize}
    			\item \emph{value: string}\\
    			Valore della variabile
    		\end{itemize}
    	\end{itemize}
\end{itemize}