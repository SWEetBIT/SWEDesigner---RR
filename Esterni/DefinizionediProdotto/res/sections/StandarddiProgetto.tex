\section{Standard di progetto}

	\subsection{Standard di progettazione architetturale}
Gli standard di progettazione sono definiti \emph{Specifica Tecnica v} \VersioneST{} .

	\subsection{Standard di documentazione del codice}
Gli standard per la scrittura della documentazione del codice sono definiti nelle Norme di Progetto \VersioneNP{}.

	\subsection{Standard di denominazione di entità e relazioni}
Tutti gli elementi definiti come package, classi, metodi o attributi, devono avere
denominazioni chiare ed esplicative. Il nome deve avere una lunghezza tale da non
pregiudicarne la leggibilità e chiarezza. È preferibile utilizzare dei sostantivi per le entità e
dei verbi per le relazioni. Le abbreviazioni sono ammesse se:
\begin{itemize}
\item immediatamente comprensibili;
\item non ambigue;
\item sufficientemente contestualizzate.
\end{itemize}

Le regole tipografiche relative ai nomi delle entità sono definite nelle \emph{Norme di Progetto v}\VersioneNP{}.

	\subsection{Standard di programmazione}
Gli standard di programmazione sono definiti e descritti nelle \emph{Norme di Progetto v}\VersioneNP{}.

	\subsection{Strumenti di lavoro}
Per gli strumenti di lavoro da utilizzare durante la codifica e le procedure per il loro corretto
funzionamento e coordinamento si rimanda al documento \emph{Norme di Progetto v}\VersioneNP{}.