\section{Specifica Front-End}
	\subsection{SWEDesigner::Client}
		\subsubsection{Informazioni generali}
			\begin{itemize}
          		\item \textbf{Descrizione:}\\
          		Questo package racchiude tutta la componente di Front-end scritta in TypeScript.
          		\item \textbf{Padre:} SWEDesigner
          		\item \textbf{Package contenuti:}\\
          		\begin{itemize}
          			\item Components\\
          			Questo package contiene tutti i components dell’applicazione
          			\item Services\\
          			Questo package contiene i servizi per le operazioni di iterazione tra i components e
il server
          		\end{itemize}
          	\end{itemize}
		\subsubsection{Classi}
	\subsection{SWEDesigner::Client::Components}
		\subsubsection{Informazioni generali}
			\begin{itemize}
          		\item \textbf{Descrizione:}\\
          		Questo package contiene tutti i components dell’applicazione.
          		\item \textbf{Padre:} SWEDesigner::Client
          		\item \textbf{Package contenuti:}\\
          		\begin{itemize}
          			\item Menu\\
          			Il package contiene tutti i components riguardanti la gestione delle funzionalità fornite dal menu.
          			\item Editor\\
          			Il package contiene tutte le components riguardanti l’editor dei diagrammi.
          			\item ActivityFrame\\
          			Il package contiene i components riguardanti la gestione dell’activity frame, per la visione del flusso del programma.
          		\end{itemize}
          	\end{itemize}
		\subsubsection{Classi}
			\paragraph{SWEDesigner::Client::Components::AppComponent}
				\begin{itemize}
          			\item \textbf{Descrizione:}\\
          			Questo component descrive un contenitore per la barra di navigazione e le altre componenti dell’applicazione le quali sono istanziate dinamicamente all’interno del template http.
          			\item \textbf{Utilizzo:}\\
          			AppComponent è il primo component che viene istanziato tramite bootsrap.
          			\item \textbf{Metodi:}\\
          		\end{itemize}
			\paragraph{SWEDesigner::Client::Components::NavbarComponent}
				\begin{itemize}
          			\item \textbf{Descrizione:}\\
          			Questo component permette la navigazione all’interno dell’applicazione tramite links.
          			\item \textbf{Utilizzo:}\\
          			NavbarComponent è istanziato per bootstrap subito dopo dell’AppComponent.
          			\item \textbf{Metodi:}\\
          		\end{itemize}
			\paragraph{SWEDesigner::Client::Components::RegistrationComponent}
				\begin{itemize}
          			\item \textbf{Descrizione:}\\
          			È il componente che descrive la pagina di registrazione dell’applicazione, mette a disposizione dell’utente un form dove iserire le informazioni necessarie alla creazione di un nuovo account utente. Gestisce le operazioni e la logica applicativa per la registrazione servendosi dei metodi forniti dal servizio AuthenticationService.
          			\item \textbf{Utilizzo:}\\
          			Questo componente viene istanziato dinamicamente dal servizio Router del framework Angular quando viene richiesta la pagina di registrazione.
          			\item \textbf{Metodi:}\\
          		\end{itemize}
			\paragraph{SWEDesigner::Client::Components::LoginComponent}
				\begin{itemize}
          			\item \textbf{Descrizione:}\\
          			È il componente che descrive la pagina di login dell’applicazione, mette a disposizione dell’utente un form dove inserire username e password. Gestisce le operazioni e la logica applicativa per il login servendosi dei metodi forniti dal servizio AuthenticationService.
          			\item \textbf{Utilizzo:}\\
          			Questo componente viene istanziato dinamicamente dal servizio Router del framework Angular qunado viene richiesta la pagina di login.
          			\item \textbf{Metodi:}\\
          		\end{itemize}
	\subsection{SWEDesigner::Client::Components::ActivityFrame}
		\subsubsection{Informazioni generali}
			\begin{itemize}
          		\item \textbf{Descrizione:}\\
          		Questo package contiene i components riguardanti la gestione dell’activity frame, per la visione del flusso del programma.
          		\item \textbf{Padre:} SWEDesigner::Client::Components
          	\end{itemize}
		\subsubsection{Classi}
			\paragraph{SWEDesigner::Client::Components::Editor::ActivityFrame::ActivityFrameComponent}
				\begin{itemize}
          			\item \textbf{Descrizione:}\\
          			Component che descrive la struttura del frame dove l’utente può visualizzare l’activity frame che rappresenta il flusso logico del programma.
          			\item \textbf{Utilizzo:}\\
          			Questo component viene istanziato per bootstrap dopo l’istanziazione del component AppComponent.
          			\item \textbf{Metodi:}\\
          		\end{itemize}
	\subsection{SWEDesigner::Client::Components::Editor}
		\subsubsection{Informazioni generali}
			\begin{itemize}
          		\item \textbf{Descrizione:}\\
          		Il package contiene tutte le components riguardanti l’editor dei diagrammi.
          		\item \textbf{Padre:} SWEDesigner::Client::Components
          	\end{itemize}
		\subsubsection{Classi}
			\paragraph{SWEDesigner::Client::Components::Editor::ClassMenuComponent}
				\begin{itemize}
          			\item \textbf{Descrizione:}\\
          			Questo component descrive il menu dal quale l’utente può selezionare gli strumenti per disegnare i diagrammi all’interno degli appositi frame. Si occupa delle operazioni e della parte logica, riguardante la costruzione dei diagrammi, servendosi dei metodi forniti delle API della libreria grafica.
          			\item \textbf{Utilizzo:}\\
          			Questo componente viene istanziato per bootstrap dopo che è stato istanziato il component AppComponent.
          			\item \textbf{Metodi:}\\
          		\end{itemize}
          	\paragraph{SWEDesigner::Client::Components::Editor::EditorComponent}
				\begin{itemize}
          			\item \textbf{Descrizione:}\\
          			Questo componente contiene la rappresentazione grafica dei diagrammi disegnati dall’utente.
          			\item \textbf{Utilizzo:}\\
          			Questo componente viene instanziato dinamicamente dal servizio Router del framework Angular quando viene richiesta la pagina dell’editor diagrammi.
          			\item \textbf{Metodi:}\\
          		\end{itemize}
          	\paragraph{SWEDesigner::Client::Components::Editor::ClassMenuComponent}
				\begin{itemize}
          			\item \textbf{Descrizione:}\\
          			Questo component permette la modifica dei campi dati di un oggetto selezionato nell’editorComponent.
          			\item \textbf{Utilizzo:}\\
          			Questo component è figlio di editorComponent viene visualizzato quando viene selezionato un elemento editabile nell’editorComponent.
          			\item \textbf{Metodi:}\\
          		\end{itemize}
			\paragraph{SWEDesigner::Client::Components::Editor::ToolbarComponent}
				\begin{itemize}
          			\item \textbf{Descrizione:}\\
          			La classe si occupa di fornire una toolbar per l'inserimento degli elementi del diagramma delle attività o del diagramma delle classi.
          			\item \textbf{Utilizzo:}\\
          			Ogni volta che viene selezionato un elemento esso viene inserito sul grafico. Nel caso dei connettori occorre selezionare, successivamente al connettore, i due elementi da collegare.
          			\item \textbf{Metodi:}\\
          			\begin{itemize}
          				\item \emph{+addClasse(): void}\\
          				Il metodo aggiunge una classe di nome "Classe" nell'area di disegno;
          				\item \emph{+addAstratta(): void}\\
          				Il metodo aggiunge una classe astratta di nome "ClasseAstratta" nell'area di disegno;
          				\item \emph{+addInterfaccia(): void}\\
          				Il metodo aggiunge un interfaccia di nome "Interfaccia" nell'area di disegno;
          				\item \emph{+addGeneralizzazione(): void}\\
          				Il metodo seleziona il tipo di connettore "Generalizzazione";
          				\item \emph{+addImplementazione(): void}\\
          				Il metodo seleziona il tipo di connettore "Implementazione";
          				\item \emph{+addCommento(): void}\\
          				Il metodo aggiunge un elemento di tipo "Commento" nell'area di disegno;
          				\item \emph{+addAssociazione(): void}\\
          				Il metodo seleziona il tipo di connettore "Associazione";
          				\item \emph{+addConnettore(cellView: any): void}\\
          				Il metodo serve, in caso venga selezionato un connettore, a selezionare i due elementi da collegare con il connettore selezionato con uno dei metodi precedenti.
          				\item \textbf{Parametri:}\\
            				\begin{itemize}
            					\item \emph{cellView: any}\\
            					Elemento da selezionare per essere collegato con il connettore selezionato
            				\end{itemize}
          			\end{itemize}
          		\end{itemize}
	\subsection{SWEDesigner::Client::Components::Menu}
		\subsubsection{Informazioni generali}
			\begin{itemize}
          		\item \textbf{Descrizione:}\\
          		Il package contiene tutti i components riguardanti la gestione delle funzionalità fornite dal menu.
          		\item \textbf{Padre:} SWEDesigner::Client::Components
          	\end{itemize}
		\subsubsection{Classi}
			\paragraph{SWEDesigner::Client::Components::Menu::MenuComponent}
				\begin{itemize}
          			\item \textbf{Descrizione:}\\
          			Component che contiene l’insieme di funzionalità fornite all’utente per la gestione dei progetti, dei propri dati personali, e della rappresentazione dei grafici su cui sta lavorando.
          			\item \textbf{Utilizzo:}\\
          			Component che viene istanziato per bootstrap dopo che è stato istanziato il component appComponent.
          			\item \textbf{Metodi:}\\
          		\end{itemize}
			\paragraph{SWEDesigner::Client::Components::Menu::FileComponent}
				\begin{itemize}
          			\item \textbf{Descrizione:}\\
          			Component che contiene l’insieme di funzionalità fornite all’utente per la gestione del progetto attualmente in uso.
          			\item \textbf{Utilizzo:}\\
          			Component che viene istanziato per bootstrap dopo che è stato istanziato il component menuComponent.
          			\item \textbf{Metodi:}\\
          		\end{itemize}
			\paragraph{SWEDesigner::Client::Components::Menu::LayerComponent}
				\begin{itemize}
          			\item \textbf{Descrizione:}\\
          			Component che contiene l’insieme di funzionalità fornite all’utente per la gestione dei layer del progetto in uso.
          			\item \textbf{Utilizzo:}\\
          			Component che viene istanziato per bootstrap dopo che è stato istanziato il component menuComponent.
          			\item \textbf{Metodi:}\\
          		\end{itemize}
			\paragraph{SWEDesigner::Client::Components::Menu::ProgettoComponent}
				\begin{itemize}
          			\item \textbf{Descrizione:}\\
          			Component che contiene l’insieme di funzionalità fornite all’utente per la gestione dei propri progetti salvati.
          			\item \textbf{Utilizzo:}\\
          			progettoComponent viene istanziato per bootstrap dopo che è stato istanziato il component menuComponent.
          			\item \textbf{Metodi:}\\
          		\end{itemize}
			\paragraph{SWEDesigner::Client::Components::Menu::ProfiloComponent}
				\begin{itemize}
          			\item \textbf{Descrizione:}\\
          			Component che contiene l’insieme di funzionalità fornite all’utente per la gestione dei propri dati personali.
          			\item \textbf{Utilizzo:}\\
          			Component che viene istanziato per bootstrap dopo che è stato istanziato il component menuComponent.
          			\item \textbf{Metodi:}\\
          		\end{itemize}
			\paragraph{SWEDesigner::Client::Components::Menu::ModificaComponent}
				\begin{itemize}
          			\item \textbf{Descrizione:}\\
          			Component che contiene l’insieme di funzionalità fornite all’utente per la modifica del progetto in uso, come ad esempio effettuare lo zoom, oppure eliminare o copiare un elemento selezionato.
          			\item \textbf{Utilizzo:}\\
          			Component che viene istanziato per bootstrap dopo che è stato istanziato il component menuComponent.
          			\item \textbf{Metodi:}\\
          		\end{itemize}
          	\paragraph{SWEDesigner::Client::Components::Menu::TemplateComponent}
				\begin{itemize}
          			\item \textbf{Descrizione:}\\
          			Component che contiene l’insieme di funzionalità fornite all’utente per l’importazione e gestione dei template.
          			\item \textbf{Utilizzo:}\\
          			Component che viene istanziato per bootstrap dopo che è stato istanziato il component menuComponent.
          			\item \textbf{Metodi:}\\
          		\end{itemize}
    \subsection{SWEDesigner::Client::Services}
		\subsubsection{Informazioni generali}
			\begin{itemize}
          		\item \textbf{Descrizione:}\\
          		Il package contiene i servizi per le operazioni di iterazione tra i component e il server.
          		\item \textbf{Padre:} SWEDesigner::Client
          		\item \textbf{Package contenuti:}\\
          		\begin{itemize}
          			\item Models\\
          			Il package contiene moduli necessari a storicizzare i dati inseriti all’interno dei diagrammi.
          		\end{itemize}
          	\end{itemize}
		\subsubsection{Classi}
			\paragraph{SWEDesigner::Client::Services::MenuService}
				\begin{itemize}
          			\item \textbf{Descrizione:}\\
          			Classe che definisce i metodi per le operazioni fornite all’utente dal menu.
          			\item \textbf{Utilizzo:}\\
          			É istaziata dal framework Angular e i suoi metodi sono utilizzati dal component menuComponent.
          			\item \textbf{Metodi:}\\
          		\end{itemize}
          	\paragraph{SWEDesigner::Client::Services::MainEditorService}
				\begin{itemize}
          			\item \textbf{Descrizione:}\\
          			Classe che definisce i metodi per le operazioni all’interno dei diagrammi e la comunicazione tra componenti e server.
          			\item \textbf{Utilizzo:}\\
          			É istaziata dal framework Angular e i suoi metodi sono utilizzati dai component editorComponent e classMenuComponent.
          			\item \textbf{Attributi:}\\
          			\begin{itemize}
          				\item \emph{-Project: Global}\\
          				Si utilizza per memorizzare e recuperare informazione riguardo il progetto corrente
          				\item \emph{-selectedClasse: Classe}\\
          				Memorizza la classe corrispondente di tipo "Classe" della classe selezionata nel canvas dell'editor
          				\item \emph{-editorComp: EditorComponent}\\
          				Si utilizza per accedere direttamente all'EditorComponent
          				\item \emph{-graph: JSON}\\
          				Si utilizza per per salvare il grafico dell'editor
          				\item \emph{-activityMode: boolean}\\
          				Indica se il diagramma delle attività è in uso
          			\end{itemize}
          			\item \textbf{Metodi:}\\
          			\begin{itemize}
          				\item \emph{+setEditorComp(editCmp: EditorComponent): void}\\
          				Questo metodo viene usato per l'istanziazione dell'EditorComponent come proprietà interna di questa classe
          				\item \textbf{Parametri:}\\
            				\begin{itemize}
            					\item \emph{editCmp: EditorComponent}\\
            					L'istanza dell'EditorComponent
            				\end{itemize}
          				\item \emph{+getClassList(): Classe[]}\\
          				Questo metodo viene usato per richiamare l'array di classi presente nel progetto
          				\item \emph{+getSelectedClasse(): void}\\
          				Questo metodo ritorna la classe selezionata di tipo "Classe"
          				\item \emph{+addClass(classe: Classe, graphElement: any): void}\\
          				Questo metodo aggiunge un oggetto di tipo classe nell'array di classi del progetto
          				\item \textbf{Parametri:}\\
            				\begin{itemize}
            					\item \emph{classe: Classe}\\
            					Questo oggetto è una rappresentazione, di tipo "Classe" o "ClasseAstratta", del parametro graphelement
            					\item \emph{graphElement: any}\\
            					Questo è un elemento della libreria grafica JointJs
            				\end{itemize}
          				\item \emph{+selectClasse(nome: string): Classe}\\
          				Questo metodo cerca, all'interno della collezione di classi del progetto, una classe con lo stesso nome di quello fornito come parametro
          				\item \textbf{Parametri:}\\
            				\begin{itemize}
            					\item \emph{nome: string}\\
            					Nome della classe da cercare
            				\end{itemize}
          				\item \emph{+setActivityMode(): void}\\
          				Questo metodo setta a True il valore di activityMode
          				\item \emph{+setClassMode(): void}\\
          				Questo metodo setta a False il valore di activityMode
          				\item \emph{+getActivityModeStatus(): boolean}\\
          				Questo metodo ritorna il valore di activityMode
          				\item \emph{+addAttributo(tipo: string, nome: string, acc: string): void}\\
          				Questo metodo richiama il metodo addAttributo della "selectedClasse"
          				\item \textbf{Parametri:}\\
            				\begin{itemize}
            					\item \emph{tipo: string}\\
            					Il tipo dell'attributo da aggiungere con addAttributo
            					\item \emph{nome: string}\\
            					Il nome dell'attributo da aggiungere con addAttributo
            					\item \emph{acc: string}\\
            					La visibilità dell'attributo da aggiungere con addAttributo
            				\end{itemize}
            			\item \emph{+removeAttributo(nome: string): void}\\
          				Questo metodo richiama il metodo removeAttr della "selectedClasse"
          				\item \textbf{Parametri:}\\
            				\begin{itemize}
            					\item \emph{nome: string}\\
            					Il nome dell'attributo da rimuovere
            				\end{itemize}
            			\item \emph{+storeGraph(graph: JSON): void}\\
          				Questo metodo salva in "this.graph" il grafico passato come parametro
          				\item \textbf{Parametri:}\\
            				\begin{itemize}
            					\item \emph{graph: JSON}\\
            					Un grafico in formato JSON
            				\end{itemize}
            			\item \emph{+enterClassMode(): void}\\
          				Questo metodo viene utilizzato per ripristinare il diagramma delle classi memorizzato in "this.graph"
          				\item \emph{+addMetodo(tipo: string, nome: string, acc: string, listArgs?: any): void}\\
          				Questo metodo aggiunge un nuovo metodo alla "selectedClasse"
          				\item \textbf{Parametri:}\\
            				\begin{itemize}
            					\item \emph{tipo: string}\\
            					Tipo di ritorno del metodo
            					\item \emph{nome: string}\\
            					Nome del metodo
            					\item \emph{acc: string}\\
            					La visibilità del metodo
            					\item \emph{listArgs?: any}\\
            					Lista dei parametri del metodo, se ce ne sono
            				\end{itemize}
            			\item \emph{+removeMetodo(nome: string): void}\\
          				Questo metodo richiama il metodo removeMetodo della "selectedClasse"
          				\item \textbf{Parametri:}\\
            				\begin{itemize}
            					\item \emph{nome: string}\\
            					Nome del metodo da eliminare
            				\end{itemize}
            			\item \emph{+enterActivityMode(name: string): void}\\
          				Questo metodo cerca un metodo nella "selectedClasse" e recupera il suo diagramma per chiamare il metodo replaceDiagram dell'editorComp, il quale carica i metodi del diagramma in Canvas
          				\item \textbf{Parametri:}\\
            				\begin{itemize}
            					\item \emph{name: string}\\
            					Nome del metodo da trovare
            				\end{itemize}
          			\end{itemize}
          		\end{itemize}
          	\paragraph{SWEDesigner::Client::Services::ToolbarService}
				\begin{itemize}
          			\item \textbf{Descrizione:}\\
          			Classe che definisce i metodi per le operazioni di inserimento di nuovi elementi all’interno dell’editor di diagrammi.
          			\item \textbf{Utilizzo:}\\
          			É istaziata dal framework Angular e i suoi metodi sono utilizzati dal component editorComponent.
          			\item \textbf{Metodi:}\\
          		\end{itemize}
          	\paragraph{SWEDesigner::Client::Services::ActivityFrameService}
				\begin{itemize}
          			\item \textbf{Descrizione:}\\
          			Classe che definisce i metodi per le operazioni di navigazione tra i metodi all’interno dell’activity frame.
          			\item \textbf{Utilizzo:}\\
          			É istaziata dal framework Angular e i suoi metodi sono utilizzati dal component activityFrameComponent.
          			\item \textbf{Metodi:}\\
          		\end{itemize}
          	\paragraph{SWEDesigner::Client::Services::ClassMenuService}
				\begin{itemize}
          			\item \textbf{Descrizione:}\\
          			Classe che definisce i metodi per le operazioni di modifica di un elemento selezionato all’interno del diagramma rappresentato.
          			\item \textbf{Utilizzo:}\\
          			É istaziata dal framework Angular e i suoi metodi sono utilizzati dal component classMenuService.
          			\item \textbf{Metodi:}\\
          		\end{itemize}
          	\paragraph{SWEDesigner::Client::Services::AccountService}
				\begin{itemize}
          			\item \textbf{Descrizione:}\\
          			Classe che definisce i metodi di registrazione, login e recupero dati utente dal server.
          			\item \textbf{Utilizzo:}\\
          			É istaziata dal framework Angular e i suoi metodi sono utilizzati dai component registrationComponent e loginComponent.
          			\item \textbf{Metodi:}\\
          		\end{itemize}
	\subsection{SWEDesigner::Client::Services::Models}
		\subsubsection{Informazioni generali}
			\begin{itemize}
          		\item \textbf{Descrizione:}\\
          		Il package contiene moduli necessari a storicizzare i dati inseriti all’interno
dei diagrammi.
          		\item \textbf{Padre:} SWEDesigner::Client::Services
          	\end{itemize}
		\subsubsection{Classi}
			\paragraph{SWEDesigner::Client::Services::Param}
				\begin{itemize}
          			\item \textbf{Descrizione:}\\
          			Classe che definisce i metodi di settaggio e richiesta dei parametri nome e tipo.
          			\item \textbf{Utilizzo:}\\
          			É istaziata dal framework Angular e i suoi metodi sono utilizzati dal model attributo.
          			\item \textbf{Metodi:}\\
          		\end{itemize}
          	\paragraph{SWEDesigner::Client::Services::Attributo}
				\begin{itemize}
          			\item \textbf{Descrizione:}\\
          			Classe derivata da Param che definisce i metodi di settaggio e richiesta dei parametri di visibilità.
          			\item \textbf{Utilizzo:}\\
          			É istaziata dal framework Angular e i suoi metodi sono utilizzati dal model classe.
          			\item \textbf{Metodi:}\\
          		\end{itemize}
          	\paragraph{SWEDesigner::Client::Services::Metodo}
				\begin{itemize}
          			\item \textbf{Descrizione:}\\
          			Classe che definisce i metodi di settaggio e richiesta dei metodi definiti all’interno dei diagrammi.
          			\item \textbf{Utilizzo:}\\
          			É istaziata dal framework Angular e i suoi metodi sono utilizzati dal model classe.
          			\item \textbf{Metodi:}\\
          		\end{itemize}
          	\paragraph{SWEDesigner::Client::Services::Classe}
				\begin{itemize}
          			\item \textbf{Descrizione:}\\
          			Classe che definisce i metodi di settaggio e richiesta di tutti gli elementi che sono contenuti in una classe. Contiene un array di metodi, con le relative rappresentazioni grafiche dei metodi implementati, e un array di attributi, oltre ai campi utili all’identificazione della classe.
          			\item \textbf{Utilizzo:}\\
          			É istaziata dal framework Angular e i suoi metodi sono utilizzati dal model global.
          			\item \textbf{Metodi:}\\
          		\end{itemize}
          	\paragraph{SWEDesigner::Client::Services::ClasseAstratta}
				\begin{itemize}
          			\item \textbf{Descrizione:}\\
          			Classe derivata da classe che definisce i metodi di settaggio e richiesta dei parametri di una classe astratta.
          			\item \textbf{Utilizzo:}\\
          			É istaziata dal framework Angular e i suoi metodi sono utilizzati dal model global.
          			\item \textbf{Metodi:}\\
          		\end{itemize}
          	\paragraph{SWEDesigner::Client::Services::Interface}
				\begin{itemize}
          			\item \textbf{Descrizione:}\\
          			Classe derivata da classe che definisce i metodi di settaggio e richiesta dei parametri di una interface.
          			\item \textbf{Utilizzo:}\\
          			É istaziata dal framework Angular e i suoi metodi sono utilizzati dal model global.
          			\item \textbf{Metodi:}\\
          		\end{itemize}
          	\paragraph{SWEDesigner::Client::Services::Global}
				\begin{itemize}
          			\item \textbf{Descrizione:}\\
          			Classe che definisce i metodi di settaggio e richiesta di tutte le classi contenenti nel diagramma delle classi.
          			\item \textbf{Utilizzo:}\\
          			É istaziata dal framework Angular e i suoi metodi sono utilizzati dal servizio editorService.
          			\item \textbf{Metodi:}\\
          		\end{itemize}