\section{Specifica Back-End}
  \subsection{SWEDesigner::Server}
    \subsubsection{Informazioni generali}
      \begin{itemize}
        \item \textbf{Descrizione:}\\
        Questo package contiente tutte le componenti del server scritte in JavaScript
        \item \textbf{Padre: } SWEDesigner
        \item \textbf{Package contenuti:}\\
        \begin{itemize}
          \item Controller \\
          Questo package contiene al suo interno tutti i controller che implementano il pattern MVVM fornito da \glossaryItem{Angular.js}.
          In particolare sono contenuti i Middleware e tutti i Servizi da essi utilizzati.
          \item Model \\
          Questo package contiene tutte le classi utili per la creazione del database, la connessione ad esso e le relative interrogazioni.
        \end{itemize}
      \end{itemize}
    \subsubsection{Classi}
      \subsubsubsection{SWEDesigner::Server::serverLoader}
        \begin{itemize}
          \item \textbf{Descrizione:}\\
          Classe che consente il caricamento di tutte le componenti e gli elementi utili al primo avvio dell'applicazione
          \item \textbf{Utilizzo:}\\
          La classe viene utilizzata per il caricamento del server e di tutti i suoi elementi.
          \item \textbf{Metodi:}\\
          \begin{itemize}
            \item \emph{+ load(db: string, mR: string, mu: string, encr: string, cb: function): void}\\
            Si tratta della funzione principale che si occupa di chiamare i metodi load contenuti in tutte le altre classi.
            \item \textbf{Parametri:}\\
            \begin{itemize}
              \item \emph{db: string}\\
              Il path del modulo che gestisce la connessione al database.
              \item \emph{mR: string}\\
              Il path del modulo che gestisce le query.
              \item \emph{mu: string}\\
              Il path del modulo che gestisce il servizio di parsing.
              \item \emph{encr: string}\\
              Il path del modulo che gestisce il servizio di encrypt.
              \item \emph{cb: function}taliano
              Callback che gestisce le rischieste asicnrone al database.
            \end{itemize}
            \item \textbf{- loadCryptParam(db: string, cb: function): void}\\
            Si tratta della funzione utilizzata da load per la richiesta dei parametri crittografici al database.
            \item \textbf{Parametri:}\\
            \begin{itemize}
              \item \emph{db: string}\\
              Il path del modulo che gestisce la connessione al database.
              \item \emph{cb: function}
              Callback che gestisce le rischieste asicnrone al database.
            \end{itemize}
          \end{itemize}
        \end{itemize}
  \subsection{SWEDesigner::Server::Model}
    \subsubsection{Informazioni generali}
      \begin{itemize}
        \item \textbf{Descrizione:}\\
        Questo package contiene tutte le classi e le funzionalità legate al database.
        \item \textbf{Padre: }SWEDesigner::Server
      \end{itemize}
    \subsubsection{Classi}
      \subsubsubsection{SWEDesigner::Server::Model::mongooseConnection}
        \begin{itemize}
          \item \textbf{Descrizione: }\\
          Classe che si occupa della connessione al database e degli errori che ne possono derivare
          \item \textbf{Utilizzo: }\\
          La classe viene utilizzata per effettuare la connessione al database all'avvio dell'applicazione.
          \item \textbf{Metodi: }\\
          \begin{itemize}
            \item \emph{+ conn() : void}\\
            Si tratta della funzione che effettua la connessione al database e ne gestisce gli eventuali errori derivanti.
          \end{itemize}
        \end{itemize}
      \subsubsubsection{SWEDesiger::Server::Model::mongooseRequest}
        Tutte le query riguardanti l'aggiornamento e la cancellazione di dati dal database verranno trattate nella successiva versione di questo documento.
        \begin{itemize}
          \item \textbf{Descrizione: }\\
          Classe che si occupa di gestire tutte le query da e vero il database.
          \item \textbf{Utilizzo: }\\
          La classe viene utilizzata per tutte le richieste, inserimento e fetch, di dati dal e nel database.
          \item \textbf{Metodi: }\\
          \begin{itemize}
            \item emph{+ins_usr(usr: Object, cb: function) : void}\\
            Si tratta della funzione che si occupa di inserire un utente all'interno del database.
            \item \textbf{Parametri: }\\
            \begin{itemize}
              \item emph{usr: Object}\\
              L'utente, in formato JSON, da inserire all'interno dello schema.
              \item emph{cb: function}\\
              Callback che gestisce le richieste asincrone al database.
            \end{itemize}
            \item emph{+ins_proj(proj: Object, cb: function) : void}\\
            Si tratta della funzione che si occupa di inserire un progetto all'interno del database.
            \item \textbf{Parametri: }\\
            \begin{itemize}
              \item emph{proj: Object}\\
              Il progetto, in formato JSON, da inserire all'interno dello schema.
              \item emph{cd: function}\\
              Callback che gestisce le richieste asincrone al database.
            \end{itemize}
            \item emph{+ins_crypt_param(k: string, i: string, cb: function) : void}\\
            Si tratta della funzione che si occupa di inserire una chiave crittografica all'interno del database.
            \item \textbf{Parametri: }\\
            \begin{itemize}
              \item emph{k: string}\\
              La chiave crittografica.
              \item emph{i: string}\\
              Valore iv per la crittografia.
              \item emph{cb: function}\\
              Callback che gestisce le richieste asincrone al database.
            \end{itemize}
            \item emph{+load_all_proj(username: string, cb: function) : void}\\
            Si tratta della funzione che si occupa di richiedere tutti i progetti di un dato utente.
            \item \textbf{Parametri: }\\
            \begin{itemize}
              \item emph{username: string}\\
              Nome dell'utente di cui sono richiesti i progetti.
              \item emph{cd: function}\\
              Callback che gestisce le richieste asincrone al database.
            \end{itemize}
            \item emph{+load_key_crypt(cb: function) : void}\\
            Si tratta della funzione che si occupa di richiedere l'unica chiave crittografica salvata nel database.
            \item \textbf{Parametri: }\\
            \begin{itemize}
              \item emph{cb: function}\\
              Callback che gestisce le richieste asincrone al database.
            \end{itemize}
            \item emph{+load_proj(projectName: string, cb: function) : void }\\
            Si tratta della funzione che si occupa di cercare e ritornare un dato progetto.
            \item \textbf{Parametri: }\\
            \begin{itemize}
              \item emph{projectName: string}\\
              Nome del progetto richiesto
              \item emph{cb: function}\\
              Callback che gestisce le richieste asincrone al database.
            \end{itemize}
            \item emph{+login(username: string, password: string, cb: function) : void}\\
            Si tratta della funzione che verifica che l'utente che cerca di loggare esiste all'interno del database.
            \item \textbf{Parametri: }\\
            \begin{itemize}
              \item emph{username: string}\\
              L'username dell'utente che cerca di loggare.
              \item emph{password: string}\\
              La password dell'utente che cerca di loggare.
              \item emph{cb: function}\\
              Callback che gestisce le richieste asincrone al database.
            \end{itemize}
            \item emph{+forgot_mail(username: string, cd: function)}\\
            Si tratta della funzione che restituisce la mail dell'utente dato.
            \item \textbf{Parametri: }\\
            \begin{itemize}
              \item emph{username: string}\\
              Nome dell'utente
              \item emph{cb: function}\\
              Callback che gestisce le richieste asincrone al database.
            \end{itemize}
            \item emph{+drop_schema() : void}//
            Si tratta della funzione che elimina il database.
          \end{itemize}
        \end{itemize}
