\section{Organigramma}
	\subsection{Redazione}
	\subsection{Approvazione}
	\subsection{Accettazione dei componenti}
	\subsection{Componenti}
	\subsection{Definizione dei ruoli}
		Durante lo sviluppo del progetto ogni membro andrà a ricoprire diversi ruoli. Ciascun ruolo rappresenta una figura aziendale specializzata, indispensabile per il buon esito del progetto. Ogni componente dovrà ricoprire almeno una volta ciascun ruolo e i ruoli saranno suddivisi in modo da massimizzare l'efficienza del gruppo. Occorre inoltre verificare che non vi siano periodi in cui una risorsa sia verificatrice di se stessa. \\
		I ruoli sono:
		\begin{itemize}
			\item \textbf{Responsabile di Progetto (PM):} rappresenta il progetto, in quanto accentra su di sé le responsabilità di scelta ed approvazione, ed il gruppo, in quanto presenta al Committente i risultati del progetto. Detiene il potere decisionale, quindi la responsabilità su: \\
			\begin{itemize}
				\item Pianificazione, coordinamento e controllo delle attività;
				\item Gestione e controllo delle risorse;
				\item Analisi e gestione dei rischi;
				\item Approvazione dei documenti;
				\item Approvazione dell'offerta economica. \\
			\end{itemize}
			Si occupa di redigere il \textit{Piano di Progetto} e collabora alla stesura del \textit{Piano di Qualifica};
			\item \textbf{Amministratore (Am):} è responsabile del controllo, dell'efficienza e dell'operatività dell'ambiente di lavoro. Le mansioni di primaria importanza che gli competono sono: \\
			\begin{itemize}
				\item Ricerca di strumenti che possano automatizzare qualsiasi compito che possa
essere tolto all'umano; \\
				\item Risoluzione dei problemi legati alle difficoltà di gestione e controllo dei processi
e delle risorse. La risoluzione di tali problemi richiede l'adozione di strumenti adatti;
				\item Controllo delle versioni e delle configurazioni del prodotto;
				\item Gestione dell'archiviazione e del versionamento della documentazione di progetto;
				\item Fornire procedure e strumenti per il monitoraggio e segnalazione per il controllo
qualità. \\
			\end{itemize}
			Redige le \textit{Norme di Progetto}, dove spiega e norma l'utilizzo degli strumenti, e la sezione del \textit{Piano di Qualifica} nella quale vengono descritti gli strumenti e i metodi di verifica;
			\item \textbf{Analista (An):} è responsabile delle attività di analisi. Le responsabilità di spicco per tale ruolo sono: \\
			\begin{itemize}
				\item Produrre una specifica di progetto comprensibile, sia per il proponente, sia per il committente che per i progettisti, e motivata in ogni suo punto;
				\item  Comprendere appieno la natura e la complessità del problema. \\
			\end{itemize}
			Redige lo \textit{Studio di Fattibilità}, l'\textit{Analisi dei Requisiti} e parte del \textit{Piano di Qualifica}. Partecipa alla redazione del \textit{Piano di Qualifica} in quanto conosce l'ambito del progetto ed ha chiari i livelli di qualità richiesta e le procedure da applicare per ottenerla;
			\item \textbf{Progettista (Pt):} è responsabile delle attività di progettazione.
Le responsabilità di tale ruolo sono: \\
			\begin{itemize}
				\item Produrre una soluzione attuabile, comprensibile e motivata;
				\item Effettuare scelte su aspetti progettuali che applichino al prodotto soluzioni
note ed ottimizzate;
				\item Effettuare scelte su aspetti progettuali e tecnologici che rendano il prodotto
facilmente manutenibile. \\
			\end{itemize}
			Redige la \textit{Specifica Tecnica}, la \textit{Definizione di Prodotto} e le sezioni inerenti le metriche di verifica della programmazione del \textit{Piano di Qualifica};
			\item \textbf{Programmatore (Pr):} è responsabile delle attività di codifica e delle componenti di ausilio necessarie per l'esecuzione delle prove di verifica e validazione. Le responsabilità di tale ruolo sono: \\
			\begin{itemize}
				\item Implementare rigorosamente le soluzioni descritte dal progettista, da cui
seguirà quindi la realizzazione del prodotto;
				\item Scrivere codice: documentato, versionato, manutenibile e che rispetti gli
standard stabiliti per la scrittura del codice;
				\item Implementare i test sul codice scritto, necessari per prove di verifica e validazione. \\
			\end{itemize}
			Redige il \textit{Manuale Utente} e produce una abbondante documentazione del codice.
			\item \textbf{Verificatore (Ve):} è responsabile delle attività di verifica. Le responsabilità di tale ruolo sono: \\
			\begin{itemize}
				\item Assicurare che l'attuazione delle attività sia conforme alle norme stabilite;
				\item Controllare la conformità di ogni stadio del ciclo di vita del prodotto. \\
			\end{itemize}
			Redige la sezione del \textit{Piano di Qualifica} che illustra l'esito e la completezza delle verifiche e delle prove effettuate;
		\end{itemize}
		Ciascun ruolo ha un costo orario, come riportato nella tabella seguente. \\
		\begin{table}[H]
		\centering
		\begin{tabular}{|c|c|}
			\hline
			\textbf{Ruolo}		& \textbf{Costo} \\
			\hline
			Project Manager		& 30 €\\
			Amministratore		& 20 €\\
			Analista			& 25 €\\
			Progettista			& 22 €\\
			Programmatore		& 15 €\\
			Verificatore		& 15 €\\
			\hline
		\end{tabular}
		\caption{Costo orario per ruolo}
		\end{table}