\section{Consuntivo}
Questa sezione contiene il prospetto economico che riporta i consuntivi dei vari periodi. Vengono riportate le ore impiegate per svolgere i compiti preventivati, sia per ruolo che per persona. In base alla differenza di ore tra il preventivo e il consuntivo, detta conguaglio, avremmo un bilancio: \\
\begin{itemize}
	\item \textbf{Positivo:} Il preventivo ha superato il consuntivo;
	\item \textbf{Negativo:} Il preventivo è inferiore al consuntivo;
	\item \textbf{In pari:} Consuntivo e preventivo coincidono. \\
\end{itemize}
	\subsection{Analisi}
	Essendo il periodo di \textbf{Analisi} considerato periodo di investimento, il consuntivo viene presentato a scopo informativo ma non conteggiato nel preventivo a finire.
	\subsubsection{Consuntivo}
	Di seguito è presentata la tabella contenente i dati del consuntivo per il periodo di \textbf{Analisi}.
	\begin{table}[H]
		\centering
		\begin{tabular}{|c|c|c|}
			\hline
			\textbf{Ruolo}		& \textbf{Ore}	& \textbf{Costo} \\
			\hline
			Project Manager		& +1		& +30	\\
			Amministratore		& +1		& +20	\\
			Analista			& -3		& -75	\\
			Progettista			& 0			& 0	\\
			Programmatore		& 0			& 0	\\
			Verificatore		& 0			& 0	\\
			\hline
			\textbf{Totale}		& -1		& -25	\\
			\hline
		\end{tabular}
		\caption{Costo per ruolo, periodo di Analisi}
	\end{table}
	Nella tabella seguente sono riportate le differenze tra le ore di lavoro previste per ogni componente e quelle realmente impiegate.
	\begin{table}[H]
		\centering
		\begin{tabular}{|c|c|c|c|c|c|c|c|}
			\hline
			\textbf{Nominativo}		& \textbf{PM}	& \textbf{Am}	& \textbf{An}	& \textbf{Pt}	& \textbf{Pr}	& \textbf{Ve}	& \textbf{Ore totali}     \\
			\hline
			Salvatore Pilò			& +1	& 		& 		&		&		&		& +1 \\
			Fabio Massignan			&		& +1	&		&		&		& 		& +1 \\
			Sebastiano Bertolin		&		& 	 	& -2	&		&		&		& -2 \\
			Davide Santimaria		&		& 		& -1	&		&		&		& -1 \\
			Malick Bodian			& 		&		&		&		&		& +1	& +1 \\
			Gianmarco Salmistraro	&		&		& 	 	&		&		& -1	& -1 \\
			\hline
		\end{tabular}
		\caption{Differenza consuntivo preventivo per componente, periodo di Analisi}
	\end{table}
	\subsubsection{Conclusioni}
	Durante il periodo di \textbf{Analisi} si è riusciti a risparmiare delle ore per quanto riguarda i ruoli di \textit{Responsabile di Progetto} e \textit{Amministratore}, mentre si è reso necessario l'impiego di un numero maggiore di ore, rispetto a quelle previste, per il ruolo di \textit{Analista}. Per quanto riguarda i ruoli di \textit{Verificatore}, le ore stimate sono risultate sufficienti. Il risultato finale del periodo è complessivamente di un'ora lavorativa oltre il previsto, con una spesa aggiuntiva di 25€.
	\subsection{Consolidamento dei Requisiti}
	Essendo il periodo di \textbf{Consolidamento dei Requisiti} considerato periodo di investimento, il consuntivo viene presentato a scopo informativo ma non conteggiato nel preventivo a finire.
		\subsubsection{Consuntivo}
		Di seguito è presentata la tabella contenente i dati del consuntivo per il periodo di \textbf{Consolidamento dei Requisiti}.
	\begin{table}[H]
		\centering
		\begin{tabular}{|c|c|c|}
			\hline
			\textbf{Ruolo}		& \textbf{Ore}	& \textbf{Costo} \\
			\hline
			Project Manager		& 0			& 0	\\
			Amministratore		& 0			& 0	\\
			Analista			& +1		& +25 \\
			Progettista			& 0			& 0	\\
			Programmatore		& 0			& 0	\\
			Verificatore		& 0			& 0	\\
			\hline
			\textbf{Totale}		& +1		& +25 \\
			\hline
		\end{tabular}
		\caption{Costo per ruolo, periodo di Consolidamento dei Requisiti}
	\end{table}
	Nella tabella seguente sono riportate le differenze tra le ore di lavoro previste per ogni componente e quelle realmente impiegate.
	\begin{table}[H]
		\centering
		\begin{tabular}{|c|c|c|c|c|c|c|c|}
			\hline
			\textbf{Nominativo}		& \textbf{PM}	& \textbf{Am}	& \textbf{An}	& \textbf{Pt}	& \textbf{Pr}	& \textbf{Ve}	& \textbf{Ore totali}     \\
			\hline
			Salvatore Pilò			& 		& 		& +1	&		&		&		& +1 \\
			Fabio Massignan			&		& 		&		&		&		& 		& 0	 \\
			Sebastiano Bertolin		&		& 	 	& -1	&		&		&		& -1 \\
			Davide Santimaria		&		& 		& 		&		&		&		& 0	 \\
			Malick Bodian			& 		&		& +1	&		&		& 		& +1 \\
			Gianmarco Salmistraro	&		&		& 	 	&		&		& 		& 0	 \\
			\hline
		\end{tabular}
		\caption{Differenza consuntivo preventivo per componente, periodo di Consolidamento dei Requisiti}
	\end{table}
		\subsubsection{Conclusioni}
		Durante il periodo di \textbf{Consolidamento dei Requisiti} si è riusciti a risparmiare un ora per quanto riguarda il ruolo di \textit{Analista}. Per quanto riguarda i ruoli di \textit{Responsabile di Progetto}, \textit{Amministratore} e \textit{Verificatore} le ore stimate sono risultate sufficienti. Il risultato finale del periodo è complessivamente di un ora lavorativa in meno rispetto al previsto, risparmiando così 25€ e bilanciando il deficit prodotto dal periodo di \textbf{Analisi}.
	\subsection{Progettazione Architetturale}
		\subsubsection{Consuntivo}
		Di seguito è presentata la tabella contenente i dati del consuntivo per il periodo di \textbf{Progettazione Architetturale}.
	\begin{table}[H]
		\centering
		\begin{tabular}{|c|c|c|}
			\hline
			\textbf{Ruolo}		& \textbf{Ore}	& \textbf{Costo} \\
			\hline
			Project Manager		& 0			& 0	\\
			Amministratore		& +1		& +20 \\
			Analista			& -1		& -25 \\
			Progettista			& +2		& +44 \\
			Programmatore		& 0			& 0	\\
			Verificatore		& +1		& +15 \\
			\hline
			\textbf{Totale}		& +3		& +54 \\
			\hline
		\end{tabular}
		\caption{Costo per ruolo, periodo di Progettazione Architetturale}
	\end{table}
	Nella tabella seguente sono riportate le differenze tra le ore di lavoro previste per ogni componente e quelle realmente impiegate.
	\begin{table}[H]
		\centering
		\begin{tabular}{|c|c|c|c|c|c|c|c|}
			\hline
			\textbf{Nominativo}		& \textbf{PM}	& \textbf{Am}	& \textbf{An}	& \textbf{Pt}	& \textbf{Pr}	& \textbf{Ve}	& \textbf{Ore totali}     \\
			\hline
			Salvatore Pilò			& 		& 		& -1	& +2	&		&		& +1 \\
			Fabio Massignan			&		& 		&		&		&		& +2	& +2 \\
			Sebastiano Bertolin		&		& 	 	& 		& -2	&		&		& -2 \\
			Davide Santimaria		&		& 		& +1	&		&		& -1	& 0	 \\
			Malick Bodian			& 		& +1	& 		& +2	&		& 		& +3 \\
			Gianmarco Salmistraro	&		&		& -1 	&		&		& 		& -1 \\
			\hline
		\end{tabular}
		\caption{Differenza consuntivo preventivo per componente, periodo di Progettazione Architetturale}
	\end{table}
		\subsubsection{Conclusioni}
		Durante il periodo di \textbf{Progettazione Architetturale} si è presentata la necessità di impiegare alcune ore per la correzione dei vari documenti e ciò ha comportato un ora aggiuntiva di lavoro per il ruolo di \textit{Analista}. Sono state invece risparmiate delle ore per quanto riguarda i ruoli di \textit{Amministratore}, \textit{Progettista} e \textit{Verificatore}. Per quanto riguarda il ruolo di \textit{Responsabile di Progetto}, le ore preventivate sono risultate sufficienti. Il risultato finale del periodo è complessivamente di tre ore lavorative in meno rispetto al previsto, con un risparmio di 54€.
\subsection{Progettazione di Dettaglio e Codifica}
		\subsubsection{Consuntivo}
		Di seguito è presentata la tabella contenente i dati del consuntivo per il periodo di \textbf{Progettazione di Dettaglio e Codifica}.
	\begin{table}[H]
		\centering
		\begin{tabular}{|c|c|c|}
			\hline
			\textbf{Ruolo}		& \textbf{Ore}	& \textbf{Costo} \\
			\hline
			Project Manager		& 0			&	\\
			Amministratore		& +2		& +40	\\
			Analista			& 0			&  \\
			Progettista			& +8		& +176	\\
			Programmatore		& +8		& +200	\\
			Verificatore		& +10		& +150	\\
			\hline
			\textbf{Totale}		& +28		& +466	\\
			\hline
		\end{tabular}
		\caption{Costo per ruolo, periodo di Progettazione di Dettaglio e Codifica}
	\end{table}
	Nella tabella seguente sono riportate le differenze tra le ore di lavoro previste per ogni componente e quelle realmente impiegate.
	\begin{table}[H]
		\centering
		\begin{tabular}{|c|c|c|c|c|c|c|c|}
			\hline
			\textbf{Nominativo}		& \textbf{PM}	& \textbf{Am}	& \textbf{An}	& \textbf{Pt}	& \textbf{Pr}	& \textbf{Ve}	& \textbf{Ore totali}     \\
			\hline
			Salvatore Pilò			& 		& 		& 		& +2	&		& +2	& +4 \\
			Fabio Massignan			& 		& 		&		&		& +2	& 		& +2 \\
			Sebastiano Bertolin		&		& 	 	& 		& +2	& 		&		& +2 \\
			Davide Santimaria		&		& 		&		&		& +3	& 		& +3 \\
			Malick Bodian			& 		& +2	& 		& +4	&		& +8	& +14 \\
			Gianmarco Salmistraro	&		&		& 	 	&		& +3	& 		& +3 \\
			\hline
		\end{tabular}
		\caption{Differenza consuntivo preventivo per componente, periodo di Verifica e Validazione}
	\end{table}
		\subsubsection{Conclusioni}
		Durante il periodo di \textbf{Progettazione di Dettaglio e Codifica} si sono verificati una serie di rischi che hanno impedito ai componenti del gruppo di svolgere il proprio lavoro come pianificato, pertanto le ore che sono state risparmiate, le quali hanno generato un risparmio di 466€, verranno reinvestite nel prossimo periodo e la differenza tra consuntivo e preventivo è destinata a diminuire.
	\subsection{Verifica e Validazione}
	\subsubsection{Consuntivo}
		Di seguito è presentata la tabella contenente i dati del consuntivo per il periodo di \textbf{Progettazione di Dettaglio e Codifica}.
	\begin{table}[H]
		\centering
		\begin{tabular}{|c|c|c|}
			\hline
			\textbf{Ruolo}		& \textbf{Ore}	& \textbf{Costo} \\
			\hline
			Project Manager		& +4		& +120 \\
			Amministratore		& -1  		& -20 \\
			Analista			& -4		& -100 \\
			Progettista			& -8 		& -176 \\
			Programmatore		& -15 		& -225 \\
			Verificatore		& -12  		& -180 \\
			\hline
			\textbf{Totale}		& -31  		& -581 \\
			\hline
		\end{tabular}
		\caption{Costo per ruolo, periodo di Verifica e Validazione}
	\end{table}
	Nella tabella seguente sono riportate le differenze tra le ore di lavoro previste per ogni componente e quelle realmente impiegate.
	\begin{table}[H]
		\centering
		\begin{tabular}{|c|c|c|c|c|c|c|c|}
			\hline
			\textbf{Nominativo}		& \textbf{PM}	& \textbf{Am}	& \textbf{An}	& \textbf{Pt}	& \textbf{Pr}	& \textbf{Ve}	& \textbf{Ore totali}     \\
			\hline
			Salvatore Pilò			& +2	& 		& -4	&   	& -6	&   	& -8 \\
			Fabio Massignan			& 		& 		&   	& -2	&   	& -5	& -7 \\
			Sebastiano Bertolin		&		& 	 	& 		& -3  	& 		&		& -3 \\
			Davide Santimaria		&		& +2	&		& -3	&   	& -5	& -6 \\
			Malick Bodian			& +2	& -3  	& 		&   	& -6	&   	& -7 \\
			Gianmarco Salmistraro	&		&		& 	 	&		& -3  	& -2	& -5 \\
			\hline
		\end{tabular}
		\caption{Differenza consuntivo preventivo per componente, periodo di Verifica e Validazione}
	\end{table}
		\subsubsection{Conclusioni}
		Durante il periodo di \textbf{Verifica e Validazione} si è recuperato il ritardo generatosi nel periodo di \textbf{Progettazione di Dettaglio e Codifica} ed inoltre è stato necessario ancora una volta un intervento di correzione di parte dei documenti. Ciò ha fatto aumentare di molto le ore di lavoro, generando una spesa aggiuntiva di 581€.
	\subsection{Consuntivo finale}
	\subsubsection{consuntivo delle ore rendicontate}
	Viene qui riportata una tabella contenente le ore a consuntivo contenente le sole ore svolte nei periodi rendicontati.
	\begin{table}[H]
		\centering
		\resizebox{\textwidth}{!}{\begin{tabular}{|c|c|c|c|c|c|c|c|}
			\hline
			\textbf{Nominativo}		& \textbf{PM}	& \textbf{Am}	& \textbf{An}	& \textbf{Pt}	& \textbf{Pr}	& \textbf{Ve}	& \textbf{Totale}     \\
			\hline
			Salvatore Pilò			&8 (+2)& 5&9 (-5)&38 (+4)&17 (-6)&28 (+2)&\textbf{105  (-3)}\\
			Fabio Massignan			&5&2&0&37 (-2)&28 (+2)&33 (-3)& \textbf{105  (-3)}\\
			Sebastiano Bertolin		&5&0&0&46 (-3)&18&36&\textbf{105  (-3)}\\
			Davide Santimaria		&5&0 (+2)&5	(+1)&22 (-3)&25 (+3)&48 (-6)&\textbf{105  (-3)}\\
			Malick Bodian			&0 (+2)&16&5&31 (+6)&18 (-6)&22 (+8)&\textbf{92  (+10)}\\
			Gianmarco Salmistraro	&5&9&3 (-1)&30&41 (+1)&17 (-3)&\textbf{105  (-3)}\\
			\hline
			\textbf{Totale}			&\textbf{28 (+4)}&\textbf{32 (+2)}&\textbf{22 (-5)}&\textbf{204 (+2)}&\textbf{147 (-6)}&\textbf{184 (-2)}&\textbf{617 (-5)}\\
			\hline
		\end{tabular}}
		\caption{Distribuzione ore rendicontate}
	\end{table}
	\newpage
	\subsubsection{Consuntivo economico}
	Viene qui presentata una tabella contenente l'attuale preventivo a finire. Vengono inseriti i valori del periodo di \textbf{Analisi} e \textbf{Consolidamento dei Requisiti} a scopo riassuntivo, tuttavia essi non verranno conteggiati nel calcolo delle ore rendicontate. Qualora il valore del consuntivo non fosse ancora presente, verrà usato il valore del preventivo.\\
	
	\def\arraystretch{1.5}
	\rowcolors{2}{D}{P}
	\begin{longtable}{p{5cm}!{\VRule[1pt]}p{4cm}!{\VRule[1pt]}p{4cm}}
	\rowcolor{I}
	\color{white} \textbf{Periodo} & \color{white} \textbf{Preventivo €} & \color{white} \textbf{Consuntivo €} \\ 
	\endfirsthead 
	\rowcolor{I} 
	\color{white} \textbf{Periodo} & \color{white} \textbf{Preventivo €} & \color{white} \textbf{Consuntivo €} \\
	\endhead
	\multicolumn{3}{c}{\textbf{Investimento}} \\
	Analisi							& 2825		& 2850	\\
	Consolidamento dei Requisiti	& 695		& 670	\\
	\multicolumn{3}{c}{\textbf{Rendicontato}} \\
	Progettazione Architetturale			& 3430				& 3376	\\
	Progettazione di Dettaglio e Codifica	& 5418				& 4952	\\
	Verifica e Validazione					& 2594				& 3175	\\
	\textbf{Totale}							& \textbf{14962}	& \textbf{15023} \\
	\textbf{Rendicontato}					& \textbf{11442}	& \textbf{11503} \\
	\rowcolor{white}
	\caption{Preventivo a finire}
	\end{longtable}
	\subsubsection{Conclusione}
	Con la conclusione del periodo di \textit{Verifica e Validazione} si può vedere come il costo finale sia leggermente maggiore rispetto al costo preventivato e ciò è stato causato soprattutto dal grosso aumento dei costi nell'ultimo periodo. Ciò nonostante il preventivo proposto era di 1.500€, pertanto, vista la differenza irrisoria, non saranno richieste spese aggiuntive al cliente.\\
	Nel complesso, si è cercato di rispettare al meglio la pianificazione ma, visto il verificarsi di alcuni rischi durante il periodo di \textit{Progettazione di Dettaglio e Codifica} e il conseguente ritardo che hanno comportato, non è stato possibile il rispetto totale della pianificazione. Inoltre, proprio per il motivo appena citato, alcuni membri del gruppo avranno raggiunto un numero di ore/persona inferiore rispetto agli altri, i quali, per cercare di recuperare questa mancanza di lavoro, hanno dovuto incrementare il proprio orario di lavoro fino al raggiungimento delle 105 ore/persona massime consentite.