\section{Preventivo}
Per il preventivo si tiene conto che i periodi di \textbf{Analisi} e di \textbf{Consolidamento dei Requisiti} sono considerati di investimento non a carico del \glossaryItem{Committente}, per cui le ore qui rendicontate non saranno conteggiate nelle ore totali da retribuire. \\
I componenti del gruppo dovranno rivestire ogni ruolo almeno una volta. Possono ricoprire più ruoli contemporaneamente purché non si presentino conflitti di interesse tra i ruoli ricoperti. \\
Per facilitare la lettura delle tabelle si è deciso che, nel caso una cella contenga un valore pari a zero, questo verrà omesso lasciando la cella vuota.
	\subsection{Analisi}
		\subsubsection{Prospetto orario}
		Nel periodo di \textbf{Analisi}, ciascun componente dovrà ricoprire i seguenti ruoli per il numero di ore indicato nella seguente tabella: \\
		\rowcolors{1}{}{}
		\begin{table}[H]
		\centering
		\begin{tabular}{|c|c|c|c|c|c|c|c|}
			\hline
			\textbf{Nominativo}		& \textbf{PM}	& \textbf{Am}	& \textbf{An}	& \textbf{Pt}	& \textbf{Pr}	& \textbf{Ve}	& \textbf{Ore totali}     \\
			\hline
			Salvatore Pilò			& 10	& 		& 12	&		&		&		& 22 \\
			Fabio Massignan			&		& 5		&		&		&		& 16	& 21 \\
			Sebastiano Bertolin		&		& 4		& 15	&		&		&		& 19 \\
			Davide Santimaria		&		& 4		& 16	&		&		&		& 20 \\
			Malick Bodian			& 9		&		&		&		&		& 13	& 22 \\
			Gianmarco Salmistraro	&		& 4		& 15	&		&		& 2		& 21 \\
			\hline
		\end{tabular}
		\caption{Ore per componente, periodo di Analisi}
		\end{table}
		I valori sono riassunti nel seguente grafico, che rappresenta in maniera visiva per quante ore un membro abbia ricoperto un determinato ruolo. \\
		\begin{figure}[H]
			\centering
			\includegraphics[scale=1]{immagini/grafici/analisi-barra.png}
			\caption{Ore per componente, periodo di Analisi}
		\end{figure}
		\subsubsection{Prospetto economico}
		Nel periodo di \textbf{Analisi}, le ore tra i ruoli sono state divise nel seguente modo: \\
	\begin{table}[H]
		\centering
		\begin{tabular}{|c|c|c|}
			\hline
			\textbf{Ruolo}		& \textbf{Ore}	& \textbf{Costo} \\
			\hline
			Project Manager		& 19			& 570	\\
			Amministratore		& 17			& 340	\\
			Analista			& 58			& 1450	\\
			Progettista			& 0				& 0	\\
			Programmatore		& 0				& 0	\\
			Verificatore		& 31			& 465	\\
			\hline
			\textbf{Totale}		& 125			& 2825	\\
			\hline
		\end{tabular}
		\caption{Ore per ruolo, periodo di Analisi}
		\end{table}
	I seguenti grafici illustrano rispettivamente come ciascun ruolo abbia influito sul totale
delle ore e dei costi del periodo di \textbf{Analisi}. \\
	\begin{figure}[H]
		\centering
		\includegraphics[scale=1]{immagini/grafici/analisi-torta.png}
		\caption{Ore per ruoli, periodo di Analisi}
	\end{figure}
	\begin{figure}[H]
		\centering
		\includegraphics[scale=1]{immagini/grafici/analisi-torta-costo.png}
		\caption{Costi per ruoli, periodo di Analisi}
	\end{figure}
	\subsection{Consolidamento dei Requisiti}
		\subsubsection{Prospetto orario}
		Nel periodo di \textbf{Consolidamento dei Requisiti}, ciascun componente dovrà ricoprire i seguenti ruoli per il numero di ore indicato nella seguente tabella: \\
		\begin{table}[H]
		\centering
		\begin{tabular}{|c|c|c|c|c|c|c|c|}
			\hline
			\textbf{Nominativo}		& \textbf{PM}	& \textbf{Am}	& \textbf{An}	& \textbf{Pt}	& \textbf{Pr}	& \textbf{Ve}	& \textbf{Ore totali}     \\
			\hline
			Salvatore Pilò			&		& 		& 4		&		&		& 2		& 6 \\
			Fabio Massignan			& 1		& 		& 4		&		&		& 		& 5 \\
			Sebastiano Bertolin		&		& 		& 5		&		&		&		& 5 \\
			Davide Santimaria		&		& 2		&		&		&		& 3		& 5 \\
			Malick Bodian			& 		& 		& 4		&		&		& 		& 4 \\
			Gianmarco Salmistraro	&		& 		& 5		&		&		& 		& 5 \\
			\hline
		\end{tabular}
		\caption{Ore per componente, periodo di Consolidamento dei Requisiti}
		\end{table}
		I valori sono riassunti nel seguente grafico, che rappresenta in maniera visiva per quante ore un membro abbia ricoperto un determinato ruolo. \\
		\begin{figure}[H]
			\centering
			\includegraphics[scale=1]{immagini/grafici/analisi_dettaglio-barra.png}
			\caption{Ore per componente, periodo di Consolidamento dei Requisiti}
		\end{figure}
		\subsubsection{Prospetto economico}
		Nel periodo di \textbf{Consolidamento dei Requisiti} le ore tra i ruoli sono state divise nel seguente modo: \\
	\begin{table}[H]
		\centering
		\begin{tabular}{|c|c|c|}
			\hline
			\textbf{Ruolo}		& \textbf{Ore}	& \textbf{Costo} \\
			\hline
			Project Manager		& 1				& 30	\\
			Amministratore		& 2				& 40	\\
			Analista			& 22			& 550	\\
			Progettista			& 0				& 0	\\
			Programmatore		& 0				& 0	\\
			Verificatore		& 5				& 75	\\
			\hline
			\textbf{Totale}		& 30			& 695	\\
			\hline
		\end{tabular}
		\caption{Ore per ruolo, periodo di Consolidamento dei Requisiti}
		\end{table}
	I seguenti grafici illustrano rispettivamente come ciascun ruolo abbia influito sul totale
delle ore e dei costi del periodo di \textbf{Consolidamento dei Requisiti}. \\
	\begin{figure}[H]
		\centering
		\includegraphics[scale=1]{immagini/grafici/analisi_dettaglio-torta.png}
		\caption{Ore per ruoli, periodo di Consolidamento dei Requisiti}
	\end{figure}
	\begin{figure}[H]
		\centering
		\includegraphics[scale=1]{immagini/grafici/analisi_dettaglio-torta-costo.png}
		\caption{Costi per ruoli, periodo di Consolidamento dei Requisiti}
	\end{figure}
	\subsection{Progettazione Architetturale}
		\subsubsection{Prospetto orario}
		Nel periodo di \textbf{Progettazione Architetturale}, ciascun componente dovrà ricoprire i seguenti ruoli per il numero di ore indicato nella seguente tabella: \\
		\begin{table}[H]
		\centering
		\begin{tabular}{|c|c|c|c|c|c|c|c|}
			\hline
			\textbf{Nominativo}		& \textbf{PM}	& \textbf{Am}	& \textbf{An}	& \textbf{Pt}	& \textbf{Pr}	& \textbf{Ve}	& \textbf{Ore totali}     \\
			\hline
			Salvatore Pilò			& 		& 5 	& 2		& 22	&		&		& 29 \\
			Fabio Massignan			&		& 		& 		& 12	&		& 15	& 27 \\
			Sebastiano Bertolin		& 5		& 		&  		& 20	&		&		& 25 \\
			Davide Santimaria		&		& 		& 6		&		&		& 20	& 26 \\
			Malick Bodian			& 		& 2		& 5		& 21	&		& 		& 28 \\
			Gianmarco Salmistraro	& 5		& 		& 2		& 20	&		& 		& 27 \\
			\hline
		\end{tabular}
		\caption{Ore per componente, periodo di Progettazione Architetturale}
		\end{table}
		I valori sono riassunti nel seguente grafico, che rappresenta in maniera visiva per quante ore un membro abbia ricoperto un determinato ruolo. \\
		\begin{figure}[H]
			\centering
			\includegraphics[scale=1]{immagini/grafici/progettazione_architetturale-barra.png}
			\caption{Ore per componente, periodo di Progettazione Architetturale}
		\end{figure}
		\subsubsection{Prospetto economico}
		Nel periodo di \textbf{Progettazione Architetturale} le ore tra i ruoli sono state divise nel seguente modo: \\
	\begin{table}[H]
		\centering
		\begin{tabular}{|c|c|c|}
			\hline
			\textbf{Ruolo}		& \textbf{Ore}	& \textbf{Costo} \\
			\hline
			Project Manager		& 10			& 300	\\
			Amministratore		& 7				& 140	\\
			Analista			& 15			& 375	\\
			Progettista			& 95			& 2090	\\
			Programmatore		& 0				& 0	\\
			Verificatore		& 35			& 525	\\
			\hline
			\textbf{Totale}		& 162			& 3430	\\
			\hline
		\end{tabular}
		\caption{Ore per ruolo, periodo di Progettazione Architetturale}
		\end{table}
	I seguenti grafici illustrano rispettivamente come ciascun ruolo abbia influito sul totale
delle ore e dei costi del periodo di \textbf{Progettazione Architetturale}. \\
	\begin{figure}[H]
		\centering
		\includegraphics[scale=1]{immagini/grafici/progettazione_architetturale-torta.png}
		\caption{Ore per ruoli, periodo di Progettazione Architetturale}
	\end{figure}
	\begin{figure}[H]
		\centering
		\includegraphics[scale=1]{immagini/grafici/progettazione_architetturale-torta-costo.png}
		\caption{Costi per ruoli, periodo di Progettazione Architetturale}
	\end{figure}
	\subsection{Progettazione di Dettaglio e Codifica}
		\subsubsection{Prospetto orario}
		Nel periodo di \textbf{Progettazione di Dettaglio e Codifica}, ciascun componente dovrà ricoprire i seguenti ruoli per il numero di ore indicato nella seguente tabella: \\
		\begin{table}[H]
		\centering
		\begin{tabular}{|c|c|c|c|c|c|c|c|}
			\hline
			\textbf{Nominativo}		& \textbf{PM}	& \textbf{Am}	& \textbf{An}	& \textbf{Pt}	& \textbf{Pr}	& \textbf{Ve}	& \textbf{Ore totali}     \\
			\hline
			Salvatore Pilò			& 		& 		& 2		& 20	&		& 30	& 52 \\
			Fabio Massignan			& 5		& 		&		& 15	& 30	& 		& 50 \\
			Sebastiano Bertolin		&		& 		& 		& 17	& 18	& 16	& 51 \\
			Davide Santimaria		& 5		& 		& 		& 16	& 28	&		& 49 \\
			Malick Bodian			& 		& 6		&		& 16	&		& 30	& 52 \\
			Gianmarco Salmistraro	&		& 		& 		& 10	& 30	& 10	& 50 \\
			\hline
		\end{tabular}
		\caption{Ore per componente, periodo di Progettazione di Dettaglio e Codifica}
		\end{table}
		I valori sono riassunti nel seguente grafico, che rappresenta in maniera visiva per quante ore un membro abbia ricoperto un determinato ruolo. \\
		\begin{figure}[H]
			\centering
			\includegraphics[scale=1]{immagini/grafici/progettazione_dettaglio_codifica-barra.png}
			\caption{Ore per componente, periodo di Progettazione di Dettaglio e Codifica}
		\end{figure}
		\subsubsection{Prospetto economico}
		Nel periodo di \textbf{Progettazione di Dettaglio e Codifica} le ore tra i ruoli sono state divise nel seguente modo: \\
	\begin{table}[H]
		\centering
		\begin{tabular}{|c|c|c|}
			\hline
			\textbf{Ruolo}		& \textbf{Ore}	& \textbf{Costo} \\
			\hline
			Project Manager		& 10			& 300	\\
			Amministratore		& 6				& 120	\\
			Analista			& 2				& 50	\\
			Progettista			& 94			& 2068	\\
			Programmatore		& 106			& 1590	\\
			Verificatore		& 86			& 1290	\\
			\hline
			\textbf{Totale}		& 304			& 5418	\\
			\hline
		\end{tabular}
		\caption{Ore per ruolo, periodo di Progettazione di Dettaglio e Codifica}
		\end{table}
	I seguenti grafici illustrano rispettivamente come ciascun ruolo abbia influito sul totale
delle ore e dei costi del periodo di \textbf{Progettazione di Dettaglio e Codifica}. \\
	\begin{figure}[H]
		\centering
		\includegraphics[scale=1]{immagini/grafici/progettazione_dettaglio_codifica-torta.png}
		\caption{Ore per ruoli, periodo di Progettazione di Dettaglio e Codifica}
	\end{figure}
	\begin{figure}[H]
		\centering
		\includegraphics[scale=1]{immagini/grafici/progettazione_dettaglio_codifica-torta-costo.png}
		\caption{Costi per ruoli, periodo di Progettazione di Dettaglio e Codifica}
	\end{figure}
	\subsection{Verifica e Validazione}
		\subsubsection{Prospetto orario}
		Nel periodo di \textbf{Verifica e Validazione}, ciascun componente dovrà ricoprire i seguenti ruoli per il numero di ore indicato nella seguente tabella: \\
		\begin{table}[H]
		\centering
		\begin{tabular}{|c|c|c|c|c|c|c|c|}
			\hline
			\textbf{Nominativo}		& \textbf{PM}	& \textbf{Am}	& \textbf{An}	& \textbf{Pt}	& \textbf{Pr}	& \textbf{Ve}	& \textbf{Ore totali}     \\
			\hline
			Salvatore Pilò			& 10	& 		& 		&		& 11	&		& 21 \\
			Fabio Massignan			&		& 2		&		& 8		&		& 15	& 25 \\
			Sebastiano Bertolin		&		& 		& 		& 6		&		& 20	& 26 \\
			Davide Santimaria		&		& 2		& 		& 3		&		& 22	& 27 \\
			Malick Bodian			& 2		& 8		&		&		& 12	& 		& 22 \\
			Gianmarco Salmistraro	&		& 9		& 		&		& 12	& 4		& 25 \\
			\hline
		\end{tabular}
		\caption{Ore per componente, periodo di Verifica e Validazione}
		\end{table}
		I valori sono riassunti nel seguente grafico, che rappresenta in maniera visiva per quante ore un membro abbia ricoperto un determinato ruolo. \\
		\begin{figure}[H]
			\centering
			\includegraphics[scale=1]{immagini/grafici/validazione-barra.png}
			\caption{Ore per componente, periodo di Verifica e Validazione}
		\end{figure}
		\subsubsection{Prospetto economico}
		Nel periodo di \textbf{Verifica e Validazione} le ore tra i ruoli sono state divise nel seguente modo: \\
	\begin{table}[H]
		\centering
		\begin{tabular}{|c|c|c|}
			\hline
			\textbf{Ruolo}		& \textbf{Ore}	& \textbf{Costo} \\
			\hline
			Project Manager		& 12			& 360	\\
			Amministratore		& 21			& 420	\\
			Analista			& 0				& 0	\\
			Progettista			& 17			& 374	\\
			Programmatore		& 33			& 525	\\
			Verificatore		& 61			& 915	\\
			\hline
			\textbf{Totale}		& 146			& 2594	\\
			\hline
		\end{tabular}
		\caption{Ore per ruolo, periodo di Verifica e Validazione}
		\end{table}
	I seguenti grafici illustrano rispettivamente come ciascun ruolo abbia influito sul totale
delle ore e dei costi del periodo di \textbf{Verifica e Validazione}. \\
	\begin{figure}[H]
		\centering
		\includegraphics[scale=1]{immagini/grafici/validazione-torta.png}
		\caption{Ore per ruoli, periodo di Verifica e Validazione}
	\end{figure}
	\begin{figure}[H]
		\centering
		\includegraphics[scale=1]{immagini/grafici/validazione-torta-costo.png}
		\caption{Costi per ruoli, periodo di Verifica e Validazione}
	\end{figure}
	\subsection{Riepilogo conclusivo}
		\subsubsection{Prospetto orario totale}
		Di seguito vengono riportate le ore totali, comprendenti sia quelle di investimento che quelle rendicontate a carico del \glossaryItem{Committente}, dedicate da ciascun componente all'intero progetto: \\
		\begin{table}[H]
		\centering
		\begin{tabular}{|c|c|c|c|c|c|c|c|}
			\hline
			\textbf{Nominativo}		& \textbf{PM}	& \textbf{Am}	& \textbf{An}	& \textbf{Pt}	& \textbf{Pr}	& \textbf{Ve}	& \textbf{Ore totali}     \\
			\hline
			Salvatore Pilò			& 20	& 5		& 20	& 42	& 11	& 32	& 130 \\
			Fabio Massignan			& 6		& 7		& 4		& 35	& 30	& 46	& 128 \\
			Sebastiano Bertolin		& 5		& 4		& 20	& 43	& 18	& 36	& 126 \\
			Davide Santimaria		& 5		& 8		& 22	& 19	& 28	& 45	& 127 \\
			Malick Bodian			& 11	& 16	& 9		& 37	& 12	& 43	& 128 \\
			Gianmarco Salmistraro	& 5		& 13	& 22	& 30	& 42	& 16	& 128 \\
			\hline
		\end{tabular}
		\caption{Ore per componente totali con investimento}
		\end{table}
		I valori sono riassunti nel seguente grafico, che rappresenta in maniera visiva per quante ore un membro abbia ricoperto un determinato ruolo. \\
		\begin{figure}[H]
			\centering
			\includegraphics[scale=1]{immagini/grafici/riepilogo_conclusivo-barra.png}
			\caption{Ore per componente totali con investimento}
		\end{figure}
		\subsubsection{Prospetto economico totale}
		Le ore totali previste per la realizzazione dell'intero progetto, comprendenti sia quelle di investimento che quelle rendicontate a carico del \glossaryItem{Committente}, sono riportate nella tabella seguente. \\
		\begin{table}[H]
		\centering
		\begin{tabular}{|c|c|c|}
			\hline
			\textbf{Ruolo}		& \textbf{Ore}	& \textbf{Costo} \\
			\hline
			Project Manager		& 52			& 1560	\\
			Amministratore		& 53			& 1060	\\
			Analista			& 97			& 2425	\\
			Progettista			& 206			& 4532	\\
			Programmatore		& 141			& 2115	\\
			Verificatore		& 218			& 2730	\\
			\hline
			\textbf{Totale}		& 767			& 14962	\\
			\hline
		\end{tabular}
		\caption{Ore totali per ruolo}
		\end{table}
		I seguenti grafici illustrano rispettivamente come ciascun ruolo abbia influito sul totale delle ore e dei costi di tutto il progetto compresa la fase di investimento. \\
		\begin{figure}[H]
		\centering
			\includegraphics[scale=1]{immagini/grafici/riepilogo_conclusivo-torta.png}
			\caption{Ore totali per ruoli}
		\end{figure}
		\begin{figure}[H]
			\centering
			\includegraphics[scale=1]{immagini/grafici/riepilogo_conclusivo-torta-costo.png}
			\caption{Costi totali per ruoli}
		\end{figure}
		\subsubsection{Prospetto orario rendicontato}
		Le ore totali rendicontate a carico del \glossaryItem{Committente}, che si riferiscono ai periodi di \textbf{Progettazione Architetturale}, \textbf{Progettazione di Dettaglio e Codifica} e \textbf{Verifica e Validazione}, che ogni componente dedicherà all'intero progetto saranno le seguenti: \\
		\begin{table}[H]
		\centering
		\begin{tabular}{|c|c|c|c|c|c|c|c|}
			\hline
			\textbf{Nominativo}		& \textbf{PM}	& \textbf{Am}	& \textbf{An}	& \textbf{Pt}	& \textbf{Pr}	& \textbf{Ve}	& \textbf{Ore totali}     \\
			\hline
			Salvatore Pilò			& 10	& 5		& 4		& 42	& 11	& 30	& 102 \\
			Fabio Massignan			& 5		& 2		& 		& 35	& 30	& 30	& 102 \\
			Sebastiano Bertolin		& 5		& 		& 		& 43	& 18	& 36	& 102 \\
			Davide Santimaria		& 5		& 2		& 6		& 19	& 28	& 42	& 102 \\
			Malick Bodian			& 2		& 16	& 5		& 37	& 12	& 30	& 102 \\
			Gianmarco Salmistraro	& 5		& 9		& 2		& 30	& 42	& 14	& 102 \\
			\hline
		\end{tabular}
		\caption{Ore per componente totali rendicontate}
		\end{table}
		I valori sono riassunti nel seguente grafico, che rappresenta in maniera visiva per quante ore un membro abbia ricoperto un determinato ruolo. \\
		\begin{figure}[H]
			\centering
			\includegraphics[scale=1]{immagini/grafici/orario_rendicontato-barra.png}
			\caption{Ore per componente totali rendicontate}
		\end{figure}
		\subsubsection{Prospetto economico rendicontato}
		Le ore totali rendicontate a carico del \glossaryItem{Committente} sono riportate nella tabella sottostante e si riferiscono ai periodi di \textbf{Progettazione Architetturale}, \textbf{Progettazione di Dettaglio e Codifica} e \textbf{Verifica e Validazione}. \\
		\begin{table}[H]
		\centering
		\begin{tabular}{|c|c|c|}
			\hline
			\textbf{Ruolo}		& \textbf{Ore}	& \textbf{Costo} \\
			\hline
			Project Manager		& 32			& 960	\\
			Amministratore		& 34			& 680	\\
			Analista			& 17			& 425	\\
			Progettista			& 206			& 4532	\\
			Programmatore		& 141			& 2115	\\
			Verificatore		& 182			& 2730	\\
			\hline
			\textbf{Totale}		& 612			& 11442	\\
			\hline
		\end{tabular}
		\caption{Ore totali retribuite per ruolo}
		\end{table}
		I seguenti grafici illustrano rispettivamente come ciascun ruolo abbia influito sul totale delle ore e dei costi retribuiti. \\
		\begin{figure}[H]
		\centering
			\includegraphics[scale=1]{immagini/grafici/orario_rendicontato-torta.png}
			\caption{Ore totali retribuite per ruoli}
		\end{figure}
		\begin{figure}[H]
			\centering
			\includegraphics[scale=1]{immagini/grafici/orario_rendicontato-torta-costo.png}
			\caption{Costi totali retribuiti per ruoli}
		\end{figure}
		\subsubsection{Conclusioni}
		Il costo totale viene arrotondato a € 11500. \\
		Si è scelto di proporre un preventivo economico maggiorato rispetto a quello calcolato poiché, nonostante la sua irrisorietà, tale maggiorazione permetterà in caso di necessità di poter disporre di ore di lavoro aggiuntive senza dover incidere sui costi proposti.