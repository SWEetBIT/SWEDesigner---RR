\section{Analisi dei rischi}
Per ottimizzare l'avanzamento del progetto, si è effettuata un'approfondita analisi dei
rischi. \\
L'analisi dei rischi si suddivide in quattro momenti:
\begin{itemize}
	\item \textbf{Identificazione:} vengono identificati i rischi potenziali e vengono suddivisi in categorie; \\
	\item \textbf{Analisi:} per ogni rischio individuato vengono valutati la possibilità di occorrenza e il livello di gravità; \\
	\item \textbf{Pianificazione di controllo:} vengono istituiti dei metodi di controllo per i rischi così da poterli evitare; \\
	\item \textbf{Mitigazione:} si cerca di prendere delle contromisure utili a ridurre gli effetti negativi nel caso che un determinato rischio si verifichi. Questa fase è richiesta solo per i rischi difficilmente controllabili. \\
\end{itemize}
	\subsection{Rischi a livello tecnologico}
		\subsubsection{Tecnologie adottate}
		\subsubsection{Rotture Hardware}
	\subsection{Rischi a livello del personale}
		\subsubsection{Problemi riguardanti i componenti del gruppo}
		\subsubsection{Problemi tra i componenti del gruppo}
		\subsubsection{Inesperienza del gruppo}
	\subsection{Problemi a livello organizzativo e della valutazione dei costi}
	\subsection{Problemi a livello dei requisiti}