\section{Analisi dei rischi}
Per ottimizzare l'avanzamento del progetto, si è effettuata un'approfondita analisi dei
rischi. \\
L'analisi dei rischi si suddivide in quattro momenti: \\
\begin{itemize}
	\item \textbf{Identificazione:} vengono identificati i rischi potenziali e vengono suddivisi in categorie;
	\item \textbf{Analisi:} per ogni rischio individuato vengono valutati la possibilità di occorrenza e il livello di gravità;
	\item \textbf{Pianificazione di controllo:} vengono istituiti dei metodi di controllo per i rischi così da poterli evitare;
	\item \textbf{Mitigazione:} si cerca di prendere delle contromisure utili a ridurre gli effetti negativi nel caso che un determinato rischio si verifichi. Questa fase è richiesta solo per i rischi difficilmente controllabili. \\
\end{itemize}
\def\arraystretch{1.5}
\rowcolors{2}{D}{P}
\begin{longtable}{p{2cm}!{\VRule[1pt]}p{2cm}!{\VRule[1pt]}p{3cm}!{\VRule[1pt]}p{3cm}!{\VRule[1pt]}p{3cm}}
\rowcolor{I}
\color{white} \textbf{Probabilità occorrenza} & \color{white} \textbf{Grado pericolosità} & \color{white} \textbf{Descrizione} & \color{white} \textbf{Strategie per la rilevazione} & \color{white} \textbf{Contromisure} \\ 
\endfirsthead 
\rowcolor{I} 
\color{white} \textbf{Probabilità occorrenza} & \color{white} \textbf{Grado pericolosità} & \color{white} \textbf{Descrizione} & \color{white} \textbf{Strategie per la rilevazione} & \color{white} \textbf{Contromisure} \\
\endhead
\hline
\multicolumn{5}{c}{Rischi a livello tecnologico} \\
\hline
Medio	&	Alto	& Possono sorgere degli inconvenienti per quanto riguarda l'utilizzo delle tecnologie adottate, nonostante siano note a buona parte del gruppo	& Il responsabile si assume il compito di verificare che ogni componente abbia una conoscenza quantomeno sufficiente per quanto riguarda le tecnologie adottate	& Ogni componente si impegnerà a documentarsi in maniera autonoma \\
\newpage
\hline
\multicolumn{5}{c}{Rotture hardware} \\
\hline
Bassa	& Basso	& Non tutti i componenti del gruppo utilizzano un portatile di tipo commerciale e non professionale, pertanto la fallibilità della componente hardware è da preventivare	& Ogni componente del gruppo è tenuto ad avere cura dei propri strumenti di lavoro	& Tutte le cartelle contenenti i dati risiedono su una repository su Github e ogni componente è tenuto ad aggiornale tale repository il prima possibile in caso di modifiche \\
\hline
\multicolumn{5}{c}{Problemi riguardanti i componenti del gruppo} \\
\hline
Media	& Medio	& All'interno del gruppo è presente uno studente lavoratore, il quale potrebbe non risultare non sempre disponibile a causa di impegni legati al lavoro. Ogni componente del gruppo ha, inoltre, delle proprie necessità e degli impegni personali. Risulta quindi inevitabile riscontrare problemi di tipo organizzativo	& Grazie ad una comunicazione tempestiva dei propri impegni, il \textit{Responsabile di Progetto} è in grado di avere sempre una visione complessiva delle disponibilità	& Quando un componente notifica un proprio impegno, il carico di lavoro che avrebbe dovuto svolgere viene ripartito tra le altre risorse disponibili \\
\newpage
\hline
\multicolumn{5}{c}{Problemi tra i componenti del gruppo} \\
\hline
Bassa	& Alto	& Ogni componente del gruppo è alla prima esperienza in un gruppo numeroso. Tutti i componenti, inoltre, hanno principi diversi. Tali fattori potrebbero causare un appesantimento del carico di lavoro e la nascita di un clima lavorativo non proficuo	& Il \textit{Responsabile di Progetto} riesce a monitorare la nascita di problematiche interpersonali grazie alla collaborazione dei membri del gruppo	& In caso di forti contrasti, il \textit{Responsabile di Progetto} dovrà tentare di mediare l'incontro dei componenti problematici. Se la discrepanza dovesse rivelarsi insormontabile, le risorse verranno allocate in modo da minimizzare il contratto tra i componenti problematici \\
\hline
\multicolumn{5}{c}{Inesperienza del gruppo} \\
\hline
Alta	& Alto	& Sono richieste capacità di analisi e di pianificazione che il gruppo non possiede. Il metodo di lavoro risulta nuovo e viene richiesto l'impiego di software che nessun componente del gruppo ha mai utilizzato, pertanto dovranno essere apprese tali conoscenze e ciò richiede tempo	& Il \textit{Responsabile di Progetto} riceve una segnalazione ogni qualvolta nasca la necessità di utilizzare un nuovo strumento. Ogni componente si dovrà occupare di trovare del materiale dove studiare la base teorica e, in caso non ne trovi, richiederà consigli al gruppo	& Ogni componente si impegna a studiare il materiale richiesto per poter affrontare in maniera ottimale il progetto, acquisendo le conoscenze necessarie prima che venga richiesto che esse siano messe in pratica \\
\newpage
\hline
\multicolumn{5}{c}{Problemi a livello organizzativo e della valutazione dei costi} \\
\hline
Medio	& Alta	& Durante la fase di pianificazione, i tempi possono essere calcolati in modo errato: un'errata stima dei tempi potrebbe comportare un aumento dei costi e un ritardo nella consegna	& Vanno controllati periodicamente gli stati dei ticket, in modo da venire subito a conoscenza di eventuali ritardi nello sviluppo delle attività	& Per ogni attività è stato stabilito un tempo prolungato per fare in modo che un eventuale ritardo non modifichi la durata totale del progetto \\
\hline
\multicolumn{5}{c}{Problemi a livello dei requisiti} \\
\hline
Media	& Medio	& Alcuni aspetti del problema possono venire studiati in modo non idoneo, causando un'incompleta comprensione del problema stesso e dei suoi requisiti oppure causando divergenze tra le aspettative del Proponente e la visione del gruppo sul prodotto	& Per ridurre al minimo la probabilità di errori nella fase di \textbf{Analisi dei Requisiti}, verranno effettuati degli incontri con il Proponente, in modo da assicurare la concordanza sulle necessità del prodotto	& Sarà indispensabile correggere eventuali errori o imprecisioni indicati dal Committente all'esito di ogni revisione \\
\rowcolor{white}
\caption{Analisi dei rischi}
\end{longtable}