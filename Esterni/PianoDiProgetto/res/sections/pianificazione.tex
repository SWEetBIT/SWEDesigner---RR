\section{Pianificazione}
	In seguito alla suddivisione delle scadenze, per eseguire una più accurata pianificazione progettuale, il progetto `è stato suddiviso nelle seguenti fasi: \\
	\begin{itemize}
		\item \textbf{Analisi} (AN); \\
		\item \textbf{Analisi di Dettaglio} (AD); \\
		\item \textbf{Progettazione Architetturale} (PA); \\
		\item \textbf{Progettazione di Dettaglio e Codifica} (PDC); \\
		\item \textbf{Verifica e Validazione} (VV). \\
	\end{itemize}
	Ognuna di queste fasi è stata poi suddivisa in più attività, a ognuna delle quali sono state associate una o più risorse. Ogni attività è stata suddivisa in sotto-attività, delle sono stati riportati i 			Diagrammi di Gantt così da evidenziare la pianificazione di dettaglio restando focalizzati sui 				concetti di maggiore importanza.
\subsection{Analisi}
	\textbf{Periodo:} da ... a ... \\
	Questa fase inizia in concomitanza con la pubblicazione dei capitolati d'appalto e termina in 			corrispondenza della Revisione dei Requisiti. \\
	Le attività della fase di analisi sono: \\
	\begin{itemize}
		\item \textbf{Norme di Progetto:} in questa attività l'Amministratore, sottoscrive tutte le regole che il gruppo è obbligatoriamente tenuto a seguire durante l'attuazione di tutte le attività progettuali. In questo documento devono quindi essere inserite tutte le norme e le scelte del software di supporto non vincolate al capitolato. Sarò poi compito dei verificatori la certificazione del rispetto di tali norme; \\
		\item \textbf{Studio di Fattibilità:} in questa fase vengono discussi e valutati dal gruppo tutti i capitolati d'appalto. Viene quindi redatto il documento Studio di Fattibilità, contenente i risultati di tali analisi. L'attività di analisi consiste nell'analisi della complessità delle varie proposte mediante l'abbozzo di Analisi dei Requisiti ad alto livello. La stesura di questo documento è necessaria per la creazione degli altri documenti in quanto è proprio da questo documento che emerge il progetto che il gruppo porterà avanti; \\
		\item \textbf{Analisi dei Requisiti:} in questa attività, dalla bozza di Analisi dei Requisiti redatta durante lo Studio di Fattibilità, si esegue un analisi più approfondita. Tale attività continuerà fino alla data di consegna stabilita; \\
		\item \textbf{Piano di Progetto:} in questa attività il responsabile di progetto, basandosi sulle date di scadenza, redige il documento Piano di Progetto, organizzando tutte le attività del gruppo per lo svolgimento del lavoro. Tale attività ha una priorità alta in quanto regola le attività svolte dall'intero gruppo; \\
		\item \textbf{Piano di qualifica:} in questa attività si individuano tutte le strategie di verifica e validazione che il gruppo dovrà adottare per il progetto. La documentazione del Piano di Qualifica viene redatta da un analista in collaborazione con l'Amministratore ed il responsabile di progetto; \\
		\item \textbf{Glossario:} in questa attività i redattori, parallelamente alla stesura degli altri documenti, creano un documento che contiene una selezione di termini usati nella stesura della documentazione che necessitano di disambiguazione. Per ognuno di questi vocaboli presenti nel Glossario si associa una definizione al fine di chiarire il significato del termine all'interno del progetto. Il documento viene quindi aggiornato in maniera incrementale ad ogni inserimento di un nuovo termine; \\
		\item \textbf{Lettera di presentazione:} in questa attività viene redatta una lettera da presentare al committente per permettere al gruppo di partecipare alla gara d'appalto per il capitolato. \\
	\end{itemize}
	In questa macro-fase i ruoli maggiormente coinvolti sono: Responsabile, Amministratore e Analista.
\subsubsection{Diagramma di Gantt delle attività}
\subsubsection{Work Breakdown Structure delle attività}
\subsubsection{Ripartizione delle risorse}
\subsection{Analisi di Dettaglio}
	\textbf{Periodo:} da ... a ... \\
	Questa fase inizia successivamente alla Revisione dei Requisiti e si conclude con l'inizio della fase di Progettazione Architetturale. \\
	Questo periodo viene utilizzato per consolidare i requisiti richiesti dal sistema e per
migliorare il documento di Analisi dei Requisiti. \\
	In questa macro-fase i ruoli maggiormente coinvolti sono: Responsabile, Amministratore e Analista.
\subsubsection{Diagramma di Gantt delle attività}
\subsubsection{Work Breakdown Structure delle attività}
\subsubsection{Ripartizione delle risorse}
\subsection{Progettazione Architetturale}
	\textbf{Periodo:} da ... a ... \\
	Questa macro-fase inizia al termine dell'Analisi Dettaglio e termina con la consegna del prodotto alla Revisione di Progetto, lasciando alla prossima fase lo stato definitivo del prodotto stesso. \\
	Le attività della fase di Progettazione Architetturale sono: \\
	\begin{itemize}
		\item \textbf{Specifica Tecnica:} in questa attività il responsabile di progetto descrive al gruppo le scelte progettuali, ad alto livello, che il prodotto dovrà rispettare. Inoltre, vengono esposti i design pattern che verranno utilizzati nella creazione del prodotto, l'architettura generale del software, i principali flussi di controllo e il tracciamento dei requisiti; \\
		\item \textbf{Incremento e verifica:} tutti i documenti verranno aggiornati in base al risultato
della Revisione dei Requisiti. \\
	\end{itemize}
	In questa macro-fase i ruoli maggiormente coinvolti sono: Responsabile, Amministratore, Progettista, Verificatore e Analista.
\subsubsection{Diagramma di Gantt delle attività}
\subsubsection{Work Breakdown Structure delle attività}
\subsubsection{Ripartizione delle risorse}
\subsection{Progettazione di Dettaglio e Codifica}
	\textbf{Periodo:} da ... a ... \\
	Questa macro-fase inizia dopo la Revisione di Progetto e termina con la consegna del prodotto alla Revisione di Qualifica. Le attività della fase di Progettazione di Dettaglio e Codifica sono: \\
	\begin{itemize}
		\item \textbf{Definizione di Prodotto:} in questa attività viene redatto il documento Definizione di Prodotto. All'interno di tale documento vengono definite approfonditamente la struttura e le relazioni dei vari componenti del prodotto, basandosi sul documento di Specifica Tecnica; \\
		\item \textbf{Codifica:} in questa attività si procede allo sviluppo del codice del software da parte dei programmatori, seguendo quanto è riportato nella Definizione di Prodotto; \\
		\item \textbf{Manuali utenti:} in questa attività si creano i documenti che hanno lo scopo
di fornire delle linee guida per l'utilizzo del sistema da parte degli utenti coinvolti; \\
		\item \textbf{Incremento e verifica:} in questa attività si devono aggiornare tutti i documenti
basandosi sui risultati della Revisione di Progettazione; \\
	\end{itemize}
	In questa macro-fase i ruoli maggiormente coinvolti sono: Responsabile, Amministratore, Progettista, Verificatore e Programmatore.
\subsubsection{Diagramma di Gantt delle attività}
\subsubsection{Work Breakdown Structure delle attività}
\subsubsection{Ripartizione delle risorse}
\subsection{Verifica e Validazione}
	\textbf{Periodo:} da ... a ... \\
	Questa macro-fase inizia dopo la Revisione di Qualifica e termina il processo di sviluppo del software. Tale fase rappresenta l'atto conclusivo delle varie attività di verifica realizzate nei singoli processi del ciclo di vita. \\
	Le attività della fase di Verifica e Validazione sono: \\
	\begin{itemize}
		\item \textbf{Collaudo del sistema:} in questa attività il prodotto viene collaudato per dare
dimostrazione che è conforme alle specifiche e soddisfa tutti i requisiti stabiliti; \\
		\item \textbf{Incremento e verifica:} in questa attività tutti i documenti vengono aggiornati in base al risultato della Revisione di Qualifica. \\
	\end{itemize}
	In questa macro-fase i ruoli maggiormente coinvolti sono: Responsabile, Amministratore, Progettista e Verificatore.
\subsubsection{Diagramma di Gantt delle attività}
\subsubsection{Work Breakdown Structure delle attività}
\subsubsection{Ripartizione delle risorse}