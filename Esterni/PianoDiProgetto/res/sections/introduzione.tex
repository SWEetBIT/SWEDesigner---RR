\section{Introduzione}
\subsection{Scopo del documento}
Il presente documento ha l'intento di esporre la pianificazione secondo la quale saranno svolti i lavori dal gruppo SWEet BIT sul progetto SWEDesigner. \\
Gli scopi del presente documento sono: \\
\begin{itemize}
	\item Presentare la pianificazione dei tempi e delle attività;
	\item Preventivare l'utilizzo delle risorse;
	\item Consuntivare l'impiego delle risorse durante l'evoluzione dei lavori;
	\item Analizzare i possibili fattori di rischio. \\
\end{itemize}
\subsection{Riferimenti}
\subsubsection{Normativi}
\begin{itemize}
	\item \textbf{Capitolato d'Appalto C6: SWEDesigner} \\
		\url{http://www.math.unipd.it/~tullio/IS-1/2015/Progetto/C6p.pdf} \\
	\item \textbf{Vincoli di organigramma e dettagli economico-tecnici:} \\
		\url{http://www.math.unipd.it/~tullio/IS-1/2015/Progetto/PD01b.html} \\
	\item \textbf{Norme di Progetto:} \textit{Norme di Progetto v\VersioneNP}. \\
\end{itemize}
\subsubsection{Informativi}
\begin{itemize}
	\item \textbf{Slide dell'insegnamento Ingegneria del Software modulo A:} \\
		\url{http://www.math.unipd.it/~tullio/IS-1/2016/}. \\
	\item \textbf{Metriche di progetto:} \\
		\url{http://it.wikipedia.org/wiki/Metriche_di_progetto}. \\
\end{itemize}
\subsection{Ciclo di vita}
Per quanto riguarda la gestione del progetto, in merito al ciclo di vita del software è stato deciso di applicare il \textbf{modello incrementale} per garantire la qualità, la conformità e la maturità del prodotto. \\
In un modello incrementale il cliente identifica, a grandi linee, i requisiti fondamentali e quelli desiderabili del prodotto software che vuole ottenere. Viene poi deciso il numero di incrementi da effettuare, tenendo conto del fatto che ogni singolo incremento costituisce un sottoinsieme delle funzionalità del prodotto software. Gli incrementi vengono decisi ordinandoli per priorità decrescente, iniziando con quelli aventi priorità più alta e lasciando per ultimi quelli con priorità minore. Una volta che gli incrementi sono stati identificati, si definiscono in dettaglio i requisiti che devono essere soddisfatti col primo incremento e quindi si comincia la fase di sviluppo dell'incremento stesso. Durante la fase di sviluppo possono essere aggiunti ulteriori requisiti che devono essere soddisfatti dagli incrementi successivi, ma non si possono andare a modificare i requisiti decisi prima di cominciare lo sviluppo dell'incremento corrente. Al termine di questa fase l'incremento viene aggiunto al prodotto software e, se il software non è completo, si procede con l'incremento successivo. Di particolare rilevanza è la fase di integrazione dell'incremento poiché dimostra il grado di efficacia e chiarisce i requisiti per gli incrementi successivi.
\subsection{Scadenze}
Di seguito vengono presentate le date delle scadenze che il gruppo SWEet BIT ha deciso di rispettare e sulle quali si baserà la pianificazione del progetto: \\
\begin{itemize}
	\item \textbf{Revisione dei Requisiti:} il 2017-04-18;
	\item \textbf{Revisione di Progettazione:} il 2017-05-15, eseguendo la \textbf{Revisione di Progettazione} minima;
	\item \textbf{Revisione di Qualifica:} il 2017-07-13;
	\item \textbf{Revisione di Accettazione:} il 2017-08-29.
\end{itemize}