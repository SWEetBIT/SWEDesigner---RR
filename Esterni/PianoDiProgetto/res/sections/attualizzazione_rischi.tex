\section{Attualizzazione dei rischi}
\def\arraystretch{1.5}
\rowcolors{2}{D}{P}
\begin{longtable}{p{3cm}!{\VRule[1pt]}p{10cm}}
\rowcolor{I}
\color{white} \textbf{Rischio} & \color{white} \textbf{Mitigazione} \\ 
\endfirsthead 
\rowcolor{I} 
\color{white} \textbf{Rischio} & \color{white} \textbf{Mitigazione} \\ 
\endhead
\multicolumn{2}{c}{\textbf{Periodo di Analisi}} \\
Problemi tra i componenti del gruppo	& Vista la necessità di utilizzare la lavagna e vista la difficoltà di organizzare riunioni alle quali tutti i membri risultassero presenti, le riunioni sono state organizzate assicurandosi che almeno cinque membri su sei risultassero presenti \\
Problemi a livello dei requisiti	& Vista la difficoltà nell'individuazione dei requisiti, sono state organizzate delle riunioni con il \glossaryItem{Proponente} allo scopo di individuare chiaramente i requisiti e quindi di minimizzare la divergenza tra le aspettative del \glossaryItem{Proponente} e la visione del gruppo sul progetto \\
Problemi a livello organizzativo	& Sono stati minimizzati i ritardi grazie alla collaborazione di tutti i membri del gruppo \\
\multicolumn{2}{c}{\textbf{Periodo di Consolidamento dei Requisiti}} \\
Nessuna rilevazione	& La brevità del periodo non ha permesso l'insorgenza di problemi \\
\multicolumn{2}{c}{\textbf{Periodo di Progettazione Architetturale}} \\
Inesperienza del gruppo	& E' risultato particolarmente impegnativo per i componenti del gruppo riuscire e reperire le informazione necessarie, ma ogni membro si è impegnato a cercare e condividere fonti attendibili da cui assimilare tali informazioni \\
\rowcolor{white}
\caption{Attualizzazione dei rischi}
\end{longtable}