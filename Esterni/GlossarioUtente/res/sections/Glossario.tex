\section{Glossario}
	\subsection{A}
		\subsubsection{Account}
			In informatica, attraverso il meccanismo dell'\glossaryItem{Account}, il sistema mette a disposizione dell'Utente un ambiente con contenuti e funzionalità personalizzabili, oltre ad un conveniente grado di isolamento dalle altre utenze parallele.
		\subsubsection{Applicazione}
			In informatica individua un programma o una serie di programmi in fase di esecuzione su un computer con lo scopo e il risultato di rendere possibile una o più funzionalità, servizi o strumenti utili e selezionabili su richiesta dall'Utente tramite interfaccia utente
	\subsection{C}
		\subsubsection{Classe}
			È un costrutto di un linguaggio di programmazione atto a rappresentare una persona, un luogo, oppure una cosa, ed è quindi l'astrazione di un concetto.
	\subsection{D}
		\subsubsection{Designer}
			È una figura professionale che si occupa di progettare qualcosa. Nel nostro caso specifico tendiamo ad indicare con questo termine un tool che ci permetta
			di disegnare, e progettare quindi, qualcosa.
	\subsection{M}
		\subsubsection{Metodo}
			In informatica, è un termine che viene usato principalmente nel contesto della programmazione orientata agli oggetti per indicare un sottoprogramma associato in modo esclusivo ad una Classe e che rappresenta (in genere) un'operazione eseguibile sugli oggetti e istanze di quella Classe
	\subsection{U}
		\subsubsection{Utente}
			È colui che usufruisce di un bene o di un servizio, generalmente collettivo, fornito da enti pubblici o strutture private.
		\subsubsection{UML}
			L'UML, o unified modeling language (linguaggio di modellizzazione unificato) è un linguaggio di modellazione basato sul paradigma dell'orientamento agli oggetti
		che mira a creare uno standard che possa unificare tutti i linguaggi che ne fanno uso
	\subsection{W}
		\subsubsection{Web App}
			Si indica con Web App, genericamente, tutte quelle applicazioni web-based, ovvero un’Applicazione
			fruibile via web tramite un network, ovvero mediante l’utilizzo di una struttura tipica Client-server


	
