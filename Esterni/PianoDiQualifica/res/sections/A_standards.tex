\section{Standard e metodi per la gestione della Qualità}
  \subsubsection{\glossaryItem{ISO}/\glossaryItem{IEC} 9126 Software engineering — Product quality}
  L'obiettivo di questo standard è quello di definire un modello per poter valutar la qualità del software.\\
  Lo standard si divide in quattro parti:
  \begin{itemize}
    \item modello della qualità del software;
    \item metriche per la qualità interna;
    \item metriche per la qualità esterna;
    \item metriche per la qualità in uso.
  \end{itemize}
  \subsubsection{Modello della qualità del software}
  Il modello della qualità del software presentato nella prima parte dello standard (\glossaryItem{ISO}/\glossaryItem{IEC} 9126-1), identifica sei caratteristiche generali e varie sottocaratteristiche:
  \begin{enumerate}
    \item \textbf{Funzionalità:}
    \begin{itemize}
        \item \textbf{Appropriatezza}: la capacità del software di offrire un appropriato insieme di funzioni per determinati compiti.
        \item \textbf{Accuratezza}: la capacità del prodotto software di rendere risultati concordati o i precisi effetti aspettati.
        \item \textbf{Interoperabilità}: capacità del prodotto software di operare e interagire con sistemi uno o più sistemi specificati.
        \item \textbf{Sicurezza}: la capacità di proteggere dati e informazioni, impedendo a persone e sistemi non autorizzati di accedervi, e garantendo sempre l'accesso ai dati a persone e sistemi autorizzati.
        \item \textbf{Conformità funzionale}: capacità del prodotto software di aderire a standard, convenzioni e regolamentazioni riguardanti il settore operativo a cui vengono applicate.
    \end{itemize}
    \item \textbf{Affidabilità:}
    \begin{itemize}
      \item \textbf{Maturità}: capacità del prodotto software di evitare il verificarsi di errori, malfunzionamenti o siano prodotto risultati errati.
      \item \textbf{Tolleranza agli errori}: è la capacità del prodotto software di mantenere livelli predeterminati di prestazioni anche in presenza di malfunzionamenti o usi scorretti del prodotto.
      \item \textbf{Recuperabilità}: capacità di ripristinare il livello appropriato di prestazioni e di recupero delle informazioni rilevanti, in seguito a un malfunzionamento.
      A seguito di un errore, il software può risultare non accessibile per un determinato periodo di tempo, questo arco di tempo è valutato proprio dalla caratteristica di recuperabilità.
      \item \textbf{Aderenza}: è la capacità di aderire a standard, regole e convenzioni inerenti all'affidabilità.
    \end{itemize}
    \item \textbf{Usabilità:}
    \begin{itemize}
      \item \textbf{Comprensibilità}: è la capacità del prodotto software di mettere l'utente in grado di comprendere se il software è appropriato, e come esso possa essere usato per scopi e condizioni d'uso particolari
      \item \textbf{Apprendibilità}: è la capacità di ridurre l'impegno richiesto agli utenti per imparare ad usare la sua applicazione.
      \item \textbf{Operabilità}: è la capacità del prodotto software di rendere l'utente in grado di operarlo e controllarlo.
      \item \textbf{Attrattiva}: è la capacità del software di essere piacevole per l'utente.
      \item \textbf{Conformità}: è la capacità del prodotto software di aderire a standard o convenzioni relativi all'usabilità.
    \end{itemize}
    \item \textbf{Efficienza:}
    \begin{itemize}
      \item \textbf{Comportamento rispetto al tempo}: è la capacità di fornire adeguati tempi di risposta, elaborazione e velocità di attraversamento, sotto condizioni determinate.
      \item \textbf{Utilizzo delle risorse}: capacità del prodotto software di usare un adeguato quntitativo e tipo di risorse quando il software esegue le sue funzioni in determinate condizioni.
      \item \textbf{Conformità}: è la capacità di aderire a standard e specifiche sull'efficienza.
    \end{itemize}
    \item \textbf{Manutenibilità:}
    \begin{itemize}
      \item \textbf{Analizzabilità}: è la del prodotto software di essere facilmente controllato per la ricerca di errori, o di facilitare l'identificazione delle parti che devono essere modificate.
      \item \textbf{Modificabilità}: capacità del prodotto software di rendere possibili eventuali implementazioni di modifiche.
      \item \textbf{Stabilità}: la capacità del prodotto software di evitare effetti indesiderati dovuti alle modifiche del software stesso.
      \item \textbf{Testabilità}: è la capacità del prodotto software di poter validare le modifiche ad esso apportate
      \item \textbf{Conformità di manutenibilità}: è la capacità di aderire a standard e specifiche riguardanti la manutenibilità.
    \end{itemize}
    \item \textbf{Portabilità:}
    \begin{itemize}
      \item \textbf{Adattabilità}: la capacità del software di essere adattato per differenti ambienti operativi senza dover applicare modifiche diverse da quelle fornite per il software considerato.
      \item \textbf{Installabilità}: la capacità del software di essere installato in uno specificato ambiente.
      \item \textbf{Sostituibilità}: è la capacità di essere utilizzato al posto di un altro software per svolgere gli stessi compiti nello stesso ambiente.
      \item \textbf{Conformità}: la capacità del prodotto software di aderire a standard e convenzioni relative alla portabilità.
    \end{itemize}
  \end{enumerate}

  \subsubsection{Metriche per la qualità interna}
    Le metriche per la qualità interna o mentriche interne, descritte nel \emph{technical report} \glossaryItem{ISO}/\glossaryItem{IEC} 9126-3,
    sono delle metriche che si applicano al software non eseguibile durante le fasi di progettazione e codifica.
    Le metriche interne permettono di individuare eventuali problemi che potrebbero influire sulla qualità finale del prodotto prima che sia realizzato il software eseguibile
    Le misure effettuate permettono di prevedere il livello di qualità esterna ed in uso del prodotto finale,
    poiché gli attributi interni influiscono su quelli esterni e quelli in uso.

  \subsubsection{Metriche per la qualità esterna}
    Le metriche per la qualità esterna o metriche esterne, descritte nel \emph{technical report} \glossaryItem{ISO}/\glossaryItem{IEC} 9126-2,
    sono delle metriche adatte alla misurazione dei comportamenti del software sulla base di misure ottenute da test, oparando e osservando il software eseguibile o sistema stesso.

  \subsubsection{Metriche per la qualità in uso}





  \subsubsection{\glossaryItem{ISO}/\glossaryItem{IEC} 15504}
  \subsubsection{}
