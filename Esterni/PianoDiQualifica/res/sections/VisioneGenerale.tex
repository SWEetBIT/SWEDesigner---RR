\section{Visione generale della strategia di gestione della qualità}
  \subsection{Definizione obiettivi di qualità}
  Questa sezione si presta a descrivere sia gli obiettivi di qualità, richiesti dal committente, riguardanti il prodotto
  che quelli relativi ai processi necessari alla sua produzione.
    \subsubsection{Qualità di prodotto}
    Il gruppo SWEet BIT si è impegnato ad aderire allo standard \glossaryItem{ISO}/\glossaryItem{IEC} 9126, in modo garantire
    le seguenti caratteristiche di qualità per il prodotto:
    \begin{description}
      \item [Funzionalità:] il prodotto deve soddisfare i requisiti indicati nel documento \emph{Analisi dei Requisiti}. Rispettando:
      \begin{itemize}
          \item Appropriatezza
          \item Accuratezza
          \item Interoperabilità
          \item Conformità
          \item Sicurezza
      \end{itemize}
      \item [Affidabilità:] il software prodotto deve essere robusto, deve essere prevista una completa gestione degli errori ed eccezioni,
       in modo da prevenire perdite di dati e facilitarne il ripristino. Il software prodotto per essere affidabile deve avere le sequenti caratteristiche:
       \begin{itemize}
         \item Maturità
         \item Tolleranza agli errori
         \item Recuperabilità
         \item Aderenza
       \end{itemize}
      \item [Usabilità:] il software prodotto deve risultare semplice e di facile utilizzo da parte dell'utente; esso deve essere di facile apprendimento,
      l'interfaccia utente deve risultare familiare e intuitiva. Il prodotto deve soddisfare al meglio le esigenze dell'utente finale. Caratteristiche fondamentali sono:
      \begin{itemize}
        \item Comprensibilità
        \item Apprendibilità
        \item Operabilità
        \item Attrattiva
        \item Conformità
      \end{itemize}
      \item [Efficienza:] Il software deve essere capaci di fornire tutti i servizi attesi con il minimo dispendio di risorse. Caratteristiche critiche sono:
      \begin{itemize}
        \item Comportamento rispetto al tempo
        \item Utilizzo delle risorse
        \item Conformità
      \end{itemize}
      \item [Manutenibilità:] Il software prodotto deve essere facilmente modificabile, in caso di correzione di errori e migliorie da apportare; esso deve essere facilmente
      adattabile in caso di cambio di ambiente operativo e/o cambio di requisiti. Il software deve possedere le seguenti caratteristiche:
      \begin{itemize}
        \item Analizzabilità
        \item Modificabilità
        \item Stabilità
        \item Testabilità
      \end{itemize}
      \item [Portabilità:] Il software prodotto deve, per la parte \glossaryItem{front end}, offrire gli stessi servizi nella maggior parte dei \glossaryItem{browser};
      la parte di \glossaryItem{back end} deve funzionare su vari sistemi operativi. Caratteristiche per la portabilità sono:
      \begin{itemize}
        \item Adattabilità
        \item Installabilità
        \item Conformità
        \item Sostituibilità
      \end{itemize}

    \end{description}
    \subsubsection{Qualità di processo}
    La qualità dei processi che portano allo sviluppo di un software hanno un ruolo fondamentale sulla qualità del prodotto;
    pertanto il gruppo SWEet BIT ha adottato lo standard \glossaryItem{ISO}/\glossaryItem{IEC} 15504 detto \glossaryItem{SPICE},
    il quale fornisce un modello per la valutazione di questi processi.
    Si deve sottoporre ogni processo a valutazione continua
