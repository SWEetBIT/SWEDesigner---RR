\section{Visione generale delle strategie di verifica}
  \subsection{Definizione obiettivi di qualità}
  Questa sezione si presta a descrivere sia gli obiettivi di qualità, richiesti dal committente, riguardanti il prodotto
  che quelli relativi ai processi necessari alla sua produzione.
    \subsubsection{Qualità di prodotto}
    Il gruppo SWEet BIT si è impegnato ad aderire allo standard \glossaryItem{ISO}/\glossaryItem{IEC} 9126, in modo garantire
    le seguenti caratteristiche di qualità per il prodotto:
    \begin{description}
      \item [Funzionalità:] il prodotto deve soddisfare i requisiti indicati nel documento \emph{Analisi dei Requisiti}.

      \item [Affidabilità:] il software prodotto deve essere robusto, deve essere prevista una completa gestione degli errori ed eccezioni,
       in modo da prevenire perdite di dati e facilitarne il ripristino.

      \item [Usabilità:] il software prodotto deve risultare semplice e di facile utilizzo da parte dell'utente; esso deve essere di facile apprendimento,
      l'interfaccia utente deve risultare familiare e intuitiva. Il prodotto deve soddisfare al meglio le esigenze dell'utente finale.

      \item [Efficienza:] Il software deve essere capace di fornire tutti i servizi attesi con il minimo dispendio di risorse.

      \item [Manutenibilità:] Il software prodotto deve essere facilmente modificabile, in caso di correzione di errori e migliorie da apportare; esso deve essere facilmente
      adattabile in caso di cambio di ambiente operativo e/o cambio di requisiti.

      \item [Portabilità:] Il software prodotto deve, per la parte \glossaryItem{front end}, offrire gli stessi servizi nella maggior parte dei \glossaryItem{browser};
      la parte di \glossaryItem{back end} deve funzionare su vari sistemi operativi.

    Per ognuna delle caratteristiche principali appena elencate esistono diverse sotto-caratteristiche,
    le quali sono approfondite nella sotto-sezione Standard e metodi per la gestione della Qualità dove è descritto lo standard \glossaryItem{ISO}/\glossaryItem{IEC} 9126
    \end{description}.

    \subsubsection{Qualità di processo}
    La qualità dei processi che portano allo sviluppo di un software hanno un ruolo fondamentale sulla qualità del prodotto.
    Il gruppo SWEet BIT ha deciso di adottare il ciclo di Deming, o ciclo \glossaryItem{PDCA}, come modello per il continuo miglioramento dei processi produttivi;
    come complemento al ciclo di Deming è stato scelto il modello descritto dallo standard \glossaryItem{ISO}/\glossaryItem{IEC} 15504 detto \glossaryItem{SPICE},
    il quale fornisce gli stumenti per la valutazione dei processi produttivi.

  \subsection{Organizzazione}
    Per onguna delle fasi descritte dal \emph{Piano di progetto} è necessario seguire una verifica mirata al tipo di processo e relativo prodotto:
    \begin{itemize}
      \item \textbf{Analisi:} in questa fase vengono controllati i processi e la documentazione prodotta,
      la verifica dovà essere eseguita secondo i metodi descritti nel documento \emph{Norme di progetto}.
      \item \textbf{Analisi nel dettaglio:} durante questa fase sono verificati i processi che portano all'incemento dei documenti, e i documenti stessi, prodotti nella precedente fase di Analisi.
      Le procedure di verifica sono descritte nel documento \emph{Norme di progetto}.
      \item \textbf{Progettazione architetturale:} in questa fase vengono verificati i processi, e relativi prodotti, che portano all'incremento dei documenti redatti nella precedente fase;
      sempre in questa fase sono verificati i processi e prodotti riguardanti la progettazione architetturale.
      Le procedure di verifica sono descritte in dettaglio nel documento \emph{Norme di progetto}.
    \end{itemize}
  \subsection{Pianificazione strategica e temporale}
    Intendendo rispettare le scadenze specificate nel \emph{Piano di progetto} è necessario un approccio organico e ben pianificato alle attività di verifica.
    Deve essere pianificato ogni incremento su codice e documentazione, di conseguenza lo dovranno essere tutte le attività di verifica
    in modo da avere il tempo di effettuare le dovute correzioni. Ogni processo di verifica dovrà essere automatizzato il più possibile, questo per poter, dove possibilie,
    risparmiare su risorse umane. L'utilizzo di software apposito, e i metodi, per la verifica sono descritti nel documento \emph{Norme di progetto}.
  \subsection{Gestione delle responsabilità}
    Il \emph{Responsabile di progetto} e i \emph{Verificatori} hanno la responsabilità delle attività di verifica;
    i compiti di queste figure sono argomento trattato nel documento \emph{Piano di progetto}.
