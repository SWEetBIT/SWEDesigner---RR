\section{Visione generale delle strategie di verifica}
  \subsection{Obiettivi di qualità}
  Questa sezione si presta a descrivere sia gli obiettivi di qualità riguardanti
  il prodotto che quelli relativi ai processi necessari alla sua produzione.
  
  \subsection{Qualità di processo}
    La qualità dei processi che portano allo sviluppo di un software hanno un
    ruolo fondamentale sulla qualità del prodotto. Il gruppo SWEet BIT ha deciso
    di adottare il \emph{Ciclo di Deming}, o \emph{Ciclo} \glossaryItem{PDCA},
    come modello per il continuo miglioramento dei processi produttivi. \\
    Come complemento al \emph{ciclo di Deming} è stato scelto il modello descritto
    dallo standard \glossaryItem{ISO}/\glossaryItem{IEC} 15504 detto
    \glossaryItem{SPICE}, il quale fornisce gli strumenti per la valutazione dei
    processi produttivi.\\
    Sia lo standard \glossaryItem{ISO}/\glossaryItem{IEC} 15504 che il
    \emph{Ciclo di Deming} sono descritti nell'Appendice \emph{A - Standard e
    \glossaryItem{Metodi} per la gestione della Qualità} del presente documento.

  \subsubsection{Procedure di controllo di qualità di processo}
  La pianificazione delle attività volte al miglioramento continuo dei processi
  sono descritte nel \emph{Piano di Progetto \VersionePP{}}. Le linee guida per
  la gestione della qualità del processo, invece, seguono il modello \glossaryItem{PDCA} e
  descrivono come devono essere attuate le procedure di controllo:
  \begin{itemize}
    \item La pianificazione deve essere dettagliata, e le attività pianificate
    devono essere monitorate;
    \item Le risorse necessarie per conseguire gli obiettivi devono essere
    definite;
    \item Il miglioramento della qualità del processo deve essere verificato
    attraverso l'utilizzo di apposite metriche, che verranno descritte in seguito.
  \end{itemize}
  
\subsubsection{Obiettivi di qualità di processo}  
Gli obiettivi di qualità di processo che il gruppo SWEet BIT vuole raggiungere nel progetto, sono un sottoinsieme di quelli definiti dallo standard ISO/IEC 12207:2008.
Sono metriche che danno un riscontro immediato sullo stato attuale del progetto, mantenendo il controllo sui processi durante il loro svolgimento.
Per ogni metrica indicata è stato definito nelle \emph{Norme di Progetto}\VersioneNP{} come si calcola e con quale strumento.

\paragraph{Controllo di pianificazione dei processi}
Un insieme di metriche che indicano se ci si trova allineati o meno tra quanto pianificato e quanto è stato effettivamente eseguito.
\begin{itemize}
\item \textbf{Schedule Variance (SV): }[];
\item \textbf{Budget Variance (BV): }[].
\end{itemize}

\paragraph{Documentazione}
Un insieme di metriche che indicano il grado di leggibilità della documentazione relativa alle funzionalità e alle caratteristiche del sistema software prodotto.
\begin{itemize}
\item \textbf{Indice di Gulpease: }[60 - 100].
\end{itemize}

\subsection{Qualità di prodotto}
    Per garantire qualitativamente un prodotto software è necessario definire
    degli obiettivi qualitativi e garantire il loro soddisfacimento. Per
    stabilire gli obiettivi di qualità inerenti al prodotto software, il gruppo
    SWEet BIT ha deciso di seguire lo standard \glossaryItem{ISO}/\glossaryItem{IEC} 9126; questo standard
    descrive un modello per definire la qualità di un prodotto software e
    suggerisce delle metriche per poterne misurare la qualità.\\
    Lo standard \glossaryItem{ISO}/\glossaryItem{IEC} 9126 è descritto nell'Appendice \emph{A - Standard e
    \glossaryItem{Metodi} per la gestione della Qualità} del presente documento.

  \subsubsection{Procedure di controllo di qualità di prodotto}
    Il controllo per la qualità del prodotto definisce i seguenti processi:
    \begin{itemize}
      \item \textbf{SQA} (Software Quality Assurance): si occupa di assicurare
      che i processi siano implementati secondo quanto pianificato e che siano
      forniti sistemi di misurazione dei processi;
      \item \textbf{Verifica}: si occupa di accertare che l'esecuzione dei
      processi non abbia introdotto degli errori, e accerta il rispetto delle
      regole, delle convenzioni e delle procedure;
      \item \textbf{Validazione}: si occupa di accertare che i prodotti
      realizzati siano conformi alle attese.
    \end{itemize}

\subsubsection{Obiettivi di qualità di prodotto}  
Gli obiettivi di qualità del software che il gruppo SWEet BIT vuole raggiungere nel progetto, sono un sottoinsieme di quelli definiti dallo standard ISO/IEC 9126:2001.
Per ogni metrica indicata è stato definito nelle \emph{Norme di Progetto} \VersioneNP{} come si calcola e con quale strumento.

\paragraph{Funzionalità} Il prodotto deve possedere tutte le funzionalità descritte dai requisiti obbligatori ed in gran parte anche quelle definite dai requisiti desiderabili. Gli obiettivi di funzionalità del prodotto e le relative metriche sono:
\begin{itemize}
\item \textbf{Copertura requisiti obbligatori: }[100];
\item \textbf{Copertura requisiti desiderabili: }[70 - 100].
\end{itemize}

\paragraph{Affidabilità} Il prodotto deve superare tutti i test per verificare che funzioni in tutte le situazioni in cui si può trovare.
Gli obiettivi di affidabilità del prodotto e le relative metriche sono:
\begin{itemize}
\item \textbf{Percentuale test superati: }[100].
\end{itemize}

\paragraph{Efficienza} Il prodotto non presenterà un alto grado di complessità. 
Gli obiettivi di efficienza del prodotto e le relative metriche sono:
\begin{itemize}
\item \textbf{Profondità di annidamento: }[0 - 4].
\end{itemize}

\paragraph{Manutenibilità} Il prodotto risulterà manutenibile nel tempo. 
Gli obiettivi di manutenibilità del prodotto e le relative metriche sono:
\begin{itemize}
\item \textbf{Complessità ciclomatica: }[0 - 10];
\item \textbf{Variabili inutilizzate: }[0 - 0];
\item \textbf{Argomenti per funzione: }[0 - 4];
\item \textbf{Linee di codice per linee di commento: }[>0.30];
\item \textbf{Copertura del codice: }[80 - 100];
\end{itemize}
