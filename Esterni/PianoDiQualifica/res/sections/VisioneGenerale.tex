\section{Visione generale delle strategie di verifica}
  \subsection{Definizione obiettivi di qualità}
  Questa sezione si presta a descrivere sia gli obiettivi di qualità riguardanti il prodotto
  che quelli relativi ai processi necessari alla sua produzione.
    \subsubsection{Qualità di prodotto}
    Il gruppo SWEet BIT si è impegnato ad aderire allo standard \glossaryItem{ISO}/\glossaryItem{IEC} 9126, in modo garantire
    le seguenti caratteristiche di qualità per il prodotto:
    \begin{itemize}
      \item \textbf{Funzionalità}: il prodotto deve soddisfare i requisiti indicati nel documento \emph{Analisi dei Requisiti \VersioneAR{}};

      \item \textbf{Affidabilità}: il software prodotto deve essere robusto e deve essere prevista una completa gestione degli errori ed eccezioni,
       in modo da prevenire perdite di dati e facilitarne il ripristino;

      \item \textbf{Usabilità}: il software prodotto deve risultare semplice e di facile utilizzo da parte dell'\glossaryItem{Utente}.
      Esso deve essere di facile apprendimento e l'interfaccia \glossaryItem{Utente} deve risultare familiare e intuitiva. Il prodotto deve soddisfare al meglio le esigenze dell'\glossaryItem{Utente} finale. \\
      La documentazione per l'utilizzo del prodotto deve essere esaustiva e di facile comprensione;

      \item \textbf{Efficienza}: Il software deve essere capace di fornire tutti i servizi attesi con il minimo dispendio di risorse;

      \item \textbf{Manutenibilità}: Il software prodotto deve essere facilmente modificabile, in caso di correzione di errori e migliorie da apportare. Esso deve essere facilmente
      adattabile in caso di cambio di ambiente operativo e/o cambio di requisiti;

      \item \textbf{Portabilità}: Il software prodotto deve, per la parte \glossaryItem{front end}, offrire gli stessi servizi nella maggior parte dei \glossaryItem{browser}.
      La parte di \glossaryItem{back end} deve poter funzionare su vari sistemi operativi.
    \end{itemize}

    Per ognuna delle caratteristiche principali appena elencate esistono diverse sotto-caratteristiche,
    le quali sono approfondite nell'Appendice \emph{A - Standard e \glossaryItem{Metodi} per la gestione della Qualità} di questo documento, dove è descritto lo standard \glossaryItem{ISO}/\glossaryItem{IEC} 9126.

    \subsubsection{Qualità di processo}
    La qualità dei processi che portano allo sviluppo di un software hanno un ruolo fondamentale sulla qualità del prodotto.
    Il gruppo SWEet BIT ha deciso di adottare il \emph{Ciclo di Deming}, o \emph{Ciclo} \glossaryItem{PDCA}, come modello per il continuo miglioramento dei processi produttivi.
    Come complemento al ciclo di Deming è stato scelto il modello descritto dallo standard \glossaryItem{ISO}/\glossaryItem{IEC} 15504 detto \glossaryItem{SPICE},
    il quale fornisce gli strumenti per la valutazione dei processi produttivi.
    Sia lo standard \glossaryItem{ISO}/\glossaryItem{IEC} 15504 che il \emph{Ciclo di Deming} sono descritti nell'Appendice \emph{A - Standard e \glossaryItem{Metodi} per la gestione della Qualità} del presente documento.

  \subsection{Organizzazione}
    Per ognuno dei periodi descritto dal \VersionePP{} è necessario seguire una verifica mirata al tipo di processo e relativo prodotto:
    \begin{itemize}
      \item \textbf{Analisi:} in questo periodo vengono controllati i processi e la documentazione prodotta.
      La verifica dovrà essere eseguita secondo i metodi descritti nel documento \emph{Norme di Progetto \VersioneNP{}};
      \item \textbf{Consolidamento dei requisiti:} durante questo periodo sono verificati i processi che portano all'incremento dei documenti, e i documenti stessi, prodotti durante il periodo di analisi.
      Le procedure di verifica sono descritte nel documento \emph{Norme di Progetto \VersioneNP{}};
      \item \textbf{Progettazione Architetturale:} in questo periodo vengono verificati i processi, e relativi prodotti, che portano all'incremento dei documenti redatti nel precedente periodo di \emph{Consolidamento dei Requisiti};
      sempre in questa fase sono verificati i processi e prodotti riguardanti la \emph{Progettazione Architetturale}.
      Le procedure di verifica sono descritte in dettaglio nel documento \emph{Norme di Progetto \VersioneNP{}};
      \item \textbf{Progettazione di Dettaglio e Codifica:} anche in questo periodo vengono verificati i processi, e relativi prodotti, che portano all'incremento dei documenti del periodo di \emph{Progettazione Architetturale}.
      Vengono verificati processi e prodotti della sotto-fase di codifica. Le procedure di verifica sono descritte nel documento \emph{Norme di Progetto \VersioneNP{}}.
      \item \textbf{Verifica e Validazione:} sono verificati i processi che portano all'aggiornamento della documentazione e i documenti stessi.
      In questo periodo vengono effettuati i test per la validazione del prodotto finale.
    \end{itemize}

  \subsection{Pianificazione strategica e temporale}
    Intendendo rispettare le scadenze specificate nel \emph{Piano di Progetto \VersionePP{}}, è necessario un approccio organico e ben pianificato alle attività di verifica.
    Deve essere pianificato ogni incremento su \glossaryItem{Codice} e documentazione e di conseguenza lo dovranno essere tutte le attività di verifica,
    in modo da avere il tempo di effettuare le dovute correzioni. Ogni processo di verifica dovrà essere automatizzato il più possibile per poter, dove possibile,
    risparmiare su risorse umane. L'utilizzo di software apposito, e dei metodi per la verifica, sono descritti nel documento \emph{Norme di Progetto \VersioneNP{}}.

  \subsection{Gestione delle responsabilità}
    Il \emph{Responsabile di Progetto} e i \emph{Verificatori} hanno la responsabilità delle attività di verifica.
    I compiti di queste figure sono argomento trattato nel documento \emph{Piano di Progetto \VersionePP{}}.

  \subsection{Risorse}
    Si possono identificare due distinti tipi di risorse necessari a garantire la qualità dei prodotti e processi e sono:
    \begin{itemize}
      \item \textbf{Risorse umane:} riguardano i ruoli rivestiti dai vari componenti del gruppo SWEet BIT:
      \begin{itemize}
        \item Responsabile di progetto;
        \item Amministratore;
        \item Analista;
        \item Progettista;
        \item Programmatore;
        \item Verificatore.
      \end{itemize}

		La responsabilità maggiore per l’attività di verifica e validazione ricade sul Responsabile di Progetto e sul Verificatore. Per una descrizione dettagliata di tutti gli altri ruoli e delle loro responsabilità fare riferimento al \emph{Piano di Progetto \VersionePP{}}.      

      \item \textbf{Risorse tecnologiche:} riguardano tutti i software utilizzati per l'automazione delle attività di verifica su prodotti e processi, e l'hardware necessario per eseguire i suddetti software. Nel documento \emph{Norme di Progetto \VersioneNP{}} sono descritti nel dettaglio tutti gli strumenti software utilizzati.
    \end{itemize}
    
\subsection{Tecniche di analisi}
    \subsubsection{Analisi statica}
    L'analisi statica è il processo di verifica del sistema o di un suo componente, senza che esso debba necessariamente poter essere eseguito.
    Questa tecnica permette di verificare sia la documentazione che il \glossaryItem{Codice}, individuandone errori ed anomalie. Questa tecnica di analisi può essere svolta in due modi distinti e complementari.
    
    \subsubsubsection{Walkthrough}
    Si svolge effettuando una lettura critica e a largo spettro del documento.
È una tecnica utilizzata soprattutto nelle prime attività del progetto, quando ancora
non è presente un'esperienza tale da permettere una verifica più mirata e precisa. 
Il verificatore genererà, infatti, una lista di controllo con gli errori più frequenti
in modo da favorire il miglioramento della verifica nei periodi successivi.
Il walkthrough è un'attività onerosa e collaborativa che richiede l'intervento di
più persone per essere efficace. Questa tecnica si svolge in tre fasi principali:
\begin{itemize}
\item \textbf{Fase uno:} lettura dei documenti ed individuazione degli errori;
\item \textbf{Fase due:} discussione dei errori riscontrati e delle correzioni da applicare;
\item \textbf{Fase tre:} correzione degli errori rilevati, e stesura di un rapporto che ne elenchi le modifiche effettuate
\end{itemize}

La lista di controllo risultante, contenente le tipologie di errori più frequenti è in appendice alle \emph{Norme di progetto \VersoneNP{}}

    \subsubsubsection{Inspection}
	Consiste nell'analisi mirata di alcune parti del documenti o del codice ritenute fonti maggiori di errore. Deve essere seguita una lista di controllo per svolgere efficacemente questa attività; tale lista deve essere redatta anticipatamente ed è sostanzialmente frutto dell'esperienza maturata dai membri del team con tecniche di walkthrough.
L'inspection è dunque più rapida del walkthrough, in quanto il documento viene
analizzato solo in alcune sue parti e con una lista di controllo ben precisa. In questa
attività sono coinvolte solo i verificatori che, dopo aver individuato gli errori, procedono
alla loro correzione e alla redazione di un rapporto che tenga traccia del lavoro svolto.

\subsubsection{Analisi dinamica}
    L'analisi dinamica è il processo che verifica il prodotto software mentre esso è in esecuzione. Viene effettuata tramite test che devono essere coerentemente pianificati in modo da evitare spreco di risorse.
    L'obiettivo dei test sul software è la realizzazione di un prodotto il più possibile esente da errori. Il principale ostacolo alla fase di test è sintetizzato nella tesi di Dijkstra, che afferma che il test può indicare la presenza di errori, ma non ne può garantire l'assenza.
Affinché tale attività sia utile e generi risultati attendibili è necessario che i test effettuati
siano ripetibili: dato un certo input deve essere prodotto sempre uno stesso output in
uno specifico ambiente. Di conseguenza, i tre elementi fondamentali di un test sono:    
    \begin{itemize}
    \item \textbf{Ambiente:}sistema hardware e software sui quali è stato pianificato l'utilizzo del
prodotto software sviluppato. Su di essi deve essere specificato uno stato iniziale
dal quale poter eseguire il test;
    \item \textbf{Specifica:}definizione di quali input sono necessari per l'esecuzione del test e quali output sono attesi;
    \item \textbf{Procedure:}definizione di come devono essere svolti i test, in che ordine devono
essere eseguiti e come devono essere analizzati i risultati.
    \end{itemize}
    
    
    Organizzazione delle attività di test:
    \begin{itemize}
      \item \textbf{Test di unità}: l'obiettivo di questo test è verificare che ogni unità del \glossaryItem{Codice} si comporti esattamente come previsto. Un unità di \glossaryItem{Codice} è la più piccola parte di software che è utile verificare singolarmente e che viene prodotta da un singolo programmatore. Attraverso tali test si vuole verificare il corretto funzionamento dei moduli che compongono l'intero sistema, in modo da esaminare possibili errori di implementazione da parte dei programmatori.

      \item \textbf{Test di integrazione}: l'obiettivo di questo test è verificare che le componenti del sistema vengano aggiunte incrementalmente al prodotto e analizzare che la combinazione di due o più unità software funzioni come previsto. Una componente è l'aggregazione di più unità software.
Questo tipo di test serve ad individuare errori residui nella realizzazione dei singoli moduli, modifiche delle interfacce e comportamenti inaspettati di componenti software preesistenti forniti da terze parti che non si conoscono a fondo.  
er effettuare tali test devono essere aggiunte delle componenti fittizie al posto di quelle che non sono ancora state sviluppate, in modo da non influenzare negativamente l'esito dell'analisi.    

      \item \textbf{Test di sistema}: l'obiettivo di questo test è verificare il comportamento del prodotto software finale. Lo scopo è quello di capire se esso rispetta i requisiti individuati nella fase di Analisi.
      
      \item \textbf{Test di regressione}: l'obiettivo di questo test è stabilire se le modifiche apportate al software hanno compromesso componenti software precedentemente funzionanti.
      Questi test vengono eseguiti ogni volta che viene apportata una modifica al software.
      Tale operazione è aiutata dal tracciamento, che permette di individuare e ripetere facilmente i test di unità, integrazione ed eventualmente sistema che sono stati potenzialmente influenzati dalla modifica;
      \item \textbf{Test di accettazione}: l'obiettivo di questo test è la realizzazione del collaudo del prodotto software eseguito in presenza del proponente. Se l'esito risulta positivo, si può procedere al rilascio ufficiale del prodotto.
    \end{itemize}