\section{Strategia di verifica nel dettaglio}
  Questa sezione del documento descrive le metriche utilizzate per la quantificazione della qualità e le tecniche di analisi adottate.

  \subsection{Metriche software e misurazioni}
  Le metrice ritenute necessarie per una corretta misurazione della qualità di processi e prodotti è descritta in questa sezione;
  l'attività di verifica si baserà su queste metriche e per ognuna di queste è stato necessario stabilire due range:
  \begin{itemize}
    \item \textbf{Range di accettazione:} l'intervallo di valori entro il quale deve trovarsi il risultato della misurazione di un processo, o prodotto, per essere ritenuto accettabile.
    \item \textbf{Range di ottimalità:} l'intervallo di valori in cui si deve trovare il risultato della misurazione di un processo, o prodotto, per essere considerato ottimale;
    questo range è l'obiettivo del gruppo SWEet BIT.
  \end{itemize}
    \subsubsection{Metriche riguardanti i processi}
    Per la verifica dei processi sono state adottate due metriche di seguito descritte:
    \begin{description}
      \item[Schedule Variance (SV):] È un indice che dà informazione necessarie a determinare se ci si trova in anticipo,
      in ritardo o in linea alle tempistiche delle attività di progetto.
      La seguente formula dà SV in termini di costo:
      \begin{center}
        \emph{SV = BCWP - BCWS}
      \end{center}
      Dove:
      \begin{itemize}
        \item BCWP = costo totale del lavoro svolto al momento della misurazione;
        \item BCWS = costo totale del lavoro pianificato al momento della misurazione.
      \end{itemize}
      SV ha tre significativi risultati:
      \begin{itemize}
        \item SV>0 indica che si è avanti rispetto alle pianificazione temporale del lavoro;
        \item SV=0 indica che si è in linea alle tempistiche delle attività di progetto;
        \item SV<0 indica che si è in ritardo rispetto alla pianificazione temporale delle attività.
      \end{itemize}
      I range stabiliti sono:
      \begin{itemize}
        \item Range di accettazione = [\(\geq\)-(\emph{preventivo fase} * 5\%)]
        \item Range di ottimalità = [\(\geq\)0]
      \end{itemize}
      \item[Cost Variance (CV):] indica se vi sono state più o meno spese rispetto al previsto.
      La seguente formula dà CV in termini di costo:
      \begin{center}
        \emph{CV = BCWP - ACWP}
      \end{center}
      Dove:
      \begin{itemize}
        \item BCWP = costo totale del lavoro svolto al momento della misurazione;
        \item ACWP = costo totale richiesto per il completamento del lavoro al momento della misurazione.
      \end{itemize}
      CV ha tre significativi risultati:
      \begin{itemize}
        \item CV>0 indica che il progetto sta avendo un costo inferiore rispetto a quanto preventivato;
        \item CV=0 indica che il progetto ha un costo in linea a quanto preventivato;
        \item CV<0 indica che il progetto ha superato il costo preventivato.
      \end{itemize}
      I range stabiliti sono:
      \begin{itemize}
        \item Range di accettazione = [\(\geq\)-(\emph{preventivo fase} * 10\%)]
        \item Range di ottimalità = [\(\geq\)0]
      \end{itemize}
    \end{description}
    \subsubsection{Metriche riguardanti i documenti}
    \begin{description}
      \item[Indice Gulpease:] è un idice di leggibilità di un testo per la lungua italiana.
      Questo indice considera due variabili linguistiche: la lunghezza della parola e la lunghezza della frase rispetto al numero delle lettere.
      La formula per il calcolo dell'indice Gulpease è:
      \begin{center}
        \( 89+\frac{300 * (\emph{numero delle frasi}) - 10 * (\emph{numero delle lettere})}{\emph{numero delle parole}} \)
      \end{center}
      Il risultato è un numero nell'intervallo [0-100], generalmente risulta che:
      \begin{itemize}
        \item inferiore a 80 il testo è difficile da leggere per chi ha la licenza elementare;
        \item inferiore a 60 il testo è difficile da leggere per chi ha la licenza media;
        \item inferiore a 40 il testo è difficile da leggere per chi ha un diploma superiore.
      \end{itemize}
      I range stabiliti sono:
      \begin{itemize}
        \item Range di accettazione = [40-100]
        \item Range di ottimalità = [50-100]
      \end{itemize}
    \end{description}
    \subsubsection{Metriche riguardanti il software prodotto}
    Come descritto nello standard \glossaryItem{ISO}/\glossaryItem{IEC} 9126, si posso identificare due tipi di metriche utilizzate per misurare la qualita del prodotto software, esse sono:
    \begin{itemize}
      \item \emph{Metriche interne}: vengono applicate al software non eseguibile durante le fasi di progettazione e codifica.
      \item \emph{Metriche esterne}: servono per poter misurare il comportamento del prodotto software attraverso test effettuati in fase di esecuzione.
    \end{itemize}
    Per ogni caratteristica identificata nella \S2.1.1 del presente documento, è stata trovata almeno una metrica interna e una esterna, esse sono:
    \begin{itemize}
      \item \textbf{Funzionalità} come metrica interna verrà utilizzata l'\textbf{accuratezza delle funzioni sviluppate}, lo scopo è qullo di poter misurare
      il livello di accuratezza con cui le funzioni sviluppate rispettano quelle indentificate nell'analisi dei requisiti. La formula è la seguente:
      \begin{center}
        AF = Fa / Fs
      \end{center}
      Dove:
      \begin{itemize}
        \item AF = accuratezza funzione;
        \item Fa = numero di funzioni sviluppate con l'accuratezza richiesta;
        \item Fs = numero totale di funzioni sviluppate.
      \end{itemize}
      come metrica esterna 
    \end{itemize}






  \subsection{Tecniche di analisi}
    \subsubsection{Analisi statica}
    L'analisi statica è il processo di verifica del sistema o di un suo componente, senza che esso debba necessariamente poter essere eseguito.
    Questa tecnica permette di verificare sia la documentazione che il codice.
    Durante il processo di analisi statica vengono utilizzate le metriche di qualità interna per la verifica del prodotto software.
    \subsubsection{Analisi dinamica}
    L'analisi dinamica è il processo che verifica il prodotto software mentre esso è in esecuzione; viene effettuata tramite test che devono essere coerentemente pianificati in modo da evitare spreco di risosrse.
    Durante l'attività di analisi dinamica vengono utilizzate le metriche di qualità esterna per la verifica del prodotto software in esecuzione.
    Organizzazione delle attività di test:
    \begin{itemize}
      \item \textbf{Test di unità}: L'obiettivo di questa fase è quello di riuscire a capire se ogni unità del codice si comporti esattamente come previsto, un unità di codice è la più piccola parte di software testabile.
      \item \textbf{Test di integrazione}: In questa fase di test viene testata una componente, una componente è l'aggragazione di più unità software.
      Questo tipo di test consente di trovare errori che si verificano nell'aggregazione di varie unità.
      \item \textbf{Test di sistema}: in questa fase viene verificato il comportamento del prodotto software finale, lo scopo è quello di capire se esso rispetta i requisiti individuati nella fase di analisi.
      \item \textbf{Test di regressione}: Questo test ha lo scopo di stabilire se le modifiche apportate al software hanno compromesso componenti software precedentemente funzionanti.
      Questi test vengono eseguiti ogni volta che viene apportata una modifica al software.
    \end{itemize}
