\section{Dettaglio esito revisioni}
Vengono di seguito elencate le modifiche applicate in seguito alle segnalazioni di problematiche riscontrate durante le revisioni. 

\subsection{Revisioni dei Requisiti}
\begin{itemize}
\item \emph{Norme di Progetto: }il documento è stato integrato con i contenuti richiesti riguardanti l'attività di progettazione, in particolare sono stati incrementati i capitoli riguardanti le attività tecniche di sviluppo e verifica.\\
Il documento è stato ristrutturato per aderire meglio alle \emph{best practice}. 

\item \emph{Analisi dei Requisiti: }sono stati rivisti e riadattati alcuni casi d'uso modificandone le relazioni di inclusione ed estensione. Sono stati rinominati alcuni titoli di casi d'uso in quanto troppo legati alla struttura dell'applicazione e non alle funzionalità esposte dal sistema.

\item \emph{Piano di Progetto: }il documento è stato ristrutturato per aderire meglio alle \emph{best practice}. É stata incrementata la sezione inerente al modello di sviluppo. Sono stati ricalcolati i preventivi erroneamente calcolati a carico del committente.

\item \emph{Piano di Qualifica: }il documento ha subito una profonda ristrutturazione in seguito alle segnalazioni fornite. Sono state fornite delle metriche più realistiche e più facili da applicare tramite sistemi automatizzati. Sono state trasferite nelle \emph{Norme di Progetto} le sezioni riguardanti gli strumenti e le procedure di verifica

\item \emph{Glossario: }sono stati rimossi i capitoli iniziali di introduzione, non necessari ai fini dell'uso del documento

\end{itemize}

\subsection{Revisione di Progettazione}
\begin{itemize}
\item \emph{Norme di Progetto: }migliorata la gestione dei processi. Sono stati definiti e spiegati come calcolare gli obiettivi di qualità definiti del \emph{Piano di Qualifica}

\item \emph{Analisi dei Requisiti: }sono stati rivisti alcuni casi d'uso non correttamente posizionati, avendo delle relazioni errate

\item \emph{Specifica Tecnica: }sono state rivisitate e migliorate le sezioni riguardanti le tecnologie utilizzate, definendo i pro e contro per tutte le tecnologie. Sono state specificate maggiormente le relazioni tra le componenti logiche, e si è scesi in un maggior livello di profondità per quanto riguarda la descrizione del front-end e del back-end;

\item \emph{Piano di Progetto: }migliorati e corretti i preventivi secondo le indicazioni, migliorata la pianificazione seguendo un modello incrementale

\item \emph{Piano di Qualifica: }spostati nelle \emph{Norme di Progetto} i contenuti non inerenti al Piano di Qualifica. Le metriche sono state rappresentate in relazione agli obiettivi di qualità richiesti dallo standard. É stato riportato lo stato di implementazione dei test eseguiti. Per essere in forma incrementale sono stati forniti gli esiti delle verifiche fatte in seguito alle revisioni.
\end{itemize}

\subsection{Revisione di Qualifica}
\begin{itemize}
\item \emph{Norme di Progetto: }sono state riorganizzate e approfondite alcune parti. Aggiunto raccordo con gli obiettivi di qualità del \emph{Piano di Qualifica};

\item \emph{Analisi dei Requisiti: }sono stati corretti i casi d'uso segnalati alla scorsa revisione;

\item \emph{Specifica Tecnica: }sono stati corretti gli errori riguardanti i diagrammi delle classi e i design pattern;

\item \emph{Piano di Progetto: }sono stati aggiunti il consuntivo di periodo e il consuntivo finale ed è stata aggiornata la tabella relativa all'attualizzazione dei rischi;

\item \emph{Piano di Qualifica: }sono state riorganizzate alcune parti al fine di renderne migliore la struttura per flusso dell'informazione;

\item \emph{Definizione di Prodotto: }sono stati corretti gli errori segnalati alla scorsa revisione.
\end{itemize}