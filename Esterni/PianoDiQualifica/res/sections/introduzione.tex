\section{Introduzione}
  \subsection{Scopo del documento}
          Questo documento descrive le strategie adottate dal gruppo SWEet BIT per il conseguimento degli obiettivi di qualità riguardanti il prodotto.
          Il raggiungimento di tali obiettivi è possibile solo tramite una precisa e continua verifica delle attività svolte; così facendo è possibile
          rilevare e correggere tempestivamente eventuali anomalie, minimizzando lo spreco di risorse.
  \subsection{Scopo del Prodotto}
          Lo scopo del progetto è la realizzazione di una \glossaryItem{Web App} che fornisca all'\glossaryItem{Utente} un \glossaryItem{UML} \glossaryItem{Designer} con il quale riuscire a disegnare correttamente \glossaryItem{Diagrammi delle classi}
          e descrivere il comportamento dei \glossaryItem{Metodi} interni alle stesse attraverso l'utilizzo del \glossaryItem{Diagramma delle attività}.
          La \glossaryItem{Web App} permetterà all'\glossaryItem{Utente} di generare \glossaryItem{Codice} \glossaryItem{Java} dal \glossaryItem{Diagramma} disegnato ed eventualmente andare a ritoccarne il risultato al fine di ottenere un \glossaryItem{Codice}
          eseguibile, funzionante e funzionale.
  \subsection{Glossario}
          Con lo scopo di evitare ambiguità di linguaggio e di massimizzare la comprensione dei documenti, il
          gruppo ha steso un documento interno che è il \emph{Glossario \VersioneG{}}. In esso saranno definiti, in modo
          chiaro e conciso, i termini che possono causare ambiguità o incomprensione del testo.
  \subsection{Riferimenti}
    \subsubsection{Informativi}
      \begin{itemize}
        \item \textbf{Analisi dei requisiti:} \VersioneAR{}
        \item \textbf{Piano di progetto:} \VersionePP{}
        \item \textbf{Lucidi dell'insegnamento di Ingegneria del \emph{Software}:}\\
        \url{http://www.math.unipd.it/~tullio/IS-1/2016/}
        \item \textbf{Libro SWEBOK v3.0: \emph{Chapter 10: Software Quality}:}\\
        \url{https://www.computer.org/web/swebok/}
        \item \textbf{\glossaryItem{ISO}/\glossaryItem{IEC} 9126 \emph{Software engineering — Product quality}:}\\
        \url{https://en.wikipedia.org/wiki/ISO/IEC_9126}
        \item \textbf{\glossaryItem{ISO}/\glossaryItem{IEC} 15504 \emph{Information technology – Process assessment}:}\\
        \url{https://en.wikipedia.org/wiki/ISO/IEC_15504}
      \end{itemize}
    \subsubsection{Normativi}
      \begin{itemize}
        \item \textbf{Norme di progetto:} \VersioneNP{}
        \item \textbf{Capitolato di appalto SWE\glossaryItem{Designer} (C6):}\\
        \url{http://www.math.unipd.it/~tullio/IS-1/2016/Progetto/C6.pdf}
      \end{itemize}
