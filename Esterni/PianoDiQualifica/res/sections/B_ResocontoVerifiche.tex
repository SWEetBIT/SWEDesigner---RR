\section{Resoconto delle attività di verifica}
  In questa sezione vengono riportati i risultati delle attività di verifica svolte.
  \subsection{Revisione dei requisiti}
  Prima della consegna relativa alla revisione dei requisiti, sono stati verificati i processi che hanno portato alla stesura dei documenti ed essi stessi. \\
  Per i documenti è stata effettuata una verifica di tipo manuale, la quale consiste nella rilettura dei documenti per l'individuazione di errori di forma ed eventuali inconsistenze;
  e una verifica automatizzata tramite gli strumenti definiti nel documento \emph{Norme di Progetto \VersioneNP{}}, di seguito citati per completezza:
  \begin{itemize}
    \item \glossaryItem{Script} \glossaryItem{Perl} per il calcolo dell’indice Gulpease
    \item \glossaryItem{Script} \glossaryItem{Perl} per la glossarizzazione dei termini
    \item Aspel
  \end{itemize}
  \subsubsection{Dettaglio delle verifiche}
    \paragraph{Analisi}
      \subparagraph{Processi}
      Nella seguente tabella sono riportati i valori per la Schedule Variance(SV) e la Budget Variance(BV) riguardanti i processi del periodo di \textbf{Analisi}.\\
      \begin{itemize}
      \item \textbf{Schedule Variance:} I range stabiliti sono:
      \begin{itemize}
        \item Range di accettabilità = [\(\geq\)-(\emph{141.25})];
        \item Range di ottimalità = [\(\geq\)0].
      \end{itemize}
      
      \item \textbf{Budget Variance:} I range stabiliti sono:
      \begin{itemize}
        \item Range di accettabilità = [\(\geq\)-(\emph{282.5})];
        \item Range di ottimalità = [\(\geq\)0].
      \end{itemize}
      
      
      \end{itemize}
     
	  \rowcolors{1}{}{}
      \begin{table}[H]
        \centering
        \begin{tabular}{|c|c|c|}
          \hline
          \textbf{Attività} & \textbf{SV}(Euro)  & \textbf{BV}(Euro) \\
          \hline
          Analisi dei requisiti  & 25 & -45  \\
          Piano di progetto & 35  & 0\\
          Piano di qualifica  & 0  & -20\\
          Norme di progetto & 0  & 0 \\
          Studio di fattibilità & 0  & 10  \\
          Glossario & 0  & 0  \\
          \hline
          \textbf{Totale} & 60  & -65  \\
          \hline
        \end{tabular}
        \caption{Indici SV e BV - Periodo di Analisi}
      \end{table}
      Stando a quanto preventivato dal prospetto economico del \emph{Piano di Progetto \VersionePP{}}, il valore di SV rientra nel range di accettabilità stabilito (SV\(\geq\)-141.25 Euro) che in quello di ottimalità;
      anche il valore di BV rientra nel range di accettabilità stabilito (BV\(\geq\)-282.5 Euro).
      
      \subparagraph{Documenti}
      Nella tabella seguente sono riportati i valori dell'indice Gulpease per ogni documento prodotto nel periodo di Analisi.\\

\begin{itemize}
\item \textbf{Indice Gulpease: }I range stabiliti sono:
      \begin{itemize}
        \item Range di accettabilità = [40-100];
        \item Range di ottimalità = [50-100].
      \end{itemize}
\end{itemize}      
      
      
      \begin{table}[H]
        \centering
        \begin{tabular}{|c|c|c|}
          \hline
          \textbf{Documento} & \textbf{Indice Gulpease} & \textbf{Esito}\\
          \hline
          Analisi dei requisiti & 74.3  & Superato \\
          Piano di progetto & 51.1  & Superato \\
          Piano di qualifica  & 51.3  & Superato \\
          Norme di progetto & 50.7  & Superato \\
          Studio di fattibilità & 49.5  & Superato \\
          Glossario & 51.6  & Superato  \\
          Verbali esterni & 51.1  & Superato \\
          \hline
        \end{tabular}
        \caption{Indici Gulpease per i documenti - Periodo di Analisi}
      \end{table}

\subparagraph{Software}
\begin{itemize}
\item \textbf{Complessità ciclomatica: }I range stabiliti sono:
      \begin{itemize}
        \item Range di accettabilità = [0-15];
        \item Range di ottimalità = [0-10].
      \end{itemize}
Per il periodo di \textbf{Analisi} non è stata applicata questa metrica.

\item \textbf{Numero di metodi per file: }I range stabiliti sono:
      \begin{itemize}
        \item Range di accettabilità = [3-10];
        \item Range di ottimalità = [0-7].
      \end{itemize}
Per il periodo di \textbf{Analisi} non è stata applicata questa metrica.

\item \textbf{Variabili non utilizzate e/o non definite: }I range stabiliti sono:
      \begin{itemize}
        \item Range di accettabilità = [0-0];
        \item Range di ottimalità = [0-0].
      \end{itemize}
Per il periodo di \textbf{Analisi} non è stata applicata questa metrica.

\item \textbf{Numero di argomenti per funzione: }I range stabiliti sono:
      \begin{itemize}
        \item Range di accettabilità = [0-6];
        \item Range di ottimalità = [0-4].
      \end{itemize}
Per il periodo di \textbf{Analisi} non è stata applicata questa metrica.

\item \textbf{Linee di codice per linee di commento: } I range stabiliti sono:
      \begin{itemize}
        \item Range di accettabilità = [>0.25];
        \item Range di ottimalità = [>0.30].
      \end{itemize}
Per il periodo di \textbf{Analisi} non è stata applicata questa metrica.
\end{itemize}

\begin{itemize}
\item \textbf{Copertura del codice: }I range stabiliti sono:
      \begin{itemize}
        \item Range di accettabilità = [70\%-100\%];
        \item Range di ottimalità = [80\%-100\%].
      \end{itemize}
Per il periodo di \textbf{Analisi} non è stata applicata questa metrica.
\end{itemize}



  \subsection{Revisione di progettazione}
  Prima della consegna relativa alla revisione di progettazione minima, sono stati verificati i processi che hanno portato alla stesura dei documenti ed essi stessi. \\
  Per i documenti è stata effettuata una verifica di tipo manuale, la quale consiste nella rilettura dei documenti per l'individuazione di errori di forma ed eventuali inconsistenze;
  e una verifica automatizzata tramite gli strumenti definiti nel documento \emph{Norme di Progetto \VersioneNP{}}, di seguito citati per completezza:
  \begin{itemize}
    \item \glossaryItem{Script} \glossaryItem{Perl} per il calcolo dell’indice Gulpease
    \item \glossaryItem{Script} \glossaryItem{Perl} per la glossarizzazione dei termini
    \item Aspel
  \end{itemize}
  \subsubsection{Dettaglio delle verifiche}
    \paragraph{Progettazione Architetturale}
      \subparagraph{Processi}
      Nella seguente tabella sono riportati i valori per la Schedule Variance(SV) e la Budget Variance(BV) riguardanti i processi del periodo di \textbf{Progettazione Architetturale}.\\
      \begin{itemize}
      \item \textbf{Schedule Variance:} I range stabiliti sono:
      \begin{itemize}
        \item Range di accettabilità = [\(\geq\)-(\emph{141.25})];
        \item Range di ottimalità = [\(\geq\)0].
      \end{itemize}
      
      \item \textbf{Budget Variance:} I range stabiliti sono:
      \begin{itemize}
        \item Range di accettabilità = [\(\geq\)-(\emph{282.5})];
        \item Range di ottimalità = [\(\geq\)0].
      \end{itemize}
      
      
      \end{itemize}
     
       
      \begin{table}[H]
        \centering
        \begin{tabular}{|c|c|c|}
          \hline
          \textbf{Attività} & \textbf{SV}(Euro)  & \textbf{BV}(Euro) \\
          \hline
          Analisi dei requisiti  & -20 & 0  \\
          Piano di progetto & 0  & -90*\\
          Piano di qualifica  & +25  & 0\\
          Norme di progetto & 0  & 0 \\
          Studio di fattibilità & 0  & 0  \\
          Glossario & 0  & 0  \\
          Specifica Tecnica & 0 & 0\\
          \hline
          \textbf{Totale} & 45  & -90  \\
          \hline
        \end{tabular}
        \caption{Indici SV e BV - Periodo di Progettazione Architetturale}
      \end{table}
      Stando a quanto preventivato dal prospetto economico del \emph{Piano di Progetto \VersionePP{}}, il valore di SV rientra nel range di accettabilità stabilito (SV\(\geq\)-141.25 Euro) che in quello di ottimalità;
      anche il valore di BV rientra nel range di accettabilità stabilito (BV\(\geq\)-282.5 Euro).\\
      * è stato necessario allocare un numero maggiore di ore per il \emph{Piano di Progetto} respetto a quanto preventivato in seguito alla valutazione negativa ottenuta in sede di revisione RR. Tali ore sono state sottratte all'attività di riadattamento del documento \emph{Piano di Qualifica}\VersionePQ{}.
      
      \subparagraph{Documenti}
      Nella tabella seguente sono riportati i valori dell'indice Gulpease per ogni documento prodotto nel periodo di \textit{Progettazione Architetturale}.\\

\begin{itemize}
\item \textbf{Indice Gulpease: }I range stabiliti sono:
      \begin{itemize}
        \item Range di accettabilità = [40-100];
        \item Range di ottimalità = [50-100].
      \end{itemize}
\end{itemize}      
      
      
      \begin{table}[H]
        \centering
        \begin{tabular}{|c|c|c|}
          \hline
          \textbf{Documento} & \textbf{Indice Gulpease} & \textbf{Esito}\\
          \hline
          Analisi dei requisiti & 71  & Superato \\
          Piano di progetto & 52  & Superato \\
          Piano di qualifica  & 63  & Superato \\
          Norme di progetto & 53  & Superato \\
          Studio di fattibilità & 50  & Superato \\
          Glossario & 49  & Superato  \\
          Verbale Interno 2017/04/24 & 55  & Superato \\
          Verbale Interno 2017/04/28 & 56 & Superato \\
          Verbale Interno 2017/05/02 & 56  & Superato \\
          Verbale Interno 2017/05/03 & 54  & Superato \\
          Verbale Esterno 2017/02/23 & 52 & Superato \\
          Verbale Esterno 2017/03/15 & 51 & Superato \\
          Specifica Tecnica & 52 & Superato\\
          \hline
        \end{tabular}
        \caption{Indici Gulpease per i documenti - Periodo di Progettazione Architetturale}
      \end{table}

\subparagraph{Software}
\begin{itemize}
\item \textbf{Complessità ciclomatica: }I range stabiliti sono:
      \begin{itemize}
        \item Range di accettabilità = [0-15];
        \item Range di ottimalità = [0-10].
      \end{itemize}
Per il periodo di \textbf{Progettazione Architetturale} non è stata applicata questa metrica.

\item \textbf{Numero di metodi per file: }I range stabiliti sono:
      \begin{itemize}
        \item Range di accettabilità = [3-10];
        \item Range di ottimalità = [0-7].
      \end{itemize}
Per il periodo di \textbf{Progettazione Architetturale} non è stata applicata questa metrica.

\item \textbf{Variabili non utilizzate e/o non definite: }I range stabiliti sono:
      \begin{itemize}
        \item Range di accettabilità = [0-0];
        \item Range di ottimalità = [0-0].
      \end{itemize}
Per il periodo di \textbf{Progettazione Architetturale} non è stata applicata questa metrica.

\item \textbf{Numero di argomenti per funzione: }I range stabiliti sono:
      \begin{itemize}
        \item Range di accettabilità = [0-6];
        \item Range di ottimalità = [0-4].
      \end{itemize}
Per il periodo di \textbf{Progettazione Architetturale} non è stata applicata questa metrica.

\item \textbf{Linee di codice per linee di commento: } I range stabiliti sono:
      \begin{itemize}
        \item Range di accettabilità = [>0.25];
        \item Range di ottimalità = [>0.30].
      \end{itemize}
Per il periodo di \textbf{Progettazione Architetturale} non è stata applicata questa metrica.
\end{itemize}

\begin{itemize}
\item \textbf{Copertura del codice: }I range stabiliti sono:
      \begin{itemize}
        \item Range di accettabilità = [70\%-100\%];
        \item Range di ottimalità = [80\%-100\%].
      \end{itemize}
Per il periodo di \textbf{Progettazione Architetturale} non è stata applicata questa metrica.

\end{itemize} 





\subsection{Revisione di Qualifica}
  Prima della consegna relativa alla revisione di qualifica, sono stati verificati i processi che hanno portato alla stesura dei documenti e sono state applicate delle misurazioni per verificare la qualità del software prodotto. \\
  Per i documenti è stata effettuata una verifica di tipo manuale, la quale consiste nella rilettura dei documenti per l'individuazione di errori di forma ed eventuali inconsistenze;
  e una verifica automatizzata tramite gli strumenti definiti nel documento \emph{Norme di Progetto \VersioneNP{}}, di seguito citati per completezza:
  \begin{itemize}
    \item \glossaryItem{Script} \glossaryItem{Perl} per il calcolo dell’indice Gulpease
    \item \glossaryItem{Script} \glossaryItem{Perl} per la glossarizzazione dei termini
    \item Aspel
  \end{itemize}
  Per la verifica del software sono stati utilizzati dei software di controllo, descritti nelle \emph{Norme di Progetto \VersioneNP{}}.
  \subsubsection{Dettaglio delle verifiche}
    \paragraph{Progettazione di Dettaglio e Codifica}
      \subparagraph{Processi}
      Nella seguente tabella sono riportati i valori per la Schedule Variance(SV) e la Budget Variance(BV) riguardanti i processi del periodo di \textbf{Progettazione di Dettaglio e Codifica}.\\
      \begin{itemize}
      \item \textbf{Schedule Variance:} I range stabiliti sono:
      \begin{itemize}
        \item Range di accettabilità = [\(\geq\)-(\emph{141.25})];
        \item Range di ottimalità = [\(\geq\)0].
      \end{itemize}
      
      \item \textbf{Budget Variance:} I range stabiliti sono:
      \begin{itemize}
        \item Range di accettabilità = [\(\geq\)-(\emph{282.5})];
        \item Range di ottimalità = [\(\geq\)0].
      \end{itemize}
      
      
      \end{itemize}
     
       
      \begin{table}[H]
        \centering
        \begin{tabular}{|c|c|c|}
          \hline
          \textbf{Attività} & \textbf{SV}(Euro)  & \textbf{BV}(Euro) \\
          \hline
          Analisi dei requisiti  & 0 & 0  \\
          Piano di progetto & 0 & 0\\
          Piano di qualifica  & +10  & 0\\
          Norme di progetto & 0  & 0 \\
          Studio di fattibilità & 0  & 0  \\
          Glossario & 0  & 0  \\
          Specifica Tecnica & 0 & -85*\\
          Definizione di Prodotto & 0 & 0\\
          Codifica & 0 & -40**\\
          Manuale Utente & 0 & 0\\
          \hline
          \textbf{Totale} & 10  & -125  \\
          \hline
        \end{tabular}
        \caption{Indici SV e BV - Periodo di Progettazione di Dettaglio e Codifica}
      \end{table}
      Stando a quanto preventivato dal prospetto economico del \emph{Piano di Progetto \VersionePP{}}, il valore di SV rientra nel range di accettabilità stabilito (SV\(\geq\)-141.25 Euro) che in quello di ottimalità;
      anche il valore di BV rientra nel range di accettabilità stabilito (BV\(\geq\)-282.5 Euro).
      * è stato necessario allocare un numero maggiore di ore per la \emph{Specifica Tecnica} rispetto a quanto preventivato in seguito alla valutazione negativa, e alla conseguente richiesta di miglioramento ottenuta in sede di RP.
      **è stato necessario allocare un numero maggiore di ore per la fase di codifica in quanto si sono riscontrati dei problemi hardware ad un terminale di uno sviluppatore.
      
      
      \subparagraph{Documenti}
      Nella tabella seguente sono riportati i valori dell'indice Gulpease per ogni documento prodotto nel periodo di \textit{Progettazione di Dettaglio e Codifica}.\\

\begin{itemize}
\item \textbf{Indice Gulpease: }I range stabiliti sono:
      \begin{itemize}
        \item Range di accettabilità = [40-100];
        \item Range di ottimalità = [50-100].
      \end{itemize}
\end{itemize}      
      
      
      \begin{table}[H]
        \centering
        \begin{tabular}{|c|c|c|}
          \hline
          \textbf{Documento} & \textbf{Indice Gulpease} & \textbf{Esito}\\
          \hline
          Analisi dei requisiti & 74.3  & Superato \\
          Piano di progetto & 51.1  & Superato \\
          Piano di qualifica  & 51.3  & Superato \\
          Norme di progetto & 50.7  & Superato \\
          Studio di fattibilità & 49.5  & Superato \\
          Glossario & 51.6  & Superato  \\
          Verbali esterni & 51.1  & Superato \\
          Specifica Tecnica &  & \\
          Definizione di Prodotto &  & \\
          Manuale Utente &  & \\
          \hline
        \end{tabular}
        \caption{Indici Gulpease per i documenti - Periodo di Progettazione di Dettaglio e Codifica}
      \end{table}

\subparagraph{Software}
\begin{itemize}
\item \textbf{Complessità ciclomatica: }I range stabiliti sono:
      \begin{itemize}
        \item Range di accettabilità = [0-15];
        \item Range di ottimalità = [0-10].
        \item \textbf{Media: }4;
        \item \textbf{Massimo: }12;
      \end{itemize}

\item \textbf{Variabili non utilizzate e/o non definite: }I range stabiliti sono:
      \begin{itemize}
        \item Range di accettabilità = [0-0];
        \item Range di ottimalità = [0-0].
        \item \textbf{Media: }0;
        \item \textbf{Massimo: }0;
      \end{itemize}

\item \textbf{Numero di argomenti per funzione: }I range stabiliti sono:
      \begin{itemize}
        \item Range di accettabilità = [0-6];
        \item Range di ottimalità = [0-4].
        \item \textbf{Media: }2
        \item \textbf{Massimo: }5;
      \end{itemize}


\item \textbf{Linee di codice per linee di commento: } I range stabiliti sono:
      \begin{itemize}
        \item Range di accettabilità = [>0.25];
        \item Range di ottimalità = [>0.30].
        \item \textbf{Media: }0.32;
      \end{itemize}

\item \textbf{Copertura requisiti desiderabili: }I range stabiliti sono:
      \begin{itemize}
        \item Range di accettabilità = [60\%-100\%];
        \item Range di ottimalità = [70\%-100\%].
        \item \textbf{Media: }65\%;
      \end{itemize}
      
		\item \textbf{Profondità di annidamento: }I range stabiliti sono:
      \begin{itemize}
        \item Range di accettabilità = [0-7];
        \item Range di ottimalità = [0-4].
        \item \textbf{Media: }4;
      \end{itemize}      
      
      
      \item \textbf{Copertura requisiti obbligatori: }I range stabiliti sono:
      \begin{itemize}
        \item Range di accettabilità = [90\%-100\%];
        \item Range di ottimalità = [100\%-100\%].
        \item \textbf{Media: }85\%;
      \end{itemize}
      
\item \textbf{Percentuale test superati: }I range stabiliti sono:
      \begin{itemize}
        \item Range di accettabilità = [100\%-100\%];
        \item Range di ottimalità = [100\%-100\%].
        \item \textbf{Media: }100\% sui test effettuati;
      \end{itemize}

\end{itemize}

\subsection{Revisione di Accettazione}
  Prima della consegna relativa alla revisione di accettazione, sono stati verificati i processi che hanno portato alla stesura dei documenti e sono state applicate delle misurazioni per verificare la qualità del software prodotto. \\
  Per i documenti è stata effettuata una verifica di tipo manuale, la quale consiste nella rilettura dei documenti per l'individuazione di errori di forma ed eventuali inconsistenze;
  e una verifica automatizzata tramite gli strumenti definiti nel documento \emph{Norme di Progetto \VersioneNP{}}, di seguito citati per completezza:
  \begin{itemize}
    \item \glossaryItem{Script} \glossaryItem{Perl} per il calcolo dell’indice Gulpease
    \item \glossaryItem{Script} \glossaryItem{Perl} per la glossarizzazione dei termini
    \item Aspel
  \end{itemize}
  Per la verifica del software sono stati utilizzati dei software di controllo, descritti nelle \emph{Norme di Progetto \VersioneNP{}}.
  \subsubsection{Dettaglio delle verifiche}
    \paragraph{Verifica e Validazione}
      \subparagraph{Processi}
      Nella seguente tabella sono riportati i valori per la Schedule Variance(SV) e la Budget Variance(BV) riguardanti i processi del periodo di \textbf{Verifica e Validazione}.\\
      \begin{itemize}
      \item \textbf{Schedule Variance:} I range stabiliti sono:
      \begin{itemize}
        \item Range di accettabilità = [\(\geq\)-(\emph{141.25})];
        \item Range di ottimalità = [\(\geq\)0].
      \end{itemize}
      
      \item \textbf{Budget Variance:} I range stabiliti sono:
      \begin{itemize}
        \item Range di accettabilità = [\(\geq\)-(\emph{282.5})];
        \item Range di ottimalità = [\(\geq\)0].
      \end{itemize}
      
      \end{itemize}
       
      \begin{table}[H]
        \centering
        \begin{tabular}{|c|c|c|}
          \hline
          \textbf{Attività} & \textbf{SV}(Euro)  & \textbf{BV}(Euro) \\
          \hline
          Analisi dei requisiti  & 0 & 0  \\
          Piano di progetto & 0 & 0\\
          Piano di qualifica  & 0  & 0\\
          Norme di progetto & 0  & 0 \\
          Studio di fattibilità & 0  & 0  \\
          Glossario & 0  & 0  \\
          Specifica Tecnica & 0 & 0\\
          Definizione di Prodotto & +15 & -25\\
          Codifica & 0 & -80\\
          Manuale Utente & 0 & 0\\
          \hline
          \textbf{Totale} & 15  & -105  \\
          \hline
        \end{tabular}
        \caption{Indici SV e BV - Periodo di Verifica e Validazione}
      \end{table}
      Stando a quanto preventivato dal prospetto economico del \emph{Piano di Progetto \VersionePP{}}, il valore di SV rientra nel range di accettabilità stabilito (SV\(\geq\)-141.25 Euro) che in quello di ottimalità;
      anche il valore di BV rientra nel range di accettabilità stabilito (BV\(\geq\)-282.5 Euro).      
      
      \subparagraph{Documenti}
      Nella tabella seguente sono riportati i valori dell'indice Gulpease per ogni documento prodotto nel periodo di \textit{Verifica e Validazione}.\\

\begin{itemize}
\item \textbf{Indice Gulpease: }I range stabiliti sono:
      \begin{itemize}
        \item Range di accettabilità = [40-100];
        \item Range di ottimalità = [50-100].
      \end{itemize}
\end{itemize}      
      
      
      \begin{table}[H]
        \centering
        \begin{tabular}{|c|c|c|}
          \hline
          \textbf{Documento} & \textbf{Indice Gulpease} & \textbf{Esito}\\
          \hline
          Analisi dei requisiti & 74.3  & Superato \\
          Piano di progetto & 51.1  & Superato \\
          Piano di qualifica  & 51.3  & Superato \\
          Norme di progetto & 50.7  & Superato \\
          Studio di fattibilità & 49.5  & Superato \\
          Glossario & 51.6  & Superato  \\
          Verbali esterni & 51.1  & Superato \\
          Specifica Tecnica &  & \\
          Definizione di Prodotto &  & \\
          Manuale Utente &  & \\
          \hline
        \end{tabular}
        \caption{Indici Gulpease per i documenti - Periodo di Progettazione di Dettaglio e Codifica}
      \end{table}

\subparagraph{Software}
\begin{itemize}
\item \textbf{Complessità ciclomatica: }I range stabiliti sono:
      \begin{itemize}
        \item Range di accettabilità = [0-15];
        \item Range di ottimalità = [0-10].
        \item \textbf{Media: }6;
        \item \textbf{Massimo: }15;
      \end{itemize}

\item \textbf{Variabili non utilizzate e/o non definite: }I range stabiliti sono:
      \begin{itemize}
        \item Range di accettabilità = [0-0];
        \item Range di ottimalità = [0-0].
        \item \textbf{Media: }0;
        \item \textbf{Massimo: }0;
      \end{itemize}

\item \textbf{Numero di argomenti per funzione: }I range stabiliti sono:
      \begin{itemize}
        \item Range di accettabilità = [0-6];
        \item Range di ottimalità = [0-4].
        \item \textbf{Media: }3
        \item \textbf{Massimo: }6;
      \end{itemize}


\item \textbf{Linee di codice per linee di commento: } I range stabiliti sono:
      \begin{itemize}
        \item Range di accettabilità = [>0.25];
        \item Range di ottimalità = [>0.30].
        \item \textbf{Media: }0.36;
      \end{itemize}

\item \textbf{Copertura requisiti desiderabili: }I range stabiliti sono:
      \begin{itemize}
        \item Range di accettabilità = [60\%-100\%];
        \item Range di ottimalità = [70\%-100\%].
        \item \textbf{Media: }80\%;
      \end{itemize}
      
		\item \textbf{Profondità di annidamento: }I range stabiliti sono:
      \begin{itemize}
        \item Range di accettabilità = [0-7];
        \item Range di ottimalità = [0-4].
        \item \textbf{Media: }5;
      \end{itemize}      
      
      
      \item \textbf{Copertura requisiti obbligatori: }I range stabiliti sono:
      \begin{itemize}
        \item Range di accettabilità = [90\%-100\%];
        \item Range di ottimalità = [100\%-100\%].
        \item \textbf{Media: }96\%;
      \end{itemize}
      
\item \textbf{Percentuale test superati: }I range stabiliti sono:
      \begin{itemize}
        \item Range di accettabilità = [100\%-100\%];
        \item Range di ottimalità = [100\%-100\%].
        \item \textbf{Media: }100\%;
      \end{itemize}

\end{itemize}