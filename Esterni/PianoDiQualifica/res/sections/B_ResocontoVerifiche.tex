\section{Resoconto delle attività di verifica}
  In questa sezione vengono riportati i risultati delle attività di verifica svolte.
  \subsection{Revisione dei requisiti}
  Prima della consegna relativa alla revisione dei requisiti, sono stati verificati i processi che hanno portato alla stesura dei documenti ed essi stessi. \\
  Per i documenti è stata effettuata una verifica di tipo manuale, la quale consiste nella rilettura dei documenti per l'individuazione di errori di forma ed eventuali inconsistenze;
  e una verifica automatizzata tramite gli strumenti definiti nel documento \emph{Norme di Progetto \VersioneNP{}}, di seguito citati per completezza:
  \begin{itemize}
    \item \glossaryItem{Script} \glossaryItem{Perl} per il calcolo dell’indice Gulpease
    \item \glossaryItem{Script} \glossaryItem{Perl} per la glossarizzazione dei termini
    \item Aspel
  \end{itemize}
  \subsubsection{Dettaglio delle verifiche}
    \paragraph{Analisi}
      \subparagraph{Processi}
      Nella seguente tabella sono riportati i valori per la Schedule Variance(SV) e la Budget Variance(BV) riguardanti i processi del periodo di \textbf{Analisi}.\\
      \begin{itemize}
      \item \textbf{Schedule Variance:} I range stabiliti sono:
      \begin{itemize}
        \item Range di accettabilità = [\(\geq\)-(\emph{141.25})];
        \item Range di ottimalità = [\(\geq\)0].
      \end{itemize}
      
      \item \textbf{Budget Variance:} I range stabiliti sono:
      \begin{itemize}
        \item Range di accettabilità = [\(\geq\)-(\emph{282.5})];
        \item Range di ottimalità = [\(\geq\)0].
      \end{itemize}
      
      
      \end{itemize}
     
       
      \begin{table}[H]
        \centering
        \begin{tabular}{|l|c|c|}
          \hline
          \textbf{Attività} &\textbf{SV}(Euro)  &\textbf{BV}(Euro) \\
          \hline
          Analisi dei requisiti  &25 &-45  \\
          Piano di progetto &35  &0\\
          Piano di qualifica  &0  &-20\\
          Norme di progetto &0  &0 \\
          Studio di fattibilità &0  &10  \\
          Glossario &0  &0  \\
          \hline
          \textbf{Totale} &60  &-65  \\
          \hline
        \end{tabular}
        \caption{Indici SV e BV - Periodo di Analisi}
      \end{table}
      Stando a quanto preventivato dal prospetto economico del \emph{Piano di Progetto \VersionePP{}}, il valore di SV rientra nel range di accettabilità stabilito (SV\(\geq\)-141.25 Euro) che in quello di ottimalità;
      anche il valore di BV rientra nel range di accettabilità stabilito (BV\(\geq\)-282.5 Euro).
      
      \subparagraph{Documenti}
      Nella tabella seguente sono riportati i valori dell'indice Gulpease per ogni documento prodotto nel periodo di Analisi.\\

\begin{itemize}
\item \textbf{Indice Gulpease: }I range stabiliti sono:
      \begin{itemize}
        \item Range di accettabilità = [40-100];
        \item Range di ottimalità = [50-100].
      \end{itemize}
\end{itemize}      
      
      
      \begin{table}[H]
        \centering
        \begin{tabular}{|l|c|c|}
          \hline
          \textbf{Documento} &\textbf{Indice Gulpease} &\textbf{Esito}\\
          \hline
          Analisi dei requisiti &74.3  &Superato \\
          Piano di progetto &51.1  &Superato \\
          Piano di qualifica  &51.3  &Superato \\
          Norme di progetto &50.7  &Superato \\
          Studio di fattibilità &49.5  &Superato \\
          Glossario &51.6  &Superato  \\
          Verbali esterni &51.1  &Superato \\
          \hline
        \end{tabular}
        \caption{Indici Gulpease per i documenti - Periodo di Analisi}
      \end{table}

\subparagraph{Software}
\begin{itemize}
\item \textbf{Complessità ciclomatica: }I range stabiliti sono:
      \begin{itemize}
        \item Range di accettabilità = [0-15];
        \item Range di ottimalità = [0-10].
      \end{itemize}
Per il periodo di \textbf{Analisi} non è stata applicata questa metrica.

\item \textbf{Numero di metodi per file: }I range stabiliti sono:
      \begin{itemize}
        \item Range di accettabilità = [3-10];
        \item Range di ottimalità = [0-7].
      \end{itemize}
Per il periodo di \textbf{Analisi} non è stata applicata questa metrica.

\item \textbf{Variabili non utilizzate e/o non definite: }I range stabiliti sono:
      \begin{itemize}
        \item Range di accettabilità = [0-0];
        \item Range di ottimalità = [0-0].
      \end{itemize}
Per il periodo di \textbf{Analisi} non è stata applicata questa metrica.

\item \textbf{Numero di argomenti per funzione: }I range stabiliti sono:
      \begin{itemize}
        \item Range di accettabilità = [0-6];
        \item Range di ottimalità = [0-4].
      \end{itemize}
Per il periodo di \textbf{Analisi} non è stata applicata questa metrica.

\item \textbf{Linee di codice per linee di commento: } I range stabiliti sono:
      \begin{itemize}
        \item Range di accettabilità = [>0.25];
        \item Range di ottimalità = [>0.30].
      \end{itemize}
Per il periodo di \textbf{Analisi} non è stata applicata questa metrica.
\end{itemize}

\begin{itemize}
\item \textbf{Copertura del codice: }I range stabiliti sono:
      \begin{itemize}
        \item Range di accettabilità = [70\%-100\%];
        \item Range di ottimalità = [80\%-100\%].
      \end{itemize}
Per il periodo di \textbf{Analisi} non è stata applicata questa metrica.
\end{itemize}



  \subsection{Revisione di progettazione}
  Prima della consegna relativa alla revisione di progettazione minima, sono stati verificati i processi che hanno portato alla stesura dei documenti ed essi stessi. \\
  Per i documenti è stata effettuata una verifica di tipo manuale, la quale consiste nella rilettura dei documenti per l'individuazione di errori di forma ed eventuali inconsistenze;
  e una verifica automatizzata tramite gli strumenti definiti nel documento \emph{Norme di Progetto \VersioneNP{}}, di seguito citati per completezza:
  \begin{itemize}
    \item \glossaryItem{Script} \glossaryItem{Perl} per il calcolo dell’indice Gulpease
    \item \glossaryItem{Script} \glossaryItem{Perl} per la glossarizzazione dei termini
    \item Aspel
  \end{itemize}
  \subsubsection{Dettaglio delle verifiche}
    \paragraph{Progettazione}
      \subparagraph{Processi}
      Nella seguente tabella sono riportati i valori per la Schedule Variance(SV) e la Budget Variance(BV) riguardanti i processi del periodo di \textbf{Progettazione}.\\
      \begin{itemize}
      \item \textbf{Schedule Variance:} I range stabiliti sono:
      \begin{itemize}
        \item Range di accettabilità = [\(\geq\)-(\emph{141.25})];
        \item Range di ottimalità = [\(\geq\)0].
      \end{itemize}
      
      \item \textbf{Budget Variance:} I range stabiliti sono:
      \begin{itemize}
        \item Range di accettabilità = [\(\geq\)-(\emph{282.5})];
        \item Range di ottimalità = [\(\geq\)0].
      \end{itemize}
      
      
      \end{itemize}
     
       
      \begin{table}[H]
        \centering
        \begin{tabular}{|l|c|c|}
          \hline
          \textbf{Attività} &\textbf{SV}(Euro)  &\textbf{BV}(Euro) \\
          \hline
          Analisi dei requisiti  &-20 &0  \\
          Piano di progetto &0  &-90*\\
          Piano di qualifica  &+25  &0\\
          Norme di progetto &0  &0 \\
          Studio di fattibilità &0  &0  \\
          Glossario &0  &0  \\
          Specifica Tecnica & 0 & 0\\
          \hline
          \textbf{Totale} &45  &-90  \\
          \hline
        \end{tabular}
        \caption{Indici SV e BV - Periodo di Progettazione}
      \end{table}
      Stando a quanto preventivato dal prospetto economico del \emph{Piano di Progetto \VersionePP{}}, il valore di SV rientra nel range di accettabilità stabilito (SV\(\geq\)-141.25 Euro) che in quello di ottimalità;
      anche il valore di BV rientra nel range di accettabilità stabilito (BV\(\geq\)-282.5 Euro).\\
      * è stato necessario allocare un numero maggiore di ore per il \emph{Piano di Progetto} respetto a quanto preventivato in seguito alla valutazione negativa ottenuta in sede di revisione RR. Tali ore sono state sottratte all'attività di riadattamento del documento \emph{Piano di Qualifica}\VersionePQ{}.
      
      \subparagraph{Documenti}
      Nella tabella seguente sono riportati i valori dell'indice Gulpease per ogni documento prodotto nel periodo di Progettazione.\\

\begin{itemize}
\item \textbf{Indice Gulpease: }I range stabiliti sono:
      \begin{itemize}
        \item Range di accettabilità = [40-100];
        \item Range di ottimalità = [50-100].
      \end{itemize}
\end{itemize}      
      
      
      \begin{table}[H]
        \centering
        \begin{tabular}{|l|c|c|}
          \hline
          \textbf{Documento} &\textbf{Indice Gulpease} &\textbf{Esito}\\
          \hline
          Analisi dei requisiti &70.7  &Superato \\
          Piano di progetto &50.3  &Superato \\
          Piano di qualifica  &62.5  &Superato \\
          Norme di progetto &52.8  &Superato \\
          Studio di fattibilità &49.5  &Superato \\
          Glossario &51.6  &Superato  \\
          Verbale Interno 2017/04/24 &55.8  &Superato \\
          Verbale Interno 2017/04/28 &52.2  &Superato \\
          Verbale Interno 2017/05/02 &53.7  &Superato \\
          Verbale Interno 2017/05/03 &55.3  &Superato \\
          Specifica Tecnica & 0 & 0\\
          \hline
        \end{tabular}
        \caption{Indici Gulpease per i documenti - Periodo di Progettazione}
      \end{table}

\subparagraph{Software}
\begin{itemize}
\item \textbf{Complessità ciclomatica: }I range stabiliti sono:
      \begin{itemize}
        \item Range di accettabilità = [0-15];
        \item Range di ottimalità = [0-10].
      \end{itemize}
Per il periodo di \textbf{Progettazione} non è stata applicata questa metrica.

\item \textbf{Numero di metodi per file: }I range stabiliti sono:
      \begin{itemize}
        \item Range di accettabilità = [3-10];
        \item Range di ottimalità = [0-7].
      \end{itemize}
Per il periodo di \textbf{Progettazione} non è stata applicata questa metrica.

\item \textbf{Variabili non utilizzate e/o non definite: }I range stabiliti sono:
      \begin{itemize}
        \item Range di accettabilità = [0-0];
        \item Range di ottimalità = [0-0].
      \end{itemize}
Per il periodo di \textbf{Progettazione} non è stata applicata questa metrica.

\item \textbf{Numero di argomenti per funzione: }I range stabiliti sono:
      \begin{itemize}
        \item Range di accettabilità = [0-6];
        \item Range di ottimalità = [0-4].
      \end{itemize}
Per il periodo di \textbf{Progettazione} non è stata applicata questa metrica.

\item \textbf{Linee di codice per linee di commento: } I range stabiliti sono:
      \begin{itemize}
        \item Range di accettabilità = [>0.25];
        \item Range di ottimalità = [>0.30].
      \end{itemize}
Per il periodo di \textbf{Progettazione} non è stata applicata questa metrica.
\end{itemize}

\begin{itemize}
\item \textbf{Copertura del codice: }I range stabiliti sono:
      \begin{itemize}
        \item Range di accettabilità = [70\%-100\%];
        \item Range di ottimalità = [80\%-100\%].
      \end{itemize}
Per il periodo di \textbf{Progettazione} non è stata applicata questa metrica.

\end{itemize} 