\section{Resoconto delle attività di verifica}
  In questa sezione vengono riportati i risultati delle attività di verifica svolte.
  \subsection{Revisione dei requisiti}
  Prima della consegna relativa alla revisione dei requisiti, sono stati verificati i processi che hanno portato alla stesura dei documenti ed essi stessi. \\
  Per i documenti è stata effettuata una verifica di tipo manuale, la quale consiste nella rilettura dei documenti per l'individuazione di errori di forma ed eventuali inconsistenze;
  e una verifica automatizzata tramite gli strumenti definiti nel documento \emph{Norme di Progetto \VersioneNP{}}, di seguito citati per completezza:
  \begin{itemize}
    \item \glossaryItem{Script} \glossaryItem{Perl} per il calcolo dell’indice Gulpease
    \item \glossaryItem{Script} \glossaryItem{Perl} per la glossarizzazione dei termini
    \item Aspel
  \end{itemize}
  \subsection{Dettaglio delle verifiche}
    \subsubsection{Analisi}
      \subsubsection{Processi}
      Nella seguente tabella sono riportati i valori per la Schedule Variance(SV) e la Budget Variance(BV) riguardanti i processi del periodo di \textbf{Analisi}.\\
      \begin{table}[H]
        \centering
        \begin{tabular}{|l|c|c|}
          \hline
          \textbf{Attività} &\textbf{SV}(Euro)  &\textbf{BV}(Euro) \\
          \hline
          Analisi dei requisiti  &0 &-45  \\
          Piano di progetto &-20  &0\\
          Piano di qualifica  &-15  &-20\\
          Norme di progetto &15  &0 \\
          Studio di fattibilità &30  &10  \\
          Glossario &0  &0  \\
          \hline
          \textbf{Totale} &10  &-55  \\
          \hline
        \end{tabular}
        \caption{Indici SV e BV - Periodo di Analisi}
      \end{table}
      Stando a quanto preventivato dal prospetto economico del \emph{Piano di Progetto \VersionePP{}}, il valore di SV rientra nel range di accettabilità stabilito (SV\(\geq\)-141.25 Euro) che in quello di ottimalità;
      anche il valore di BV rientra nel range di accettabilità stabilito (BV\(\geq\)-282.5 Euro).
      \subsubsection{Documenti}
      Nella tabella seguente sono riportati i valori dell'indice Gulpease per ogni documento prodotto nel periodo di Analisi.\\
      \begin{table}[H]
        \centering
        \begin{tabular}{|l|c|c|}
          \hline
          \textbf{Documento} &\textbf{Indice Gulpease} &\textbf{Esito}\\
          \hline
          Analisi dei requisiti &0  &Superato \\
          Piano di progetto &0  &Superato \\
          Piano di qualifica  &0  &Superato \\
          Norme di progetto &0  &Superato \\
          Studio di fattibilità &0  &Superato \\
          Glossario &0  &Superato  \\
          Verbali esterni &0  &Superato \\
          \hline
        \end{tabular}
        \caption{Indici Gulpease per i documenti - Periodo di Analisi}
      \end{table}
