\def\arraystretch{1.5}
\rowcolors{2}{D}{P}
\section{Pianificazione dei test}
La strategia di verifica del software che si vuole adottare è l'utilizzo di una serie di test opportunamente  predeterminati che garantiscano almeno un test per requisito. I test devono poter essere ripetibili, ossia, tramite delle specifiche su come riprodurre i test, si vuole che il loro output sia deterministico. Per le tempistiche di esecuzione dei test si fa riferimento al \emph{Piano di Progetto v}\VersionePP.
Nelle tabelle sottostanti lo stato dei test \textbf{N.E.} è da intendersi come non eseguito, in quanto saranno applicati in secondo momento.

\subsection{Livelli di testing}
I test effettuati vengono divisi in livelli differenti  e si concretizzano in un esecuzione bottom-up che avanza sequenzialmente alle attività di codifica e validazione. I test che saranno applicati sono di cinque tipi:
\begin{itemize}
\item \textbf{Test di Accettazione (TA): } verificano che il prodotto soddisfi quanto richiesto dal proponente individuando delle macro azioni eseguite tipicamente dall'utente sul sistema;
\item \textbf{Test di Sistema (TS): } verificano che il comportamento dell'intero sistema funzioni complessivamente bene;
\item \textbf{Test di Integrazione (TI): } verificano le componenti del sistema contenute nella \emph{Specifica Tecnica v}\VersioneST siano funzionanti e in grado di funzionare nel loro insieme;
\item \textbf{Test di Unità (TU): } verificano ogni unità assegnata ad un programmatore. In questo progetto una unità dovrebbe rappresentare una \emph{function} o un \emph{method}. Saranno aggiornati nelle prossime consegne;
\item \textbf{Test di Regressione (TR): } verificano che una modifica non abbia compromesso componenti software precedentemente funzionanti. Saranno aggiornati nelle prossime consegne.
\end{itemize}


\subsection{Test di Sistema}
Vengono descritti i test per verificare il comportamento del sistema rispetto ai requisiti deginiti nell'\emph{Analisi dei Requisiti v}\VersioneAR.
I test sotto riportati sono quelli relativi ai requisiti rilevanti meritevoli di test.
\begin{longtable}{p{2.5cm}!{\VRule[1pt]}p{6.5cm}!{\VRule[1pt]}p{1cm}!{\VRule[1pt]}p{3cm}}
\rowcolor{I}
\color{white} \textbf{Test} & \color{white} \textbf{Descrizione}  & \color{white} \textbf{Stato}  & \color{white} \textbf{Requisito}\\ 
\endfirsthead 
\rowcolor{I} 
\color{white} \textbf{Test} & \color{white} \textbf{Descrizione}  & \color{white} \textbf{Stato}  & \color{white} \textbf{Requisito}\\  
\endhead 
TS1 & Verificare che l'utente possa registrarsi al sistema inserendo username, password ed email  & N.E. & R0F1, R0F1.1\newline
R0F1.2, R0F1.3\newline
R0F1.4\\
TS1.5 & Verificare che l'applicazione visualizzi messaggi di errore in caso di username, password o email non conformi & N.E. & R0F1.5, R0F1.6 \newline
R0F1.7\\
TS2 & Verificare che l'utente possa effettuare l'autenticazione inserendo username e password & N.E. & R0F2.1, R0F2.2\\
TS2.3 & Verificare che l'applicazione visualizzi un messaggio di errore in caso di username o password errati & N.E. & R0F2.3\\
TS4 & Verificare che l'utente possa effettuare il logout & N.E. & R0F4\\
TS5 & Verificare che il sistema visualizzi l'elenco dei progetti realizzati dall'utente & N.E. & R0F5 \\
TS5.1.1 & Verificare che il sistema permetta la creazione di un progetto vuoto & N.E. & R0F5.1 R0F5.1.1\\
TS5.1.2 & Verificare che il sistema permetta l'importazione di un progetto esistente & N.E. & R0F5.1 R0F5.1.2\\
TS5.2 & Verificare che il sistema permetta l'apertura di un progetto salvato & N.E. & R0F5.2\\
TS5.3 & Verificare che il sistema permetta la cancellazione di un progetto salvato & N.E. & R0F5.3\\
TS6 & Verificare che il sistema visualizzi correttamente l'editor dei diagrammi & N.E. & R0F6\\
TS6.1.1.1 & Verificare che il sistema effettui il salvataggio del progetto in uso su richiesta dell'utente & N.E. & R0F6.1.1.1\\
TS6.1.1.2 & Verificare che il sistema effettui la chiusura del progetto in uso su richiesta dell'utente & N.E. & R0F6.1.1.2\\
TS6.1.1.3 & Verificare che il sistema effettui l'esportazione del progetto in uso su richiesta dell'utente & N.E. & R0F6.1.1.3\\
TS6.1.1.4 & Verificare che il sistema effettui la generazione del codice su richiesta dell'utente & N.E. & R0F6.1.1.4\\
TS6.1.1.5 & Verificare che il sistema effettui il salvataggio come Template del progetto in uso su richiesta dell'utente & N.E. & R0F6.1.1.5\\
TS6.1.2.1 & Verificare che il sistema permetta di annullare l'ultima operazione effettuata & N.E. & R1F6.1.2.1\\
TS6.1.2.2 & Verificare che il sistema permetta di ripristinare l'ultima operazione effettuata & N.E. & R1F6.1.2.2\\
TS6.1.2.3 & Verificare che il sistema permetta effettuare l'operazione "taglia" di un oggetto selezionato & N.E. & R1F6.1.2.3\\
TS6.1.2.4 & Verificare che il sistema permetta effettuare l'operazione "copia" di un oggetto selezionato & N.E. & R1F6.1.2.4\\
TS6.1.2.5 & Verificare che il sistema permetta effettuare l'operazione "incolla" di un oggetto copiato & N.E. & R1F6.1.2.5\\
TS6.1.2.6 & Verificare che il sistema permetta effettuare l'operazione di "zoom-in" della schermata & N.E. & R1F6.1.2.6\\
TS6.1.2.7 & Verificare che il sistema permetta effettuare l'operazione di "zoom-out" della schermata & N.E. & R1F6.1.2.7\\
TS6.1.3 & Verificare che il sistema visualizzi l'elenco dei template disponibili & N.E. & R1F6.1.3 \\
TS6.1.3.1 & Verificare che il sistema permetta l'inserimento di un template nel proprio progetto& N.E. & R1F6.1.3.1 \\
TS6.1.3.2 & Verificare che il sistema permetta l'eliminazione di un template template precedentemente salvato & N.E. & R1F6.1.3.2 \\
TS6.1.4.1 & Verificare che il sistema permetta la creazione di un nuovo layer & N.E. & R2F6.1.4.1 \\
TS6.1.4.2 & Verificare che il sistema visualizzi l'elenco dei layer esistenti permettendo la selezione di quelli che l'attore vuole visualizzare & N.E. & R2F6.1.4.2\\
TS6.1.4.3.1 & Verificare che il sistema permetta la modifica del nome del layer da parte dell'utente & N.E. & R2F6.1.4.3.1 \\
TS6.1.4.3.2 & Verificare che il sistema permetta l'eliminazione del layer da parte dell'utente & N.E. & R2F6.1.4.3.2 \\
TS6.2.1.1 & Verificare che il sistema permetta l'inserimento di un nuovo elemento "classe" & N.E. & R0F6.2.1.1\\
TS6.2.1.2 & Verificare che il sistema permetta l'inserimento di un nuovo elemento "pacchetto" & N.E. & R0F6.2.1.2\\
TS6.2.1.3 & Verificare che il sistema permetta l'inserimento di un nuovo elemento "relazione" con le relative classi di partenza e destinazione & N.E. & R0F6.2.1.3 R0F6.2.1.3.1\newline
R0F6.2.1.3.2\\
TS6.2.1.4 & Verificare che il sistema permetta l'inserimento di un nuovo elemento "commento" associandolo ad un altro elemento & N.E. & R0F6.2.1.4\\
TS6.2.2.1 & Verificare che il sistema permetta l'inserimento di un nuovo elemento "operazione" & N.E. & R0F6.2.2.1\\
TS6.2.1.2 & Verificare che il sistema permetta l'inserimento di un nuovo elemento "chiamata a metodo" & N.E. & R0F6.2.1.2\\
TS6.2.1.3 & Verificare che il sistema permetta l'inserimento di un nuovo elemento "variabile" & N.E. & R0F6.2.1.3\\
TS6.2.1.4 & Verificare che il sistema permetta l'inserimento di un nuovo elemento "connettore" & N.E. & R0F6.2.1.4\\
TS6.2.1.5 & Verificare che il sistema permetta l'inserimento di un nuovo elemento "nodo di decisione" & N.E. & R0F6.2.1.5\\
TS6.2.1.5 & Verificare che il sistema permetta l'inserimento di un nuovo elemento "nodo merge" & N.E. & R0F6.2.1.5\\
TS6.2.1.6 & Verificare che il sistema permetta l'inserimento di un nuovo elemento "commento" & N.E. & R0F6.2.1.6\\
TS6.2.1.7 & Verificare che il sistema permetta l'inserimento di un nuovo elemento "output pin" & N.E. & R0F6.2.1.7\\
TS6.3.1.1 & Verificare che il sistema permetta l'eliminazione di un elemento "classe" & N.E. & R0F6.3.1.1\\
TS6.3.1.2 & Verificare che il sistema permetta l'eliminazione di un elemento "relazione" & N.E. & R0F6.3.1.2\\
TS6.3.1.3 & Verificare che il sistema permetta l'eliminazione di un elemento "pacchetto" & N.E. & R0F6.3.1.3\\
TS6.3.1.4 & Verificare che il sistema permetta la visualizzazione di più o meno dettagli di un elemento & N.E. & R0F6.3.1.4\\
TS6.3.1.5 & Verificare che il sistema permetta la modifica di un elemento "classe" & N.E. & R0F6.3.1.5\\
TS6.3.1.5.1 & Verificare che il sistema permetta la modifica di un elemento "nome" in una classe & N.E. & R0F6.3.1.5.1\\
TS6.3.1.5.2 & Verificare che il sistema permetta la modifica di un elemento "attributo" in una classe & N.E. & R0F6.3.1.5.2\\
TS6.3.1.5.2.1 & Verificare che il sistema permetta la modifica di un elemento "visibilità" in un attributo & N.E. & R0F6.3.1.5.2.1\\
TS6.3.1.5.2.2 & Verificare che il sistema permetta la modifica di un elemento "nome" in un attributo & N.E. & R0F6.3.1.5.2.2\\
TS6.3.1.5.2.3 & Verificare che il sistema permetta la modifica di un elemento "tipo" in un attributo & N.E. & R0F6.3.1.5.2.3\\
TS6.3.1.5.2.4 & Verificare che il sistema permetta la modifica di un elemento "valore di default" in un attributo & N.E. & R0F6.3.1.5.2.4\\
TS6.3.1.5.3 & Verificare che il sistema permetta l'aggiunta di un nuovo elemento "attributo" in una classe & N.E. & R0F6.3.1.5.3\\
TS6.3.1.5.4 & Verificare che il sistema permetta la modifica di un elemento "metodo" in una classe & N.E. & R0F6.3.1.5.4\\
TS6.3.1.5.4.1 & Verificare che il sistema permetta la modifica di un elemento "visibilità" in un metodo & N.E. & R0F6.3.1.5.4.1\\
TS6.3.1.5.4.2 & Verificare che il sistema permetta la modifica di un elemento "nome" in un metodo & N.E. & R0F6.3.1.5.4.2\\
TS6.3.1.5.4.3 & Verificare che il sistema permetta la modifica di un elemento "tipo di ritorno" in un metodo & N.E. & R0F6.3.1.5.4.3\\
TS6.3.1.5.4.4 & Verificare che il sistema permetta la modifica di un elemento "lista di parametri" in un metodo & N.E. & R0F6.3.1.5.4.4\\
TS6.3.1.5.5 & Verificare che il sistema permetta l'aggiunta di un nuovo elemento "metodo" in una classe & N.E. & R0F6.3.1.5.5\\
TS6.3.1.5.6 & Verificare che il sistema permetta l'assegnazione "classe astratta" ad una classe & N.E. & R0F6.3.1.5.6\\
TS6.3.1.5.7 & Verificare che il sistema permetta l'assegnazione "interfaccia" ad una classe & N.E. & R0F6.3.1.5.7\\
TS6.3.1.5.8 & Verificare che il sistema permetta l'assegnazione ad un nuovo layer una classe & N.E. & R0F6.3.1.5.8\\
TS6.3.1.6 & Verificare che il sistema permetta la modifica di un elemento "relazione" & N.E. & R0F6.3.1.6\\
TS6.3.1.7 & Verificare che il sistema permetta la modifica di un elemento "pacchetto" & N.E. & R0F6.3.1.7\\
TS6.3.1.7.1 & Verificare che il sistema permetta la modifica di un elemento "nome" in un pacchetto & N.E. & R0F6.3.1.7.1\\
TS6.3.1.7.2 & Verificare che il sistema permetta l'eliminazione di una classe all'interno di un pacchetto & N.E. & R0F6.3.1.7.2\\
TS6.3.1.7.3 & Verificare che il sistema permetta l'estrazione di una classe dall'interno del pacchetto all'esterno & N.E. & R0F6.3.1.7.3\\
TS6.3.1.8 & Verificare che il sistema permetta la modifica del testo di un elemento "commento" & N.E. & R0F6.3.1.8 R0F6.3.1.8.1\\
TS6.3.1.9 & Verificare che il sistema permetta l'eliminazione un elemento "commento" & N.E. & R0F6.3.1.9\\
TS6.3.1.10 & Verificare che il sistema permetta l'eliminazione un elemento "metodo" presente in una classe & N.E. & R0F6.3.1.10\\
TS6.3.1.11 & Verificare che il sistema permetta l'eliminazione un elemento "attributo" presente in una classe & N.E. & R0F6.3.1.11\\
TS6.3.4 & Verificare che il sistema permetta di effettuare l'operazione "drag" degli oggetti selezionati & N.E. & R0F6.3.4\\
TS6.3.5.1 & Verificare che il sistema permetta l'eliminazione un elemento "operazione" & N.E. & R0F6.3.5.1\\
TS6.3.5.2 & Verificare che il sistema permetta l'eliminazione un elemento "chiamata a metodo" & N.E. & R0F6.3.5.2\\
TS6.3.5.3 & Verificare che il sistema permetta l'eliminazione un elemento "variabile" & N.E. & R0F6.3.5.3\\
TS6.3.5.4 & Verificare che il sistema permetta l'eliminazione un elemento "connettore" & N.E. & R0F6.3.5.4\\
TS6.3.5.5 & Verificare che il sistema permetta l'eliminazione un elemento "nodo di decisione" & N.E. & R0F6.3.5.5\\
TS6.3.5.6 & Verificare che il sistema permetta l'eliminazione un elemento "nodo merge" & N.E. & R0F6.3.5.6\\
TS6.3.5.7 & Verificare che il sistema permetta l'eliminazione un elemento "commento" & N.E. & R0F6.3.5.7\\
TS6.3.5.8 & Verificare che il sistema permetta l'eliminazione un elemento "output pin" & N.E. & R0F6.3.5.8\\
TS6.3.5.9 & Verificare che il sistema permetta la modifica di un elemento "commento" nel diagramma dei metodi & N.E. & R0F6.3.5.9\\
TS6.3.5.10 & Verificare che il sistema permetta la modifica di un elemento "operazione" inserendo un'operazione tra variabili & N.E. & R0F6.3.5.10 R0F6.3.5.10.1\\
TS6.3.5.11 & Verificare che il sistema permetta la modifica di un elemento "chiamata a metodo" selezionando un metodo precedentemente dichiarato & N.E. & R0F6.3.5.11 R0F6.3.5.11.1\\
TS6.3.5.12 & Verificare che il sistema permetta la modifica di un elemento "variabile" selezionando una variabile precedentemente dichiarata & N.E. & R0F6.3.5.12 R0F6.3.5.12.1\\
TS6.3.5.13 & Verificare che il sistema permetta la modifica di un elemento "connettore" modificandone la condizione di guardia & N.E. & R0F6.3.5.13 R0F6.3.5.13.1\\
TS6.4.1 & Verificare che il sistema permetta la navigazione tra le chiamate di metodi nel pannello laterale & N.E. & R0F6.4.1\\
TS6.4.2 & Verificare che il sistema visualizzi la breadcrumb, che rappresenta il percorso tra le chiamate a metodo selezionate dall'utente nel pannello laterale & N.E. & R0F6.4.2\\
TS3.1 & Verificare che il sistema permetta all'utente di modificare la password & N.E. & R1F3.1 \\
TS3.2 & Verificare che il sistema permetta all'utente di modificare l'email & N.E. & R1F3.2 \\
TS3.3 & Verificare che il sistema permetta all'utente di eliminare il proprio profilo & N.E. & R1F3.3 \\
TS3.4 & Verificare che il sistema visualizzi un messaggio di errore in caso di password non conforme & N.E. & R1F3.2 \\
TS3.5 & Verificare che il sistema visualizzi un messaggio di errore in caso di email non conforme & N.E. & R1F3.5 \\
TS13 & Verificare che il sistema permetta il recupero della password & N.E. & R1F13 \\
TS14 & Verificare che il sistema invii la nuova password all'indirizzo email dell'utente & N.E. & R1F14 \\
TS7 & Verificare che il codice generato rispetti le metriche riportate nelle \emph{Norme di Progetto v}\VersioneNP e \emph{Piano di Progetto v}\VersionePP & N.E. & R0Q7\\
TS8 & Verificare che la progettazione rispetti le norme riportate nelle \emph{Norme di Progetto v}\VersioneNP e \emph{Piano di Qualifica v}\VersionePQ & N.E. & R0Q8\\
TS9 & Verificare che sia stato rilasciato il manuale d'uso per l'applicazione & N.E. & R0Q9\\
TS10 & Verificare che il codice generato sia nel linguaggio java & N.E. & R0V10\\
TS11 & Verificare che il progetto sia rilasciato con licenza opensource & N.E. & R0V11\\
TS12 & Verificare che le siano utilizzate le tecnologie web HTML, CSS, JavaScript & N.E. & R0V12\\
TS13 & Verificare che i browser in uso siano Google Chrome versione 49.X o Mozilla Firefox versione 45.Y & N.E. & R0V13\\
 
\rowcolor{white}
\caption{Tracciamento Test di Sistema - Requisiti}
\end{longtable}


\subsection{Test di Integrazione}
\begin{longtable}{p{2cm}!{\VRule[1pt]}p{10cm}!{\VRule[1pt]}p{1cm}}
\rowcolor{I}
\color{white} \textbf{Test} & \color{white} \textbf{Descrizione}  & \color{white} \textbf{Stato}\\ 
\endfirsthead 
\rowcolor{I} 
\color{white} \textbf{Test} & \color{white} \textbf{Descrizione}  & \color{white} \textbf{Requisito}\\  
\endhead 
TI1 & Test di integrazione finale per le componenti client e server. & N.E.\\

TI2 & Test di integrazione tra le componenti interne SWEDesigner::Server::Controller, SWEDesigner::Server::Model e SWEDesigner::Client e le librerie esterne Express.Js, Mongoose, MongoDB, BodyParcer, Passport, Moustache, Forge, PassportJWT e Bcrypt & N.E.\\

TI3 & Verificare che il sistema gestisca correttamente le componenti relative al package SWEDesigner::Server::Controller. In particolare che gestisca correttamente l'iterazione con il SWEDesigner::Client e le librerie esterne Express.Js, BodyParcer, Passport, Moustache, Forge, PassportJWT e Bcrypt. & N.E.\\

TI4 & Verificare che il sistema gestisca correttamente le componenti relative al package SWEDesigner::Server::Model. In particolare che gestisca correttamente l'iterazione con il SWEDesigner::Controller::Middleware e le librerie esterne Mongoose e MongoDB  & N.E.\\

TI5 & Verificare che il sistema gestisca correttamente le componenti relative al package SWEDesigner::Server::Controller::Middleware. In particolare che gestisca correttamente l'iterazione con SWEDesigner::Controller::Services e le librerie esterne Express.Js, BodyParcer, Passport, Moustache, Forge, PassportJWT e Bcrypt. & N.E.\\

TI6 & Verificare che il sistema gestisca correttamente le componenti relative al package SWEDesigner::Server::Controller::Services. In particolare che gestisca correttamente l'iterazione con SWEDesigner::Controller::Middleware e la libreria esterna Express.Js. & N.E.\\

TI7 & Verificare che il sistema gestisca correttamente le componenti relative al package SWEDesigner::Server::Controller::Services::UserService. In particolare che gestisca correttamente l'iterazione con la libreria esterna Express. & N.E.\\

TI8 & Test di integrazione tra le componenti interne SWEDesigner::Client::Component, SWEDesigner::Client::Services, SWEDesigner::Server e la libreria esterna Draw2D. & N.E.\\

TI9 & Verificare che il sistema gestisca correttamente le componenti relative al package SWEDesigner::Client::Components. In particolare che gestisca correttamente l'iterazione con SWEDesigner::Client::Services & N.E.\\

TI10 & Verificare che il sistema gestisca correttamente le componenti relative al package SWEDesigner::Client::Components::EditorComponents. & N.E.\\

TI11 & Verificare che il sistema gestisca correttamente le componenti relative al package SWEDesigner::Client::Components::DashComponents. & N.E.\\

TI12 & Verificare che il sistema gestisca correttamente le componenti relative al package SWEDesigner::Client::Service. In particolare che gestisca correttamente l'iterazione con SWEDesigner::Client::Components, SWEDesigner::Server e la libreria esterna Draw2D. & N.E.\\

TI13 & Verificare che il sistema gestisca correttamente le componenti relative al package SWEDesigner::Client::Service::UserServices. & N.E.\\

TI14 & Verificare che il sistema gestisca correttamente le componenti relative al package SWEDesigner::Client::Service::ProjectServices. & N.E.\\

TI15 & Verificare che il sistema gestisca correttamente le componenti relative al package SWEDesigner::Client::Service::GraphLib. & N.E.\\

\rowcolor{white}
\caption{Descrizione test di Integrazione}
\end{longtable}


\begin{longtable}{p{2cm}!{\VRule[1pt]}p{10cm}}
\rowcolor{I}
\color{white} \textbf{Test} & \color{white} \textbf{Componenti aggiunte} \\ 
\endfirsthead 
\rowcolor{I} 
\color{white} \textbf{Test} & \color{white} \textbf{Componenti aggiunte} \\ 
\endhead 
TI1 & SWEDesigner \\

TI2 & SWEDesigner::Server\\

TI3 & SWEDesigner::Server::Controller\\

TI4 & SWEDesigner::Server::Model \\

TI5 & SWEDesigner::Server::Controller::Middleware \\

TI6 & SWEDesigner::Server::Controller::Services \\

TI7 & SWEDesigner::Server::Controller::Services::UserService\\

TI8 & SWEDesigner::Client\\

TI9 & SWEDesigner::Client::Components\\

TI10 & SWEDesigner::Client::Components::EditorComponents \\

TI11 & SWEDesigner::Client::Components::DashComponents\\

TI12 & SWEDesigner::Client::Service\\

TI13 & SWEDesigner::Client::Service::UserServices\\

TI14 & SWEDesigner::Client::Service::ProjectServices\\

TI15 & SWEDesigner::Client::Service::GraphLib\\

\rowcolor{white}
\caption{Tracciamento test di Integrazione - Componenti}
\end{longtable}



\newpage
\subsection{Test di Accettazione}
\begin{longtable}{p{2cm}!{\VRule[1pt]}p{7.5cm}!{\VRule[1pt]}p{1cm}!{\VRule[1pt]}p{2.5cm}}
\rowcolor{I}
\color{white} \textbf{Test} & \color{white} \textbf{Descrizione}  & \color{white} \textbf{Stato}  & \color{white} \textbf{Requisito}\\ 
\endfirsthead 
\rowcolor{I} 
\color{white} \textbf{Test} & \color{white} \textbf{Descrizione}  & \color{white} \textbf{Stato}  & \color{white} \textbf{Requisito}\\  
\endhead 
TA1 & L'utente che non ha mai effettuato la registrazione deve prima registrarsi al sistema. All'utente è richiesto: \begin{itemize}
\item inserire l'username secondo i criteri specificati;
\item inserire la password secondo i criteri specificati;
\item inserire l'email;
\item confermare la registrazione.
\end{itemize} & N.E. & R0F1, R0F1.1 \newline R0F1.2 R0F1.3 \newline R0F1.4\\

TA2 & L'utente non autenticato intende accedere all'applicazione, per farlo deve inserire le proprie credenziali composte da username o email e password. All'utente è richiesto: \begin{itemize}
\item inserire l'username o la mail nel campo apposito;
\item inserire la password;
\item procedere con l'autenticazione.
\end{itemize} & N.E. & R0F2 R0F2.1 \newline R0F2.2\\

TA3 & L'utente autenticato deve poter eseguire il logout dall'applicazione. All'utente è richiesto di:\begin{itemize}
\item essere autenticato;
\item selezionare l'opzione di logout.
\end{itemize} & N.E. & R0F4\\

TA4 & L'utente autenticato ha la possibilità di gestire i propri progetti aggiungendone di nuovi, importandone di esistenti, esportandone o eliminandone. All'utente è richiesto di: \begin{itemize}
\item essere autenticato;
\item accedere alla pagina della gestione dei progetti.
\item selezionare "Nuovo Progetto" se vuole creare uno di nuovo;
\item selezionare "Importa" se vuole importare un progetto già esistente;
\item selezionare "Esporta" se vuole salvare nel proprio sistema il progetto;
\item selezionare "Elimina" se vuole eliminare un progetto.
\end{itemize} & N.E. & R0F5 R0F5.1 R0F5.1.1 R0F5.1.2 R0F5.2 R0F5.3 \\
TA5 & L'utente autenticato ha la possibilità di salvare, chiudere, esportare, generere il codice sorgente, salvare come template il progetto in uso dal menu File. All'utente è richiesto di: \begin{itemize}
\item essere autenticato;
\item avere un progetto aperto;
\item selezionare "Salva" per salvare nel database il progetto;
\item selezionare "Chiudi" per chiudere il progetto;
\item selezionare "Esporta" per esportare il progetto per una futura condivisione;
\item selezionare "Genera" per effettuare la generazione del codice sorgente;
\item selezionare "Salva template" per salvare il progetto come template per un futuro riutilizzo.
\end{itemize}  & N.E. & R0F6.1.1 R0F6.1.1.1 R0F6.1.1.2 R0F6.1.1.3 R0F6.1.1.4 R0F6.1.1.5\\
TA6 & L'utente autenticato ha la possibilità di annullare, ripristinare una operazione, oppure tagliare, copiare, incollare un oggetto  o fare operazioni di zoom, dal menu Edit. All'utente è richiesto di: \begin{itemize}
\item essere autenticato;
\item avere un progetto aperto;
\item selezionare "Annulla" per annullare l'ultima operazione effettuata;
\item selezionare "Ripristina" per ripristinare l'ultima operazione effettuata;
\item selezionare "Taglia" per tagliare un elemento selezionato;
\item selezionare "Copia" per copiare un elemento selezionato;
\item selezionare "Incolla" per incollare un elemento copiato;
\item selezionare "Zoom-in" per effettuare una operazione di zoom in
\item selezionare "Zoom-out" per effettuare una operazione di zoom out
\end{itemize} & N.E. & R1F6.1.2.1 R1F6.1.2.2 R1F6.1.2.3 R1F6.1.2.4 R1F6.1.2.5 R1F6.1.2.6 R1F6.1.2.7\\
TA7 & L'utente autenticato ha la possibilità di visionare i template forniti dal dominio applicativo, e gestire i propri, dal menu Template. All'utente è richiesto di:\begin{itemize}
\item essere autenticato;
\item avere un progetto aperto;
\item accedere alla pagina di gestione dei template;
\item selezionare "Aggiungi" per aggiuungere al proprio progetto un template;
\item selezionare "Elimina" per eliminare un proprio template precedentemente salvato;
\end{itemize} & N.E. & R1F6.1.3 R1F6.1.3.1 R1F6.1.3.2 \\
TA8 & L'utente autenticato ha la possibilità di visionare e gestire i layer disponibili, dal menu Layer. All'utente è richiesto di:\begin{itemize}
\item essere autenticato;
\item avere un progetto aperto;
\item selezionare "Aggiungi" per creare un nuovo layer;
\item selezionare "Visualizza" per visualizzare i layer desiderati;
\item selezionare "Modifica" per modificare il nome del layer;
\item selezionare "Elimina" per eliminare un layer.
\end{itemize} & N.E. & R2F6.1.4 R2F6.1.4.1 R2F6.1.4.2 R2F6.1.4.3.1 R2F6.1.4.3.2\\
TA9 & L'utente autenticato ha la possibilità di aggiungere al proprio progetto gli elementi dei diagrammi delle classi, tramite la barra degli strumenti. All'utente è richiesto di:\begin{itemize}
\item essere autenticato;
\item avere un progetto aperto;
\item selezionare l'icona "Classe" per inserire il disegno di una Classe;
\item selezionare l'icona "Pacchetto" per inserire il disegno di un Pacchetto;
\item selezionare l'icona "Relazione" per inserire il disegno di una Relazione indicandone classe di  partenza e destinazione e il tipo di relazione;
\item selezionare l'icona "Commento" per inserire il disegno di un Commento, indicandone a quali elemento è riferito.
\end{itemize} & N.E. & R0F6.2.1 R0F6.2.1.1 R0F6.2.1.2 R0F6.2.1.3 R0F6.2.1.3.1 R0F6.2.1.3.2 R0F6.2.1.3.3 R0F6.2.1.4 R0F6.2.1.4.1\\
TA10 & L'utente autenticato ha la possibilità di aggiungere al proprio progetto gli elementi dei diagrammi dei metodi, tramite la barra degli strumenti. All'utente è richiesto di: \begin{itemize}
\item essere autenticato;
\item avere un progetto aperto;
\item avere selezionato un metodo da implementare;
\item selezionare l'icona "Operazione" per inserire il disegno di una operazione;
\item selezionare l'icona "Chiamata a Metodo" per inserire il disegno di una chiamata a metodo;
\item selezionare l'icona "Variabile" per inserire il disegno di una variabile;
\item selezionare l'icona "Connettore" per inserire il disegno di un connettore;
\item selezionare l'icona "Nodo di Decisione" per inserire il disegno di un nodo di decisione;
\item selezionare l'icona "Nodo di Merge" per inserire il disegno di un nodo di merge;
\item selezionare l'icona "Commento" per inserire il disegno di un commento;
\item selezionare l'icona "Output pin" per inserire il disegno di un output pin;
\end{itemize} & N.E. & R0F6.2.2 R0F6.2.2.1 R0F6.2.2.2 R0F6.2.2.3 R0F6.2.2.4 R0F6.2.2.5 R0F6.2.2.6 R0F6.2.2.7 R0F6.2.2.8\\
TA11 & L'utente autenticato ha la possibilità di eliminare gli elementi dei diagrammi delle classi aggiunti del disegnatore. All'utente è richiesto di:\begin{itemize}
\item essere autenticato;
\item avere un progetto aperto;
\item avere almeno un elemento del diagramma delle classi inserito nel progetto;
\item eliminare un elemento "Classe";
\item eliminare un elemento "Relazione";
\item eliminare un elemento "Pacchetto";
\item eliminare un elemento "Commento";
\item eliminare un elemento "Metodo" in una classe;
\item eliminare un elemento "Attributo" in una classe;
\end{itemize} & N.E. & R0F6.3.1 R0F6.3.1.1 R0F6.3.1.2 R0F6.3.1.3 R0F6.3.1.9 R0F6.3.1.10 R0F6.3.1.11\\
TA12 & L'utente autenticato ha la possibilità di modificare un elemento Classe. All'utente è richiesto di:
\begin{itemize}
\item essere autenticato;
\item avere un progetto aperto;
\item avere un elemento classe disegnato; 
\item modificare campo "nome";
\item modificare campo "attributo";
\item modificare campo "visibilità attributo";
\item modificare campo "nome attributo";
\item modificare campo "tipo attributo";
\item modificare campo "valore di default attributo";
\item modificare campo "metodo";
\item modificare campo "visibilità metodo";
\item modificare campo "nome metodo";
\item modificare campo "tipo di ritorno metodo";
\item modificare campo "lista parametri metodo";
\item aggiungere campo "metodo";
\item aggiungere campo "attributo";
\item modificare campo "classe astratta";
\item modificare campo "interfaccia";
\item modificare campo "assegnazione layer";
\item modificare campo "relazione";
\end{itemize} & N.E. & R0F6.3.1.5.1 R0F6.3.1.5.2 R0F6.3.1.5.2.1 R0F6.3.1.5.2.2 R0F6.3.1.5.2.3 R0F6.3.1.5.2.4 R0F6.3.1.5.3 R0F6.3.1.5.4 R0F6.3.1.5.4.1 R0F6.3.1.5.4.2 R0F6.3.1.5.4.3 R0F6.3.1.5.4.4 R0F6.3.1.5.5 R0F6.3.1.5.6 R0F6.3.1.5.7 R0F6.3.1.5.8\\
TA13 & L'utente autenticato ha la possibilità di modificare un elemento relazione tra classi. All'utente è richiesto di:\begin{itemize}
\item essere autenticato;
\item avere un progetto aperto;
\item avere un elemento relazione disegnato;
\item modificare tipo relazione;
\end{itemize} & N.E. & R0F6.3.1.6\\
TA14 & L'utente autenticato ha la possibilità di modificare un elemento pacchetto presente nel disegnatore. All'utente è richiesto di:\begin{itemize}
\item essere registrato;
\item avere un progetto aperto;
\item avere un elemento pacchetto disegnato;
\item modificare campo "nome";
\item modificare campo "nome";
\item eliminare una classe all'interno del pacchetto;
\item estrarre una classe dal pacchetto per portarla all'esterno;
\end{itemize} & N.E. & R0F6.3.1.7 R0F6.3.1.7.1 R0F6.3.1.7.2 R0F6.3.1.7.3\\
TA15 & L'utente autenticato ha la possibilità di modificare un elemento commento presente nel disegnatore. All'utente è richiesto di:\begin{itemize}
\item essere autenticato;
\item avere un progetto aperto;
\item avere un elemento commento disegnato;
\item modificare il campo "testo".
\end{itemize} & N.E. & R0F6.3.1.8 R0F6.3.1.8.1 \\
TA16 & L'utente autenticato ha la possibilità di spostare all'interno dell'editor ogni elemento disegnato. All'utente è richiesto di:\begin{itemize}
\item essere registrato;
\item avere un progetto aperto;
\item avere un qualsiasi elemento disegnato.
\end{itemize} & N.E. & R0F6.3.4\\
TA17 & L'utente autenticato ha la possibilità di eliminare gli elementi dei diagrammi dei metodi aggiunti del disegnatore. All'utente è richiesto di:\begin{itemize}
\item essere autenticato;
\item avere un progetto aperto;
\item avere aperto la creazione di un diagramma dei metodi;
\item eliminare un elemento "Operazione";
\item eliminare un elemento "Chiamata a metodo";
\item eliminare un elemento "Variabile";
\item eliminare un elemento "Connettore";
\item eliminare un elemento "Nodo decisione";
\item eliminare un elemento "Nodo merge";
\item eliminare un elemento "Commento";
\item eliminare un elemento "Output pin";
\end{itemize} & N.E. & R0F6.3.5 R0F6.3.5.1 R0F6.3.5.2 R0F6.3.5.3 R0F6.3.5.4 R0F6.3.5.5 R0F6.3.5.6 R0F6.3.5.7 R0F6.3.5.8\\
TA18 & L'utente autenticato ha la possibilità di modificare gli elementi dei diagrammi dei metodi aggiunti del disegnatore. All'utente è richiesto di: \begin{itemize}
\item essere autenticato;
\item avere un progetto aperto;
\item avere aperto la creazione di un diagramma dei metodi;
\item modificare il campo "testo" di un commento;
\item modificare il campo "operazione" tra variabili di un elemento operazione;
\item modificare un campo "metodo" di un elemento chiamata a metodo;
\item modificare un campo "variabile" di un elemento variabile;
\item modificare il campo "guardia" di un connettore;
\end{itemize} & N.E. & R0F6.3.5.9 R0F6.3.5.10 R0F6.3.5.10.1 R0F6.3.5.11 R0F6.3.5.11.1 R0F6.3.5.12 R0F6.3.5.12.1 R0F6.3.5.13 R0F6.3.5.13.1\\
TA18 & L'utente autenticato ha la possibilità di visualizzare il pannello laterale contenente il diagramma dei metodi generale, permettendo la navigazione tra le varie chiamate a metodo. All'utente è richiesto di: \begin{itemize}
\item essere autenticato;
\item avere un progetto aperto;
\item selezionare un elemento "Chiamata a metodo" per navigare tra le varie chiamate a metodo;
\item selezionare dalla "Breadcrumb" una tra le classi percorse durante la navigazione tra le chiamate a metodo.
\end{itemize} & N.E. & R0F6.4 R0F6.4.1 R0F6.4.2\\
TA19 & L'utente autenticato ha la possibilità di gestire i propri dati personali. All'utente è richiesto di:\begin{itemize}
\item essere autenticato;
\item avere aperto la pagina di gestione profilo;
\item modificare la propria password;
\item modificare la propria email;
\item eliminare il proprio profilo.
\end{itemize} & N.E. & R1F3 R1F3.1 R1F3.2 R1F3.3\\
TA20 & L'utente non ancora autenticato, che ha smarrito la password ha la possibilità di richiedere al sistema una nuova password di reset. All'utente è richiesto di:\begin{itemize}
\item essere registrato;
\item accedere alla pagina di recupero password;
\item inserire l'indirizzo email fornito in fase di registrazione;
\item confermare l'invio della nuova password.
\end{itemize} & N.E. & R1F13 \\
\rowcolor{white}
\caption{Tracciamento Test di Accettazione - Requisiti}
\end{longtable}
