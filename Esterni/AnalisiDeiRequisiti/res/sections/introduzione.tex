\section{Introduzione}
\subsection{Scopo del documento}
Lo scopo del documento è mostrare e descrivere le funzionalità che il prodotto dovrà avere. Tali requisiti sono emersi dal \glossaryItem{Capitolato} presentato, da discussioni interne, e da incontri svolti con il \glossaryItem{Proponente}.

\subsection{Glossario}
          Con lo scopo di evitare ambiguità di linguaggio e di massimizzare la comprensione dei documenti, il
          gruppo ha steso un documento interno che è il \emph{Glossario v\VersioneG{}}. In esso saranno definiti, in modo
          chiaro e conciso i termini che possono causare ambiguità o incomprensione del testo.
  \newpage
\section{Riferimenti}
\subsection{Normativi}
\begin{itemize}
\item \textbf{Capitolato d'appalto C6:} SWEDesigner: editor di \glossaryItem{Diagrammi} \glossaryItem{UML} con generazione di \glossaryItem{Codice} \\
\url{http://www.math.unipd.it/~tullio/IS-1/2016/Progetto/C6p.pdf};
\item \textbf{Verbale di incontro} con il \glossaryItem{Proponente} Zucchetti del 23-02-2017;
\item \textbf{Verbale di incontro} con il \glossaryItem{Proponente} Zucchetti del 15-03-2017;
\item \textbf{Norme di Progetto v\VersioneNP{}}.
\end{itemize}
\subsection{Informativi}
\begin{itemize}
\item \textbf{Studio di Fattibilità v\VersioneSF{}};
\item \textbf{IEEE 830-1998}: \url{http://en.wikipedia.org/wiki/Software_requirements_specification}.
\item \textbf{Presentazione \glossaryItem{Capitolato} d'appalto}: \url{http://www.math.unipd.it/~tullio/IS-1/2016/Progetto/C6p.pdf}.
\end{itemize}
\newpage
\section{Descrizione generale}
\subsection{Contesto d'utilizzo}
Con il progetto SWEDesigner si vuole far avvicinare la fase di progettazione delle strutture delle \glossaryItem{Classi}, realizzata utilizzando i \glossaryItem{Diagrammi delle Classi} previsti dell'\glossaryItem{UML}, alla fase di creazione del corpo dei \glossaryItem{Metodi}, realizzata utilizzando i \glossaryItem{Diagrammi delle attività} previsti dall'\glossaryItem{UML}, in modo da  rendere più forte la sincronizzazione tra questi due standard. In particolare il progetto intende creare un ambiente di sviluppo online per la creazione dei \glossaryItem{Diagrammi} sopra citati e la conseguente realizzazione del \glossaryItem{Codice} applicativo descritto dall'\glossaryItem{Utente} in fase di disegno.

\subsection{Funzione di Prodotto}
Le funzioni che saranno disponibili all'interno di SWEDesigner sono:
\begin{itemize}
\item Registrazione dell'\glossaryItem{Utente} per la visione dei propri progetti e dei \glossaryItem{Template} salvati;
\item Creazione di un \glossaryItem{Diagramma delle Classi};
\item Creazione di un \glossaryItem{Diagramma delle attività} per ogni \glossaryItem{Metodo} definito nelle \glossaryItem{Classi};
\item Realizzazione del \glossaryItem{Codice} applicativo descritto tramite i \glossaryItem{Diagrammi};
\item Gestione dei propri progetti;
\item Gestione dei propri \glossaryItem{Template}.
\end{itemize}

\subsection{Descrizione degli utenti}
Gli utenti che possono accedere al progetto SWEDesigner sono delle persone che hanno già esperienze nell'ambito della programmazione, in particolare conoscono le convenzioni dello standard \glossaryItem{UML} riguardanti i \glossaryItem{Diagrammi delle Classi} e delle attività. 
Per accedere alle funzionalità dell'\glossaryItem{Applicazione} l'\glossaryItem{Utente} dovrà effettuare una registrazione e successivamente un \glossaryItem{Login}, in modo da poter creare nuovi progetti e salvarli sul suo \glossaryItem{Account}.
