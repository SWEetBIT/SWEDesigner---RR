\section{Introduzione}
\subsection{Scopo del documento}
Lo scopo del documento è mostrare e descrivere le funzionalità che il prodotto dovrà avere. Tali requisiti sono emersi dal capitolato presentato, da discussioni interne, e da incontri svolti con il Proponente.

\subsection{Glossario}
          Con lo scopo di evitare ambiguità di linguaggio e di massimizzare la comprensione dei documenti, il
          gruppo ha steso un documento interno che è il \emph{Glossario v1.0.0}. In esso saranno definiti, in modo
          chiaro e conciso i termini che possono causare ambiguità o incomprensione del testo.
  \newpage        
\section{Riferimenti}
\subsection{Normativi}
\begin{itemize}
\item \textbf{Capitolato d'appalto C6:} SWEDesigner: editor di diagrammi \glossaryItem{UML} con generazione di codice \\
\url{http://www.math.unipd.it/~tullio/IS-1/2016/Progetto/C6p.pdf};
\item \textbf{Verbale di incontro} con il Proponente Zucchetti del 23-02-2017;
\item \textbf{Verbale di incontro} con il Proponente Zucchetti del 15-03-2017;
\item \textbf{Norme di Progetto v1.0.0}.
\end{itemize}
\subsection{Informativi}
\begin{itemize}
\item \textbf{Studio di Fattibilità v1.0.0};
\item \textbf{IEEE 830-1998}: \url{http://en.wikipedia.org/wiki/Software_requirements_specification}.
\item \textbf{Presentazione capitolato d'appalto}: \url{http://www.math.unipd.it/~tullio/IS-1/2016/Progetto/C6p.pdf}.
\end{itemize}
\newpage
\section{Descrizione generale}
\subsection{Contesto d'utilizzo}
Con il progetto SWEDesigner si vuole far avvicinare la fase di progettazione delle strutture delle classi, realizzata utilizzando i diagrammi delle classi previsti dell'UML, alla fase di creazione del corpo dei metodi, realizzata utilizzando i diagrammi delle attività previsti dall'UML, in modo da  rendere più forte la sincronizzazione tra questi due standard. In particolare il progetto intende creare un ambiente di sviluppo online per la creazione dei diagrammi sopra citati e la conseguente realizzazione del codice applicativo descritto dall'utente in fase di disegno.
 
\subsection{Funzione di Prodotto}
Le funzioni che saranno disponibili all'interno di SWEDesigner sono:
\begin{itemize}
\item Registrazione dell'utente per la visione dei propri progetti e dei template salvati; 
\item Creazione di un diagramma delle classi; 
\item Creazione di un diagramma delle attività per ogni metodo definito nelle classi;
\item Realizzazione del codice applicativo descritto tramite i diagrammi;
\item Gestione dei propri progetti;
\item Gestione dei propri template.
\end{itemize}

\subsection{Descrizione degli utenti}
Gli utenti che possono accedere al progetto SWEDesigner sono delle persone che hanno già esperienze nell'ambito della programmazione, in particolare conoscono le convenzioni dello standard UML riguardanti i diagrammi delle classi e delle attività. 
Per accedere alle funzionalità dell'applicazione l'utente dovrà effettuare una registrazione e successivamente un login, in modo da poter creare nuovi progetti e salvarli sul suo account.