%%%%%%%%%%%%%%
%  COSTANTI  %
%%%%%%%%%%%%%%

% In questa prima parte vanno definite le 'costanti' utilizzate da due o pi� documenti.

% Meglio non mettere gli \emph dentro le costanti, in certi casi creano problemi
\newcommand{\GroupName}{SWEet BIT}
\newcommand{\GroupEmail}{sweet.bit.group@gmail.com}
\newcommand{\ProjectName}{SWEDesigner}
\newcommand{\ProjectVersion}{v1.0.0}

\newcommand{\Proponente}{Zucchetti}
\newcommand{\Committente}{Prof. Tullio Vardanega \\ Prof. Riccardo Cardin}
\newcommand{\Responsabile}{Da Decidere}

% La versione dei documenti deve essere definita qui in global, perch� serve anche agli altri documenti
\newcommand{\VersioneG}{1.0.0}
\newcommand{\VersionePQ}{1.0.0}
\newcommand{\VersioneNP}{1.0.0}
\newcommand{\VersionePP}{1.0.0}
\newcommand{\VersioneAR}{1.0.0}
\newcommand{\VersioneSF}{1.0.0}
\newcommand{\VersioneST}{1.0.0}
\newcommand{\VersioneMA}{1.0.0}
\newcommand{\VersioneMS}{1.0.0}
\newcommand{\VersioneMU}{1.0.0}
\newcommand{\VersioneDP}{1.0.0}
% Il verbale non ha versionamento.

% Quando serve riferirsi a ``Nome del Documento + ultima versione x.y.z'' usiamo queste costanti:
\newcommand{\Glossario}{\emph{Glossario v\VersioneG{}}}
\newcommand{\PianoDiQualifica}{\emph{Piano di Qualifica v\VersionePQ{}}}
\newcommand{\NormeDiProgetto}{\emph{Norme di Progetto v\VersioneNP{}}}
\newcommand{\PianoDiProgetto}{\emph{Piano di Progetto v\VersionePP{}}}
\newcommand{\StudioDiFattibilita}{\emph{Studio di Fattibilit� v\VersioneSF{}}}
\newcommand{\AnalisiDeiRequisiti}{\emph{Analisi dei Requisiti v\VersioneAR{}}}
\newcommand{\SpecificaTecnica}{\emph{Specifica Tecnica v\VersioneST{}}}
\newcommand{\ManualeAdmin}{\emph{Manuale Admin v\VersioneMA{}}}
\newcommand{\ManualeSviluppatore}{\emph{Manuale Sviluppatore v\VersioneMS{}}}
\newcommand{\ManualeUtente}{\emph{Manuale Utente v\VersioneMU{}}}
\newcommand{\DefinizioneDiProdotto}{\emph{Definizione di Prodotto v\VersioneDP{}}}

\newcommand{\ScopoDelProdotto}{
	Lo scopo del progetto � la realizzazione del progetto \glossaryItem{SWEDesigner}.}

%%%%%%%%%%%%%%
%  FUNZIONI  %
%%%%%%%%%%%%%%

% In questa seconda parte vanno definite le 'funzioni' utilizzate da due o pi� documenti.

% Serve a dare la giusta formattazione alle parole presenti nel glossario
% il nome del comando \glossary � gi� usato da LaTeX
\newcommand{\glossaryItem}[1]{\textit{#1\ped{\ped{G}}}}

% Serve a dare la giusta formattazione per indicare il tipo di verbale in cui e' stata presa una decisione
% Uso: \verbalRI{data}{punto}
% RI = Riunione Interna
\newcommand{\verbalRI}[2]{\textit{RI-#1-#2}}
% RE = Riunione Esterna
\newcommand{\verbalRE}[2]{\textit{RE-#1-#2}}

% Serve a dare la giusta formattazione al codice inline
\newcommand{\code}[1]{\flextt{\texttt{#1}}}

% Serve a dare la giusta formattazione a tutte le path presenti nei documenti
\newcommand{\file}[1]{\flextt{\texttt{#1}}}

% Permette di andare a capo all'interno di una cella in una tabella
\newcommand{\multiLineCell}[2][c]{\begin{tabular}[#1]{@{}l@{}}#2\end{tabular}}

% Genera automaticamente la pagina di copertina
\newcommand{\makeFrontPage}{
  % Declare new goemetry for the title page only.
  \newgeometry{top=3.5cm}
  
  \begin{titlepage}
  \begin{center}

  \begin{center}
  \includegraphics[width=10cm]{../../common/logo.jpg}
  \end{center}
  
  \vspace{1cm}

  \begin{Huge}
  \textbf{\DocTitle{}}
  \end{Huge}
  
  \textbf{\emph{Gruppo \GroupName{} \, \texttwelveudash{} \, Progetto \ProjectName{}}}
  
  \vspace{11pt}

  \bgroup
  \def\arraystretch{1.3}
  \begin{tabular}{ r|l }
    \multicolumn{2}{c}{\textbf{Informazioni sul documento}} \\
    \hline
		% differenzia a seconda che \DocVersion{} stampi testo o no
		\setbox0=\hbox{\DocVersion{}\unskip}\ifdim\wd0=0pt
			% nulla (non ho trovato come togliere l'a capo)
			\\
		\else
			\textbf{Versione} & \DocVersion{} \\
		\fi
    \textbf{Redazione} & \multiLineCell[t]{\DocRedazione{}} \\
    \textbf{Verifica} & \multiLineCell[t]{\DocVerifica{}} \\
    \textbf{Approvazione} & \multiLineCell[t]{\DocApprovazione{}} \\
    \textbf{Uso} & \DocUso{} \\
    \textbf{Distribuzione} & \multiLineCell[t]{\DocDistribuzione{}} \\
  \end{tabular}
  \egroup

  \vspace{22pt}

  \textbf{Descrizione} \\
  \DocDescription{}

  \end{center}
  \end{titlepage}
  
  % Ends the declared geometry for the titlepage
  \restoregeometry
}
\input{../../common/layout.tex}
%%%%%%%%%%%%%%
%  COSTANTI  %
%%%%%%%%%%%%%%

% In questa prima parte vanno definite le 'costanti' utilizzate soltanto da questo documento.
% Devono iniziare con una lettera maiuscola per distinguersi dalle funzioni.

\newcommand{\DocTitle}{Definizione di Prodotto}
\newcommand{\DocVersion}{\VersioneDP{}}

\newcommand{\DocRedazione}{}
\newcommand{\DocVerifica}{}
\newcommand{\DocApprovazione}{}

\newcommand{\DocUso}{Esterno}
\newcommand{\DocDistribuzione}{
	\Committente{} \\
	\Proponente
}

% La descrizione del documento
\newcommand{\DocDescription}{
	Questo documento descrive la struttura e le relazioni tra le parti del prodotto SWEDesigner del gruppo SWEet BIT.
	}

%%%%%%%%%%%%%%
%  FUNZIONI  %
%%%%%%%%%%%%%%

% In questa seconda parte vanno definite le 'funzioni' utilizzate soltanto da questo documento.

% Pacchetti per gli accenti
\usepackage[utf8]{inputenc}

\usepackage{tocloft}% http://ctan.org/pkg/tocloft

\setlength{\cftsubsecnumwidth}{3em}

%\usepackage{glossaries}
\usepackage[nonumberlist]{glossaries}
\usepackage{titletoc}



\titlepage{}
\newglossarystyle{myaltlistgroup}{%
  \setglossarystyle{altlistgroup}%
  \renewcommand*{\glsgroupheading}[1]{%
   \newpage
    \section{##1}%
    \vspace*{-\baselineskip}%
	
    \item\makebox[\linewidth]{\hspace*{4cm}\hrulefill\hspace*{2cm}}%
  	
  }%
}

\newcommand\invisiblesection[1]{%
  \refstepcounter{subsection}%
  \addcontentsline{toc}{subsection}{\protect\numberline{\thesection}#1}%
  \sectionmark{#1}
}

\makeglossaries
\loadglsentries{res/sections/glossario}
%\glstoctrue

\title{\textbf{Glossario}}
\author{SWEet BIT}

\date{1 marzo 2017}

%COMPILARE IL FILE DA TERMINALE CON:
%makeindex -s main.ist -o main.gls main.glo

\begin{document}

%\maketitle

\makeFrontPage

\section*{Registro delle modifiche}

\begin{center}

    \begin{longtable}{ >{\centering}p{1.8cm} | >{\centering}p{2.2cm} | >{\centering}p{3cm} | >{\centering}p{6cm} }
      \textbf{Versione} & \textbf{Data} & \textbf{Persone coinvolte} & \textbf{Descrizione} \tabularnewline \hline
      	%periodo di verifica e validazione
      	
      	1.4.0 & 2017/04/26 & Pilò Salvatore & Approvazione documento\tabularnewline \hline %
      	
      	1.3.0 & 2017/04/24 & Santimaria Davide & Verifica documento\tabularnewline \hline %
      	
      	1.2.1 & 2017/04/23 & Massignan Fabio & Modifica sezione critica capitoli: Suddivisione del Lavoro, Prospetto economico e Consuntivo di periodo\tabularnewline \hline %

		1.2.0 & 2017/03/30 & Bodian Malick & Approvazione documento\tabularnewline \hline %  

		1.1.0 & 2017/03/27 & Massignan Fabio & Verifica documento\tabularnewline \hline %       	
      	
		1.0.3 & 2017/03/21 & Bodian Malick & Stesura capitoli: Analisi dei rischi e Consuntivo finale \tabularnewline \hline %       	
      	
		1.0.2 & 2017/03/14 & Pilò Salvatore & Stesura capitoli: Suddivisione del lavoro e Progetto economico \tabularnewline \hline %       	
      	
		1.0.1 & 2017/03/05 & Bodian Malick & Stesura capitoli: Organigramma, Introduzione e Pianificazione \tabularnewline \hline %      	
      	
		1.0.0 & 2017/03/02 & Pilò Salvatore & Creazione scheletro del documento \tabularnewline \hline %
    \end{longtable}
  
\end{center}

		\tableofcontents
	\label{LastFrontPage}
	\titleformat{\section}[block]{\bfseries\filcenter\fontsize{25pt}{25pt}\selectfont}{}{1em}{}	
			

%\tableofcontents
%\listoffigures
%\listoftables


%how to: \input{res/chapters/argumentOfChapter}
\section{Introduzione}
  \subsection{Scopo del documento}
          In questo documento sono definite le norme che i membri del gruppo SWEetBIT adotteranno durante lo sviluppo del progetto SWEDesigner.
          Tutti i membri sono tenuti a leggere e seguire le norme per migliorare l’uniformità del materiale prodotto, migliorare l’efficienza
          e ridurre il numero di errori oltre che, ovviamente, evitarli il più possibile.
          In particolare verranno definite norme riguardanti:
            \begin{\begin{itemize}
              \item Interazioni fra membri del gruppo.
              \item Stesura e convenzioni dei documenti.
              \item Modalità di lavoro durante le fasi di sviluppo del progetto.
              \item Ambiente di lavoro.
            \end{itemize}
  \subsection{Scopo del Prodotto}
          Lo scopo del progetto è la realizzazone di una \glossaryItem{Web App} che fornisca all'utente un \glossaryItem{UML} \glossaryItem{Designer} con il quale riuscire a disegnare correttamente diagrammi delle classi
          e descrivere il comportamento dei metodi interni alle stsse attraverso l'utilizzo di -da decidere il tipo di schema-.
          La \glossaryItem{Web App} permetterà all'utente di generare codice Java o Javascript dal diagramma disegnato ed eventualmente andare a ritoccarne il risultato al fine di ottenere un codice
          eseguibile, funzonante e funzionale.
  \subsection{Glossario}
          Con lo scopo di evitare ambiguità di linguaggio e di massimizzare la comprensione dei documenti, il
          gruppo ha steso un documento interno che è il \emph{Glossario v1.0.0}. In esso saranno definiti, in modo
          chiaro e conciso i termini che possono causare ambiguità o incomprensione del testo.
  \subsection{Riferimenti}
    \subsubsection{Informativi}
      \begin{itemize}
        \item \textbf{Specifiche UTF-8:}\\
        \url{http://unicode.org/faq/utf_bom.html}
        \item \textbf{ISO 8601:2004:} \\
        \url{https://www.iso.org/standard/40874.html}\\
        \item \textbf{Licenza MIT:}
        \url{https://opensource.org/licenses/MIT}\\
        \item \textbf{GitHUB:}\\
        \url{https://github.com/}
        \item \textbf{UML:} \\
        \url{http://www.uml.org/}
        \item \textbf{Atom:}\\
        \url{https://atom.io/}
        \item \textbf{TexLive:}\\
        \url{https://www.tug.org/texlive/}
        \item \LaTeX\\
        \url{https://www.latex-project.org/}
        \item \textbf{Telegram:}\\
        \url{https://telegram.org/}
        \item INSERIRE ALTRI
        \item \textbf{Piano di progetto:} \emph{Piano di progetto v1.0.0}
        \item \textbf{Piano di qualifica:} \emph{Piano di qualifica v1.0.0}
      \end{itemize}
    \subsubsection{Normativi}
      \begin{itemize}
        \item \textbf{Capitolato di appalto SWEDesigner (C6):}\\
        \url{http://www.math.unipd.it/~tullio/IS-1/2016/Progetto/C6.pdf}
      \end{itemize}

\section{Scelta del Capitolato C6}
  \subsection{Descrizione del capitolato}
    Il capitolato C6, proposto dall'azienda \emph{Zucchetti}, propone lo sviluppo di una \glossaryItem{Web App} costituita da un \glossaryItem{designer} di \glossaryItem{UML} che utilizzi sia gli schemi tipici
    del linguaggio, come ad esempio il diagramma delle classi, sia alcuni ibridi ideati appositamente per lo scopo.
    Dal diagramma \glossaryItem{UML} prodotto sarà possibile generare automaticamente del codice \emph{Java} e/o \emph{Javascript} chee può e deve essere modificabil dall'utilizzatore.
    In particolare è rchiesta una certa coerenza fra il codice scritto e i diagrammi presenti all'interno del \glossaryItem{designer}.
    Le richieste principali del capitolato sono le seguenti:
    \begin{itemize}
      \item La trasformazione degli \glossaryItem{UML} in linguaggio \emph{Java} e/o \emph{Javascript};
      \item L'utilizzo di strutture tipiche del linguaggio \glossaryItem{UML};
      \item L'utilizzo di \textbf{TOMCAT} o \emph{Node.js} per quanto riguarda il lato \glossaryItem{Server};
      \item Il corretto funzionamento del prodotto finale su browser supportanti \emph{Html 5.0} e \emph{CSS 3};
     \end{itemize}
   \subsection{Dominio applicativo}
    Il capitolato pone come obbiettivo quello di creare uo strumento che possa automatizzare, nei limtii del possibile, il processo di generazione di codice.
    Negli ultimi anni si sente sempre di più l'esigenza di sviluppare \emph{software} in tempi esigui e spendendo meno risorse possibili nella mano d'opera.
    Oltre a tutto questo si sente la necessità di avere davanti del codice quanto più pulito possibile da errori umani, pertanto l'esigenza di un tool in grado di
    automatizzare questo processo macchinoso, rendendo meno influente l'azione umana (e relativi errori) sul prodotto finale. \\
    Nella pratica un tale sistema sarebbe impossibile da realizzare per via della mole di varibili in gioco, pertanto si deve provare a ridimensionare il problema ponendolo
    all'internodi un dominio specifico.
    In questo caso il dominio indicato dal proponente è quello dei giochi da tavolo -inserire altri domini qualora volessimo- così da ridimensionare notevolmente il problema:
    si tratta di un dominio molto specifico in cui è più "semplice" riuscire a generare del codice adatto alla situazione molto più particolare.
    Ad esempio è noto a tutti che un gioco da tavolo mette sempre a disposizione una plancia di gioco, la quale, nonostante ne esistano varie versioni, ha sempre degli
    elementi fissi che possono essere utilizzati a nostro vantaggio.
  \subsection{Dominio tecnologico}
    Vista la natura di \glossaryItem{Web App} del capitolato e sopratutto alla luce dei requisiti richiesti dal proponente si è reso necessario uno studio approfondito in diversi campi:
      \begin{itemize}
        \item \textbf{\glossaryItem{Server} TOMCAT:} conoscenza delle strumentazioni offerte da questa particolare tecnologia Apache con conseguenti pro e contro del caso.
        \item \textbf{Node.js:} conoscenza di questa piattaforma: in particolare si rendono necessarie le conoscenze della sua offerta e della possibili applicazioni
        all'interno del progetto.
        \item \textbf{JVM:} conoscenze di base del funzionamento della macchina virtuale di Java.
        \item \textbf{Java/Javascript:} conoscenza abbastanza approfondita dei due linguagg necessaria per la generazione del codice automatico a partire dagli \glossaryItem{UML}.
        \item \textbf{Diagrammi \glossaryItem{UML}:} conoscenza dei principali schemi utilizzati all'interno dello standard \glossaryItem{UML}.
        \item \textbf{Meteor:} conoscenza basilare della piattaorma per agevolare la scrittura del lato \glossaryItem{Client} della \glossaryItem{Web App}.
      \end{itemize}
  \subsection{Criticità potenziali e costi}
    Tutte le tecnologie richieste per la realizzazione del progetto sono gratuite quindi non è richiesto
    un'impegno monetario per utilizzarle, tuttavia essendo in gran parte nuove per i membri del gruppo
    l'acquisizione delle competenze necessarie richiederà un investimento non banale in termini di
    tempo.
    \\ \\
    In maniera più specifica le tecnologie che possono ssere fonti di forti criticità sono le seguenti:
      \begin{itemize}
        \item \textbf{Diagrammi \glossaryItem{UML}:} nessun componente del gruppo ha mai avuto a che fare con la progettazione di diagrammi \glossaryItem{UML}, salvo che durante i corsi didattici ancora
        in corso. Lo studio approfondito di tale strumenti è fondamentale per la realizzazione del progetto.
        \item \textbf{Java/Javascript:} il gruppo possiede una conoscenza piuttosto generale dei linguaggi in questione. Si rende quindi necessario un approfondimento di tali
        conoscenze.
        \item \textbf{Node.js/TOMCAT:} nessun componente del gruppo ha avuto a che fare con tali tecnologie per lo sviluppo del lato \glossaryItem{Server}, si rende pertanto necessaria una conoscenza
        generale per la scelta della tecnologia da adoperare da approfondire maggiormente in seguito.
        \item \textbf{Meteor:} nonostante i compoenti del gruppo abbiano una conoscenza piuttosto basilare e generica della piattaforma è necessario uno studio più approfondito della stessa.
      \end{itemize}
  \subsection{Analisi del mercato e benefici}
    Attualment sul mercato non sono disponibili strumenti di questo genere di si offrono di generare del codice in maniera automatizzata. I pochi esempi che possiamo ritrovare
    prevedono un sistema poco funzionale di \glossaryItem{Drag-and-drop} che genera del codice non sempre ottimale.
    Oltre a questo si sente molto l'esigenza di un mabiente che possa diminuire drasticamente i tempi di sviluppo software all'interno di un'azienda permettendo quindi al progetto
    di rispondere ad una richiesta piuttosto importante all'interno del mercato. \\
    Il rilascio su licenza MIT permetterà infine una potenziale rapida crescita del progetto grazie al possibile apporto della comunità.
  \subsection{Considerazioni e valutazioni finali}
    Conseguentemente alle considerazioni esposte nelle sezioni precedenti il gruppo ha definito un
    insieme di aspetti positivi e negativi del capitolato:
    \subsubsection{Aspetti positivi}
      \begin{itemize}
        \item \textbf{Interesse:} i componenti del gruppo hanno manifestato un interesse elevato nei confronti del dominio applicativo e delle tecnologie necessarie
        allo sviluppo, soprattutto per via dell'enorme potenzialità creativa dello stesso;
        \item \textbf{Novità:} il prodotto rappresenta un'interessante novità per il mercato che ha stimolato particolarmente i componenti del gruppo;
        \item \textbf{Esperienza:} lo sviluppo del prodotto permetterà ai membri del gruppo di acquisire competenze utili nel proseguimento della carriera
        grazie a tecnologie come \emph{Node.Js}.
        \item \textbf{Licenza:} il rilascio del prodotto con licenza MIT fornisce interessanti
         prospettive future di utilizzo e sviluppo;
     \end{itemize}
   \subsubsection{Aspetti negativi}
    Gli aspetti negativi del progetto sono da legarsi principlamente alle tecnologie da utilizzare che sono poco familiari agli elementi del gruppo.
    La criticità maggiore è da riscontrarsi invece sulla fattibilità del progetto stesso che cerca una soluzione ad un problema piuttosto complesso che richiede grandi
    capacità di pensiero e di sviluppo.

\section{Altri capitolati}
  \subsection{Capitolato C1 - An API Market Platform (APIM)}
    \subsubsection{Valutazione Generale}
    Il capitolato propone la creazione di una \glossaryItem{Web App}, che consiste in un \glossaryItem{API} market in grado di registrare, consultare ed effettuare operazioni di compravendita di \glossaryItem{microservizi}.
    Il gruppo ha ritenuto il capitolato C1 fattibile e stimolante per la possibilità di interagire con una tecnologia di recente espansione, ovvero l'architettura a \glossaryItem{microservizi}. Tuttavia non ha suscitato molto interesse la limitazione delle tecnologie necessarie, si è deciso di orientare la propria scelta su altri capitolati.
    \subsubsection{Potenziali Criticità}
     \begin{itemize}
      \item Difficoltà nel garantire la correttezza delle \glossaryItem{API} registrate dagli utenti.
     \end{itemize}
  \subsection{Capitolato C2 - Accoglienza tramite Assistente Virtuale (AtAVi)}
    \subsubsection{Valutazione Generale}
    Il capitolato propone la creazione di una \glossaryItem{Web App} che permetta, ad un ospite dell'ufficio dei proponenti, di interrogare un assistente virtuale per annunciare la propria presenza, in modo che l'applicativo lo accolga e comunichi l'arrivo a chi di dovere.
    Il gruppo ha ritenuto il capitolato C2 molto affascinante ed eccitante, sopratutto perché si affronta la tematica dell'\glossaryItem{Intelligenza Artificiale}, una tematica fortemente attuale e che è destinata a diventare sempre più fondamentale in innumerevoli settori. Nonostante ciò, il gruppo ha optato per altri capitolati poichè l'inesperienza dei componenti su un argomento così complesso avrebbe potuto aumentare notevolmente la difficoltà del capitolato fuoriuscendo obbiettivi iniziali di quest'ultimo.
     \subsubsection{Potenziali Criticità}
      \begin{itemize}
       \item Difficoltà nel creare un programma di \glossaryItem{IA} efficiente;
       \item Scarse conoscenze riguardo gli \glossaryItem{SDK} per assistenti virtuali, quindi difficoltà nell'effettuare paragoni ed analisi;
       \item Conoscenze basilari solamente da parte di alcuni membri del gruppo di NodeJS, con conseguente incremento del tempo per l'apprendimento dello stesso.
      \end{itemize}
  \subsection{Capitolato C3 - A Designer and Geo-localizer Web App for Organizational Plants (DeGeOP)}
    \subsubsection{Valutazione Generale}
    Il capitolato richiede la creazione di un'interfaccia \glossaryItem{Web App}, erogabile anche su dispositivi mobili, per inserire i processi produttivi delle aziende (macchinari, magazzini, fornitori, distributori) su mappa geografica e per disegnare i vari scenari di danno che possono interessare l'azienda. 
    Questo capitolato non è stato ritenuto interessante dal gruppo sia dal punto di vista del dominio applicativo, sia delle tecnologie da utilizzare. Di conseguenza si è preferito scegliere altro.
    \subsubsection{Potenziali Criticità}
    \begin{itemize}
     \item Difficoltà di definizione di tutti gli scenari di danno possibili.
    \end{itemize}
\subsection{Capitolato C4 - Applicazione di lettura per dislessici (eBread)}
    \subsubsection{Valutazione Generale}
    L'obiettivo di questo capitolato è quello di realizzare un'applicazione in ambiente \glossaryItem{Android} che agevoli la lettura alle persone affette da dislessia, grazie all'aiuto di tecnologie appropriate, fra cui la sintesi vocale.
Il gruppo ha deciso di non approfondire questo capitolato perché, considerando l'alto numero di applicazioni appartenenti allo stesso dominio, con il tempo a disposizione sarebbe stato complicato ottenere innovazioni degne di nota. Si è quindi preferito puntare su capitolati più originali.
\subsection{Potenziali Criticità}
 \begin{itemize}
 \item Difficoltà di implementazione di un motore di sintesi vocale che sia sincronizzato con il testo;
 \item Difficoltà di implementazione di supporto multilingua;
 \item Scarsa conoscenza da parte del gruppo delle tecnologie da utilizzare.
 \end{itemize}
    
\subsection{Capitolato C5 - An interactive bubble provider (Monolith)}
    \subsubsection{Valutazione Generale}
    Il capitolato prevede la creazione di un \glossaryItem{framework} che permetta l'istanziazione delle cosiddette bolle per la piattaforma di \glossaryItem{Web Chat} denominata Rocket.chat, dove per bolle si intendono delle funzionalità che possono venire aggiunte alla piattaforma senza nessuna nuova installazione.
    Il gruppo ha reputato questo capitolato poco interessante dato che ormai esistono numerose piattaforme di Web Chat affermate e note a milioni di utenti, quindi si sarebbe difficilmente arrivati ad una vera innovazione.
    \subsubsection{Potenziali Criticità}
    \begin{itemize}
     \item Difficoltà di contatto con i proponenti, vista la locazione della loro sede.    
    \end{itemize}
 
%\appendix



%\printglossary[style=myaltlistgroup]

\end{document}