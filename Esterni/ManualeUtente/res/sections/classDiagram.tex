\section{Diagramma delle classi}
L'applicazione SWEDesigner permette di generare un codice Java funzionante a partire da Diagrammi delle classi e diagrammi delle attività.\\
In questa sezione verranno presentati le funzionalità principali offerte dall'applicazione per quanto riguarda la realizzazione di diagrammi delle classi.

\subsection{Inserimento elementi classi}
Per inserire un elemento per la realizzazione del diagramma delle classi è sufficiente selezionare dalla barra laterale di sinistra, uno degli elementi desiderati, e posizionarlo nell'area di disegno effettuando un clic sulla posizione voluta.

\subsection{Associazioni tra classi}
Per realizzare una associazione tra due classi, una volta selezionato il tipo di relazione dal menu laterale di sinistra, bisogna cliccare prima sulla classe di partenza e successivamente sulla classe di destinazione. In seguito verrà visualizzata la freccia rappresentante l'associazione tra le classi.

\subsection{Eliminazione elemento}
Per eliminare un elemento disegnato, ma non desiderato, bisogna selezionare l'elemento, e dal menu superiore, all'interno della voce "Modifica" selezionare "Elimina".

\subsection{Modificare una classe}
Per modificare una classe, una volta cliccato sulla classe desiderata, a destra compare un menu di modifica.

\subsubsection{Gestione Attributi}
É possibile visualizzare una lista di tutti gli attributi attualmente inseriti, potendoli modificare, o aggiungerne di nuovo, compilando i form con i relativi dati.

\subsubsection{Gestione Metodi}
É possibile visualizzare una lista di tutti i metodi attualmente inseriti, potendoli modificare, o aggiungerne di nuovo, compilando i form con i relativi dati. Selezionando il l'icona di modifica, è possibile entrare nella modalità di disegno dei diagrammi delle attività.