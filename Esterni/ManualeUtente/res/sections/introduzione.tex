\section{Introduzione}
  \subsection{Scopo del documento}
          Questo documento rappresenta il manuale utente per l'applicazione SWEDesigner.
          All'interno delle sue sezioni verranno descritte dettagliatamente le caratteristiche dell'applicazione utilizzabili dall'utente e ne verrà spiegato il funzionamento
          al fine da garantire una semplice fruizione da parte dell'utente finale.
  \subsection{Scopo del Prodotto}
          Lo scopo del progetto è la realizzazione di una Web App che fornisca all'Utente un UML Designer con il quale riuscire a disegnare correttamente Diagrammi delle classi
          e descrivere il comportamento dei Metodi interni alle stesse attraverso l'utilizzo del Diagramma delle attività.
          La Web App permetterà all'Utente di generare Codice Java dal Diagramma disegnato ed eventualmente andare a ritoccarne il risultato al fine di ottenere un Codice
          eseguibile, funzionante e funzionale.
  \subsection{Glossario}
          Con lo scopo di evitare ambiguità di linguaggio e di massimizzare la comprensione dei documenti, il
          gruppo ha steso un documento interno che è il \emph{Glossario \VersioneG{}}. In esso saranno definiti, in modo
          chiaro e conciso, i termini che possono causare ambiguità o incomprensione del testo.
  \subsection{Comunicazione malfunzionamenti}
  Nel caso di malfunzionamenti o comportamenti indesiderati dell'applicazione SWEDesigner, si invitano gli utilizzatori a contattare il fornitore all'indirizzo email:\\
  \begin{center}
  \emph{sweet.bit.group@gmail.com}
  \end{center}
Occorrerà fornire une descrizione dettagliata del problema e, se possibile, si verrà ricontattati il prima possibile.
In alternativa è pissibile aprire un ticket all'interno della sezione \emph{Iussues} sulla pagina \emph{GitHub} all'indirizzo \url{https://github.com/SWEetBITGroup/SWEDesigner-App/issues}. Il fornitore provvederà quanto prima
a trovare una solzuione per il problema indicato.
