\section{Introduzione}
  \subsection{Scopo del documento}
          Questo documento `rappresenta il manuale utente per l'applicazione SWEDesigner, nel quale verranno descritte dettagliatamente le caratteristiche dell'applicativo utilizzabili dall'utente.
  \subsection{Scopo del Prodotto}
          Lo scopo del progetto è la realizzazione di una \glossaryItem{Web App} che fornisca all'\glossaryItem{Utente} un \glossaryItem{UML} \glossaryItem{Designer} con il quale riuscire a disegnare correttamente \glossaryItem{Diagrammi delle classi}
          e descrivere il comportamento dei \glossaryItem{Metodi} interni alle stesse attraverso l'utilizzo del \glossaryItem{Diagramma delle attività}.
          La \glossaryItem{Web App} permetterà all'\glossaryItem{Utente} di generare \glossaryItem{Codice} \glossaryItem{Java} dal \glossaryItem{Diagramma} disegnato ed eventualmente andare a ritoccarne il risultato al fine di ottenere un \glossaryItem{Codice}
          eseguibile, funzionante e funzionale.
  \subsection{Glossario}
          Con lo scopo di evitare ambiguità di linguaggio e di massimizzare la comprensione dei documenti, il
          gruppo ha steso un documento interno che è il \emph{Glossario \VersioneG{}}. In esso saranno definiti, in modo
          chiaro e conciso, i termini che possono causare ambiguità o incomprensione del testo.
  \subsection{Comunicazione malfunzionamenti}
  Nel caso di malfunzionamenti o comportamenti indesiderati dell'applicazione SWEDesigner, si invitano gli utilizzatori a contattare il fornitore all'indirizzo email:\\
  \begin{center}
  \emph{sweet.bit.group@gmail.com}
  \end{center}
Occorrerà fornire une descrizione dettagliata del problema e si sarà ricontattati il prima possibile.