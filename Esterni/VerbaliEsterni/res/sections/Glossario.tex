	\newcommand{\glossaryElement}[1]{
	 \textbf{#1}
	 
%A
	  \glossaryElement{activity diagram}: è un diagramma definito all'interno dell’UML che definisce le attività da svolgere per realizzare una data funzionalità. Può essere utilizzato durante la progettazione del software per dettagliare un determinato algoritmo.
	  
 %C 
 
	   \glossaryElement{capitolato}: atto allegato a un contratto d'appalto che intercorre tra il cliente ed una ditta in cui vengono indicate modalità, costi e tempi di realizzazione dell'opera oggetto del contratto.
	   
	   	%TEMPORANEO
		\glossaryElement{capitolati}: plurale di capitolato.
	
	   \glossaryElement{classe}: è un costrutto di un linguaggio di programmazione atto a rappresentare una persona, un luogo, oppure una cosa, ed è quindi l'astrazione di un concetto.
	     
	    %TEMPORANEO
		\glossaryElement{classi}: plurale di classe.
	
	    \glossaryElement{client}: in informatica, indica una componente che accede ai servizi o alle risorse di un'altra componente detta server, la quale fornisce il servizio richiesto.
	     
	    \glossaryElement{codice}: è una rappresentazione di un insieme di simboli in grado di rappresentare l'informazione che viene così codificata.
	
%D     
	
	    \glossaryElement{desktop}: si intende il processo di scrittura di software che verrà eseguito in un computer standard (desktop, portatile o generico). Il software sviluppato potrebbe essere software applicativo, concepito per l'esecuzione di una o più attività e include elementi quali giochi, elaboratori di testo e applicazioni aziendali personalizzate, oppure software di supporto al sistema operativo. Solitamente una applicazione desktop richiede una installazione prima di poter esser utilizzata.
	     
	    \glossaryElement{dominio}: nel contesto utilizzato si intende focalizzarsi su di un specifico ambito; ovvero la dove si è deciso l’ambito su cui rappresentare i diagrammi (ad esempio i giochi da tavolo), tutto ciò che riguarda argomenti esterni viene ignorato perché non fa parte di tale dominio.

		\glossaryElement{diagramma}: è una rappresentazione simbolica di dati che si prefigge lo scopo di renderli facilmente consultabili, elaborato graficamente secondo convenzioni prestabilite. I diagrammi si differenziano in base al metodo di rappresentazione e allo scopo specifico che viene prefissato.
	
		%TEMPORANEO
		\glossaryElement{diagrammi}: plurale di diagramma.

%F
	
		\glossaryElement{file}: traducibile come "archivio", ma comunemente chiamato anche "documento"; in informatica, viene utilizzato per riferirsi a un contenitore di informazioni/dati in formato digitale. Le informazioni scritte/codificate al suo interno sono leggibili solo tramite uno specifico software in grado di effettuare l'operazione.
	
		\glossaryElement{flow chart}: indica una rappresentazione grafica, o diagramma, usato in informatica e altre discipline, per riferirsi a un particolare tipo di rappresentazione grafica del flusso di controllo negli algoritmi e nei processi.
	
%J
	
		\glossaryElement{Java}: è un linguaggio di programmazione ad alto livello, orientato agli oggetti e a tipizzazione statica, specificatamente progettato per essere il più possibile indipendente dalla piattaforma di esecuzione.
	
		\glossaryElement{JavaScript}: è un linguaggio di \glossaryItem{scripting} orientato agli oggetti e agli eventi, comunemente utilizzato nella programmazione Web lato \glossaryItem{client} per la creazione, in siti web e web-app, di effetti dinamici interattivi tramite funzioni di script invocate da eventi innescati a loro volta in vari modi dall'utente sulla pagina web in uso (mouse, tastiera, caricamento della pagina ecc...).
	
		\glossaryElement{JSON}: acronimo di JavaScript Object Notation, è un formato adatto all'interscambio di dati fra applicazioni client-server.
	
%L
	
		\glossaryElement{layer}: sinonimo di strato.
	
%M
	
		\glossaryElement{metodo}: in informatica, è un termine che viene usato principalmente nel contesto della programmazione orientata agli oggetti per indicare un sottoprogramma associato in modo esclusivo ad una classe e che rappresenta (in genere) un'operazione eseguibile sugli oggetti e istanze di quella classe. È formato da:
- una firma ovvero la definizione/dichiarazione del metodo con tipo di ritorno, nome del metodo, tipo e nome degli eventuali parametri passati in input.
- un corpo, opportunamente delimitato da inizio e fine, con all'interno una o più sequenze o blocchi di istruzioni scritte per eseguire una determinata azione.

		%TEMPORANEO
		\glossaryElement{metodi}: plurale di metodo.
		
%P

		\glossaryElement{pattern}: è un termine inglese, che può essere tradotto, a seconda del contesto, con "disegno, modello, schema, schema ricorrente, struttura ripetitiva" e, in generale, può essere utilizzato per indicare una regolarità che si riscontra all'interno di un insieme di oggetti osservati.

%S

		\glossaryElement{script}: in informatica, designa un tipo particolare di programma, scritto in una particolare classe di linguaggi di programmazione, detti linguaggi di scripting.

%T

		\glossaryElement{template}: in informatica indica un documento o programma nel quale, come in un foglio semicompilato cartaceo, su una struttura generica o standard esistono spazi temporaneamente "bianchi" da riempire successivamente.

%U

		\glossaryElement{UML}: L'UML, o unified modeling language (linguaggio di modellizzazione unificato) è un linguaggio di modellazione basato sul paradigma dell'orientamento agli oggetti
che mira a creare uno standard che possa unificare tutti i linguaggi che ne fanno uso.

		\glossaryElement{utente}: è colui che usufruisce di un bene o di un servizio, generalmente collettivo, fornito da enti pubblici o strutture private. In ambito informatico è colui che interagisce con un computer.

%W

		\glossaryelement{Wen App}: Si indica con Web App, genericamente, tutte quelle applicazioni web-based, ovvero un'applicazione fruibile via web tramite un network, ovvero
mediante l'utilizzo di una struttura tipica client-server.

%Z

		\glossaryElement{zoom}: sinonimo di ingrandimento.
	 }
	 
