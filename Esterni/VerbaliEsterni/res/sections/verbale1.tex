	\section{Riunione 1}
	  \subsection{Informazioni sulla riunione}
	    \begin{itemize}
	      \item \textbf{Data: } 23/02/2017;
	      \item \textbf{Luogo: } Zucchetti - sede di Padova , via Giovanni Cittadella 7;
	      \item \textbf{Ora: } 16:30;
	      \item \textbf{Durata: } 90 min;
	      \item \textbf{Argomento: } Chiarimenti sui requisiti del \glossaryItem{capitolato};
	      \item \textbf{Partecipanti Interni:} Santimaria Davide - Massignan Fabio - Salmistraro Gianmarco - Bodian Malick - Pilò Salvatore - Bertolin Sebastiano;
	      \item \textbf{Partecipanti Esterni:} Piccoli Gregorio.
	    \end{itemize}
	  \subsection{Domande e Risposte}
	   \begin{itemize}
	   
	   	\item 
	   	%dDOMANDA
	   		\textbf{Come dev'essere la qualità del \glossaryItem{codice} generato dai \glossaryItem{diagrammi}\glossaryItem{UML}?}
	   	%RISPOSTA
	    	\justifying     		
Per rispondere al meglio alla domanda, ci si vuole dapprima focalizzare su un altro punto, ovvero il perché sia stato suggerito il tema dei giochi da tavolo in questo \glossaryItem{capitolato}.\\
Ebbene nel \glossaryItem{dominio} dei giochi da tavolo, la ripetitività è molto elevata; basti pensare al gioco della Dama o degli Scacchi, entrambi condividono la stessa scacchiera di gioco, anche il Monopoli ha una scacchiera, che si diversifica per colori e tipi di caselle, ma è pur sempre una scacchiera. Focalizzarsi su un specifico ambito, in questo caso i giochi da tavolo, aiuta a migliorare la qualità del \glossaryItem{codice} (vista anche la possibilità del suo riutilizzo), in quanto se si decidesse di rappresentare ogni contesto, risulterebbe difficoltoso generare del \glossaryItem{codice} che si adatti al meglio in ogni situazione.\\Sfruttare il fattore del riutilizzo sicuramente aiuta, ma la scelta che incide è il modo in cui si sceglie di disporre i \glossaryItem{diagrammi}, ovvero se visualizzarli per \glossaryItem{layer}, suddividendoli per categorie ad esempio, oppure disporre il tutto su un unico \glossaryItem{layer} ed utilizzare un sistema di evidenziazione, \glossaryItem{zoom} o altre tecniche. Tale scelta è importante perchè riflette quanto più complesso e specifico potrà essere il progetto che l'\glossaryItem{utente} vorrà realizzare, e da questo ne deriva anche la qualità del \glossaryItem{codice}.
      		La vera difficoltà sta nel generare \glossaryItem{codice} dai \glossaryItem{diagrammi} che rappresenteranno i \glossaryItem{metodi}, per essi si suggerisce un approccio utilizzando il\glossaryItem{diagramma delle attività}.\\

\newpage			  	
	      \item
	      %domanda
	      \textbf{Nel nostro \glossaryItem{designer} dobbiamo includere dei \glossaryItem{template}?\\}
		%risposta
		\justifying
		Si, a patto che venga scelto un \glossaryItem{dominio} su cui il \glossaryItem{designer} si basi.\\
		Mettere a disposizione ad esempio una scacchiera 8x8, evitando che l'\glossaryItem{utente} finale debba crearsela da zero.
		Ovviamente i giochi da tavolo non son l'unico \glossaryItem{dominio} su cui ci si può basare, la scelta potrebbe ricadere su altri settori, ma tale scelta deve esser fatta per poter avere dei \glossaryItem{diagrammi} e quindi del \glossaryItem{codice}.\\
	Il \glossaryItem{dominio} fin ora citato è consigliato perchè offre molti approfondimenti sull'adattamento in \glossaryItem{diagrammi} ed inoltre garantirà una fase di testing più piacevole. \\
	
	    \item
	    %domanda
	    \textbf{Se viene modificato il \glossaryItem{codice} generato, il \glossaryItem{diagramma} deve aggiornarsi anch'esso?\\}
	    \justifying
	   È un aspetto sicuramente interessante se si riuscisse ad implementare. Solitamente dopo la creazione dei \glossaryItem{diagrammi} ed il rispettivo \glossaryItem{codice}, le successive modifiche vengono apportate solo al \glossaryItem{codice}; in quanto il \glossaryItem{diagramma} ha lo scopo principale di fungere da linea guida, rappresentando solo le \glossaryItem{classi} principali, tralasciando nella visualizzazione quelle di supporto. Rappresentare tutte le \glossaryItem{classi} che costituiscono il progetto, potrebbe ridurre la leggibilità del \glossaryItem{diagramma}; una soluzione potrebbe essere quella di celare o inserire in un \glossaryItem{layer} diverso le \glossaryItem{classi} di supporto.\\
     Visualizzare solo lo scheletro della \glossaryItem{classe} senza l'implementazione dei \glossaryItem{metodi} può esser una soluzione alla domanda posta.\\
	     
	   	 \item
	   	 %domanda
	   	 \textbf{L'\glossaryItem{Applicazione} deve essere solo \glossaryItem{desktop} o deve essere anche una \glossaryItem{Web-App}?\\}
	   	 %risposta
	   	 \justifying
	   	 È preferita la \glossaryItem{Web-App}, ma la scelta non è vincolante. Lo scopo principale è entrare nell'ottica di fare un progetto usando molto i \glossaryItem{diagrammi}.\\
			 	
		   	 \item
	   	 %domanda
	   	 \textbf{Il \glossaryItem{codice} prodotto deve esser in formato \glossaryItem{Java} o \glossaryItem{JavaScript}?\\}
	   	 %risposta
	   	 \justifying
	   	 Nella fase di disegno dei \glossaryItem{diagrammi}, con particolare riferimento a quelli dei \glossaryItem{metodi}, si deve procedere in modo astratto, ovvero tracciando solo l'algoritmo necessario al \glossaryItem{metodo} in esame. Procedendo con quest'ottica si può generare \glossaryItem{codice} in entrambi i linguaggi. Proseguire in quest'ottica di pensare per specifiche risulta interessante ed una buona sfida, ma se si decidesse di pensare per programma, iI consiglio è quello di procedere inizialmente con \glossaryItem{Java}, che è un linguaggio più controllato e verificato.\\
		
		   	 \item
	   	 %domanda
	   	 \textbf{L'\glossaryItem{Applicazione} dovrà esser disponibile su di un \glossaryItem{Server} oppure è sufficiente in locale?\\}
	   	 %risposta
	   	\justifying
	   	 Non è necessario l'acquisto di uno spazio su cui ospitare l'\glossaryItem{Applicazione}, eventualmente si possono utilizzare dei \glossaryItem{Server} gratuiti a tempo limitato come ad esempio Amazon, Heroku.\\
		
		\item
	   	 %domanda
	   	 \textbf{Considerando che il numero di gruppi che aderiscono a tale \glossaryItem{capitolato} è aumentato, alcuni dei requisiti opzionali son diventati obbligatori?\\}
	   	 %risposta
	   	\justifying
	   	I requisiti opzionali rimangono invariati, un aspetto su cui focalizzarsi è lo studio dell'\glossaryItem{UML}. Il progetto consiste nel creare un disegnatore che abbia il \glossaryItem{diagramma} della \glossaryItem{classe} con i rispettivi \glossaryItem{diagrammi} dei \glossaryItem{metodi}; la soluzione di come rappresentare il collegamento tra questi \glossaryItem{diagrammi}, è sicuramente il punto su cui ci si deve concentrare, proponendo anche modelli ibridi dei \glossaryItem{diagrammi}.\\
	   	 
	   	 		\item
	   	 %domanda
	   	 \textbf{Ci devono essere dei controlli nella costruzione del \glossaryItem{diagramma}? E come devo restituire il \glossaryItem{codice} generato? \\}
	   	 %risposta
	   	\justifying
	   	 Ciò che viene restituito può essere un pacchetto contenente i \glossaryItem{file} oppure una visualizzazione testuale del \glossaryItem{codice}; l'importante è che venga visualizzato il risultato che l'\glossaryItem{utente} si aspetta. Per quanto riguarda i controlli, sono un lavoro di cortesia, evitare che l'\glossaryItem{utente} durante la fase di "disegno" faccia degli errori come ad esempio dimenticarsi il nome della \glossaryItem{classe} è sicuramente apprezzato. Ovviamente se si desse la possibilità di aggiungere nel \glossaryItem{diagramma} un rettangolo dove si possa aggiungere del \glossaryItem{codice} manualmente, in quella situazione risulterà impegnativo gestire automaticamente l'errore. 
	   	 \end{itemize}
