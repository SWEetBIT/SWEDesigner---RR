	\section{Riunione 2}
	  \subsection{Informazioni sulla riunione}
	    \begin{itemize}
	      \item \textbf{Data: } 15/03/2017
	      \item \textbf{Luogo: } Zucchetti - sede di Padova , via Giovanni Cittadella 7
	      \item \textbf{Ora: } 11:30
	      \item \textbf{Durata: } 20 min
	      \item \textbf{Argomento: } Consulenza sui \glossaryItem{diagrammi}
	      \item \textbf{Partecipanti Interni: } Davide Santimaria - Fabio Massignan - Gianmarco Salmistraro - Malick Bodian - Salvatore Pilò - Sebastiano Bertolin
	      \item \textbf{Partecipanti Esterni: } Gregorio Piccoli
	    \end{itemize}

	  \subsection{Domande e Risposte}
	   \begin{itemize}
	   
	   	   	\item 
	   	%dDOMANDA
	   		\textbf{Per la gestione tra i diagrammi delle classi e dei metodi, abbiamo pensato di mettere a disposizione dell'utente una finestra principalmente divisa in 3 parti. La prima parte comprende gli strumenti necessari a disegnare i vari diagrammi, la seconda finestra è quella principale e consiste nello schema delle classi e, in caso di doppio click sul metodo di una classe, si aprirà un \glossaryItem{activity diagram}, col quale l'utente potrà creare e definire il corpo del metodo. Nella terza finestra, l'utente ha a disposizione un \glossaryItem{activity diagram} in sola lettura, sul quale è possibile visualizzare il flusso dell'intero progetto. Come le sembra la nostra soluzione? } \\
	   	%RISPOSTA
	    	\justifying     		
Questa disposizione è apprezzata, l' importante è non saltare il diagramma delle classi e trattare il \glossaryItem{activity diagram} in modo da poter generare il codice..\\

	   	\item 
	   	%dDOMANDA
	   		\textbf{Data la scelta del \glossaryItem{dominio} dei giochi da tavolo, possiamo aggiungere dei \glossaryItem{template} di \glossaryItem{classi} e funzioni legate al \glossaryItem{dominio}?} \\
	   	%RISPOSTA
	    	\justifying     		
L'\glossaryItem{UML} non ha un grande supporto dei \glossaryItem{template}, solitamente si usano le collaborazioni, che son un ente che descrivono un insieme statico di relazioni tra istanze, e i ruoli che queste istanze svolgono in queste relazioni. Questa soluzione è abbastanza scomoda; suggerisco di approfondire i \glossaryItem{pattern} che utilizza l'\glossaryItem{UML} per capire da sé che inizialmente l'\glossaryItem{UML} è nato per disegnare \glossaryItem{classi} e quindi con l'uso dei \glossaryItem{pattern} c'è contrasto. 
Ci son due rappresentazioni di \glossaryItem{pattern} quella della sua definizione e quella del suo uso; implementare la possibilità all'\glossaryItem{utente} di crearsi i propri \glossaryItem{template}. Quando l'\glossaryItem{utente} crerà il suo \glossaryItem{template}, le \glossaryItem{classi} che ne fanno parte dovranno avere una etichetta che ne definisce il ruolo e la funzionalità del cambio del nome; altrimenti se in un \glossaryItem{diagramma} si richiama più volte lo stesso \glossaryItem{template} darà luogo ad errori visto che il nome è un attributo univoco. Un esempio può essere un videogioco di calcio, dove per rappresentare la testa dei giocatori si usa un \glossaryItem{pattern}, che a sua volta può esser un contenitore di altri \glossaryItem{pattern} come occhi, e naso. Già questo \glossaryItem{pattern} si può utilizzare 22 volte per il numero di giocatori in campo ed ognuno di essi avrà delle caratteristiche differenti, ma seguono tutti lo stesso modello.
\\
	  		   	\item 
	   	%dDOMANDA
	   		\textbf{Il passaggio tra \glossaryItem{codice} e disegno è un requisito richiesto?} \\
	   	%RISPOSTA
	    	\justifying     		
No, non è necessario questa caratteristica, si potrebbe valutare eventualmente la sincronizzazione , o meglio dare la possibilità a due persone di lavorare allo stesso \glossaryItem{diagramma} , ma anche questo è una caratteristica che può esser inserita nei nei requisiti opzionali. Ci si potrebbe orientare su un disegnatore collaborativo, che supporti \glossaryItem{JSON} e librerie come \emph{diff-match-Patch}, così da permetter di far lavorare in maniera collaborativa più utenti, grazie allo scambio di \glossaryItem{file} in formato \glossaryItem{JSON}; resta comunque una caratteristica che diventa un progetto all'interno del progetto attuale e quindi, come già detto, è un requisito opzionale.
\\
	  		   	\item 
	   	%dDOMANDA
	   		\textbf{È obbligatorio fornire l'auto compilazione del \glossaryItem{codice}?} \\
	   	%RISPOSTA
	    	\justifying     		
Se si intende che il \glossaryItem{codice} generato debba compilare questo è banalmente ovvio. La compilazione automatica sarebbe gradita.
\\
		  		   	\item 
	   	%dDOMANDA
	   		\textbf{Dobbiamo aggiungere la presenza di un \glossaryItem{utente} amministratore?} \\
	   	%RISPOSTA
	    	\justifying     		
Credo che la mole di lavoro sia sufficiente e che quindi aggiungere nuove funzionalità al momento si possa esculdere, basta concentrarsi sui requisiti già definiti. Avete accennato alla possibilità di aggiungere dei gruppi di lavoro, ma preferisco che vi concentriate sull'inserimento dei \glossaryItem{template} anzichè di questa funzionalità.
\\
	   	 \end{itemize}
