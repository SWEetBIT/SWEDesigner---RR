\section{Descrizione architettura }
	\subsection{Metodo e formalismo di specifica}
Le scelte architetturali per lo sviluppo di SWEDesigner sono state fortemente influenzate dallo stack tecnologico utilizzato. \\

Nell’esposizione dell’architettura dell'applicazione si procederà con un approccio \glossaryItem{top-down}, descrivendo l'architettura iniziando dal generale ed andando al particolare; si è partiti suddividendo il sistema in \glossaryItem{front-end} e \glossaryItem{back-end}, definendo l'interfaccia di comunicazione, scegliendo di seguire in ciascuno l'organizzazione suggeritaci dai framework (Express e Angular.js). 

La descrizione dell’architettura di SWEDesigner è suddivisa in quattro sezioni:
\begin{itemize}
\item \S\ref{3.2}: che illustra gli aspetti generali dell’architettura del software;
\item \S\ref{3.3}: che descrive il protocollo che lega le due interfacce tra \glossaryItem{Client} e \glossaryItem{Server};
che descrive l’architettura del front end dell’applicazione;
\item \S\ref{3.4}: che descrive l’architettura del \glossaryItem{back-end} dell’applicazione;
\item \S\ref{3.5}: che descrive l’architettura del \glossaryItem{front-end} dell’applicazione.

\end{itemize}

I vari tipi di diagrammi presentati di seguito utilizzano la specifica \glossaryItem{UML} 2.0.

	\subsection{Architettura generale}
	\label{3.2}
	\subsection{Interfaccia REST-like}
	\label{3.3}
	\subsection{Architettura del Server}
	\label{3.4}
	\subsection{Architettura del Client}
	\label{3.5}
