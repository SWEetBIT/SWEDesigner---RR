\section{Tecnologie utilizzate}
L'architettura è stata progettata utilizzando lo stack di \textbf{\glossaryItem{MEAN}} (\url {http://mean.io/ } ),il quale comprende 4 tecnologie, alcune delle quali espressamente richieste nel \glossaryItem{capitolato} d’appalto. Vengono di seguito elencate e descritte le principali tecnologie impiegate comprese in \textbf{\glossaryItem{MEAN}} e le motivazioni del loro utilizzo:
\begin{itemize}
\item \textbf{Node.js}: piattaforma per il \glossaryItem{back-end};
\item \textbf{Expressjs}: \glossaryItem{framework} per la realizzazione dell’applicazione web in \glossaryItem{Node.js} ;
\item \textbf{MongoDB}: \glossaryItem{database} di tipo \glossaryItem{NoSQL} per la parte di recupero e salvataggio dei dati;
\item \textbf{Mongoose}: \glossaryItem{libreria} per interfacciarsi con il driver di \textbf{MongoDB};
\item \textbf{Angular 4.0}: \glossaryItem{framework} \glossaryItem{JavaScript} per la realizzazione del \glossaryItem{front-end} .
\end{itemize}

	\subsection{Server}
	\subsubsection{Node.js}
\textbf{Node.js} è una \glossaryItem{piattaforma} software costruita sul motore \glossaryItem{JavaScript} di \glossaryItem{Chrome} che permette di realizzare facilmente applicazioni di rete scalabili e veloci. \glossaryItem{Node.js} utilizza \glossaryItem{JavaScript} come linguaggio di programmazione, e grazie al suo modello \glossaryItem{event-driven} con chiamate di input/output non bloccanti risulta essere leggero e e ciente.

\paragraph{Vantaggi}
\begin{itemize}
\item \textbf{Approccio asincrono}: \glossaryItem{Node.js} permette di accedere alle risorse del sistema operativo in modalità \glossaryItem{event-driven} e non sfruttando il classico modello basato su processi concorrenti utilizzato dai classici web \glossaryItem{server}. Ciò garantisce una maggiore efficienza in termini di prestazioni, poiché durante le attese il runtime può gestire qualcos’altro in maniera asincrona;
\item \textbf{Architettura modulare}: Lavorando con \glossaryItem{Node.js} è molto facile organizzare il lavoro in librerie, importare i \glossaryItem{moduli} e combinarli fra loro.Questo è reso molto comodo attraverso il \glossaryItem{node package manager} (\textbf{npm}) attraverso il quale lo sviluppatore può contribuire e accedere ai \glossaryItem{package} messi a disposizione dalla community.
\end{itemize}

\paragraph{Svantaggi}
\begin{itemize}
\item \textbf{Supporto incompleto alle feature di \glossaryItem{ES6}}: Molte delle feature di ES6 non
sono supportate in Node nella versione 4.4 scelta come versione di riferimento per
lo sviluppo del progetto.
\end{itemize}

	\subsubsection{Expressjs}
\textbf{Expressjs} è un \glossaryItem{framework} minimale per creare  \glossaryItem{Web App} con \glossaryItem{Node.js}. Richiede \glossaryItem{moduli} Node di terze parti per applicazioni che prevedono l'interazione con le \glossaryItem{database}.
È stato utilizzato il \glossaryItem{framework} \glossaryItem{Expressjs} per supportare lo sviluppo dell'application \glossaryItem{server} grazie alle utili e robuste caratteristiche da esso offerte, le quali sono pensate per non oscurare le funzionalità fornite da \glossaryItem{Node.js} aprendo così le porte all'utilizzo di moduli per \glossaryItem{Node.js} atti a supportare specifiche funzionalità.

\paragraph{Vantaggi}
\begin{itemize}
\item \textbf{Minimale}: si basa su \glossaryItem{Node.js} e permette di estenderlo a seconda dei bisogni
dell’applicazione;
\item\textbf{Documentazione}: esaustiva e completa;
\item\textbf{Apprendimento}: facile da imparare.
\end{itemize}

\paragraph{Svantaggi}
\begin{itemize}
\item \textbf{Integrazione}: richiede di integrare \glossaryItem{moduli} diversi per comporre l’applicazione
finale. Altri \glossaryItem{framework} permettono di definire \glossaryItem{API} (Application Programming
Interface) \glossaryItem{REST} (REpresentational State Transfer) in modo semplice, ma vincolano
maggiormente nelle scelte progettuali.
\end{itemize}


	\subsubsection{MongoDB}
\textbf{\glossaryItem{MongoDB}} è un \glossaryItem{database} \glossaryItem{NoSQL} \glossaryItem{open source} scalabile e altamente performante di tipo document-oriented, in cui i dati sono archiviati sotto forma di documenti in stile \glossaryItem{JSON} con schemi dinamici, secondo una struttura semplice e potente. \\

\paragraph{Vantaggi}
 \begin{itemize}

\item \textbf{Alte performance}: non ci sono join che possono rallentare le operazioni di lettura o scrittura. L'indicizzazione include gli indici di chiave anche sui documenti innestati e sugli array, permettendo una rapida interrogazione al \glossaryItem{database};
\item \textbf{Affidabilità}: alto meccanismo di replicazione su server;
\item \textbf{Schemaless}: non esiste nessuno \glossaryItem{schema} , è più flessibile e può essere facilmente traspostoin un modello ad oggetti;
\item Permette di definire query complesse utilizzando un linguaggio che non è \glossaryItem{SQL};
\item Permette di processare parallelamente i dati (\glossaryItem{Map-Reduce});
\item Tipi di dato più flessibili.
\end{itemize}

\paragraph{Svantaggi}
\begin{itemize}
\item \textbf{Flessibilità}: per i tipi di dato. Sebbene questo possa essere visto come vantaggio,
è opinione del team che un’eccessiva flessibilità possa portare più problemi che
benefici: allo scopo di aggiungere rigidità è stato infatti scelto, come verrà descritto
in seguito, Mongoose, che introduce una costruzione a schemi per le collections di
\glossaryItem{MongoDB} e quindi vincola i documenti inseriti ad avere una struttura uniforme;
\item \textbf{Nessun supporto per le transazioni}: sono supportate alcune operazioni atomiche,
ma a livello di documento;
\item \textbf{Nessun \glossaryItem{join}}: va simulato via codice attraverso query multiple;
\item \textbf{Problemi di concorrenza}: per le operazioni di scrittura viene creato un lock
sull’intero database. Questo lock blocca anche le operazioni di lettura.
\end{itemize}
	\subsubsection{Mongoose}
\textbf{Mongoose} è una \glossaryItem{libreria} per interfacciarsi a \glossaryItem{MongoDB} che permette di definire degli schemi per modellare i dati del \glossaryItem{database}, imponendo una certa struttura per la creazione di nuovi Document . Inoltre fornisce molti strumenti utili per la validazione dei dati, per la definizione di query e per il cast dei tipi predefiniti.
Per interfacciare l'application \glossaryItem{server} con \glossaryItem{MongoDB} sono disponibili diversi progetti \glossaryItem{open source}. Per questo progetto è stato scelto di utilizzare \glossaryItem{Mongoose.js} , attualmente il più di uso.

\paragraph{Vantaggi}
\begin{itemize}
\item \textbf{Diffusione}: è la libreria più diffusa per interfacciarsi con \glossaryItem{MongoDB};
\item \textbf{Funzionalità aggiuntive}: permette di definire strumenti per la validazione dei
dati e per il cast dei tipi;
\item \textbf{Permette di eseguire dei \glossaryItem{join} tra collections}: Sebbene non sia previsto da
\glossaryItem{MongoDB}, \glossaryItem{mongoose} prevede la funzione populate per imitare la funzione di \glossaryItem{join} in modo completamente trasparente per l’utilizzatore;
\item \textbf{Rapido ed intuitivo}: La strutturazione dei dati con questa libreria è rapida ed
inutitiva, ciò dovuto anche dalla sintassi dichiarativa della libreria stessa.
\end{itemize}

\paragraph{Svantaggi}
\begin{itemize}
\item \textbf{Schema-based}: è basato sulla creazione di una forte schematizzazione per i
documenti, e questo limita l’estrema flessibilità di \glossaryItem{MongoDB}.
\end{itemize}

\subsection{Librerie}
Vengono di seguito descritte le \glossaryItem{librerie} aggiuntive utilizzate dal \glossaryItem{back-end}. La scelta è stata effettuata cercando di valutare la diffusione, il livello di stabilità, l'assenza di errori noti.

	\subsubsection{Mustache}
\textbf{Mustache} è un \glossaryItem{template} engine che permette di espandere tags all'interno di un \glossaryItem{template}, racchiusi da 2 parentesi graffe,usando valori forniti da oggetti.

	\subsubsection{Passport:} \label{passport} È un \glossaryItem{middleware} di autenticazione per \glossaryItem{Node.js}. Estremamente flessibile e modulare, Passport può essere facilmente inserito in qualsiasi applicazione web basata su \glossaryItem{Expressjs}.

	\subsubsection{BodyParser:} \glossaryItem{Modulo} di terze parti per la corretta lettura delle informazioni contenute nel body delle richieste \glossaryItem{HTTP}; viene utilizzato come \glossaryItem{middleware} per {Expressjs} e si occupa della corretta lettura delle informazioni contenute nel body di una richiesta
\glossaryItem{HTTP}. Nel nostro caso verrà impiegato per la lettura dei dati del body in formato \glossaryItem{JSON}.

	\subsubsection{Forge:} È un \glossaryItem{modulo} in \glossaryItem{Javascript} che fornisce funzionalità di criptografia per la sicurezza e un set di strumenti per la realizzazione di \glossaryItem{Web app}.

	\subsubsection{PassportJWT:} È un \glossaryItem{modulo} che fornisce una strategia Passport(\S\ref{3.2}) per l'autenticazione con un \glossaryItem{JSON} Web Token, che è una tecnica compatta per trasmettere in modo sicuro le informazioni tra 2 oggetti \glossaryItem{JSON}.

	\subsubsection{Bcrypt:} È una \glossaryItem{libreria} in \glossaryItem{Javascript} che utilizza la funzione di \glossaryItem{hash}ing per criptare le password.

\subsection{Client}

	\subsubsection{Angular 4.0}
	\textbf{Angular 4.0} è un \glossaryItem{framework} web \glossaryItem{open source} per lo sviluppo di applicazioni Web lato client; utile a semplificare la realizzazione di applicazioni web, come ad esempio le Single Page Application, cioè applicazioni le cui risorse vengono caricate dinamicamente su richiesta, senza necessità di ricaricare l’intera pagina.

\paragraph{Vantaggi}
\begin{itemize}
\item \textbf{Velocità}: Riduce in maniera considerevole il codice necessario a realizzare applicazioni \glossaryItem{HTML}/\glossaryItem{JavaScript};
\item \textbf{Ampia documentazione disponibile};
\item \textbf{Data Binding bidirezionale}: approccio automatico per aggiornare la vista ogniqualvolta il model cambia e viceversa. Ciò semplifica lo sviluppo eliminando la necessità di manipolare il \glossaryItem{DOM};
\item \textbf{Sviluppato per facilitare la fase di test};
\item \textbf{Direttive}: caratteristica peculiare di Angular e permettono di estendere la sintassi \glossaryItem{HTML} , creando dei componenti specifici per la propria applicazione e facilmente riutilizzabili;
\end{itemize}

\paragraph{Svantaggi}
\begin{itemize}
\item \textbf{Maggior studio}: curva di apprendimento più ripida rispetto ad altri \glossaryItem{framework};
\item \textbf{Codice articolato}: ciò potrebbe comportare, in caso di variazione dei requisiti, delle difficoltà delle successive modifiche.
\end{itemize}

	\subsubsection{Draw2D}
	\textbf{Drwa2D} è una libreria \glossaryItem{JavaScript} di \glossaryItem{diagrammi} che consente di creare rapidamente applicazioni di grafici interattive e grafici che vengono eseguiti in modo nativo su tutti i browser principali.

\paragraph{Vantaggi}
\begin{itemize}
\item \textbf{Non sono necessari altri plug-in}: Ciò elimina i plug-in di dipendenza dai fornitori;
\item \textbf{\glossaryItem{Open source}}:Le tecnologie coinvolte sono libere e ci sono molte implementazioni aperte, nessun fornitore può rimuovere un prodotto o una tecnologia che lascia in pratica la tua applicazione inoperabile;
\item \textbf{Le tecnologie sono standardizzate}: L'applicazione è distribuibile al numero massimo di utenti del browser senza bisogno di ulteriori configurazioni o installazione sul computer \glossaryItem{client}. Gli ambienti aziendali di grandi dimensioni spesso non amano consentire agli individui di installare plug-in del \glossaryItem{browser} e non amano cambiare la build standard creata su tutte le macchine.
\end{itemize}

\paragraph{Svantaggi}
\begin{itemize}
\item \textbf{Aumento rapido di celle}: Poiché il numero di celle visibili sullo schermo degli utenti aumenta di centinaia, la valutazione rallenta oltre i limiti accettabili sulla maggior parte dei \glossaryItem{browser}. Nella teoria della gestione delle informazioni, visualizzare diverse centinaia di celle è generalmente sbagliato, in quanto l'utente non può interpretare i dati.
\end{itemize}

\subsubsection{HTML5}
È un linguaggio di markup per la strutturazione delle pagine web, pubblicato come W3C Recommendation da ottobre 2014. L’uso di HTML5 rispetto a XHTML
(e\textbf{X}tensible \textbf{H}yper\textbf{T}ext \textbf{M}arkup \textbf{L}anguage) è stato deciso all'unanimità dal gruppo.

\paragraph{Vantaggi}
\begin{itemize}
\item Raccomandazione W3C;
\item reazione di pagine interattive: soprattutto se usato insieme a CSS.
\end{itemize}

\paragraph{Svantaggi}
\begin{itemize}
\item Supporto: non tutti i browser lo supportano allo stesso modo, e non tutte le caratteristiche definite sono ancora completamente supportate.
\end{itemize}

\subsubsection{CSS3}
È un linguaggio utilizzato per definire la formattazione di documenti HTML.
Le regole per la composizione di un foglio di stile CSS sono definite dal W3C
a partire dal 1996. Inoltre permette di separare i contenuti delle pagine HTML dalla loro formattazione, assicurando una maggiore manutenibilità e riutilizzo.

\paragraph{Vantaggi}
\begin{itemize}
\item Separazione tra contenuto e presentazione;
\item Raccomandato da W3C.
\end{itemize}
