\section{Tecnologie utilizzate}
L'architettura è stata progettata utilizzando lo stack di \textbf{\glossaryItem{MEAN}} (\url {http://mean.io/ } ),il quale comprende 4 tecnologie, alcune delle quali espressamente richieste nel \glossaryItem{capitolato} d’appalto. Vengono di seguito elencate e descritte le principali tecnologie impiegate comprese in \textbf{\glossaryItem{MEAN}} e le motivazioni del loro utilizzo:
\begin{itemize}
\item \textbf{Node.js}: piattaforma per il \glossaryItem{back-end};
\item \textbf{Express.js}: \glossaryItem{framework} per la realizzazione dell’applicazione web in \glossaryItem{Node.js} ;
\item \textbf{MongoDB}: \glossaryItem{database} di tipo \glossaryItem{NoSQL} per la parte di recupero e salvataggio dei dati; 
\item \textbf{Mongoose}: \glossaryItem{libreria} per interfacciarsi con il driver di \textbf{MongoDB};
\item \textbf{Angular.js}: \glossaryItem{framework} \glossaryItem{JavaScript} la realizzazione del \glossaryItem{front-end} .
\end{itemize}

	\subsection{Server}
	\subsubsection{Node.js}
\textbf{Node.js} è una \glossaryItem{piattaforma} software costruita sul motore \glossaryItem{JavaScript} di \glossaryItem{Chrome} che permette di realizzare facilmente applicazioni di rete scalabili e veloci. \glossaryItem{Node.js} utilizza \glossaryItem{JavaScript} come linguaggio di programmazione, e grazie al suo modello \glossaryItem{event-driven} con chiamate di input/output non bloccanti risulta essere leggero e e ciente.
I principali vantaggi dell'utilizzo di \glossaryItem{Node.js} sono:

\begin{itemize}

\item \textbf{Approccio asincrono}: \glossaryItem{Node.js} permette di accedere alle risorse del sistema operativo in modalità \glossaryItem{event-driven} e non sfruttando il classico modello basato su processi concorrenti utilizzato dai classici web \glossaryItem{server}. Ciò garantisce una maggiore efficienza in termini di prestazioni, poiché durante le attese il runtime può gestire qualcos’altro in maniera asincrona.

\item \textbf{Architettura modulare}: Lavorando con \glossaryItem{Node.js} è molto facile organizzare il lavoro in librerie, importare i \glossaryItem{moduli} e combinarli fra loro.Questo è reso molto comodo attraverso il \glossaryItem{node package manager} (\textbf{npm}) attraverso il quale lo sviluppatore può contribuire e accedere ai \glossaryItem{package} messi a disposizione dalla community.
\end{itemize}

	\subsection{Client}
	\subsubsection{Express.js}
\textbf{Express.js} è un \glossaryItem{framework} minimale per creare  \glossaryItem{Web App} con \glossaryItem{Node.js}. Richiede \glossaryItem{moduli} Node di terze parti per applicazioni che prevedono l'interazione con le \glossaryItem{database}.
È stato utilizzato il \glossaryItem{framework} \glossaryItem{Express.js} per supportare lo sviluppo dell'application \glossaryItem{server} grazie alle utili e robuste caratteristiche da esso offerte, le quali sono pensate per non oscurare le funzionalità fornite da \glossaryItem{Node.js} aprendo così le porte all'utilizzo di moduli per \glossaryItem{Node.js} atti a supportare specifiche funzionalità.

	\subsubsection{MongoDB}
\textbf{\glossaryItem{MongoDB}} è un \glossaryItem{database} \glossaryItem{NoSQL} \glossaryItem{open source} scalabile e altamente performante di tipo document-oriented, in cui i dati sono archiviati sotto forma di documenti in stile \glossaryItem{JSON} con schemi dinamici, secondo una struttura semplice e potente. \\

I principali vantaggi derivati dal suo utilizzo sono:

 \begin{itemize}

\item \textbf{Alte performance}: non ci sono join che possono rallentare le operazioni di lettura o scrittura. L'indicizzazione include gli indici di chiave anche sui documenti innestati e sugli array, permettendo una rapida interrogazione al \glossaryItem{database};
\item \textbf{Affidabilità}: alto meccanismo di replicazione su server;
\item \textbf{Schemaless}: non esiste nessuno \glossaryItem{schema} , è più flessibile e può essere facilmente traspostoin un modello ad oggetti;
\item Permette di definire query complesse utilizzando un linguaggio che non è \glossaryItem{SQL};
\item Permette di processare parallelamente i dati (\glossaryItem{Map-Reduce});
\item Tipi di dato più flessibili.
\end{itemize}  
	\subsubsection{Mongoose}
\textbf{Mongoose} è una \glossaryItem{libreria} per interfacciarsi a \glossaryItem{MongoDB} che permette di definire degli schemi per modellare i dati del \glossaryItem{database}, imponendo una certa struttura per la creazione di nuovi Document . Inoltre fornisce molti strumenti utili per la validazione dei dati, per la definizione di query e per il cast dei tipi predefiniti.
Per interfacciare l'application \glossaryItem{server} con \glossaryItem{MongoDB} sono disponibili diversi progetti \glossaryItem{open source}. Per questo progetto è stato scelto di utilizzare \glossaryItem{Mongoose.js} , attualmente il più di uso.

	\subsubsection{mxGrafh}