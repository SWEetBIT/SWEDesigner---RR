\section{Introduzione}
\subsection{Scopo del documento}
Questo documento ha come scopo quello di definire la \glossaryItem{progettazione ad alto livello} per il prodotto. Verrà presentata la strttura generale secondo la quale saranno organizzate le varie componenti software e i \glossaryItem{Design Pattern} utilizzati nella creazione del prodotto SWEDesigner. Verrà dettagliato il tracciamento tra le componenti software individuate ed i requisiti. 
\subsection{Scopo del prodotto}
          Lo scopo del progetto è la realizzazone di una \glossaryItem{Web App} che fornisca all'\glossaryItem{Utente} un \glossaryItem{UML} \glossaryItem{Designer} con il quale riuscire a disegnare correttamente \glossaryItem{Diagrammi} delle \glossaryItem{Classi}
          e descrivere il comportamento dei \glossaryItem{Metodi} interni alle stesse attraverso l'utilizzo di \glossaryItem{Diagrammi} delle attività.
          La \glossaryItem{Web App} permetterà all'\glossaryItem{Utente} di generare \glossaryItem{Codice} \glossaryItem{Java} dall'insieme dei \glossaryItem{diagrammi classi} e dei rispettivi \glossaryItem{metodi}.
\subsection{Glossario}
          Con lo scopo di evitare ambiguità di linguaggio e di massimizzare la comprensione dei documenti, il
          gruppo ha steso un documento interno che è il \emph{Glossario v}\VersioneG{}. In esso saranno definiti, in modo
          chiaro e conciso i termini che possono causare ambiguità o incomprensione del testo.
\subsection{Riferimenti}
\subsubsection{Normativi}
\begin{itemize}
	\item \textbf{Capitolato d'Appalto C6: SWEDesigner} \\
		\url{http://www.math.unipd.it/~tullio/IS-1/2016/Progetto/C6p.pdf}; \\
	\item \textbf{Norme di Progetto:} \textit{Norme di Progetto v\VersioneNP}. \\
	
		\item \textbf{Analisi dei Requisiti:} \textit{Analisi dei Requisiti v\VersioneAR}. \\
\end{itemize}
\subsubsection{Informativi}
\begin{itemize}
	\item \textbf{Slide dell'insegnamento Ingegneria del Software modulo A:} \\
		\url{http://www.math.unipd.it/~tullio/IS-1/2016/}. \\
\begin{itemize}
\item \textbf{Slides del corso di Ingegneria del Software mod. A: \glossaryItem{Diagrammi delle classi}}:
\url{http://www.math.unipd.it/~tullio/IS-1/2015/Dispense/E03.pdf};

\item  \textbf{Slides del corso di Ingegneria del Software mod. A: Diagrammi dei package}:
\url{http://www.math.unipd.it/~tullio/IS-1/2015/Dispense/E04.pdf};

\item \textbf{Slides del corso di Ingegneria del Software mod. A: Diagrammi di sequenza}:
\url{http://www.math.unipd.it/~tullio/IS-1/2015/Dispense/E05.pdf};

\item \textbf{Slides del corso di Ingegneria del Software mod. A: Diagrammi di attività}:
\url{http://www.math.unipd.it/~tullio/IS-1/2015/Dispense/E06.pdf};

\item \textbf{Slides del corso di Ingegneria del Software mod. A: \glossaryItem{Design pattern} strutturali:
Decorator, Proxy, Facade, Adapter}:\url{ http://www.math.unipd.it/~tullio/
IS-1/2015/Dispense/E07.pdf};

\item \textbf{Slides del corso di Ingegneria del Software mod. A: \glossaryItem{Design pattern} creazionali:
Singleton, Builder, Abstract Factory}: \url{http://www.math.unipd.it/~tullio/
IS-1/2015/Dispense/E08.pdf};

\item \textbf{Slides del corso di Ingegneria del Software mod. A: \glossaryItem{Design pattern} comportamentali:
Observer, Template Method, Command, Strategy, Iterator}: \url{http:
//www.math.unipd.it/~tullio/IS-1/2015/Dispense/E09.pdf};
\end{itemize}

\item \textbf{Design Patterns - E. Gamma, R. Helm, R. Johnson, J. Vlissides (Pearson Education, Addison-Wesley, 1995;};
\item \textbf{\glossaryItem{Node.js}}: \url{https://nodejs.org/dist/latest-v6.x/docs/api/};

\item\textbf{ MongoDB}: \url{https://docs.mongodb.org/manual/};
\item \textbf{HTML5}: \url{http://www.w3schools.com/html/html5_intro.asp};
\item \textbf{CSS3}: \url{http://www.w3schools.com/css/css3_intro.asp};

\item \textbf{ExpressJS}: \url{http://expressjs.com/en/4x/api.html}.

\item \textbf{Mustache}: \url{http://mustache.github.io/}.
\end{itemize}


