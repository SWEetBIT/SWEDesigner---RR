\section{Introduzione}
\subsection{Scopo del documento}
Questo documento ha come scopo quello di definire la \glossaryItem{progettazione ad alto livello} per il prodotto. Verrà presentata la strttura generale secondo la quale saranno organizzate le varie componenti software e i \glossaryItem{Design Pattern} utilizzati nella creazione del prodotto SWEDesigner. Verrà dettagliato il tracciamento tra le componenti software individuate ed i requisiti. 
\subsection{Scopo del prodotto}
          Lo scopo del progetto è la realizzazone di una \glossaryItem{Web App} che fornisca all'\glossaryItem{Utente} un \glossaryItem{UML} \glossaryItem{Designer} con il quale riuscire a disegnare correttamente \glossaryItem{Diagrammi} delle \glossaryItem{Classi}
          e descrivere il comportamento dei \glossaryItem{Metodi} interni alle stesse attraverso l'utilizzo di \glossaryItem{Diagrammi} delle attività.
          La \glossaryItem{Web App} permetterà all'\glossaryItem{Utente} di generare \glossaryItem{Codice} \glossaryItem{Java} dall'insieme dei \glossaryItem{diagrammi classi} e dei rispettivi \glossaryItem{metodi}.
\subsection{Glossario}
          Con lo scopo di evitare ambiguità di linguaggio e di massimizzare la comprensione dei documenti, il
          gruppo ha steso un documento interno che è il \emph{Glossario v}\VersioneG{}. In esso saranno definiti, in modo
          chiaro e conciso i termini che possono causare ambiguità o incomprensione del testo.
\subsection{Riferimenti}
\subsubsection{Normativi}
\begin{itemize}
	\item \textbf{Capitolato d'Appalto C6: SWEDesigner} \\
		\url{http://www.math.unipd.it/~tullio/IS-1/2015/Progetto/C6p.pdf} \\
	\item \textbf{Norme di Progetto:} \textit{Norme di Progetto v\VersioneNP}. \\
	
		\item \textbf{Analisi dei Requisiti:} \textit{Analisi dei Requisiti v\VersioneAR}. \\
\end{itemize}
\subsubsection{Informativi}
\begin{itemize}
	\item \textbf{Slide dell'insegnamento Ingegneria del Software modulo A:} \\
		\url{http://www.math.unipd.it/~tullio/IS-1/2016/}. \\

\end{itemize}


