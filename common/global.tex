%%%%%%%%%%%%%%
%  COSTANTI  %
%%%%%%%%%%%%%%

% In questa prima parte vanno definite le 'costanti' utilizzate da due o pi� documenti.

% Meglio non mettere gli \emph dentro le costanti, in certi casi creano problemi
\newcommand{\GroupName}{SWEet BIT}
\newcommand{\GroupEmail}{sweet.bit.group@gmail.com}
\newcommand{\ProjectName}{SWEDesigner}
\newcommand{\ProjectVersion}{v1.0.0}

\newcommand{\Proponente}{Zucchetti}
\newcommand{\Committente}{Prof. Tullio Vardanega \\ Prof. Riccardo Cardin}
\newcommand{\Responsabile}{Da Decidere}

% La versione dei documenti deve essere definita qui in global, perch� serve anche agli altri documenti
\newcommand{\VersioneG}{1.0.0}
\newcommand{\VersionePQ}{1.0.0}
\newcommand{\VersioneNP}{1.0.0}
\newcommand{\VersionePP}{1.0.0}
\newcommand{\VersioneAR}{1.0.0}
\newcommand{\VersioneSF}{1.0.0}
\newcommand{\VersioneST}{1.0.0}
\newcommand{\VersioneMA}{1.0.0}
\newcommand{\VersioneMS}{1.0.0}
\newcommand{\VersioneMU}{1.0.0}
\newcommand{\VersioneDP}{1.0.0}
% Il verbale non ha versionamento.

% Quando serve riferirsi a ``Nome del Documento + ultima versione x.y.z'' usiamo queste costanti:
\newcommand{\Glossario}{\emph{Glossario v\VersioneG{}}}
\newcommand{\PianoDiQualifica}{\emph{Piano di Qualifica v\VersionePQ{}}}
\newcommand{\NormeDiProgetto}{\emph{Norme di Progetto v\VersioneNP{}}}
\newcommand{\PianoDiProgetto}{\emph{Piano di Progetto v\VersionePP{}}}
\newcommand{\StudioDiFattibilita}{\emph{Studio di Fattibilit� v\VersioneSF{}}}
\newcommand{\AnalisiDeiRequisiti}{\emph{Analisi dei Requisiti v\VersioneAR{}}}
\newcommand{\SpecificaTecnica}{\emph{Specifica Tecnica v\VersioneST{}}}
\newcommand{\ManualeAdmin}{\emph{Manuale Admin v\VersioneMA{}}}
\newcommand{\ManualeSviluppatore}{\emph{Manuale Sviluppatore v\VersioneMS{}}}
\newcommand{\ManualeUtente}{\emph{Manuale Utente v\VersioneMU{}}}
\newcommand{\DefinizioneDiProdotto}{\emph{Definizione di Prodotto v\VersioneDP{}}}

\newcommand{\ScopoDelProdotto}{
	Lo scopo del progetto � la realizzazione del progetto \glossaryItem{SWEDesigner}.}

%%%%%%%%%%%%%%
%  FUNZIONI  %
%%%%%%%%%%%%%%

% In questa seconda parte vanno definite le 'funzioni' utilizzate da due o pi� documenti.

% Serve a dare la giusta formattazione alle parole presenti nel glossario
% il nome del comando \glossary � gi� usato da LaTeX
\newcommand{\glossaryItem}[1]{\textit{#1\ped{\ped{G}}}}

% Serve a dare la giusta formattazione per indicare il tipo di verbale in cui e' stata presa una decisione
% Uso: \verbalRI{data}{punto}
% RI = Riunione Interna
\newcommand{\verbalRI}[2]{\textit{RI-#1-#2}}
% RE = Riunione Esterna
\newcommand{\verbalRE}[2]{\textit{RE-#1-#2}}

% Serve a dare la giusta formattazione al codice inline
\newcommand{\code}[1]{\flextt{\texttt{#1}}}

% Serve a dare la giusta formattazione a tutte le path presenti nei documenti
\newcommand{\file}[1]{\flextt{\texttt{#1}}}

% Permette di andare a capo all'interno di una cella in una tabella
\newcommand{\multiLineCell}[2][c]{\begin{tabular}[#1]{@{}l@{}}#2\end{tabular}}

% Genera automaticamente la pagina di copertina
\newcommand{\makeFrontPage}{
  % Declare new goemetry for the title page only.
  \newgeometry{top=3.5cm}
  
  \begin{titlepage}
  \begin{center}

  \begin{center}
  \includegraphics[width=10cm]{../../common/logo.jpg}
  \end{center}
  
  \vspace{1cm}

  \begin{Huge}
  \textbf{\DocTitle{}}
  \end{Huge}
  
  \textbf{\emph{Gruppo \GroupName{} \, \texttwelveudash{} \, Progetto \ProjectName{}}}
  
  \vspace{11pt}

  \bgroup
  \def\arraystretch{1.3}
  \begin{tabular}{ r|l }
    \multicolumn{2}{c}{\textbf{Informazioni sul documento}} \\
    \hline
		% differenzia a seconda che \DocVersion{} stampi testo o no
		\setbox0=\hbox{\DocVersion{}\unskip}\ifdim\wd0=0pt
			% nulla (non ho trovato come togliere l'a capo)
			\\
		\else
			\textbf{Versione} & \DocVersion{} \\
		\fi
    \textbf{Redazione} & \multiLineCell[t]{\DocRedazione{}} \\
    \textbf{Verifica} & \multiLineCell[t]{\DocVerifica{}} \\
    \textbf{Approvazione} & \multiLineCell[t]{\DocApprovazione{}} \\
    \textbf{Uso} & \DocUso{} \\
    \textbf{Distribuzione} & \multiLineCell[t]{\DocDistribuzione{}} \\
  \end{tabular}
  \egroup

  \vspace{22pt}

  \textbf{Descrizione} \\
  \DocDescription{}

  \end{center}
  \end{titlepage}
  
  % Ends the declared geometry for the titlepage
  \restoregeometry
}