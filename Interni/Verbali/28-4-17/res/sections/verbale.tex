\section{Riunione 24/04/2017}
  \subsection{Informazioni sulla riunione}
    \begin{itemize}
      \item \textbf{Data: }28/04/2017
      \item \textbf{Luogo: }Aula Torre Archimede
      \item \textbf{Ora: }10:00
      \item \textbf{Durata: }4h
      \item \textbf{Argomento: }Specifica Tecnica - Design Pattern
      \item \textbf{Partecipanti Interni: }Santimaria Davide - Massignan Fabio - Salmistraro Gianmarco - Bodian Malick - Pilò Salvatore - Bertolin Sebastiano;
      \item \textbf{Partecipanti Esterni: }/
    \end{itemize}
  \subsection{Decisioni prese}
		\begin{itemize}
			\item Abbiamo deciso come correggere (e corretto) il Piano di Progetto in modo che fosse conforme alle segnalazioni del Committente.
      \item Abbiamo deciso, viste le esigenze, di operare un'archittettura client-server disegnando una prima bozza della stessa specificando le varie interazioni
        fra client e server.
      \item Abbiamo disegnato una bozza del back-end cercando di capire in che modo le varie componenti comunicassero fra di loro per svolgere le varie operazioni.
      \item Abbiamo lavorato sui design pattern cercando di selezionare quelli utili per le nostre esigenze trovando nel Depency Injection una soluzione obbligatoria
        per via di Angular.js, nel Factory un valido aiuto per la creazione dei progetti e nell'Observer un'ottima struttura vista la libreria grafica.
      \item Abbiamo provato ad abbozzare il comportamento dell'applicazione incontrando qualche criticità che ci ha fatto ripensare ai pattern selezionati e alle tecnologie
        indicate.
      \item Per quanto rigurda la comunicazione con il database abbiamo deciso di introdurre Mongose nelle tecnologie in modo da ridurre gli errori nella popolazione del
        database.
      \item Abbiamo provato ad integrare la nuova tecnologia nella bozza architetturale disegnata incontrando qualche criticità nel suo collocamento.
		\end{itemize}
