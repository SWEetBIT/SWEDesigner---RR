\section{Riunione 24/04/2017}
  \subsection{Informazioni sulla riunione}
    \begin{itemize}
      \item \textbf{Data: }24/04/2017
      \item \textbf{Luogo: }Aula Torre Archimede
      \item \textbf{Ora: }09:30
      \item \textbf{Durata: }3h
      \item \textbf{Argomento: }Specifica Tecnica
      \item \textbf{Partecipanti Interni: }Santimaria Davide - Massignan Fabio - Salmistraro Gianmarco - Bodian Malick - Pilò Salvatore - Bertolin Sebastiano;
      \item \textbf{Partecipanti Esterni: }/
    \end{itemize}
  \subsection{Decisioni prese}
		\begin{itemize}
			\item Dopo aver valutato le funzioni del server
				e i pro e i contro fra TOMCAT e Node.js abbiamo optato per l'utilizzo di Node.js per il lato back-end per via delle motivazioni descritte nella Specifica Tecnica.
			\item Abbiamo ricercato tecnologie utili per il lato front-end dell'applicazione ed infine abbiamo optato
				per l'utilizzo di MEAN, uno stack comprendente MongoDB, Express.js, Angular.js e Node.js per via delle motivazioni descritte nella Specifica Tecnica.
			\item Dopo aver analizzato le scarse  proposte open-surce per quanto abbiamo scelto di
				utilizzare la libreria grafica JGraph per via delle motivazioni descritte nella Specifica Tecnica.
			\item Abbiam naalizzato diversi template engine compatibili con Node.js che potessero fare al caso nostro,
				quindi essere utili con la sintassi Java.
				Abbiamo optato per l'utilizzo di Mustache come template engine per via delle motivazioni descritte nella Specifica Tecnica.
			\item Abbiamo afrontato il problema della validità dei file importati dagli utenti dopo l'esportazione.
				Abbiamo deciso di criptare tali file per essere certi della loro validità e dopo molte discussioni in merito al metodo di criptazione
				abbiamo optato per l'utilizzo di una tabella di hash.
		\end{itemize}
