\section{Riunione 03/05/2017}
  \subsection{Informazioni sulla riunione}
    \begin{itemize}
      \item \textbf{Data: }03/05/2017
      \item \textbf{Luogo: }Aula Torre Archimede
      \item \textbf{Ora: }09:30
      \item \textbf{Durata: }4h
      \item \textbf{Argomento: }Specifica Tecnica - \glossaryItem{Design Pattern}
      \item \textbf{Partecipanti Interni: }Santimaria Davide - Massignan Fabio - Salmistraro Gianmarco - Bodian Malick - Pilò Salvatore - Bertolin Sebastiano;
      \item \textbf{Partecipanti Esterni: }/
    \end{itemize}
  \subsection{Decisioni prese}
		\begin{itemize}
			\item Abbiamo deciso, e scritto, la struttura della \versioneST.
      \item Abbiamo deciso di cambiare libreria grafica passando da mxGraph a Draw2D per via delle interazioni con il server: Draw2D ci permette un'esportazione più semplice
      nel formati \glossaryItem{SVG} e \glossaryItem{PNG} e permette la comunicazione con il server attraverso \glossaryItem{JSON} a differenza di mxGraph che generava degli xml da convertire.
      \item Abbiamo deciso tutti i \glossaryItem{design pattern} da utilizzare per la progettazione dell'\glossaryItem{applicazione}, sia per il \glossaryItem{front-end} che per il \glossaryItem{back-end}.
		\end{itemize}
