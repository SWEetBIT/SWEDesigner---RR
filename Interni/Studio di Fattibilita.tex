%Document-Author: Salvatore Pilò
%Document-Date: 2017-02-27
%Document-Description: Studio di fattibilità relativo al capitolato C6


\documentclass[a4paper]{report}
\usepackage[english, italian]{babel}
\usepackage[T1]{fontenc}
\usepackage[utf8]{inputenc}
\usepackage{url}
\usepackage{graphicx}
\graphicspath{{Immagini/}}
\usepackage[hidelinks]{hyperref}
\usepackage{lipsum}
\usepackage{booktabs}
\usepackage{tabularx}
\usepackage{pifont}
\usepackage[table]{xcolor}
\usepackage{float}
\usepackage{geometry}
\geometry{margin=1in}

\newcolumntype{s}{>{\hsize=.21\hsize}X}
\newcolumntype{f}{>{\hsize=.37\hsize}X}
\newcolumntype{m}{>{\hsize=.42\hsize}X}

\newcommand{\mychapter}[2]{
	\setcounter{chapter}{#1}
	\setcounter{section}{0}
	\setcounter{subsection}{1}
	\chapter*{#2}
	\addcontentsline{toc}{chapter}{#2}
}

\renewcommand{\abstractname}{Tabella contenuti}

\begin{document}

  \begin{titlepage}
    % Defines a new command for the horizontal lines, change thickness here
		\newcommand{\HRule}{\rule{\linewidth}{0.5mm}}
		\center

		% HEADING SECTION
		\textsc{\LARGE SWEetBIT}\\[1.5cm]
		\textsc{\Large SWEDesigner}\\[0.5cm]
		\textsc{\large  editor di diagrammi UML con generazione di codice}\\[0.5cm]

    % TITLE SECTION
		\HRule \\[0.4cm]
		{ \huge \bfseries Studio di fattibilità}\\[0.4cm]
		\HRule \\[1.5cm]

		% AUTHOR SECTION
		\begin{minipage}{0.4\textwidth}
			\begin{flushleft} \large
				\emph{Redattori:}\\
				Salvatore Pilò \\
			\end{flushleft}
		\end{minipage}
		~
		\begin{minipage}{0.4\textwidth}
			\begin{flushright} \large
				\emph{Approvazione:} \\
				NomeCognome \\
				\emph{Verifica:} \\
				NomeCognome \\
			\end{flushright}
		\end{minipage}

    %immagine
		\begin{figure}[H]
			\centering
			\includegraphics[scale=0.8]{sweet.png}
		\end{figure}
		\begin{center}
			Versione 1.0.0
		\end{center}
		% Date, change the \today to a set date if you want to be precise
		{\large \today}\\[3cm]
		% Fill the rest of the page with whitespace
		\vfill
	\end{titlepage}

  \tableofcontents

	\mychapter{0}{Diario delle modifiche}
		\begin{table}[H]
			\begin{tabularx}{\textwidth}{s f m X}
				\noalign{\hrule height 1.5pt}
				\rowcolor{orange!85} Versione & Data & Autore & Descrizione \\
				\noalign{\hrule height 1.5pt}
        1.0.0 & 2017-02-27 & \emph{Analista} Salvatore Pilò & Creazione scheletro del documento e stesura introduzione \\
				\noalign{\hrule height 1.5pt}
			\end{tabularx}
			\caption{Diario delle modifiche \label{tab:table_label}}
		\end{table}

    \mychapter{1}{Introduzione}
	\section{Scopo del documento}
          Lo scopo di questo documento è quello di descrivere le motivazioni dietro la scelta del capitolato C6, SWEDesigner, da parte del gruppo SWEtBIT.
  \section{Scopo del Prodotto}
          Lo scopo del progetto è la realizzazone di una Web App che fornisca all'utente un UML Designer con il quale riuscire a disegnare correttamente diagrammi delle classi
          e descrivere il comportamento dei metodi interni alle stsse attraverso l'utilizzo di -da decidere il tipo di schema-.
          La Web App permetterà all'utente di generare codice Java o Javascript dal diagramma disegnato ed eventualmente andare a ritoccarne il risultato al fine di ottenere un codice
          eseguibile, funzonante e funzionale.
  \section{Glossario}
		      Con lo scopo di evitare ambiguità di linguaggio e di massimizzare la comprensione dei documenti, il
          gruppo ha steso un documento interno che è il \emph{Glossario v1.0.0}. In esso saranno definiti, in modo
          chiaro e conciso i termini che possono causare ambiguità o incomprensione del testo.
          \section{Riferimenti}
        	\subsection{Informativi}
        		\begin{itemize}
        			\item \textbf{Capitolato d'appalto C1:} APIM: An API Market Platform \\
        			\url{http://www.math.unipd.it/~tullio/IS-1/2016/Progetto/C1.pdf}
        			\item \textbf{Capitolato d'appalto C2:} AtAVi: Accoglienza tramite Assistente Virtuale \\
        			\url{http://www.math.unipd.it/~tullio/IS-1/2016/Progetto/C2p.pdf}
        			\item \textbf{Capitolato d'appalto C3:} DeGeOP: A Designer and Geo-localizer Web App for
              Organizational Plants \\
        			\url{http://www.math.unipd.it/~tullio/IS-1/2016/Progetto/C3p.pdf}
        			\item \textbf{Capitolato d'appalto C4:} eBread: applicazione di lettura per dislessici \\
        			\url{http://www.math.unipd.it/~tullio/IS-1/2016/Progetto/C4p.pdf}
        			\item \textbf{Capitolato d'appalto C5:} Monolith: an interactive bubble provider \\
        			\url{http://www.math.unipd.it/~tullio/IS-1/201/Progetto/C5p.pdf}
        			\item \textbf{Capitolato d'appalto C6:} SWEDesigner: editor di diagrammi UML con generazione di codice \\
        			\url{http://www.math.unipd.it/~tullio/IS-1/2016/Progetto/C6p.pdf}
        		\end{itemize}
        	\subsection{Normativi}
        		\begin{itemize}
        			\item \textbf{Norme di progetto:} \emph{Norme di progetto v1.0.0}
        		\end{itemize}
\end{document}
