\section{Riunione 28/04/2017}
  \subsection{Informazioni sulla riunione}
    \begin{itemize}
      \item \textbf{Data: }28/04/2017
      \item \textbf{Luogo: }Aula Torre Archimede
      \item \textbf{Ora: }10:00
      \item \textbf{Durata: }4h
      \item \textbf{Argomento: }Specifica Tecnica - \glossaryItem{\glossaryItem{Design-Pattern}}
      \item \textbf{Partecipanti Interni: }Santimaria Davide - Massignan Fabio - Salmistraro Gianmarco - Bodian Malick - Pilò Salvatore - Bertolin Sebastiano;
      \item \textbf{Partecipanti Esterni: }/
    \end{itemize}
  \subsection{Decisioni prese}
		\begin{itemize}
			\item Abbiamo deciso come correggere (e corretto) il \versionePP in modo che fosse conforme alle segnalazioni del \glossaryItem{Committente}.
      \item Abbiamo deciso, viste le esigenze, di operare un'archittettura \glossaryItem{client}-\glossaryItem{server} disegnando una prima bozza della stessa specificando le varie interazioni
        fra \glossaryItem{client} e \glossaryItem{server}.
      \item Abbiamo disegnato una bozza del \glossaryItem{back-end} cercando di capire in che modo le varie componenti comunicassero fra di loro per svolgere le varie operazioni.
      \item Abbiamo lavorato sui \glossaryItem{Design-Pattern} cercando di selezionare quelli utili per le nostre esigenze trovando nel Depency Injection una soluzione obbligatoria
        per via di \glossaryItem{Angular} 4.0, nel Factory un valido aiuto per la creazione dei progetti e nell'Observer un'ottima struttura vista la libreria grafica.
      \item Abbiamo provato ad abbozzare il comportamento dell'\glossaryItem{applicazione} incontrando qualche criticità che ci ha fatto ripensare ai pattern selezionati e alle tecnologie
        indicate.
      \item Per quanto riguarda la comunicazione con il database abbiamo deciso di introdurre \glossaryItem{Mongoose} nelle tecnologie in modo da ridurre gli errori nella popolazione del
        database.
      \item Abbiamo provato ad integrare la nuova tecnologia nella bozza architetturale disegnata incontrando qualche criticità nel suo collocamento.
		\end{itemize}
