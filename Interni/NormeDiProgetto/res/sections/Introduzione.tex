\section{Introduzione}
  \subsection{Scopo del documento}
          In questo documento sono definite le norme che i membri del gruppo SWEet BIT adotteranno durante lo sviluppo del progetto \emph{SWEDesigner}.
          Tutti i membri sono tenuti a leggere e seguire le norme per migliorare l’uniformità del materiale prodotto, migliorare l’efficienza
          e ridurre il numero di errori.
          In particolare verranno definite norme riguardanti:
            \begin{itemize}
              \item Interazioni fra membri del gruppo;
              \item Stesura e convenzioni dei documenti;
              \item Modalità di lavoro durante le fasi di sviluppo del progetto;
              \item Ambiente di lavoro.
            \end{itemize}
  \subsection{Scopo del Prodotto}
          Lo scopo del progetto è la realizzazone di una \glossaryItem{Web App} che fornisca all'utente un \glossaryItem{UML} \glossaryItem{Designer} con il quale riuscire a disegnare correttamente diagrammi delle classi
          e descrivere il comportamento dei metodi interni alle stesse attraverso l'utilizzo di diagrammi delle attività.
          La \glossaryItem{Web App} permetterà all'utente di generare codice Java dall'insieme dei diagrammi delle classi e dei rispettivi metodi.
  \subsection{Glossario}
          Con lo scopo di evitare ambiguità di linguaggio e di massimizzare la comprensione dei documenti, il
          gruppo ha steso un documento interno che è il \emph{Glossario v1.0.0}. In esso saranno definiti, in modo
          chiaro e conciso i termini che possono causare ambiguità o incomprensione del testo.
  \subsection{Riferimenti}
    \subsubsection{Informativi}
      \begin{itemize}
        \item \textbf{Specifiche UTF-8:}\\
        \url{http://unicode.org/faq/utf_bom.html}
        \item \textbf{ISO 8601:2004:} \\
        \url{https://www.iso.org/standard/40874.html}\\
        \item \textbf{Licenza MIT:}\\
        \url{https://opensource.org/licenses/MIT}\\
        \item \textbf{GitHUB:}\\
        \url{https://github.com/}
        \item \textbf{UML:} \\
        \url{http://www.uml.org/}
        \item \textbf{Atom:}\\
        \url{https://atom.io/}
        \item \textbf{TexLive:}\\
        \url{https://www.tug.org/texlive/}
        \item \LaTeX\\
        \url{https://www.latex-project.org/}
        \item \textbf{Telegram:}\\
        \url{https://telegram.org/}
        \item INSERIRE ALTRI
        \item \textbf{Piano di progetto:} \emph{Piano di progetto v1.0.0}
        \item \textbf{Piano di qualifica:} \emph{Piano di qualifica v1.0.0}
      \end{itemize}
    \subsubsection{Normativi}
      \begin{itemize}
        \item \textbf{Capitolato di appalto SWEDesigner (C6):}\\
        \url{http://www.math.unipd.it/~tullio/IS-1/2016/Progetto/C6.pdf}
      \end{itemize}
