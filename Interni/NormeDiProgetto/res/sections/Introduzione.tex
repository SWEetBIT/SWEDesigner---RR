\section{Introduzione}
  \subsection{Scopo del documento}
          In questo documento sono definite le norme che i membri del gruppo SWEet BIT adotteranno durante lo sviluppo del progetto \emph{SWEDesigner}.
          Tutti i membri sono tenuti a leggere e seguire le norme per migliorare l’uniformità del materiale prodotto, migliorare l’efficienza
          e ridurre il numero di errori.
          In particolare verranno definite norme riguardanti:
            \begin{itemize}
              \item Interazioni fra membri del gruppo;
              \item Stesura e convenzioni dei documenti;
              \item Modalità di lavoro durante le fasi di sviluppo del progetto;
              \item Ambiente di lavoro.
            \end{itemize}
  \subsection{Struttura del documento}
    Il presente documento è diviso in tre sezioni principali:
    \begin{itemize}
      \item La sezione che regola i due procssi primari di:
      \begin{itemize}
      \item \textbf{fornitura}
      \item \textbf{sviluppo}
      \end{itemize}
      \item La sezione che regola i processi di supporto:
      \begin{itemize}
      \item \textbf{documentazione}
      \item \textbf{configurazione}
      \item \textbf{verifica}
      \item \textbf{validazione}
      \end{itemize}
      \item La sezione riguardante i processi organizzativi:
	  \begin{itemize}     
      \item \textbf{amministrazione}
      \end{itemize} 
    \end{itemize}
  \subsection{Scopo del Prodotto}
          Lo scopo del progetto è la realizzazone di una \glossaryItem{Web App} che fornisca all'\glossaryItem{Utente} un \glossaryItem{UML} \glossaryItem{Designer} con il quale riuscire a disegnare correttamente \glossaryItem{Diagrammi} delle \glossaryItem{Classi}
          e descrivere il comportamento dei \glossaryItem{Metodi} interni alle stesse attraverso l'utilizzo di \glossaryItem{Diagrammi} delle attività.
          La \glossaryItem{Web App} permetterà all'\glossaryItem{Utente} di generare \glossaryItem{Codice} \glossaryItem{Java} dall'insieme dei \glossaryItem{diagrammi classi} e dei rispettivi \glossaryItem{metodi}.
  \subsection{Glossario}
          Con lo scopo di evitare ambiguità di linguaggio e di massimizzare la comprensione dei documenti, il
          gruppo ha steso un documento interno che è il \emph{Glossario v}\VersioneG{}. In esso saranno definiti, in modo
          chiaro e conciso i termini che possono causare ambiguità o incomprensione del testo.
  \subsection{Riferimenti}
    \subsubsection{Informativi}
      \begin{itemize}
        \item \textbf{Specifiche \glossaryItem{UTF-8}:}\\
        \url{http://unicode.org/faq/utf_BOM.html}
        \item \textbf{ISO 8601:2004:} \\
        \url{https://www.iso.org/standard/40874.html}\\
        \item \textbf{Licenza MIT:}\\
        \url{https://opensource.org/licenses/MIT}
        \item \textbf{GitHUB:}\\
        \url{https://GitHub.com/}
        \item \textbf{UML:} \\
        \url{http://www.UML.org/}
        \item \textbf{Atom:}\\
        \url{https://atom.io/}
        \item \textbf{TexLive:}\\
        \url{https://www.tug.org/texlive/}
        \item \textbf{\glossaryItem{\LaTeX}:}\\
        \url{https://www.latex-project.org/}
        \item \textbf{Telegram:}\\
        \url{https://Telegram.org/}
        \item \textbf{Piano di progetto:} \emph{Piano di progetto v}\VersionePP{}
        \item \textbf{Piano di qualifica:} \emph{Piano di qualifica v}\VersionePQ{}
      \end{itemize}
    \subsubsection{Normativi}
      \begin{itemize}
        \item \textbf{Capitolato di appalto SWE\glossaryItem{Designer} (C6):}\\
        \url{http://www.math.unipd.it/~tullio/IS-1/2016/Progetto/C6.pdf}
      \end{itemize}
