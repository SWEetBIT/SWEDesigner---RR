\section{Processi Organizzativi}
  I processi normatizzati in questa sezione servono all'organizzazione che li istanzia a progetto, quindi servon al gruppo stesso.
  \subsection{Gestione di progetto}
    \subsubsection{Ruoli}
      Verranno assegnati dei ruoli corrispondendi a quelli professionali per lo sviluppo del progetto. Ogni membro del gruppo ricoprirà
      tutti i ruoli almeno una volta durante ogni periodo di sviluppo.\\
      Ogni attività di un processo verrà svolta solo ed esclusivamente dal membro del gruppo che ricopre il ruolo adatt all'attività rispettando la pianificazione
      specificata all'interno del Piano di Progetto.
        \paragraph{Responsabile di Progetto}
          Il Responsabile di Progetto detiene la resposabilità di tutto il team e potere decisionale su ogni attività. Si occuperà inoltre delle comunicazioni esterne
          ed avrà il pieno accesso alla mail del gruppo.
          In particolare è di responsabilità del Responsabile di Progetto:
          \begin{itemize}
            \item Coordinamento, pianificazione e controllo delle attività;
            \item Approvazione dei documenti;
            \item Comunicazioni esterne;
            \item Assegnazione compiti ai vari individui;
            \item Convocazione riuinioni interne/esterne.
          \end{itemize}
          Nello specifico i suoi compiti sono:
          \begin{itemize}
            \item Redazine Piano di Progetto;
            \item Collaborazione alla stesura del Piano di Qualifica;
            \item Assicurarsi che le attività svolte siano conformi alle Norme di Progetto;
            \item Garantire il rispetto dei ruoli;
            \item Assegnare e gestire task agli altri membri del gruppo;
            \item Approvare definitivamente i documenti.
          \end{itemize}
        \paragraph{Amministratore}
          L'Amministratore è responsabile dell'ambiente di lavoro.\\
          È sua responsabilità:
          \begin{itemize}
            \item Gestione ambiente di lavoro attrezzando il team di strumenti necessari;
            \item Aggiungere all’ambiente di lavoro strumenti di automazione del lavoro;
            \item Gestione del versionamento della documentazione;
            \item Controllo delle versioni del prodotto;
            \item Risoluzione dei problemi in merito alla gestione di risorse e processi.
          \end{itemize}
          L'Amministratore, inoltre, redige le Norme di Progetto, collabora alla stesura del Piano di Progetto e al Piano di Qualifica.
        \paragraph{Analista}
          L'Analista è il responsabile di tutte le analisi del problema.
          È sua responsabilità:
          \begin{itemize}
            \item Studiare a fondo la natura del prodotto;
            \item Classificare i requisiti;
            \item Redigere diagramma dei Casi d'Uso;
            \item Produrre una specifica di progetto precisa in ogni sui punto e comprensibile dal proponente, dal committente e dai progettisti;
            \item Redigere lo Studio di Fattibilità e l'Analisi dei Requisiti.
          \end{itemize}
        \paragraph{Progettista}
          Il Progettista è il responsabile delle attività di progettazione.
          È sua responsabilità:
          \begin{itemize}
            \item Effettuare scelte progettuali volte ad applicare al prodotto soluzionni ottimali;
            \item Effettuare scelte procedurali che ne garantiscano la manutenibilità;
            \item Produrre una soluzione soddisfacente per il Committente.
          \end{itemize}
          Il Progettista, inoltre, redige la Specifica Tecnica, la Definizione di Prodotto e collabora alla stesura del Piano di Qualifica.
        \paragraph{Programmatore}
          Il Programmatore è responsabile delle attività di codifica e delle commenti ausiliarie necessari per il processo di verifica.
          È sua responsabilità:
          \begin{itemize}
            \item Implementare le soluzioni descritte dal Progettista;
            \item Scrivere codice documentato che rispetti le metriche stabilite;
            \item Implementare i test da eseguire sul codice scritto durante i processi di verifica.
          \end{itemize}
          Il Programmatore ha, inoltre, il compito di redigere il Manuale Utente.
        \paragraph{Verificatore}
          Il verificatore è responsabile delle attività di verifica.
          È sua responsabilità:
          \begin{itemize}
            \item Controllare la conformità del prodotto ad ogni stadio del sup ciclo di vita;
            \item Garantire  che le attività seguano le norme stabilite.
          \end{itemize}
          Il Verificatre, inoltre, collabora alla stesura del Pian di Qualifica.
    \subsubsection{Pianificazione}
      \paragraph{Strumenti di Pianificazione}
        Per pianificare le attività legate allo sviluppo del progetto e la gestione delle risorse si è scelto di utilizzare \textbf{ProjectLibre}.\\
        Si tratta di un ottimo software \glossaryItem{Open-source} basato su \glossaryItem{Java} per il project management.\\
        La scelta è ricaduta su questo software principlamente per quattro motivi:
        \begin{itemize}
          \item Si tratta di un software portabile essendo basato su \glossaryItem{Java};
          \item È \glossaryItem{Open-source};
          \item Genera automaticamente digrammi di \glossaryItem{Gannt};
          \item Il salvataggio dei \glossaryItem{File} è in \glossaryItem{XML}, quindi un formato testuale che permette di utilizzare i \glossaryItem{Merge} senza causare troppi conflitti.
        \end{itemize}
      \paragraph{Ticketing}
        La piattaforma che è stata scelta per la gestione del progetto è \textbf{Redmine} che fornisce:
        \begin{itemize}
          \item Un sistema flessibile di gestione dei \glossaryItem{Ticket};
          \item Il grafico \glossaryItem{Gantt} delle attività;
          \item Un calendario per l'organizzazione e la distribuzione dei compiti;
          \item La visualizzazione della \glossaryItem{Repository} relativa al progetto;
          \item Un sistema di rendicontazione del tempo.
        \end{itemize}
        Sono state analizzate altre altenrative a \textbf{Redmine} che, dopo una fase di analisi iniziale, non sono risultate idonee allo scopo:
        \begin{itemize}
          \item \textbf{Teamworks: }si tratta del software probabilmente più adatto per il project management vista la sua grande versatilità e
          la strumentazione offerta. Purtroppo i suoi costi non hanno permesso un suo utilizzo in ambito universitario;
          \item \textbf{Zohoo: }A differenza di Redimine o di Teamworks, questa piattaforma non offre un servizio di rendicontazione del tempo
          e la generazione di grafici \glossaryItem{Gantt};
        \end{itemize}
        Tutti i digrammi di \glossaryItem{Gantt} presenti sono stati invece disegnati tramite il sotware \textbf{Ganttproject} sulla base dei dati inseriti e formiti da \textbf{Redmine}
        per permettere una stesura più corretta e una maggiore flessibilità nella stessa.
      \paragraph{Protocollo di Sviluppo}
        Per procedere con una stesura controllata dei documenti e con uno sviluppo controllato del \glossaryItem{Codice} si è scelto di adottare
        il sistema di \glossaryItem{Ticketing} \textbf{Redmine}.\\
        La scelta di tale piattaforma è spiegata all'interno del capitolo 9.
        In questa sezine si faranno molti riferimenti impliciti al \emph{Piano di Progetto} e al \emph{Piano di  Qualifica}.
        \paragraph{Creazione di un nuovo progetto}
          La creazione di un progetto è un compito del \emph{Responsabile di Progetto}.\\
          Un nuovo progetto è una macro-attività formata da molte sotto-attività coordinate da un responsabile.\\
          Per la creazione di un nuovo progetto la prassi da seguire è la seguente:
          \begin{itemize}
            \item Aprire \textbf{Progetti};
            \item Selezionare \textbf{Nuovo Progetto};
            \item Assegnare un \textbf{Nome} breve ma significativo;
            \item Nel caso in cui sia necessario creare un sotto-progetto, indicare il nome del progetto padre nell'omonimo campo;
            \item \textbf{Identificativo: }scrivere in minuscolo ed indicare il nome della fase a cui si riferisce (es. SdF-rr).
          \end{itemize}
        \paragraph{Creazione ticket}
          I \glossaryItem{Ticket} vengono creati da:
          \begin{itemize}
            \item \textbf{\emph{Responsabile di Progetto}: }crea i \glossaryItem{Ticket} più importanti che rappresentano le macro-fasi evideziate
            all'interno della pianificazione;
            \item \textbf{\emph{Responsabile di Sotto-progetto}: }crea i \glossaryItem{Ticket} per i processi non pianificati inizialmente ma che si rivelano necessari
            per l'avanzamento del sotto-progetto assegnato;
            \item \textbf{\emph{Verificatore}: }crea i \glossaryItem{Ticket} per segnalare errori emersi durante il processo di verifica.
          \end{itemize}
          I \glossaryItem{Ticket} possono essere di tre categorie:
          \begin{itemize}
            \item \textbf{Ticket di pianificazione: }rappresentano le macro-attività di maggiore importanza e sono organizzati in una gerarchia
            basata sul livello di importanza.\\
            Tali attività vengono create da:
            \bgroup
              \begin{itemize}
                \item \emph{Responsabile di Progetto} che durante la pianificazione individua le attività più importanti da svolgere;
                \item \emph{Responsabile di Sotto-progetto} che durante lo svolgimento dell'attività principale può scomporla in sotto-problemi.
              \end{itemize}
            \egroup
            \item \textbf{Ticket di realizzazione e controllo: }ogni documento, durante la sua stesura, passa attraverso due stadi:
            \bgroup
              \begin{itemize}
                \item \textbf{Realizzazione: }un redattore realizzerà la prima stesura dell'interno documento;
                \item \textbf{Controllo: }un redattore, diverso dal precedente, eseguirà un primo controllo di tutta la parte scritta.
              \end{itemize}
            \egroup
            \item \textbf{Ticket di verifica: }rappresentano gli errori evidenziati dai \emph{Verificatori} durante l'operazione di controllo dell'intero documento.
          \end{itemize}
            \subsubsection{Ticket di pianificazione}
              \begin{itemize}
                \item Selezionare \textbf{Nuova segnalazione} dal menù principale;
                \item \textbf{Tracker: }Indicare la natura del \glossaryItem{Ticket}:
                \bgroup
                  \begin{itemize}
                    \item \textbf{Documento: }Attività legata alla stesura di un documento;
                    \item \textbf{Codifica: }Attività legata alla  codifica del software;
                    \item \textbf{Verifica: }Macro-attività legata alla verifica del prodotto delle macro-attività.
                  \end{itemize}
                \egroup
                \item \textbf{Oggetto: }Descrizione breve e significativa della natura del \glossaryItem{Ticket};
                \item \textbf{Descrizione: }Descrizione comprensibile dell'attività da svolgere;
                \item \textbf{Stato: }Plan;
                \item \textbf{Attività principale: }Se si vuole identificare una \textbf{sotto-attività} indicare l'id del \glossaryItem{Ticket} padre;
                \item \textbf{Categoria: }\glossaryItem{PDCA} se e solo se il \glossaryItem{Ticket} viene generato dal \emph{Responsabile di Progetto};
                \item \textbf{Assegnato a: }Indicare il nome del responsabile;
                \item \textbf{Osservatori: }Aggiungere eventuali collaboratori.
              \end{itemize}
            \paragraph{Ticket di realizzazione e controllo}
              \begin{itemize}
                \item Selezionare \textbf{Nuova segnalazione} dal menù principale;
                \item \textbf{Tracker: }Indicare la natura del \glossaryItem{Ticket}:
                \bgroup
                  \begin{itemize}
                    \item \textbf{Documento: }Attività legata alla stesura di un documento;
                    \item \textbf{Codifica: }Attività legata alla  codifica del software;
                    \item \textbf{Verifica: }Macro-attività legata alla verifica del prodotto delle macro-attività;
                  \end{itemize}
                \egroup
                  \item \textbf{Oggetto: }Descrizione breve e significativa della natura del \glossaryItem{Ticket};
                \item \textbf{Descrizione: }Descrizione comprensibile dell'attività da svolgere;
                \item \textbf{Stato: }New;
                \item \textbf{Attività principale: }Se si vuole identificare una \textbf{sotto-attività} indicare l'id del \glossaryItem{Ticket} padre;
                \item \textbf{Inizio: }Dare una presunta data di inizio;
                \item \textbf{Scadenza: }Dare una presunta data di fine;
                \item \textbf{Assegnato a: }Indicare il nome del responsabile;
                \item \textbf{Osservatori: }Aggiungere eventuali collaboratori.
              \end{itemize}
            \paragraph{Ticket di verifica}
              Un \emph{Verificatore} per creare un \emph{ticket di verifica} deve:
              \begin{enumerate}
                \item Assicurarsi che esista all'interno del progetto l'attività \emph{Verifica}.\\
                Su questa attività devono essere presenti due sotto-attività: \emph{Verifica - realizzazione} e \emph{Verifica - approvazione}.\\
                Tutti i \glossaryItem{Ticket} devono essere creati come sotto-attività di \emph{Verifica - realizzazione}.
                \item Creare il \glossaryItem{Ticket} seguendo le seguenti direttive:
                \bgroup
                  \begin{itemize}
                    \item Selezionare \textbf{Nuova segnalazione} dal menù principale;
                    \item \textbf{Tracker: }\glossaryItem{Bug};
                    \item \textbf{Oggetto: }Descrizione breve e significativa della natura del \glossaryItem{Ticket};
                    \item \textbf{Descrizione: }Descrizione comprensibile dell'attività da svolgere;
                    \item \textbf{Stato: }New;
                    \item \textbf{Attività principale: }Se si vuole identificare una \textbf{sotto-attività} indicare l'id del \glossaryItem{Ticket} padre;
                    \item \textbf{Assegnato a: }Indicare il nome del responsabile.
                  \end{itemize}
                \egroup
              \end{enumerate}
            Tutti i campi non segnalati sono da lasciarsi vuoti.\\
            Il compito di assegnare la correzione dell'errore è dato al Responsabile del progetto padre.
          \paragraph{Dipendenze temporali}
            Dopo la creazione dei \glossaryItem{Ticket} è necessario assegnare le \textbf{dipendenze temporali} fra gli stessi.\\
            La procedura da seguire è la seguente:
            \begin{itemize}
              \item Spostarsi su \textbf{Segnalazioni};
              \item Aprire il link alla segnalazione a cui aggiungere la dipendenza;
              \item Nella sezione \textbf{Segnalazioni correlate} premere su \textbf{Aggiungi};
              \item Scegliere \textbf{segue} ed indicare il numero della segnalazione bloccante con eventuali giorni di \glossaryItem{Slack}.
            \end{itemize}
            Tutti i campi non segnalati sono da lasciarsi vuoti.
        \paragraph{Aggiornamento ticket}
          L'aggiornamento dei \glossaryItem{Ticket} avviene tramite il cambiamento del loro stato da:
          \begin{itemize}
            \item \textbf{In Progress: }segnala che uno o più membri del gruppo stanno lavorando al completamento di quel \glossaryItem{Ticket}.\\
            In questo caso la percentuale di completamento deve essere compresa fra 0\% e 99\%;
            \item \textbf{Closed: }segnala che l'attività è stata conclusa.\\
            La percentuale di completamento in questo caso è fissata a 100\%;
          \end{itemize}
    \subsection{Riunioni}
      \subsubsection{Frequenza}
        Tutte le riunioni interne, salvo casi eccezionali, si svolgeranno settimanalmente mentre tutte quelle esterne saranno convocate solo quando sorgerà la necessità di contattare il \glossaryItem{Proponente} o con il \glossaryItem{Committente}.
        \paragraph{Interna}
          Il \emph{Responsabile di Progetto} ha il compito di convocare le riunioni generali valutando, di volta in volta, la possibilità di anticipare o posticipare la data
          designata per la riunione del gruppo.\\
          Qualora un membro del gruppo lo ritenesse necessario potrà fare richiesta, attraverso il gruppo \textbf{\glossaryItem{Telegram}}, di una riunione extra.\\
          È auspicabile, infine, che diversi membri del gruppo possano organizzarsi fra di loro per svolgere alcuni compiti che non richiedono la presenza  del gruppo di lavoro
          al completo. Ad esempio, è interessante e utile la collaborazione fra \emph{Progettista} e \emph{Analista}, senza che vengano coinvolte altre persone esterne ai compiti
          da loro svolti.\\
          Il responsabile avviserà tutti i membri del gruppo attraverso un messaggio sul gruppo \textbf{\glossaryItem{Telegram}} che verrà fissato in alto e conterrà luogo, data ed ora della riunione.\\
          Ogni membro del gruppo è tenuto a confermare o meno la sua presenza nelle 24h successive. Il responsabile è tenuto ad avvertire telefonicamente tutti i membri che
          non hanno ancora risposto al messaggio.\\
          Ogni cambiamento nell'orario di convocazione deve essere comunicato per tempo dal responsabile attraverso le modalità sopra elencate.
        \paragraph{Esterna}
          Concordando con gli altri membri del gruppo la necessità di effettuare una riuinione con il \glossaryItem{Proponente} o con il \glossaryItem{Committente}, il \emph{Responsabile di Progetto}
          si metterà in contatto con i diretti interessati per fissare una data consona a tutti.
        \paragraph{Svolgimento riunione}
          All’apertura della riunione, verificata la presenza dei membri previsti, viene scelto un segretario che avrà il compito di annotare ogni argomento trattato e di redigere il verbale
          dell’assemblea, che dovrà poi essere inviato ai restanti componenti del gruppo.\\
          Tutti i partecipanti devono osservare un comportamento consono al miglior svolgimento della riunione e al raggiungimento degli obbiettivi della stessa. Il segretario deve inoltre
          controllare che venga seguito l’ordine del giorno in modo da non tralasciare alcun punto.
        \subsubsection{Verbale}
          \paragraph{Riunione interna}
            Il verbale di una riunione interna è un documento informale che traccia semplicemente tutti gli argomenti trattati all'interno della riunione.\\
            Verrà redatto dal segretario della riuione, ruolo scelto di volta in volta e a rotazione fra i presenti, e dovrà essere condiviso attraverso il gruppo \textbf{\glossaryItem{Telegram}}
            per essere a disposizione, in qualsiasi momento (\textbf{\glossaryItem{Telegram}} offre la possibilità di tracciare istantaneamente i media condivisi) da ogni membro del gruppo.\\
            Il verbale dovrà essere inoltre inviato via e-mail ad ogni componente del gruppo il quale avrà cura di mantenerlo localmente al fine di avere sempre a disposizione
            gli argomenti trattati nel corso di una riunione.
          \paragraph{Esterna}
            Il verbale generato da una riunione esterna con il \glossaryItem{Proponente} o il \glossaryItem{Committente} è un documento ufficiale che può assumere il valore di normativo, quindi deve essere redatto
            seguendo dei criteri specifici.\\
            Per agevolarne la scrittura è stato creato un \glossaryItem{Template} con \glossaryItem{\LaTeX}\ che ne definisce la struttura.
            Vi è, ovviamente, l'obbligo di seguire tale schema per la stesura del verbale di una riunione esterna che dovrà poi essere inviato a tutti i membri del gruppo seguendo
            le stesse regole del verbale per una riunione interna.
    \subsection{Comunicazioni}
      \subsubsection{Comunicazioni esterne}
        Per le comunicazioni esterne è stata creata un'apposita casella di posta elettronica:\\
          \begin{center}
            \href{mailto:sweet.bit.group@gmail.com}{sweet.bit.group@gmail.com}
          \end{center}
        Tale indirizzo deve essere l’unico canale di comunicazione esistente tra il gruppo di lavoro e l’esterno.\\
        L'unico membro del gruppo ad avere accesso alla mail, e quindi alle comunicazioni con il \glossaryItem{Committente}, è il \emph{Responsabile di Progetto}.
        Suo anche il compito di informare i membri del gruppo delle discussioni avvenute tramite la casella di posta eletronica e inoltrare, qualora dovesse ritenerlo necessario,
        tali discussioni alle caselle postali dei vari membri del gruppo.
      \subsection{Comunicazioni interne}
        Per tutte le comunicazioni interne è stato creato un gruppo \textbf{\glossaryItem{Telegram}} in cui ogni membro può comunicare i propri impegni e il proprio stato dei lavori.\\
        Il gruppo serve, inoltre, per la comunicazione delle date delle riuinioni, per comunicare tempestivamente le decisioni prese e per scambiare materiale utile al progetto.
      \subsection{Composizione e-mail}
        \subsubsection{Destinatario}
          Il destinatario delle mail esterne può variare a seconda che ci si debba riferire al \glossaryItem{Proponente} del progetto, al Prof. Vardanega Tullio o al Prof. Cardin Riccardo.
        \subsubsection{Mittente}
          L’unico indirizzo utilizzabile è \href{mailto:sweet.bit.group@gmail.com}{sweet.bit.group@gmail.com} e deve essere usato solamente dal \emph{Responsabile di Progetto}.
        \subsubsection{Oggetto}
          L'oggetto di una comunicazione deve essere chiaro, stringato e possibilmente diverso da altri oggetti preesistenti.\\
          Nel caso in cui si dovesse comporre un messaggio di risposta, vi è l'obbligo di utilizzare la particella "RE:" prima dell'oggetto in modo tale da identificare
          il livello di risposta; Qualora si trattasse invece di un inoltro è obbligatorio utilizzare la particella "I:".\\
          In ogni caso l'oggetto di una comunicazione già avviata non potrà essere cambiato.
        \subsubsection{Corpo}
          Il corpo di un messaggio deve contenere tutte le informazioni necessarie a rendere facilmente comprensibile l’argomento trattato a tutti i destinatari. Se alcune parti del messaggio hanno uno o più destinatari specifici, il loro nome dovrà essere aggiunto all’inizio del paragrafo tramite la seguente segnatura: \emph{@destinatario}.
          In caso di risposta od inoltro del messaggio, il contenuto aggiunto deve essere sempre messo in testa (per non costringere gli altri membri a dover scorrere tutta la mail).
          È caldamente consigliato, inoltre, evitare la cancellazione delle precedenti parti del messaggio in modo tale da rendere ogni partecipante alla discussione consapevole del contesto.
        \subsubsection{Allegati}
          L'uso di allegati è permesso qualora si ritenga necessario.\\
          Un esempio è il resoconto di un incontro con il \glossaryItem{Proponente} o il \glossaryItem{Committente}.
