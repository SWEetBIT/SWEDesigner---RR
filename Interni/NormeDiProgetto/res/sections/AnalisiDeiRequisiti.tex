\section{Analisi dei requisiti}
  \subsection{Studio di Fattibilità e Analisi dei Rischi}
    Alla pubblicazione dei capitolati è compito del \emph{Responsabile di Progetto} convocare un numero di riunioni tale da consentire
    al gruppo un confronto su tutti i capitolati disponibili.\\
    Tali riunioni saranno d'aiuto agli \emph{Analisti} per farsi un'idea delle conoscenze e preferenze di ogni membro del gruppo così da
    poter redigere uno \emph{Studio di Fattibilità} dei capitolati disponibili basandosi su:\\
    \begin{itemize}
      \item \textbf{Dominio tecnologico e applicativo: }Conoscenza delle tecnologie richieste, esperienze precedenti con le problematiche poste dal capitolato, conoscenza del
      dominio applicativo;
      \item \textbf{Rapporto Costi/Benefici: }Competitori e prodotti simili già presenti sul mercato, quantità di requisiti obbligatori, costo della realizzazione rapportato al
      risultato previsto;
      \item \textbf{Individuazione dei rischi: }Comprensione dei punti critici della realizzazione, individuazione di eventuali lacune tecniche o di conoscenza del dominio applicativo
      dei membri del gruppo, analisi delle difficoltà nell’individuazione dei requisiti e loro verificabilità.
    \end{itemize}
    Un'ulteriore riunione, a \emph{Studio di Fattibilità} concluso, determinerà la scelta del capitolato.
  \subsection{Analisi dei requisiti}
    La stesura del documento di \emph{Analisi dei Requisiti} è compito degli \emph{Analisti} e si divide nelle fasi di seguito riportante.\\
    Il documento dovrà seguire, inoltre, le norme specificate di seguito.
    \subsubsection{Classificazione dei requisiti}
      È compito degli \emph{Analisti} stilare una lista dei requisiti emersi dal capitolato e da eventuali riunioni con il Proponente. Questi dovranno essere
      classificati per tipo e per importanza utilizzando la seguente codifica:
      \begin{center}
        R[importanza][tipo][codice]
      \end{center}
      \begin{itemize}
        \item \textbf{Importanza} può assumere i seguenti valori:
          \bgroup
            \begin{labeling}{}
              \item [0]: Requisito obbligatorio;
              \item [1]: Requisito desiderabile;
              \item [2]: Requisito opzionale.
            \end{labeling}
          \egroup
        \item \textbf{Tipo} può assumere i seguenti valori:
          \bgroup
            \begin{labeling}{}
              \item [F]: Funzionale;
              \item [Q]: Di Qualità;
              \item [P]: Prestazionale;
              \item [V]: Vincolo.
            \end{labeling}
          \egroup
        \item \textbf{Codice} è il codice univoco di ogni requisito espresso in modo gerarchico.
      \end{itemize}
      Ogni requisito è poi esplicato nel seguente modo:
      \begin{itemize}
        \item \textbf{Relazioni} di dipendenza con altri requisiti;
        \item \textbf{Descrizione} sintetica del requisito.
      \end{itemize}
    \subsubsection{Modellazione concettuale del sistema e Allocazione}
      Successivamente al riconoscimento e definizione del requisiti emersi dal capitolato si procede all'analisi dei casi d'uso, denominati anche come
      \emph{use case} o con l'acronimo UC.\\
      È richiesta agli analisti l'identificazione dei vari casi d'uso, procedendo dal generale al particolare che verranno inseriti nel software di tracciamento
      \glossaryItem{Trender}.\\
      Per ogni UC è richiesto l'inserimento, all'interno del software, di:
      \begin{itemize}
        \item \textbf{Titolo: }Nome del Caso d'Uso;
        \item \textbf{Descrizione: }Descrizione breve e coincisa dell'UC;
        \item \textbf{Precondizione: }Condizione d'accesso al Caso d'Uso;
        \item \textbf{Postcondizione: }Condizione d'uscita al Caso d'Uso.
      \end{itemize}
      Gli altri campi da compilare, da non ritenere obbligatori ma desiderabili, sono:
      \begin{itemize}
        \item \textbf{Padre: }Indicare codice univoco del Caso d'Uso padre;
        \item \textbf{Tipo: }Può essere di tre tipi differenti:
        \bgroup
          \begin{itemize}
            \item \textbf{Inclusione};
            \item \textbf{Estensione};
            \item \textbf{Gerarchia}.
          \end{itemize}
        \egroup
        \item \textbf{Scenario: }Descrizione dello scenario rappresentato;
        \item \textbf{Scenario alternativo: }Descrizione scenari alternativi, se presenti;
        \item \textbf{Percorso immagine: }Il percorso dell'immagine rappresentate l'UML;
        \item \textbf{Descrizione immagine: }Descrizione breve e sommaria dell'immagine.
      \end{itemize}
      Una volta creato l'UC, selezionarlo dalla lista per:
      \begin{itemize}
        \item Modificare i campi dati inseriti;
        \item Osservare i figli dell'UC corrente;
        \item Associare un \textbf{Attore} selezionandolo dal menù a tendina;
        \item Associare un \textbf{Requisito}.
      \end{itemize}
      Il caso d'uso dovrà essere accompagnato da un grafico riassuntivo in \glossaryItem{UML}2.x, titolato come il caso d'uso in questione.\\
      È compito del software di tracciamento tracciare gli UC con un codice univoco e gerarchico nella forma:
      \begin{center}
        UC[codice univoco del padre].[codice univoco del figlio]
      \end{center}
      Il software provvederà a generare anche i file \LaTeX\ corrispondenti.
