\section{Processi primari}
	\subsection{Fornitura}
		\subsubsection{Studio di fattibilità}
		Alla pubblicazione dei \glossaryItem{capitolati} è compito del \emph{Responsabile di Progetto} convocare un numero di riunioni tale da consentire
    al gruppo un confronto su tutti i \glossaryItem{capitolati} disponibili.\\
    Tali riunioni saranno d'aiuto agli \emph{Analisti} per farsi un'idea delle conoscenze e preferenze di ogni membro del gruppo così da
    poter redigere uno \emph{Studio di Fattibilità} dei \glossaryItem{capitolati} disponibili basandosi su:\\
    \begin{itemize}
      \item \textbf{Dominio tecnologico e applicativo: }Conoscenza delle tecnologie richieste, esperienze precedenti con le problematiche poste dal \glossaryItem{capitolato}, conoscenza del
      \glossaryItem{\glossaryItem{Dominio}} applicativo;
      \item \textbf{Rapporto Costi/Benefici: }Competitori e prodotti simili già presenti sul mercato, quantità di requisiti obbligatori, costo della realizzazione rapportato al
      risultato previsto;
      \item \textbf{Individuazione dei rischi: }Comprensione dei punti critici della realizzazione, individuazione di eventuali lacune tecniche o di conoscenza del \glossaryItem{Dominio} applicativo
      dei membri del gruppo, analisi delle difficoltà nell’individuazione dei requisiti e loro verificabilità.
    \end{itemize}
    Un'ulteriore riunione, a \emph{Studio di Fattibilità} concluso, determinerà la scelta del \glossaryItem{capitolato}.
	\subsection{Sviluppo}
		\subsubsection{Analisi dei requisiti}
		 La stesura del documento di \emph{Analisi dei Requisiti} è compito degli \emph{Analisti} e si divide nelle fasi di seguito riportante.\\
		 Per semplificare il tracciamento degli \glossaryItem{Use Case} e dei Requisiti si è scelto di utilizzare \textbf{Trender}, uno strumento \glossaryItem{Open-source} che permette
        di gestire al meglio entrambi gli elementi e le realizioni fra gli stessi.\\
        Il tool permette l'esportazione di questi direttamente in \glossaryItem{\LaTeX}\, così da semplificare la scrittura degli stessi all'interno della documentazione.
    Il documento dovrà seguire, inoltre, le norme specificate di seguito.
    \paragraph{Classificazione dei requisiti}
      È compito degli \emph{Analisti} stilare una lista dei requisiti emersi dal \glossaryItem{capitolato} e da eventuali riunioni con il \glossaryItem{Proponente}. Questi dovranno essere
      classificati per tipo e per importanza utilizzando la seguente codifica:
      \begin{center}
        R[importanza][tipo][codice]
      \end{center}
      \begin{itemize}
        \item \textbf{Importanza} può assumere i seguenti valori:
          \bgroup
            \begin{labeling}{}
              \item [0]: Requisito obbligatorio;
              \item [1]: Requisito desiderabile;
              \item [2]: Requisito opzionale.
            \end{labeling}
          \egroup
        \item \textbf{Tipo} può assumere i seguenti valori:
          \bgroup
            \begin{labeling}{}
              \item [F]: Funzionale;
              \item [Q]: Di Qualità;
              \item [P]: Prestazionale;
              \item [V]: Vincolo.
            \end{labeling}
          \egroup
        \item \textbf{Codice} è il codice univoco di ogni requisito espresso in modo gerarchico.
      \end{itemize}
      Ogni requisito è poi esplicato nel seguente modo:
      \begin{itemize}
        \item \textbf{Relazioni} di dipendenza con altri requisiti;
        \item \textbf{Descrizione} sintetica del requisito.
      \end{itemize}
    \paragraph{Modellazione concettuale del sistema e Allocazione}
      Successivamente al riconoscimento e definizione del requisiti emersi dal \glossaryItem{capitolato} si procede all'analisi dei casi d'uso, denominati anche come
      \glossaryItem{Use case} o con l'acronimo UC.\\
      È richiesta agli analisti l'identificazione dei vari casi d'uso, procedendo dal generale al particolare che verranno inseriti nel software di tracciamento
      \textbf{\glossaryItem{Trender}}.\\
      Per ogni UC è richiesto l'inserimento, all'interno del software, di:
      \begin{itemize}
        \item \textbf{Titolo: }Nome del Caso d'Uso;
        \item \textbf{Descrizione: }Descrizione breve e coincisa dell'UC;
        \item \textbf{Precondizione: }Condizione d'accesso al Caso d'Uso;
        \item \textbf{Postcondizione: }Condizione d'uscita al Caso d'Uso.
      \end{itemize}
      Gli altri campi da compilare, da non ritenere obbligatori ma desiderabili, sono:
      \begin{itemize}
        \item \textbf{Padre: }Indicare codice univoco del Caso d'Uso padre;
        \item \textbf{Tipo: }Può essere di tre tipi differenti:
        \bgroup
          \begin{itemize}
            \item \textbf{Inclusione};
            \item \textbf{Estensione};
            \item \textbf{Gerarchia}.
          \end{itemize}
        \egroup
        \item \textbf{Scenario: }Descrizione dello scenario rappresentato;
        \item \textbf{Scenario alternativo: }Descrizione scenari alternativi, se presenti;
        \item \textbf{Percorso immagine: }Il percorso dell'immagine rappresentate l'\glossaryItem{UML};
        \item \textbf{Descrizione immagine: }Descrizione breve e sommaria dell'immagine.
      \end{itemize}
      Una volta creato l'UC, selezionarlo dalla lista per:
      \begin{itemize}
        \item Modificare i campi dati inseriti;
        \item Osservare i figli dell'UC corrente;
        \item Associare un \textbf{Attore} selezionandolo dal menù a tendina;
        \item Associare un \textbf{Requisito}.
      \end{itemize}
      Il caso d'uso dovrà essere accompagnato da un grafico riassuntivo in \glossaryItem{UML}2.x, titolato come il caso d'uso in questione.\\
      È compito del software di tracciamento tracciare gli UC con un \glossaryItem{Codice} univoco e gerarchico nella forma:
      \begin{center}
        UC[codice univoco del padre].[codice univoco del figlio]
      \end{center}
      Il software provvederà a generare anche i \glossaryItem{File} \glossaryItem{LaTeX} corrispondenti.
\subsubsection{Progettazione}
	\paragraph{Attività progettuali} Il lavoro dei progettisti consiste nel progettare
struttura e comportamento del software in modo da implementare tutti i requisiti
obbligatori individuati dagli analisti; inoltre, i progettisti dovranno prevedere
anche il massimo della copertura dei requisiti desiderabili. Nello specifico, il lavoro dei progettisti si articola in due
attività:
\begin{itemize}
\item \textbf{Progettazione architetturale} (descritta in §5.3.5 di ISO/IEC 12207:1995): definire l’architettura del sistema, cioè individuare e descrivere le componenti
del sistema, esplicitandone ruoli e relazioni tra essi; ogni progettista
documenta i prodotti di questa attività nel documento di \emph{Specifica Tecnica}.
\item \textbf{Progettazione di dettaglio} (descritta in §5.3.6 di ISO/IEC 12207:1995): definire, in dettaglio, il comportamento delle componenti del sistema;
ogni progettista documenta i prodotti di questa attività nel documento di
\emph{Definizione di Prodotto}.
\end{itemize}
Per ognuna di queste due attività, ogni progettista deve definire dei test che
permettano la verifica del sistema:
\begin{itemize}
\item durante la progettazione architetturale, i progettisti definiscono i test di
integrazione, cioè dei test che verificano il corretto interfacciarsi delle
componenti del sistema;
\item durante la progettazione di dettaglio, la specifica del comportamento di
una componente dev’essere accompagnata dalla definizione di un test di
unità che verifichi il corretto funzionamento della componente isolata.

\end{itemize}
\paragraph{Diagrammi} Per descrivere l’architettura e il comportamento del
sistema i progettisti utilizzano i diagrammi \glossaryItem{UML} descritti qui di seguito, seguendo
lo standard \glossaryItem{UML} 2.0:
\begin{itemize}
\item\textbf{ Diagramma dei package: }raggruppa un numero di elementi \glossaryItem{UML} in
una sola unità di livello più alto.
\item \textbf{Diagramma delle classi: }descrive le interfacce delle componenti del
sistema e le relazioni tra esse.
\item \textbf{Diagramma delle attività: }descrive i passi di una procedura.
\item \textbf{Diagramma di sequenza: }descrive uno scenario, dove le azioni sono disposte
in sequenza e le varie scelte sono già state prese.
\end{itemize}
Per disegnare i diagrammi appena esposti, i progettisti devono usare il software
gratuito Astah ed esportare i diagrammi in formato \glossaryItem{PNG}. Astah supporta la
generazione di tutti i diagrammi che vengono usati dal gruppo (classi, attività,
sequenza e package). Le alternative analizzate sono state \emph{Dia}, \emph{LuchiChart} e \emph{Papyrus} che si sono rivelati però troppo deboli in confronto ad \textbf{Astah} che offre
      diverse funzionalità aggiuntive raggruppandone diverse dei tre software citati al suo interno.

\paragraph{Stile di progettazione} L’architettura di SWEDesigner segue lo
stile dell’orientazione agli oggetti; perciò, le componenti software descritte
nei diagrammi delle classi sono:
\begin{itemize}
\item prototipi di oggetti \glossaryItem{JavaScript};
\end{itemize}

Al fine di aumentare la manutenibilità i progettisti devono seguire le seguenti regole:
\begin{itemize}
\item La progettazione dovrà usare quanto più possibile \glossaryItem{design pattern};
\item L'architettura deve rispettare le regole di progettazione che ne garantiscono la correttezza rispetto allo standard \glossaryItem{UML};
\item Suddividere il progetto in \glossaryItem{moduli}, in accordo con lo stile di progettazione dell'ambiente \glossaryItem{Node.js};
\item ogni componente deve avere un compito ben precisio, ricavabile dal proprio nome e dai metodi che costituiscono l'interfaccia;
\end{itemize}

\paragraph{Nomenclatura}
I progettisti devono osservare le seguenti norme, al fine di uniformare il codice che verrà derivato dalla progettazione:
\begin{itemize}
\item le componenti vanno raggruppate in \glossaryItem{moduli}. Il nome di tale \glossaryItem{modulo} deve:
\begin{itemize}
\item contenere solo caratteri alfanumerici;
\item essere in lingua inglese;
\item avere il nome del \glossaryItem{modulo} "padre" che lo contiene preposto al nome del metodo, seguito da una coppia di due punti (::);
\item essere rappresentativo del ruolo che rappresenta.
\end{itemize}
\item ogni nome di componente di un \glossaryItem{modulo} deve:
\begin{itemize}
\item contenere solo caratteri alfanumerici;
\item essere in lingua inglese;
\item se composto da più termini, l'iniziale di ogni termine deve essere maiuscola;
\item non può essere un termine riservato del linguaggio \glossaryItem{JavaScript}.
\end{itemize}
\item ogni nome di funzione deve:
\begin{itemize}
\item contenere solo caratteri alfanumerici;
\item essere in lingua inglese;
\item se composto da più termini, l'iniziale di ogni termine deve essere maiuscola;
\item essere rappresentativo per il servizio che offre;
\item non può essere un termine riservato del linguaggio \glossaryItem{JavaScript}.
\end{itemize}
\end{itemize}

\paragraph{Specifica Tecnica}
I Progettisti devono descrivere la progettazione ad alto livello dell’architettura dell’applicazione e dei singoli componenti nella “Specifica Tecnica \VersioneST ”. Devono inoltre
provvedere alla progettazione di opportuni test di integrazione.
\begin{itemize}
\item \emph{Diagrammi \glossaryItem{UML}} \\
Devono essere realizzati i seguenti diagrammi:
\begin{itemize}
\item diagrammi delle classi;
\item diagrammi dei \glossaryItem{package};
\item diagrammi di attività;
\item diagrammi di sequenza.
\end{itemize}

\item \emph{\glossaryItem{Design pattern}}\\
I Progettisti devono descrivere i design pattern utilizzati per realizzare l’architettura.
Di tali design pattern, si deve includere una breve descrizione e un diagramma
che ne esemplifichi il funzionamento e la struttura.
\item \emph{Tracciamento componenti}\\
Ogni requisito deve essere tracciato al componente che lo soddisfa. L’applicazione
web \glossaryItem{Trender} genera automaticamente le tabelle di tracciamento.
In questo modo sarà possibile garantire che ogni requisito venga soddisfatto e,
al tempo stesso, misurare il progresso nell’attività di progettazione.
\item \emph{Test di integrazione}\\
I Progettisti devono definire delle classi di verifica. Tali classi sono necessarie
per verificare che i componenti del sistema funzionino nella maniera prevista.

\end{itemize}

\paragraph{Definizione di Prodotto}
I Progettisti devono produrre la “Definizione di Prodotto”. In essa viene descritta la progettazione di dettaglio del sistema, ampliando quanto scritto nella “Specifica Tecnica \VersioneST ”.
\begin{itemize}
\item \emph{Diagrammi UML}\\
Devono essere redatti i seguenti diagrammi:
\begin{itemize}
\item diagrammi delle classi;
\item diagrammi di attività;
\item diagrammi di sequenza.
\end{itemize}

\item \emph{Definizione di classe}\\
Ogni classe progettata deve essere descritta all’interno della “Definizione di Prodotto”. Tale descrizione deve comprendere una spiegazione sullo scopo della classe e deve specificare quale funzionalità essa modella. Nella descrizione devono inoltre essere presenti l’elenco di metodi e attributi della classe.
\item \emph{Tracciamento classi}\\
Ogni requisito deve essere tracciato alle classi che lo soddisfano. Il software
Trender genera in automatico le tabelle di tracciamento. In questo modo sarà possibile misurare il progresso nell’attività di progettazione e garantire che ogni classe soddisfi almeno un requisito.
\item \emph{Test di unità}\\
I Progettisti devono definire i test di unità necessari per verificare che i componenti del sistema funzionino nel modo previsto.
\end{itemize}


\subsubsection{Codifica dei file e documentazione}
   \paragraph{Codifica e convenzioni}
   Di seguito vengono elencate le convenzioni stabilite; è contemplata la modifica delle stesse, previa autorizzazione del \emph{Project Manager};
\begin{itemize}
\item Tutti i file contenenti codice o documentazione dovranno essere in codifica \glossaryItem{UTF-8} senza \glossaryItem{BOM};
\item È ammessa la possibilità di effettuare modifiche alle convenzioni stabilite in seguito ad una decisione del \emph{Responsabile di Progetto};
\item L'unica lingua ammessa per i nomi di variabili, classi e funzioni è l'inglese;
\item I commenti devono essere fatti in italiano;
\item I nomi delle classi devono avere l’iniziale maiuscola;
\item I nomi di variabili e metodi devono avere la prima lettera minuscola, mentre devono essere maiuscole le iniziali delle eventuali altre parole che compongono il nome stesso;
\item Identazione effettuata tramite tabs; 
\item I comandi if / else / for / while / try vengono sempre seguiti le parentesi graffe e bisogna sempre andare a capo dopo la loro apertura;
\item Nessun spazio di riempimento in costruzioni vuote (ad esempio, {}, [], fn ());
\item I nomi delle variabili e dei metodi devono essere il più descrittivi possibili, ad eccezioni degli iteratori;
\item I nomi di metodi e funzioni devono essere adottati quelli definiti nel documento \emph{Specifica Tecnica};
\item I programmatori possono introdurre funzioni ausiliarie non presenti nella progettazione.
\end{itemize}   
     
   
   \paragraph{Documentazione del codice}
     I file contenti codice dovranno essere provvisti di un'intestazione contenente:
     \begin{lstlisting}[frame=single]
       /*!
       * \file Nome del file
       * \author Autore (indirizzo e-mail dell'autore)
       * \date Data di creazione
       * \brief Breve descrizione del file
       *
       * Descrizione dettagliata del file
       */
     \end{lstlisting}
     \begin{itemize}
     \item Nome: è il nome del file, estensione compresa;
     \item Autore: è l'indirizzo e-mail dell'autore;
     \item Data di creazione: è la data di creazione del file;
     \item Breve descrizione del file: qui viene inserita una descrizione sintetica del file;
     \item Descrizione dettagliata del file: qui viene inserita una descrizione più approfondita di quella precedente, illustrando tutti i dettagli che compongono il file.
     \end{itemize}
     Prima di ogni classe dovrà esserci un commento contenente:
     \begin{lstlisting}[frame=single]
       /*!
       * \class Nome della classe
       * \brief Breve descrizione della classe
       */
     \end{lstlisting}

     Prima di ogni metodo dovrà essere inserito un commento contenente:
     \begin{lstlisting}[frame=single]
       /*!
       * \brief Breve descrizione della funzione
       * \param Nome del primo parametro
       * \param Nome del secondo parametro
       * \return Valore ritornato dalla funzione
      */
  \end{lstlisting}
      
\subsubsection{Strumenti e ambienti di sviluppo}
\paragraph{Stesura del codice}
Per la stesura del codice è consigliabile usare uno dei seguenti editor:
\begin{itemize}
\item gedit \href{http://projects.gnome.org/gedit}{(http://projects.gnome.org/gedit)};
\item \glossaryItem{Microsoft Visual Studio} \href{https://www.visualstudio.com/it/}{(https://www.visualstudio.com/it/)}.
\end{itemize}

Microsoft Visual Studio è un \glossaryItem{IDE} per lo sviluppo di applicazioni per tablet, smartphone e computer, oltre a siti e servizi web. E' fortemente consigliato per lo sviluppo di questo progetto dato che supporta numerosi linguaggi di programmazione tra cui \glossaryItem{Node.js}, \glossaryItem{HTML}, \glossaryItem{Java} e \glossaryItem{Javascript}, tutti linguaggi fondamentali per la realizzazione del capitolato richiesto.

\paragraph{\glossaryItem{Framework}}
Per lo sviluppo del progetto è previsto l'utilizzo dello stack \glossaryItem{MEAN}, contenente le seguenti tecnologie:
\begin{itemize}
\item \glossaryItem{Mongodb};
\item \glossaryItem{express};
\item \glossaryItem{Angular js};
\item \glossaryItem{Node js}.
\end{itemize}