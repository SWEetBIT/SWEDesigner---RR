\section{Comunicazioni}
  \subsection{Comunicazioni esterne}
    Per le comunicazioni esterne è stata creata un'apposita casella di posta elettronica:\\
      \begin{center}
        \href{mailto:sweet.bit.group@gmail.com}{sweet.bit.group@gmail.com}
      \end{center}
    Tale indirizzo deve essere l’unico canale di comunicazione esistente tra il gruppo di lavoro e l’esterno.\\
    L'unico membro del gruppo ad avere accesso alla mail, e quindi alle comunicazioni con il \glossaryItem{Committente}, è il \emph{Responsabile di Progetto}.
    Suo anche il compito di informare i membri del gruppo delle discussioni avvenute tramite la casella di posta eletronica e inoltrare, qualora dovesse ritenerlo necessario,
    tali discussioni alle caselle postali dei vari membri del gruppo.
  \subsection{Comunicazioni interne}
    Per tutte le comunicazioni interne è stato creato un gruppo \textbf{\glossaryItem{Telegram}} in cui ogni membro può comunicare i propri impegni e il proprio stato dei lavori.\\
    Il gruppo serve, inoltre, per la comunicazione delle date delle riuinioni, per comunicare tempestivamente le decisioni prese e per scambiare materiale utile al progetto.
  \subsection{Composizione e-mail}
    \subsubsection{Destinatario}
      Il destinatario delle mail esterne può variare a seconda che ci si debba riferire al \glossaryItem{Proponente} del progetto, al Prof. Vardanega Tullio o al Prof. Cardin Riccardo.
    \subsubsection{Mittente}
      L’unico indirizzo utilizzabile è \href{mailto:sweet.bit.group@gmail.com}{sweet.bit.group@gmail.com} e deve essere usato solamente dal \emph{Responsabile di Progetto}.
    \subsubsection{Oggetto}
      L'oggetto di una comunicazione deve essere chiaro, stringato e possibilmente diverso da altri oggetti preesistenti.\\
      Nel caso in cui si dovesse comporre un messaggio di risposta, vi è l'obbligo di utilizzare la particella "RE:" prima dell'oggetto in modo tale da identificare
      il livello di risposta; Qualora si trattasse invece di un inoltro è obbligatorio utilizzare la particella "I:".\\
      In ogni caso l'oggetto di una comunicazione già avviata non potrà essere cambiato.
    \subsubsection{Corpo}
      Il corpo di un messaggio deve contenere tutte le informazioni necessarie a rendere facilmente comprensibile l’argomento trattato a tutti i destinatari. Se alcune parti del messaggio hanno uno o più destinatari specifici, il loro nome dovrà essere aggiunto all’inizio del paragrafo tramite la seguente segnatura: \emph{@destinatario}.
      In caso di risposta od inoltro del messaggio, il contenuto aggiunto deve essere sempre messo in testa (per non costringere gli altri membri a dover scorrere tutta la mail).
      È caldamente consigliato, inoltre, evitare la cancellazione delle precedenti parti del messaggio in modo tale da rendere ogni partecipante alla discussione consapevole del contesto.
    \subsubsection{Allegati}
      L'uso di allegati è permesso qualora si ritenga necessario.\\
      Un esempio è il resoconto di un incontro con il \glossaryItem{Proponente} o il \glossaryItem{Committente}.
