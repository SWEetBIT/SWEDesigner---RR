\section{Processi di supporto}
Questo capitolo descrive tutte le convenzioni scelte ed adottate da SWEet BIT riguardo alla stesura, verifica e approvazione della documentazione da produrre.

\subsection{Documentazione di progetto}
La documentazione del progetto è contenuta in un \emph{repository} distinto da quello del codice sorgente. Tale \emph{repository} contiene i seguenti documenti:
\begin{itemize}
\item \textbf{Studio di Fattibilità: } Documento interno che valuta i pro e contro di ogni capitolato d'appalto;
\item \textbf{Norme di Progetto: } Documento interno che descrive le norme utilizzate da SWEet BIT per la gestione dei processi;
\item \textbf{Verbali: } Documento interno che riporta i verbali delle riunioni che interessano il gruppo;
\item \textbf{Piano di Progetto: } Documento esterno che dichiara come il gruppo intende gestire le risorse umane e temporali;
\item \textbf{Piano di Qualifica: } Documento esterno che descrive come il gruppo gestisce la qualità del prodotto;
\item \textbf{Analisi dei Requisiti: } Documento esterno che elenca, descrive e traccia i requisiti ed i casi d'uso del prodotto;
\item \textbf{Specifica Tecnica: } Documento esterno che descrive l'architettura logica del sistema, mostrandone l'interfaccia di ogni componente utilizzato;
\item \textbf{Glossario: } Documento esterno che riporta la definizione di termini che possono portare ad ambiguità;
\end{itemize} 

\subsubsection{Ambiente di lavoro}
    \paragraph{Stesura documenti}
      \subparagraph{\LaTeX}
        Per la stesura dei documenti si è scelto di utilizzare il sistema \glossaryItem{\LaTeX}\ poiché permette una netta separazione fra contenuti
        e formattazione; con \glossaryItem{\LaTeX}\ è possibile definire i \glossaryItem{Template} di layout in \glossaryItem{File} condivisi da ogni documento rendendo la lavorazione
        degli stessi altamente più flessibile e ottimale.\\
        \glossaryItem{\LaTeX}\ consente poi l'utilizzo di funzioni e variabili locali definiti dall'\glossaryItem{Utente}, in maniera tale da semplificare ulteriormente
        il lavoro di stesura dei documenti, una volta definita la struttura degli stessi.\\
        Per la scrittura dei documenti \glossaryItem{\LaTeX}\ l'editor utilizzato è \textbf{TexMaker}.
      \subparagraph{Strumentazione esterna}
        Per ridurre al minimo gli errori di calcolo di alcuni indici o di formattazione del testo si è scelto di utilizzare alcuni strumenti
        automatici che si occupano di svolgere alcuni semplici compiti di calcolo e formattazione:
        \begin{itemize}
          \item \textbf{Script \glossaryItem{Perl} per il calcolo dell'indice di Gulpease: }All'interno della directory \path{tools\gulpease} dentro la \glossaryItem{Repository} è presente
          uno \glossaryItem{Script} in \glossaryItem{Perl} per il calcolo automatico dell'indice di Gulpease che segue la seguente formula:
          \begin{center}
            \begin{equation}
              Gulpease Index = 89+\frac{300*( numero delle frasi ) - 10*( numero delle lettere )}{numero delle parole}
            \end{equation}
          \end{center}
          \item \textbf{Script \glossaryItem{Perl} per la glossarizzazione dei termini: }Per ridurre al minimo gli errori nell'inserimento di un termine all'interno
          del \emph{Glossario} si è deciso di utilizzare uno \glossaryItem{Script} in \glossaryItem{Perl}, reperibile nella directory \path{tools\glossary}, che si occupa di inserire
          il comando \glossaryItem{\LaTeX}\ \textbackslash glossaryItem{} su tutti i termini presenti all'interno del \emph{Glossario} dati in input un \emph{Glossario}
          e un documento;
          \item \textbf{Aspell: }si tratta di un tool offerto dall'editor stesso o presente in forma \glossaryItem{Stand-alone} che effettua un controllo ortografico
          su tutto il documento.
        \end{itemize}
   
  \subsubsection{Template}
    Per agevolare la redazione di un documento è stato prodotto un \glossaryItem{Template} e delle regole da seguire per la stesura degli stessi.\\
    Tale modello e tali regole sono inseriti all'interno di una cartella \path{documents\template} sulla \glossaryItem{repository}.

\subsubsection{Formattazione generale delle pagine}
        L’intestazione di ogni pagina contiene:
        \begin{itemize}
          \item Logo del gruppo;
          \item Nome del gruppo;
          \item Nome del progetto;
          \item Sezione corrente del documento;
        \end{itemize}
        A piè di pagina invece è presente:
        \begin{itemize}
          \item Nome e versione del documento;
          \item Pagina corrente nel formato \emph{N} di \emph{T} dove \emph{N} è il numero di pagina corrente e \emph{T} è il numero di pagine totali.
        \end{itemize}

\subsubsection{Struttura del documento}
      \paragraph{Prima pagina}
        Ogni documento è caratterizzato da una prima pagina che contiene le seguenti informazioni sul documento:\\
        \begin{itemize}
          \item Nome del gruppo;
          \item Nome del progetto;
          \item Logo del gruppo;
          \item Titolo del documento;
          \item Versione del documento;
          \item Cognome e nome dei redattori del documento;
          \item Cognome e nome dei verificatori del documento;
          \item Cognome e nome del responsabile approvatore del documento;
          \item Destinazione d'uso del documento;
          \item Lista di distribuzione del documento;
          \item Breve descrizione del documento.
        \end{itemize}
      \paragraph{Diario delle modifiche}
        La seconda pagina di ogni documento contiene il diario delle modifiche.\\
        Ogni riga del diario delle modifiche contiene:
        \begin{itemize}
          \item Un breve sommario delle modifiche svolte;
          \item Cognome e nome dell’autore;
          \item Data della modifica;
          \item Versione del documento dopo la modifica.
        \end{itemize}
        La tabella è ordinata per data in ordine decrescente, in modo che la prima riga corrisponda alla versione attuale del documento.
      \paragraph{Indici}
        In ogni documento è presente un indice delle sezioni, un indice delle figure e un indice delle tabelle. Nel caso non siano presenti figure o tabelle i rispettivi indici verranno omessi.        

    \subsubsection{Ciclo di vita}\label{subsec:ciclovita}
      Ogni documento prodotto segue un preciso iter che scandisce le fasi in cui si trova in ogni istante. Un documento può trovarsi in tre stati diversi:\\
      \begin{itemize}
        \item \textbf{In lavorazione: }Un documento entra in questa fase nel momento della sua creazione e vi rimane per tutto il periodo della sua stesura
          o per eventuali successive modifiche;
        \item \textbf{Da verificare: }Un documento entra in questa fase alla fine della sua stesura quando entra in possesso dei verificatori che avranno il compito
          di individuare e correggere eventuali errori sintattici o semantici;
        \item \textbf{Approvato: }Un documento entra in questa fase una volta che il \emph{Responsabile di Progetto} lo ha approvato dopo la fase di verifica.\\
          L'approvazione sancisce la fine del ciclo di vita del documento per la data versione.
      \end{itemize}
      Ogni fase del ciclo di vita può essere affrontata anche più volte da parte di un documento.
\subsubsection{Glossario}
  Il Glossario conterrà tutte le parole presenti negli altri documenti che fanno parte del contesto dell'applicazione o che possono essere fraintese. Le definizioni, presentate in
  ordine alfabetico, dovranno essere concise e comprensibili.\\
  I termini verranno inseriti nel glossario parallelamente al processo di stesura degli altri documenti, in modo da limitare l'errore umano.\\
  È preferibile inserire un termine inizialmente privo di definizione, piuttosto che rimandare la stesura del glossario.

    \subsubsection{Norme tipografiche}
      Questa sezione racchiude le convenzioni riguardanti tipografia, ortografia e uno stile uniforme per tutti i documenti.
      \paragraph{Punteggiatura}
        \begin{itemize}
          \item \textbf{Parentesi: }Il testo racchiuso tra parentesi non deve aprirsi o chiudersi con un carattere di spaziatura e non deve terminare con un carattere di punteggiatura;
          \item \textbf{Punteggiatura: }Un carattere di punteggiatura non deve mai esser preceduto da un carattere di spaziatura;
          \item \textbf{Lettere maiuscole: }Le lettere maiuscole vanno poste solo dopo il punto, il punto di domanda, il punto esclamativo e all’inizio di ogni elemento di un elenco puntato ed
          oltre a dove sia previsto dalla lingua italiana. È inoltre utilizzata l’iniziale maiuscola nel nome del team, del progetto, dei documenti, dei ruoli di progetto, delle fasi di
          lavoro e nelle parole \glossaryItem{Proponente} e \glossaryItem{Committente}.
        \end{itemize}
      \paragraph{Stile di testo}
        \begin{itemize}
          \item \textbf{Corsivo: }Il corsivo deve essere utilizzato nei seguenti casi:
            \bgroup
              \begin{itemize}
                \item \textbf{Citazioni: }Quando si deve citare una frase questa sarà scritta in corsivo;
                \item \textbf{Nomi particolari: }Il corsviso deve essere utilizzato quando ci si rierisce a figure particolari (es. \emph{Analista});
                \item \textbf{Documenti: }Il corsivo deve essere utilizzato quando ci si riferisce a documenti particolari (es. \emph{Glossario});
                \item \textbf{Altri casi: }Il corsivo sarà utilizzato in tutte quelle situazioni in cui è necessario dare rilievo ad una parola o passaggio
                significativo;
              \end{itemize}
            \egroup
          \item \textbf{Grassetto: }Il grassetto deve essere utilizzato nei seguenti casi:
            \bgroup
              \begin{itemize}
                \item \textbf{Elenchi puntati: }In questo caso il grassetto può essere utilizzato per mettere in evidenza i punti sviluppati nella loro continuazione;
                \item \textbf{Altri casi: }Il grasstto dovrà essere sempre utilizzato per evidenziare passaggi o parole chiave;
              \end{itemize}
            \egroup
          \item \textbf{\textbackslash path: }Il comando \textbackslash path deve essere utilizzato per indicare i percorsi all'interno di directory;
          \item \textbf{Maiuscolo: }L'utilizzo di parole completamente in maiuscolo è riservato solo ed esplusivamente alle sigle o alle macro \glossaryItem{\LaTeX}\ riportate
          nei documenti;
          \item \textbf{\glossaryItem{\LaTeX}\ : }Ogni riferimento a\glossaryItem{\LaTeX}\ deve essere scritto utilizzano la macro \textbackslash \glossaryItem{\LaTeX};
        \end{itemize}
      \paragraph{Composizione del testo}
        \begin{itemize}
          \item \textbf{Elenchi puntati: }Ogni punto dell'elenco puntato deve essere scritto in grassetto e con la prima lettera in maiuscolo.\\
            Nella definizione del punto la prima lettera dovrà essere maiuscola ad eccezione di casi isolati (es. nome di file) e dovrà terminare sempre con un ";" ; mentre l'ultimo elemento terminerà con un "." ;
          \item \textbf{Note a piè di pagina: }Ogni nota dovrà cominciare con l’iniziale della prima parola maiuscola e non deve essere preceduta da alcun carattere di spaziatura.\\
            Ogni nota deve terminare con un punto.
        \end{itemize}
      \paragraph{Formati}
        \begin{itemize}
          \item \textbf{Percorsi: }Per tutti gli indirizzi e-mail e web completi dovrà essere utilizzato il comando \glossaryItem{\LaTeX}\ \textbackslash url mentre per i percorsi
            relativi si utilizzerà il comando \textbackslash path;
          \item \textbf{Date: }Tutte le date presenti all'interno della documentazione devono seguire la notazione definiti nello standard \glossaryItem{ISO} 8601:2004:
            \begin{center}
              AAAA-MM-GG\\
            \end{center}
            dove:
            \bgroup
              \begin{itemize}
                \item AAAA: rappresenta l'anno utilizzando quattro cifre;
                \item MM: rappresenta il mese utilizzando due cifre;
                \item GG: rappresenta il giorno utilizzando due cifre.
              \end{itemize}
            \egroup
          \item \textbf{Nomi propri: }L'utilizzo dei nomi propri dei membri del team (e non) deve seguire la notazione "Cognome Nome";
          \item \textbf{Nome gruppo: }Ci si riferirà al gruppo solo come "SWEet BIT";
          \item \textbf{Nome del \glossaryItem{Proponente}: }Ci si riferirà al \glossaryItem{Proponente} come "Zucchetti s.r.l" o semplicemente come "\glossaryItem{Proponente}";
          \item \textbf{Nome del \glossaryItem{Committente}: }Ci si riferià al \glossaryItem{Committente} come "prof. Vardanega Tullio" o semplicemente come "\glossaryItem{Committente}";
          \item \textbf{Nome del progetto: }Ci si riferirà al progetto solo come "SWEDesigner".
        \end{itemize}
      \paragraph{Sigle}
        Le sigle dei documenti potranno essere utilizzate solo ed esclusivamente all'interno di tabelle o diagrammi. Sono previste le seguenti sigle:
        \begin{itemize}
          \item \textbf{AdR} = Analisi dei Requisiti;
          \item \textbf{GL} = Glossario;
          \item \textbf{NdP} = Norme di Progetto;
          \item \textbf{PdP} = Piano di Progetto;
          \item \textbf{PdQ} = Piano di Qualifica;
          \item \textbf{SdF} = Studio di Fattibilità;
          \item \textbf{ST} = Specifica Tecnica;
          \item \textbf{RA} = Revisione d'Accettazione;
          \item \textbf{RP} = Revisione di Progettazione;
          \item \textbf{RQ} = Revisione di Qualifica;
          \item \textbf{RR} = Revisione dei Requisiti.
        \end{itemize}
    \paragraph{Componenti grafiche}
      \subparagraph{Tabelle}
        Ogni tabella presente all’interno dei documenti dev’essere accompagnata da una didascalia, in cui deve comparire un numero identificativo incrementale per la tracciabilità
        della stessa all’interno del documento.
      \subparagraph{Immagini}
        Le immagini da includere all'interno del documento devono avere preferibilmente il formato Portable Network Graphics (\glossaryItem{PNG}).
        
        \subsubsection{Classificazione dei documenti}
  \paragraph{Documenti formali}
        Un documento viene definito formale quando viene approvato dal \emph{Responsabile di Progetto} ed è quindi pronto per essere inviato ai richiedenti.\\
        Per raggiungere questo stato il documento deve seguire l'iter descritto nel \emph{Norme di Progetto} e nel paragrafo \ref{subsec:ciclovita} riguardante il ciclo di vita dei documenti.
      \paragraph{Documenti informali}
        Un documento è definito informale fino a quando non approvato dal \emph{Responsabile di Progetto}, fino ad allora il suo uso è da considerarsi unicamente interno.
      \paragraph{Versionamento}
        La documentazione prodotta deve essere corredata dal numero di versione attuale utilizzando la codifica:
        \begin{center}
          \emph{v.X.Y.Z}
        \end{center}
        dove:
        \begin{itemize}
          \item X: indica il numero crescente di uscite formali del documento;
          \item Y: indica il numero crescente di modifiche sotanziali al documento;
          \item Z: indica il numero crescente di modifiche minori apportate al documento;
        \end{itemize}

\subsection{Coordinamento}
    Il coordinamento del gruppo avviene tramite:
    \begin{itemize}
      \item \glossaryItem{Repository} su \textbf{\glossaryItem{GitHub}};
      \item \glossaryItem{Google Drive};
      \item \glossaryItem{Telegram}
    \end{itemize}
    \subsubsection{Repository}
      Sulla \glossaryItem{Repository} di \textbf{\glossaryItem{GitHub}}, raggiungibile all'indirizzo \url{https://GitHub.com/SWEetBIT}, sono caricati i vari \glossaryItem{Template} da utilizzare
      durante la stesura dei documenti e la strumentazione da utilizzare per la formattazione dei termini del \emph{Glossario}.

\subsubsection{Google Drive}
      Lo strumento di \glossaryItem{Cloud storage} di \glossaryItem{Google}, è stato utilizzato principalmente per tenere traccia di verbali interni e di documentazione interna
      non formale, che può essere utile a tutti i membri del gruppo, durante le varie fasi di lavorazione del progetto.
       
    
    \subsection{Configurazione}
      
    \subsubsection{Versionamento}
      Dopo aver preso in considerazione diverse opzioni per il versionamento, alla fine si è scelto di utilizzare \textbf{\glossaryItem{GitHub}} per via della
      sua enrome flessibilità e per via delle esperienze pregresse di tutti i membri del gruppo che hanno manifestato una certa familiarità con
      tale strumento.\\
      È stata creata una sola \glossaryItem{Repository}, alla quale si aggiungeranno le altre legate alle fasi successive, contenente tutte le cartelle necessarie
      alla stesura dei documenti \glossaryItem{\LaTeX}\.\\
      Una volta terminato il lavoro di redazione dei documenti sarà creato un \emph{branch} di verifica per permettere ai \emph{Verificatori} di
      lavorare in parallelo agli altri membri del gruppo.
      
      \subsubsection{Controllo della configurazione}
      I cambiamenti nella configurazione del software e della documentazione vanno controllati e tracciati, possono nascere dal \emph{Piano di Progetto} o dall'iniziativa di un membro.
      \paragraph{Cambiamenti pianificati}
	Nascono dal \emph{Piano di Progetto}, il cambiamento è già controllato, e quindi basta registrarlo aggiornando la versione dell'elemento che è stato cambiato.
	\paragraph{Cambiamenti non pianificati}
	Nascono dall'iniziativa di un membro, va proposto agli altri membri del gruppo tramite \emph{Telegram}, nel caso la maggioranza sia d'accordo, il responsabile di progetto incarica un amministratore di implementarlo e motivarlo nella documentazione.



      
  \subsubsection{Verifica}
    Vengono qui elencati, e sommariamente descritti, gli strumenti automatizzati per effettuare la verifica dei documenti redatti e del \glossaryItem{Codice} prodotto.

\paragraph{Analisi Statica}
Per l'analisi statica del codice, descritta nel documento \emph{Piano di Qualifica v}\VersionePQ, e per aiutare l'identificazione automatica di errori, vengono usati i seguenti strumenti:
\begin{itemize}
\item \textbf{JSHint: }uno strumento che aiuta ad identificare gli errori e i potenziali problemi nel codice \glossaryItem{JavaScript};
\item \textbf{JSLint: }uno strumento usato per verificare che il codice \glossaryItem{JavaScript} compila secondo le regole del linguaggio;
\item \textbf{Closure Compiler: }strumento utilizzato per  migliorare il codice, controllando la sintassi ed i riferimenti alle variabili.
\end{itemize}

\paragraph{Analisi Dinamica}
Per l'analisi dinamica, descritta nel documento \emph{Piano di Qualifica v}\VersionePQ, per verificare tramite test se il prodotto software non abbia errori, viene utilizzato:
\begin{itemize}
\item \textbf{Mocha: } una libreria ricca di funzionalità per l'esecuzione di test \glossaryItem{JavaScript}.
\end{itemize}
I tipi di test implementati sono:
\begin{itemize}
\item \textbf{Test di Unità: }Verificano che ogni singola unità funzioni correttamente; un'unità è la più piccola quantità di software che conviene verificare da sola. Vengono identificati tramite la seguente notazione:
\begin{center}
\emph{TU[CodiceTest]}
\end{center}
\item \textbf{Test di Integrazione: }Verificano che più unità funzionino assieme nel modo corretto. Vengono identificati tramite la seguente notazione:
\begin{center}
\emph{TI[CodiceTest]}
\end{center}
\item \textbf{Test di Sistema: }Verificano che il prodotto software copra i requisiti. Vengono identificati tramite la seguente notazione:
\begin{center}
\emph{TS[CodiceProgressivoRequisito]}
\end{center}
\item \textbf{Test di Accettazione: }Verificano in presenza del Proponente la completezza del prodotto durante il collaudo. Vengono identificati tramite la seguente notazione:
\begin{center}
\emph{TA[CodiceTest]}
\end{center}
\end{itemize}

\paragraph{Metriche}
Durante l'attività di verifica dei processi vengono definite delle metriche, di seguito riportate, utili alla misurazione quantitativa degli obiettivi di qualità prefissati dal documento \emph{Piano di Qualifica v}\VersionePQ. Per ognuna delle metriche definite dal documento \emph{Piano di Qualifica v}\VersionePQ{} viene definito il nome ed il range o il valore di accettazione stabilito. In appendice viene specificato come calcolare tali metriche.\\
\begin{table}[H]
		\centering
		\begin{tabular}{|p{7cm}|p{2cm}|}
\hline
\textbf{Metrica} & \textbf{Range}\\ \hline
Schedule Variance & [\(\geq\) -141]\\ \hline
Budget Variance & [\(\geq\) -282.5]\\ \hline
Indice Gulpease & [40-100]\\ \hline
Complessità Ciclomatica & [0-15]\\ \hline
Numero metodi per file & [3-10] \\ \hline
Variabili non utilizzate e/o non definite & [0-0] \\ \hline
Numero argomenti per funzione & [0-6]\\ \hline
Linee di codice per linee di commento & [\(\geq\)0.25]\\ \hline
Copertura del codice & [70\%-100\%]\\ \hline
\end{tabular}
\caption {Metriche - Range accettazione}
\end{table}



  

    
      
      
      
      
      

        
        
    

