\section{Ambiente di lavoro}
  \subsection{Coordinamento}
    Il coordinamento del gruppo avviene tramite:
    \begin{itemize}
      \item Repository su \glossaryItem{GitHub};
      \item \glossaryItem{Google Drive};
      \item \glossaryItem{Telegram}
    \end{itemize}
    \subsubsection{Repository}
      Suulla repository di GitHub, raggiungibile all'indirizzo \url{https://github.com/SWEetBIT}, sono caricati i vari template da utilizzare
      durante la stsura dei documenti e la strumentazione d autilizzare per la formattazione dei termini del \emph{Glossario}.
    \subsubsection{Gestione del progetto}
      La piattaforma che è stata scelta per la gestione del progetto è \textbf{Redmine} che fornisce:
      \begin{itemize}
        \item Un sistema flessibile di gestione dei ticket;
        \item Il grafico Gantt delle attività;
        \item Un calendario per l'organizzazione e la distribuzione dei compiti;
        \item La visualizzazione della repository relativa al progetto;
        \item Un sistema di rendicontazione del tempo;
      \end{itemize}
      Sono state anlizzate altre altenrative a \textbf{Redmine} che, dopo una fase di analisi iniziale, non sono risultate idonee allo scopo:
      \begin{itemize}
        \item \textbf{Teamworks: }si tratta del software probabilmente più adatto per il project management vista la sua grande versatilità e
        la strumentazione offerta. Purtroppo i suoi costi non hanno permesso un suo utilizzo in ambiito universitario;
        \item \textbf{Zohoo: }A differenza di Redimine o di Teamworks, questa piattaforma non offre un servizio di rendicontazione del tempo
        e la generazione di grafici Gantt;
      \end{itemize}
    \subsubsection{Versionamento}
      Dopo aver preso in considerazione diverse opzioni per il versionamento ma alla fine si è scelto di utilizzare \textbf{GitHub} per via della
      sua enrome flessibilità e per via delle esperienze pregresse di tutti i membri del gruppo che hanno manifestato una certa familiarità con
      lo strumento.\\
      È stata creata una sola repository, alla quale si aggiungeranno le altre legate alle fasi successive, contenente tutte le cartelle necessarie
      alla stesura dei documenti \LaTeX.\\
      Una volta terminato il lavoro di redazione dei documenti sarà creato un \emph{branch} di verifica per permettere ai \emph{Verificatori} di
      lavorare in parallelo agli altri membri del gruppo.
    \subsubsection{Google Drive}
      Lo strumento di cloud storage di Google è stato utilizzato principalmente per tenere traccia di verbali interni e di documentazione interna
      non formale che può essere utile a tutti i membri del gruppo durante le varie fasi di lavorazione del progetto.
  \subsection{Ambiente documentale}
    \subsubsection{Pianificazione}
      Per pianificare le attività legate allo sviluppo del progetto e la gestione delle risorse si è scelto di utilizzare \textbf{ProjectLibre}.\\
      Si tratta di un ottimo software open-source basato su Java per il project management.\\
      La scelta è ricaduta su questo software principlamente per quattro motivi:
      \begin{itemize}
        \item Si tratta di un software portabile essendo basato su Java;
        \item È open-source;
        \item Genera automaticamente digrammi di Gannt;
        \item Il salvataggio dei file è in XML, quindi un formato testuale che permette di utilizzare i merge senza causare troppi conflitti;
      \end{itemize}
    \subsubsection{Stesura documenti}
      \paragraph{\LaTeX}
        Per la stesura dei documenti si è scelto di utilizzare il sistema \LaTeX poiché permette una netta seprazione fra contenuti
        e formattazione: con \LaTeX è possible definire i template di layout in file condivisi da ogni documento rendendo la lavorazione
        degli stessi altamente più flessibile e ottimale.\\
        \LaTeX consente poi l'utilizzo di funzioni e variabli locali defintii dall'utente in maniera tale da semplificare ulteriormente
        il lavoro di stesura dei documenti una volta definita la struttura degli stessi.\\
        Per la scrittura dei documenti \LaTeX l'editor utilizzato è \textbf{TexMaker}.
      \paragraph{Strumentazione esterna}
        Per ridurre al minimo gli errori di calcolo di alcuni indici o di formattazione del testo si è scelto di utilizzare alcuni strumenti
        automatici che si occupano di svolgere alcuni semplici compiti di calcolo e formattazione:
        \begin{itemize}
          \item \large{TOOL INDICE DI GULPEASE}
          \item \large{GLOSSARIZZAZIONE DEI TERMINI}
          \item \textbf{Aspell: }si tratta di un tool offerto dall'editor stesso o presente in forma stand-alone che effettua un controllo ortografico
          su tutto il documento;
        \end{itemize}
      \paragraph{Gestione Use Case e Requisiti}
        Per semplificare il tracciamento degli Use Case e dei Requisiti si è scelto di utilizzare \textbf{Trender}, uno strumento open-source che permette
        di gestire al meglio entrambi gli elementi e le realizioni fra gli stessi.\\
        Il tool permette l'esportazione di questi direttamente in \LaTeX così da semplificare la scrittura degli stessi all'interno della documentazione.
    \paragraph{Grafici UML}
      Per il disegno dei grafici UML si è scelto di tuilizzare \textbf{Astah Professional} per via del suo enorme potenziale e della mole di strumenti offerti
      rispetto agli altri editor presenti sul mercato.\\
      Le alternative analizzate sono state Dia, LuchiChart e Papyrus che si sono rivelati però troppo deboli in confronto ad \textbf{Astah Professional} che offre
      diverse funzionalità aggiuntive raggruppandone diverse dei tre software citati al suo interno.
  \subsection{Ambiente di verifica}
    Vengono qui elencati e sommariamente descritti gli strumenti automatizzati per effettuare la verifica dei documenti redatti e del codice prodotto.\\
    Le metriche ed i metodi per effettuare verifica sono ampiamente e dettagliatamente descritti nel \emph{Piano di Qualifica}.\\
    A tale documento si fa inoltre riferimento per le caratteristiche di fondamentale importanza per la verifica degli strumenti qui riportati.
    \subsubsection{Documenti}
      \begin{itemize}
        \item \textbf{TexMaker: }Per la stesura dei documenti è stato utilizzato TexMaker per via dell'intrazione con i dizionari di OpenOffice.org che integra
        che consentono un controllo ortografico in real-time;
        \item \textbf{Aspell: }strumento per la correzione tipografica dei documenti redatti in \LaTeX;
      \end{itemize}
