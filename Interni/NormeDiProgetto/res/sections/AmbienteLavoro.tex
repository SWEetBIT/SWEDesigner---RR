\section{Ambiente di lavoro}
  \subsection{Coordinamento}
    Il coordinamento del gruppo avviene tramite:
    \begin{itemize}
      \item \glossaryItem{Repository} su \textbf{\glossaryItem{GitHub}};
      \item \glossaryItem{Google Drive};
      \item \glossaryItem{Telegram}
    \end{itemize}
    \subsubsection{Repository}
      Sulla \glossaryItem{Repository} di \textbf{\glossaryItem{GitHub}}, raggiungibile all'indirizzo \url{https://GitHub.com/SWEetBIT}, sono caricati i vari \glossaryItem{Template} da utilizzare
      durante la stesura dei documenti e la strumentazione da utilizzare per la formattazione dei termini del \emph{Glossario}.
    \subsubsection{Gestione del progetto}
      
    \subsubsection{Versionamento}
      Dopo aver preso in considerazione diverse opzioni per il versionamento, alla fine si è scelto di utilizzare \textbf{\glossaryItem{GitHub}} per via della
      sua enrome flessibilità e per via delle esperienze pregresse di tutti i membri del gruppo che hanno manifestato una certa familiarità con
      tale strumento.\\
      È stata creata una sola \glossaryItem{Repository}, alla quale si aggiungeranno le altre legate alle fasi successive, contenente tutte le cartelle necessarie
      alla stesura dei documenti \glossaryItem{\LaTeX}\.\\
      Una volta terminato il lavoro di redazione dei documenti sarà creato un \emph{branch} di verifica per permettere ai \emph{Verificatori} di
      lavorare in parallelo agli altri membri del gruppo.
    \subsubsection{Google Drive}
      Lo strumento di \glossaryItem{Cloud storage} di \glossaryItem{Google}, è stato utilizzato principalmente per tenere traccia di verbali interni e di documentazione interna
      non formale, che può essere utile a tutti i membri del gruppo, durante le varie fasi di lavorazione del progetto.
  \subsection{Ambiente documentale}

    \subsubsection{Stesura documenti}
      \paragraph{\LaTeX}
        Per la stesura dei documenti si è scelto di utilizzare il sistema \glossaryItem{\LaTeX}\ poiché permette una netta seprazione fra contenuti
        e formattazione; con \glossaryItem{\LaTeX}\ è possible definire i \glossaryItem{Template} di layout in \glossaryItem{File} condivisi da ogni documento rendendo la lavorazione
        degli stessi altamente più flessibile e ottimale.\\
        \glossaryItem{\LaTeX}\ consente poi l'utilizzo di funzioni e variabli locali definti dall'\glossaryItem{Utente}, in maniera tale da semplificare ulteriormente
        il lavoro di stesura dei documenti, una volta definita la struttura degli stessi.\\
        Per la scrittura dei documenti \glossaryItem{\LaTeX}\ l'editor utilizzato è \textbf{TexMaker}.
      \paragraph{Strumentazione esterna}
        Per ridurre al minimo gli errori di calcolo di alcuni indici o di formattazione del testo si è scelto di utilizzare alcuni strumenti
        automatici che si occupano di svolgere alcuni semplici compiti di calcolo e formattazione:
        \begin{itemize}
          \item \textbf{Script \glossaryItem{Perl} per il calcolo dell'indice di Gulpease: }All'interno della directory \path{tools\gulpease} dentro la \glossaryItem{Repository} è presente
          uno \glossaryItem{Script} in \glossaryItem{Perl} per il calcolo automatico dell'indice di Gulpease che segue la seguente formula:
          \begin{center}
            \begin{equation}
              Gulpease Index = 89+\frac{300*( numero delle frasi ) - 10*( numero delle lettere )}{numero delle parole}
            \end{equation}
          \end{center}
          \item \textbf{Script \glossaryItem{Perl} per la glossarizzazione dei termini: }Per ridurre al minimo gli errori nell'inserimento di un termine all'interno
          del \emph{Glossario} si è deciso di utilizzare uno \glossaryItem{Script} in \glossaryItem{Perl}, reperibile nella directory \path{tools\glossary}, che si occupa di inserire
          il comando \glossaryItem{\LaTeX}\ \textbackslash glossaryItem{} su tutti i termini presenti all'interno del \emph{Glossario} dati in input un \emph{Glossario}
          e un documento;
          \item \textbf{Aspell: }si tratta di un tool offerto dall'editor stesso o presente in forma \glossaryItem{Stand-alone} che effettua un controllo ortografico
          su tutto il documento.
        \end{itemize}
      \paragraph{Gestione \glossaryItem{Use Case} e Requisiti}
        Per semplificare il tracciamento degli \glossaryItem{Use Case} e dei Requisiti si è scelto di utilizzare \textbf{Trender}, uno strumento \glossaryItem{Open-source} che permette
        di gestire al meglio entrambi gli elementi e le realizioni fra gli stessi.\\
        Il tool permette l'esportazione di questi direttamente in \glossaryItem{\LaTeX}\, così da semplificare la scrittura degli stessi all'interno della documentazione.
    \paragraph{Grafici UML}
      Per il disegno dei grafici \glossaryItem{UML} si è scelto di utilizzare \textbf{Astah Professional} per via del suo enorme potenziale e della mole di strumenti offerti
      rispetto agli altri editor presenti sul mercato.\\
      Le alternative analizzate sono state \emph{Dia}, \emph{LuchiChart} e \emph{Papyrus} che si sono rivelati però troppo deboli in confronto ad \textbf{Astah Professional} che offre
      diverse funzionalità aggiuntive raggruppandone diverse dei tre software citati al suo interno.
  \subsection{Ambiente di verifica}
    Vengono qui elencati, e sommariamente descritti, gli strumenti automatizzati per effettuare la verifica dei documenti redatti e del \glossaryItem{Codice} prodotto.\\
    Le metriche ed i metodi per effettuare verifica sono ampiamente e dettagliatamente descritti nel \emph{Piano di Qualifica}.\\
    \subsubsection{Documenti}
      \begin{itemize}
        \item \textbf{TexMaker: }Per la stesura dei documenti è stato utilizzato \glossaryItem{TexMaker} per via dell'integrazione con i dizionari di OpenOffice.org, i quali consentono un controllo ortografico in real-time;
        \item \textbf{Aspell: }strumento per la correzione tipografica dei documenti redatti in \glossaryItem{\LaTeX}.
      \end{itemize}
