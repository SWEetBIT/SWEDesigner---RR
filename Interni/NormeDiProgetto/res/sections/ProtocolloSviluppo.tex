\section{Protocollo per lo sviluppo dell’applicazione}
  Per procedere con una stesura controllata dei documenti e con uno sviluppo controllato del codice si è scelto di adottare
  il sistema di ticketing \textbf{Redmine}.\\
  La scelta di tale piattaforma è spiegata all'interno del capitolo 9.
  In questa sezine si faranno molti riferimenti impliciti al \emph{Piano di Progetto} e al \emph{Piano di  Qualifica}.
  \subsection{Creazione di un nuovo progetto}
    La creazione di un progetto è un compito del \emph{Responsabile di Progetto}.\\
    Un nuovo progetto è una macro-attività formata da molte sotto-attività coordinate da un responsabile.\\
    Per la creazione di un nuovo progetto la prassi da seguire è la seguente:
    \begin{itemize}
      \item Aprire \textbf{Progetti};
      \item Selezionare \textbf{Nuovo Progetto};
      \item Assegnare un \textbf{Nome} breve ma significativo;
      \item Nel caso in cui è necessario creare un sotto-progetto, indicare il nome del progetto padre nell'omonimo campo;
      \item \textbf{Identificativo: }scrivere in minuscolo ed indicare il nome della fase a cui si riferisce (es. SdF-rr);
    \end{itemize}
  \subsection{Creazione ticket}
    I ticket vengono creati da:
    \begin{itemize}
      \item \textbf{\emph{Responsabile di Progetto}: }crea i ticket più importanti che rappresentano le macro-fasi evideziate
      all'interno della pianificazione;
      \item \textbf{\emph{Responsabile di Sotto-progetto}: }crea i ticket per i processi non pianificati inizialmente ma che si rivelano necesari
      per l'avanzamento del sotto-progetto assegnato;
      \item \textbf{\emph{Verificatore}: }crea i ticket per segnalare errori emersi durante il processo di verifica;
    \end{itemize}
    I ticket possono essere di tre categorie:
    \begin{itemize}
      \item \textbf{Ticket di pianificazione: }rappresentano le macro-attività di maggiore importanza e sono organizzati in una gerarchia
      basata sul livello di importanza.\\
      Tali attività vengono create da:
      \bgroup
        \begin{itemize}
          \item \emph{Responsabile di Progetto} che durante la pianificazione individua le attività più importanti da svolgere;
          \item \emph{Responsabile di Sotto-progetto} che durante lo svolgimento dell'attività principale può scomporla in sotto-problemi;
        \end{itemize}
      \egroup
      \item \textbf{Ticket di realizzazione e controllo: }ogni documento, durante la sua stesura, passa attraverso due stadi:
      \bgroup
        \begin{itemize}
          \item \textbf{Realizzazione: }un redattore realizzerà la prima stesura dell'interno documento;
          \item \textbf{Controllo: }un redattore, diverso dal precedente, eseguirà un primo controllo di tutta la parte scritta;
        \end{itemize}
      \egroup
      \item \textbf{Ticket di verifica: }rappresentano gli errori evidenziati dai \emph{Verificatori} durante l'operazione di controllo dell'intero documento;
    \end{itemize}
      \subsubsection{Ticket di pianificazione}
        \begin{itemize}
          \item Selezionare \textbf{Nuova segnalazione} dal menù principale;
          \item \textbf{Tracker: }indicare la natura del ticket:
          \bgroup
            \begin{itemize}
              \item \textbf{Documento: }attività legata alla stesura di un documento;
              \item \textbf{Codifica: }attività legata alla  codifica del software;
              \item \textbf{Verifica: }macro-attività legata alla verifica del prodotto delle macro-attività;
            \end{itemize}
          \egroup
          \item \textbf{Oggetto: }descrizione breve e significativa della natura del ticket;
          \item \textbf{Descrizione: }descrizione comprensibile dell'attività da svolgere;
          \item \textbf{Stato: }Plan;
          \item \textbf{Attività principale: }se si vuole identificare una \textbf{sotto-attività} indicare l'id del ticket padre;
          \item \textbf{Categoria: }\glossaryItem{PDCA} se e solo se il ticket viene generato dal \emph{Responsabile di Progetto};
          \item \textbf{Assegnato a: }indicare il nome del responsabile;
          \item \textbf{Osservatori: }aggiungere eventuali collaboratori;
        \end{itemize}
      \subsubsection{Ticket di realizzazione e controllo}
        \begin{itemize}
          \item Selezionare \textbf{Nuova segnalazione} dal menù principale;
          \item \textbf{Tracker: }indicare la natura del ticket:
          \bgroup
            \begin{itemize}
              \item \textbf{Documento: }attività legata alla stesura di un documento;
              \item \textbf{Codifica: }attività legata alla  codifica del software;
              \item \textbf{Verifica: }macro-attività legata alla verifica del prodotto delle macro-attività;
            \end{itemize}
          \egroup
            \item \textbf{Oggetto: }descrizione breve e significativa della natura del ticket;
          \item \textbf{Descrizione: }descrizione comprensibile dell'attività da svolgere;
          \item \textbf{Stato: }New;
          \item \textbf{Attività principale: }se si vuole identificare una \textbf{sotto-attività} indicare l'id del ticket padre;
          \item \textbf{Inizio: }dare una presunta data di inizio;
          \item \textbf{Scadenza: }dare una presunta data di fine;
          \item \textbf{Assegnato a: }indicare il nome del responsabile;
          \item \textbf{Osservatori: }aggiungere eventuali collaboratori;
        \end{itemize}
      \subsubsection{Ticket di verifica}
        Un \emph{Verificatore} per creare un \emph{ticket di verifica} deve:
        \begin{enumerate}
          \item Assicurarsi che esista all'interno del progetto l'attività \emph{Verifica}.\\
          Su questa attività devono essere presenti due sotto-attività: \emph{Verifica - realizzazione} e \emph{Verifica - approvazione}.\\
          Tutti i ticket devono essere creati come sotto-attività di \emph{Verifica - realizzazione}.
          \item Creare il ticket seguendo le seguenti direttive:
          \bgroup
            \begin{itemize}
              \item Selezionare \textbf{Nuova segnalazione} dal menù principale;
              \item \textbf{Tracker: }Bug;
              \item \textbf{Oggetto: }descrizione breve e significativa della natura del ticket;
              \item \textbf{Descrizione: }descrizione comprensibile dell'attività da svolgere;
              \item \textbf{Stato: }New;
              \item \textbf{Attività principale: }se si vuole identificare una \textbf{sotto-attività} indicare l'id del ticket padre;
              \item \textbf{Assegnato a: }indicare il nome del responsabile;
            \end{itemize}
          \egroup
        \end{enumerate}
      Tutti i campi non segnalati sono da lasciarsi vuoti.\\
      Il compito di assegnare la correzione dell'errore è dato al Responsabile del progetto padre.
    \subsubsection{Dipendenze temporali}
      Dopo la creazione dei ticket è necessario assegnare le \textbf{dipendenze temporali} fra gli stessi.\\
      La procedura da seguire è la seguente:
      \begin{itemize}
        \item Spostarsi su \textbf{Segnalazioni};
        \item Aprire il link alla segnalazione a cui aggiungere la dipendenza;
        \item Nella sezione \textbf{Segnalazioni correlate} premere su \textbf{Aggiungi};
        \item Scegliere \textbf{segue} ed indicare il numero della segnalazione bloccante con eventuali giorni di slack;
      \end{itemize}
      Tutti i campi non segnalati sono da lasciarsi vuoti.
  \subsection{Aggiornamento ticket}
    L'aggiornamento die ticket avviene tramite il cambiamento del loro stato da:
    \begin{itemize}
      \item \textbf{In Progress: }segnala che uno o più membri del gruppo stanno lavorando al completamento di quel ticket.\\
      In questo caso la percentuale di completamento deve essere compresa fra 0\% e 90\%;
      \item \textbf{Closed: }segnala che l'attività è stata conclusa.\\
      La percentuale di completamento in questo caso è fissata a 100\&;
    \end{itemize}
