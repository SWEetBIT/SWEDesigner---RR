\section{Protocollo per lo sviluppo dell’applicazione}
  Per procedere con una stesura controllata dei documenti e con uno sviluppo controllato del codice si è scelto di adottare
  il sistema di ticketing \textbf{Redmine}.\\
  La scelta di tale piattaforma è spiegata all'interno del capitolo 9.
  In questa sezine si faranno molti riferimenti impliciti al \emph{Piano di Progetto} e al \emph{Piano di  Qualifica}.
  \subsection{Creazione di un nuovo progetto}
    La creazione di un progetto è un compito del \emph{Responsabile di Progetto}.\\
    Un nuovo progetto è una macro-attività formata da molte sotto-attività coordinate da un responsabile.\\
    Per la creazione di un nuovo progetto la prassi da seguire è la seguente:
    \begin{itemize}
      \item Aprire \textbf{Progetti};
      \item Selezionare \textbf{Nuovo Progetto};
      \item Assegnare un \textbf{Nome} breve ma significativo;
      \item Nel caso in cui è necessario creare un sotto-progetto, indicare il nome del progetto padre nell'omonimo campo;
      \item \textbf{Identificativo: }scrivere in minuscolo ed indicare il nome della fase a cui si riferisce (es. SdF-rr);
    \end{itemize}
  \subsection{Creazione ticket}
    I ticket vengono creati da:
    \begin{itemize}
      \item \textbf{\emph{Responsabile di Progetto}: }crea i ticket più importanti che rappresentano le macro-fasi evideziate
      all'interno della pianificazione;
      \item \textbf{\emph{Responsabile di Sotto-progetto}: }crea i ticket per i processi non pianificati inizialmente ma che si rivelano necesari
      per l'avanzamento del sotto-progetto assegnato;
      \item \textbf{\emph{Verificatore}: }crea i ticket per segnalare errori emersi durante il processo di verifica;
    \end{itemize}
    I ticket possono essere di tre categorie:
    
