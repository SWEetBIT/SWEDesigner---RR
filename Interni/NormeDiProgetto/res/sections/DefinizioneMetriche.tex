\section{Definizione delle metriche}

Il processo di verifica, per essere informativo, deve esse quantificabile oggettivamente. Le misure rilevate dal processo di verifica devono quindi essere basate sulle metriche stabilite a priori nel \textit{Piano di Qualifica}. Per automatizzare il più possibile il lavoro di misurazione saranno utilizzati degli strumenti automatizzati, con lo scopo di avere un resoconto affidabile e quantitativo che permetta di assicurare il grado di qualità voluto.

  \subsection{Schedule Variance (SV)}
  È un indice che dà informazioni necessarie a determinare se ci si trova in anticipo, in ritardo o in linea alle tempistiche delle attività di progetto. Si tratta di un indicatore di efficacia nei confronti del cliente. Se il risultato di tale metrica risulta positivo significa che il progetto sta procedendo con maggior velocità rispetto a quanto pianificato, viceversa
se negativo. Alla fine del progetto questo indice assumerà il valore 0, perchè in quel momento tutte le attività saranno state realizzate.
La seguente formula dà SV in termini di costo:
  \begin{center}
    \emph{SV = BCWP - BCWS}
  \end{center}
  Dove:
  \begin{itemize}
    \item BCWP = costo totale del lavoro svolto al momento della misurazione;
    \item BCWS = costo totale del lavoro pianificato al momento della misurazione.
  \end{itemize}
  SV ha tre significativi risultati:
  \begin{itemize}
    \item SV>0 indica che si è avanti rispetto alle pianificazione temporale del lavoro;
    \item SV=0 indica che si è in linea alle tempistiche delle attività di progetto;
    \item SV<0 indica che si è in ritardo rispetto alla pianificazione temporale delle attività.
  \end{itemize}

  \subsection{Budget Variance (BV)}
  È un indice che dà informazioni necessarie a determinare se vi sono state più o meno spese rispetto al previsto. Si tratta di un indicatore che ha un valore contabile e finanziario. Se il risultato di tale metrica risulta positivo significa che l'attuazione del progetto sta consumando il proprio budget con minor velocità rispetto a quanto pianificato, viceversa se negativo.

  La seguente formula dà BV in termini di costo:
  \begin{center}
    \emph{BV = BCWS - ACWP}
  \end{center}
  Dove:
  \begin{itemize}
    \item BCWS = costo pianificato per realizzare il lavoro al momento della misurazione;
    \item ACWP = costo totale richiesto per il completamento del lavoro al momento della misurazione.
  \end{itemize}
  BV ha tre significativi risultati:
  \begin{itemize}
    \item BV>0 indica che il progetto sta avendo un costo inferiore rispetto a quanto preventivato;
    \item BV=0 indica che il progetto ha un costo in linea a quanto preventivato;
    \item BV<0 indica che il progetto ha superato il costo preventivato.
  \end{itemize}

  \subsection{Indice Gulpease}
  È un indice di leggibilità di un testo per la lingua italiana.
Rispetto ad altri ha il vantaggio di utilizzare la lunghezza delle parole in lettere anziché in sillabe, semplificandone il calcolo automatico. Permette di misurare la complessità dello stile di un documento.

  Questo indice considera due variabili linguistiche: la lunghezza della parola e la lunghezza della frase rispetto al numero delle lettere.
  La formula per il calcolo dell'indice Gulpease è:
  \begin{center}
    \( 89+\frac{300 * (\emph{numero delle frasi}) - 10 * (\emph{numero delle lettere})}{\emph{numero delle parole}} \)
  \end{center}
  Il risultato è un numero nell'intervallo [0-100], generalmente risulta che:
  \begin{itemize}
    \item inferiore a 80 il testo è difficile da leggere per chi ha la licenza elementare;
    \item inferiore a 60 il testo è difficile da leggere per chi ha la licenza media;
    \item inferiore a 40 il testo è difficile da leggere per chi ha un diploma superiore.
  \end{itemize}

  \subsection{Complessità ciclomatica}
Una metrica per la misura della complessità di funzioni, moduli, metodi o classi di un programma; la quale è calcolata mediate il grafo di controllo di flusso relativo al \glossaryItem{Codice}.
            I nodi del grafo sono gruppi di istruzioni indivisibili. Se nel \glossaryItem{Codice} non sono presenti punti decisionali o cicli, allora la complessità ciclomatica sarà pari a 1.
            Alti valori di complessità ciclomatica implicano una ridotta manutenibilità del codice. Valori bassi di complessità ciclomatica potrebbero però determinare scarsa efficienza dei metodi.
  
  La formula per il
  calcolo della complessità ciclomatica è:
  \begin{center}
    \emph{Cc = a - n + 2p}
  \end{center}
  Dove:
  \begin{itemize}
    \item Cc = indice di complessità ciclomatica;
    \item a = numero di archi nel grafo del \glossaryItem{Codice};
    \item n = numero di nodi nel grafo del \glossaryItem{Codice};
    \item p = numero di componenti connesse (per un singolo programma, \glossaryItem{Metodo} o funzione p è sempre 1).
  \end{itemize}

  \subsection{Variabili non utilizzate e/o non definite}
Rappresenta il numero di variabili che vengono definite, ma non utilizzate, o viceversa. Questo viene considerato pollution, e pertanto considerato inaccettabile. La misurazione avviene mediante un’analisi dell’AST (Abstract Syntax Tree).
  \begin{center}
    \emph{VNU = numero variabili non utilizzate e/o non definite}
  \end{center}

  \subsection{Numero di argomenti per funzione}
  Rappresenta il numero di argomenti di una funzione. Una funzione con troppi argomenti risulta complessa e poco mantenibile; pertanto è necessario che tale numero sia contenuto.
  \begin{center}
    \emph{NAF = numerio argomenti per funzione}
  \end{center}

  \subsection{Linee di codice per linee di commento}
  Rappresenta il rapporto tra le linee di codice e linee di commento, utile per stimare la manutenibilità del codice.
  \begin{center}
    \( LCC = \frac{numero linee di codice}{numero linee di commento} \)
  \end{center}

  \subsection{Copertura del codice}
  Rappresenta la percentuale di istruzioni che sono eseguite durante i test. Maggiore è la percentuale di istruzioni coperte dai test eseguiti, maggiore sarà la probabilità
che le componenti testate abbiano una ridotta quantità di errori. Il valore di tale indice può essere abbassato da metodi molto semplici che non richiedono testing.
  \begin{center}
    \( PCC = {\frac{numero test eseguiti}{numero test pianificati}}*100 \)
  \end{center}
