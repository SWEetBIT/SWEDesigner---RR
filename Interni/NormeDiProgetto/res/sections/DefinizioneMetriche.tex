\section{Definizione delle metriche}

  \subsection{Schedule Variance (SV)}
  È un indice che dà informazioni necessarie a determinare
  se ci si trova in anticipo, in ritardo o in linea alle tempistiche delle attività di
  progetto. La seguente formula dà SV in termini di costo:
  \begin{center}
    \emph{SV = BCWP - BCWS}
  \end{center}
  Dove:
  \begin{itemize}
    \item BCWP = costo totale del lavoro svolto al momento della misurazione;
    \item BCWS = costo totale del lavoro pianificato al momento della misurazione.
  \end{itemize}
  SV ha tre significativi risultati:
  \begin{itemize}
    \item SV>0 indica che si è avanti rispetto alle pianificazione temporale del lavoro;
    \item SV=0 indica che si è in linea alle tempistiche delle attività di progetto;
    \item SV<0 indica che si è in ritardo rispetto alla pianificazione temporale delle attività.
  \end{itemize}

  \subsection{Budget Variance (BV)}
  Indica se vi sono state più o meno spese rispetto al previsto.
  La seguente formula dà BV in termini di costo:
  \begin{center}
    \emph{BV = BCWS - ACWP}
  \end{center}
  Dove:
  \begin{itemize}
    \item BCWS = costo pianificato per realizzare il lavoro al momento della misurazione;
    \item ACWP = costo totale richiesto per il completamento del lavoro al momento della misurazione.
  \end{itemize}
  BV ha tre significativi risultati:
  \begin{itemize}
    \item BV>0 indica che il progetto sta avendo un costo inferiore rispetto a quanto preventivato;
    \item BV=0 indica che il progetto ha un costo in linea a quanto preventivato;
    \item BV<0 indica che il progetto ha superato il costo preventivato.
  \end{itemize}

  \subsection{Indice Gulpease}
  È un indice di leggibilità di un testo per la lingua italiana.
  Questo indice considera due variabili linguistiche: la lunghezza della parola e la lunghezza della frase rispetto al numero delle lettere.
  La formula per il calcolo dell'indice Gulpease è:
  \begin{center}
    \( 89+\frac{300 * (\emph{numero delle frasi}) - 10 * (\emph{numero delle lettere})}{\emph{numero delle parole}} \)
  \end{center}
  Il risultato è un numero nell'intervallo [0-100], generalmente risulta che:
  \begin{itemize}
    \item inferiore a 80 il testo è difficile da leggere per chi ha la licenza elementare;
    \item inferiore a 60 il testo è difficile da leggere per chi ha la licenza media;
    \item inferiore a 40 il testo è difficile da leggere per chi ha un diploma superiore.
  \end{itemize}

  \subsection{Dimensione del prodotto software}
  Rappresenta le dimensioni del prodotto software, è misurata in termini di migliaia di linee di codice (KLOC, Thousands Line Of Code),
  ma da alcuni anni è stata introdotto una nuova misura, legata al numero di funzionalità offerte, e quindi dal valore che il prodotto ha per l’utente.
  Questa seconda misurazione è espressa come numero di punti funzione (FP, Function Points).

  \subsection{Complessità ciclomatica}
  è una metrica per la misura della complessità del \glossaryItem{Codice};
  la quale è calcolata mediate il grafo di controllo di flusso relativo al \glossaryItem{Codice}.
  I nodi del grafo sono gruppi di istruzioni indivisibili. Se nel \glossaryItem{Codice} non sono
  presenti punti decisionali o cicli, allora la complessità ciclomatica sarà pari a 1. La formula per il
  caclolo della complessità ciclomatica è:
  \begin{center}
    \emph{Cc = a - n + 2p}
  \end{center}
  Dove:
  \begin{itemize}
    \item Cc = indice di complessità ciclomatica;
    \item a = numero di archi nel grafo del \glossaryItem{Codice};
    \item n = numero di nodi nel grafo del \glossaryItem{Codice};
    \item p = numero di componenti connesse (per un singolo programma, \glossaryItem{Metodo} o funzione p è sempre 1).
  \end{itemize}

  \subsection{Variabili non utilizzate e/o non definite}
  rappresenta il numero di variabili che vengono definite, ma non utilizzate, o viceversa.
  Questo viene considerato pollution, e pertanto considerato inaccettabile.
  La misurazione avviene mediante un’analisi dell’AST (Abstract Syntax Tree).

  \subsection{Numero di argomenti per funzione}
  Rappresenta il numero di argomenti di una funzione. Una funzione con troppi argomenti
  risulta complessa e poco mantenibile; pertanto è necessario che tale numero sia contenuto.

  \subsection{Linee di codice per linee di commento}
  Rappresenta il rapporto tra le linee di codice e linee di commento, utile per stimare la manutenibilità del codice.

  \subsection{Copertura del codice}
  Rappresenta la percentuale di istruzioni che sono eseguite durante i test. Maggiore è la percentuale di istruzioni coperte dai test eseguiti, maggiore sarà la probabilità
  che le componenti testate abbiano una ridotta quantità di errori. Il valore di tale indice può essere abbassato da metodi molto semplici che non richiedono testing.
