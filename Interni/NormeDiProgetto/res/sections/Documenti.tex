\section{Documenti}
  Questo capitolo descrive tutte le convenzioni scelte ed adottate da SWEetBIT riguardo alla stesura, verifica e approvazione della documentazione da produrre.
  \subsection{Template}
    Per agevolare la redazione di un documento è stato prodotto un template e delle regole da seguire per la stesura degli stessi.\\
    Tale modello e tali regole sono inseriti all'interno di una cartella \path{documents\template} sulla \glossaryItem{repository}
    \subsection{Struttura del documento}
      \subsubsection{Prima pagina}
        Ogni documento è caratterizzato da una prima pagina che contiene le seguenti informazioni sul documento:\\
        \begin{itemize}
          \item Nome del gruppo;
          \item Nome del progetto;
          \item Logo del gruppo;
          \item Titolo del documento;
          \item Versione del documento;
          \item Cognome e nome dei redattori del documento;
          \item Cognome e nome dei verificatori del documento;
          \item Cognome e nome del responsabile approvatore del documento;
          \item Destinazione d’uso del documento;
          \item Lista di distribuzione del documento;
          \item Breve descrizione del documento;
        \end{itemize}
      \subsubsection{Diario delle modifiche}
        La seconda pagina di ogni documento contiene il diario delle modifiche del documento.\\
        Ogni riga del diario delle modifiche contiene:
        \begin{itemize}
          \item Un breve sommario delle modifiche svolte;
          \item Cognome e nome dell’autore;
          \item Data della modifica;
          \item Versione del documento dopo la modifica;
        \end{itemize}
        La tabella è ordinata per data in ordine decrescente, in modo che la prima riga corrisponda alla versione attuale del documento.
      \subsubsection{Indici}
        In ogni documento è presente un indice delle sezioni, un indice delle figure e un indice delle tabelle.
        Nel caso non siano presenti figure o tabelle i rispettivi indici verranno omessi.
      \subsubsection{Formattazione generale delle pagine}
        L’intestazione di ogni pagina contiene:
        \begin{itemize}
          \item Logo del gruppo;
          \item Nome del gruppo;
          \item Nome del progetto;
          \item Sezione corrente del documento;
        \end{itemize}
        A piè di pagina invece è presente:
        \begin{itemize}
          \item Nome e versione del documento;
          \item Pagina corrente nel formato \emph{N} di \emph{T} dove \emph{N} è il numero di pagina corrente e \emph{T} è il numero di pagine totali.
        \end{itemize}
    \subsection{Norme tipografiche}
      Questa sezione racchiude le convenzioni riguardanti tipografia, ortografia e uno stile uniforme per tutti i documenti.
      \subsubsection{Punteggiatura}
        \begin{itemize}
          \item \textbf{Parentesi: }il testo racchiuso tra parentesi non deve aprirsi o chiudersi con un carattere di spaziatura e non deve terminare con un carattere di punteggiatura;
          \item \textbf{Punteggiatura: }un carattere di punteggiatura non deve mai seguire un carattere di spaziatura;
          \item \textbf{Lettere maiuscole: }le lettere maiuscole vanno poste solo dopo il punto, il punto di domanda, il punto esclamativo e all’inizio di ogni elemento di un elenco puntato,
          oltre che dove previsto dalla lingua italiana. È inoltre utilizzata l’iniziale maiuscola nel nome del team, del progetto, dei documenti, dei ruoli di progetto, delle fasi di
          lavoro e nelle parole Proponente e Committente.
        \end{itemize}
