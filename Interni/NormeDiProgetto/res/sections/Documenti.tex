\section{Documenti}
  Questo capitolo descrive tutte le convenzioni scelte ed adottate da SWEet BIT riguardo alla stesura, verifica e approvazione della documentazione da produrre.
  \subsection{Template}
    Per agevolare la redazione di un documento è stato prodotto un template e delle regole da seguire per la stesura degli stessi.\\
    Tale modello e tali regole sono inseriti all'interno di una cartella \path{documents\template} sulla \glossaryItem{repository}
    \subsection{Struttura del documento}
      \subsubsection{Prima pagina}
        Ogni documento è caratterizzato da una prima pagina che contiene le seguenti informazioni sul documento:\\
        \begin{itemize}
          \item Nome del gruppo;
          \item Nome del progetto;
          \item Logo del gruppo;
          \item Titolo del documento;
          \item Versione del documento;
          \item Cognome e nome dei redattori del documento;
          \item Cognome e nome dei verificatori del documento;
          \item Cognome e nome del responsabile approvatore del documento;
          \item Destinazione d’uso del documento;
          \item Lista di distribuzione del documento;
          \item Breve descrizione del documento.
        \end{itemize}
      \subsubsection{Diario delle modifiche}
        La seconda pagina di ogni documento contiene il diario delle modifiche del documento.\\
        Ogni riga del diario delle modifiche contiene:
        \begin{itemize}
          \item Un breve sommario delle modifiche svolte;
          \item Cognome e nome dell’autore;
          \item Data della modifica;
          \item Versione del documento dopo la modifica.
        \end{itemize}
        La tabella è ordinata per data in ordine decrescente, in modo che la prima riga corrisponda alla versione attuale del documento.
      \subsubsection{Indici}
        In ogni documento è presente un indice delle sezioni, un indice delle figure e un indice delle tabelle.
        Nel caso non siano presenti figure o tabelle i rispettivi indici verranno omessi.
      \subsubsection{Formattazione generale delle pagine}
        L’intestazione di ogni pagina contiene:
        \begin{itemize}
          \item Logo del gruppo;
          \item Nome del gruppo;
          \item Nome del progetto;
          \item Sezione corrente del documento;
        \end{itemize}
        A piè di pagina invece è presente:
        \begin{itemize}
          \item Nome e versione del documento;
          \item Pagina corrente nel formato \emph{N} di \emph{T} dove \emph{N} è il numero di pagina corrente e \emph{T} è il numero di pagine totali.
        \end{itemize}
    \subsection{Norme tipografiche}
      Questa sezione racchiude le convenzioni riguardanti tipografia, ortografia e uno stile uniforme per tutti i documenti.
      \subsubsection{Punteggiatura}
        \begin{itemize}
          \item \textbf{Parentesi: }Il testo racchiuso tra parentesi non deve aprirsi o chiudersi con un carattere di spaziatura e non deve terminare con un carattere di punteggiatura;
          \item \textbf{Punteggiatura: }Un carattere di punteggiatura non deve mai esser preceduto da un carattere di spaziatur
          \item \textbf{Lettere maiuscole: }Le lettere maiuscole vanno poste solo dopo il punto, il punto di domanda, il punto esclamativo e all’inizio di ogni elemento di un elenco puntato ed
          oltre a dove sia previsto dalla lingua italiana. È inoltre utilizzata l’iniziale maiuscola nel nome del team, del progetto, dei documenti, dei ruoli di progetto, delle fasi di
          lavoro e nelle parole Proponente e Committente.
        \end{itemize}
      \subsubsection{Stile di testo}
        \begin{itemize}
          \item \textbf{Corsivo: }Il corsivo deve essere utilizzato nei seguenti casi:
            \bgroup
              \begin{itemize}
                \item \textbf{Citazioni: }Quando si deve citare una frase questa sarà scritta in corsivo;
                \item \textbf{Nomi particolari: }Il corsviso deve essere utilizzato quando ci si rierisce a figure particolari (es. \emph{Analista});
                \item \textbf{Documenti: }Il corsivo deve essere utilizzato quando ci si riferisce a documenti particolari (es. \emph{Glossario});
                \item \textbf{Altri casi: }Il corsivo sarà utilizzato in tutte quelle situazioni in cui è necessario dare rilievo ad una parola o passaggio
                significativo;
              \end{itemize}
            \egroup
          \item \textbf{Grassetto: }Il grassetto deve essere utilizzato nei seguenti casi:
            \bgroup
              \begin{itemize}
                \item \textbf{Elenchi puntati: }In questo caso il grassetto può essere utilizzato per mettere in evidenza i punti sviluppati nella loro continuazione;
                \item \textbf{Altri casi: }Il grasstto dovrà essere sempre utilizzato per evidenziare passaggi o parole chiave;
              \end{itemize}
            \egroup
          \item \textbf{\textbackslash path: }Il comando \textbackslash path deve essere utilizzato per indicare i percorsi all'interno di directory;
          \item \textbf{Maiuscolo: }L'utilizzo di parole completamente in maiuscolo è riservato solo ed esplusivamente alle sigle o alle macro \LaTeX\ riportate
          nei documenti;
          \item \textbf{\LaTeX\ : }Ogni riferimento a \LaTeX\ deve essere scritto utilizzano la macro \textbackslash LaTeX;
        \end{itemize}
      \subsubsection{Composizione del testo}
        \begin{itemize}
          \item \textbf{Elenchi puntati: }Ogni punto dell'elenco puntato deve essere scritto in grassetto e con la prima lettera in maiuscolo.\\
            Nella definizione del punto la prima lettera dovrà essere maiuscola ad eccezione di casi isolati (es. nome di file) e dovrà terminare sempre con un ";" ; mentre l'ultimo elemento terminerà con un "." ;
          \item \textbf{Note a piè di pagina: }Ogni nota dovrà cominciare con l’iniziale della prima parola maiuscola e non deve essere preceduta da alcun carattere di spaziatura.\\
            Ogni nota deve terminare con un punto.
        \end{itemize}
      \subsubsection{Formati}
        \begin{itemize}
          \item \textbf{Percorsi: }Per tutti gli indirizzi e-mail e web completi dovrà essere utilizzato il comando \LaTeX\ \textbackslash url mentre per i percorsi
            relativi si utilizzerà il comando \textbackslash path;
          \item \textbf{Date: }Tutte le date presenti all'interno della documentazione devono seguire la notazione definiti nello standard \glossaryItem{ISO} 8601:2004:
            \begin{center}
              AAAA-MM-GG\\
            \end{center}
            dove:
            \bgroup
              \begin{itemize}
                \item AAAA: rappresenta l'anno utilizzando quattro cifre;
                \item MM: rappresenta il mese utilizzando due cifre;
                \item GG: rappresenta il giorno utilizzando due cifre.
              \end{itemize}
            \egroup
          \item \textbf{Nomi propri: }l'utilizzo dei nomi propri dei membri del team (e non) deve seguire la notazione "Cognome Nome";
          \item \textbf{Nome gruppo: }ci si riferirà al gruppo solo come "SWEet BIT";
          \item \textbf{Nome del Proponente: }ci si riferirà al Proponente come "Zucchetti s.r.l" o semplicemente come "Proponente";
          \item \textbf{Nome del Committente: }ci si riferià al Committente come "prof. Vardanega Tullio" o semplicemente come "Committente";
          \item \textbf{Nome del progetto: }ci si riferirà al progetto solo come "SWEDesigner".
        \end{itemize}
      \subsubsection{Sigle}
        Le sigle dei documenti potranno essere utilizzate solo ed esclusivamente all'interno di tabelle o diagrammi. Sono previste le seguenti sigle:
        \begin{itemize}
          \item \textbf{AdR} = Analisi dei Requisiti;
          \item \textbf{GL} = Glossario;
          \item \textbf{NdP} = Norme di Progetto;
          \item \textbf{PdP} = Piano di Progetto;
          \item \textbf{PdQ} = Piano di Qualifica;
          \item \textbf{SdF} = Studio di Fattibilità;
          \item \textbf{ST} = Specifica Tecnica;
          \item \textbf{RA} = Revisione d'Accettazione;
          \item \textbf{RP} = Revisione di Progettazione;
          \item \textbf{RQ} = Revisione di Qualifica;
          \item \textbf{RR} = Revisione dei Requisiti.
        \end{itemize}
    \subsection{Componenti grafiche}
      \subsubsection{Tabelle}
        Ogni tabella presente all’interno dei documenti dev’essere accompagnata da una didascalia, in cui deve comparire un numero identificativo incrementale per la tracciabilità
        della stessa all’interno del documento.
      \subsubsection{Immagini}
        Le immagini da includere all'interno del documento devono avere preferibilmente il formato Portable Network Graphics (\glossaryItem{PNG}).
    \subsection{Classificazione dei documenti}
      \subsubsection{Documenti formali}
        Un documento viene definito formale quando viene approvato dal \emph{Responsabile di Progetto} ed è quindi pronto per essere inviato ai richiedenti.\\
        Per raggiungere questo stato il documento deve seguire l'iter descritto nel \emph{Piano di Qualifica} e nel paragrafo \ref{subsec:ciclovita} riguardante il ciclo di vita dei documenti.
      \subsubsection{Documenti informali}
        Un documento è definito informale fino a quando non approvato dal \emph{Responsabile di Progetto}, fino ad allora il suo uso è da considerarsi unicamente interno.
      \subsubsection{Versionamento}
        La documentazione prodotta deve essere corredata dal numero di versione attuale utilizzando la codifica:
        \begin{center}
          \emph{v.X.Y.Z}
        \end{center}
        dove:
        \begin{itemize}
          \item X: indica il numero crescente di uscite formali del documento;
          \item Y: indica il numero crescente di modifiche sotanziali al documento;
          \item Z: indica il numero crescente di modifiche minori apportate al documento;
        \end{itemize}
    \subsection{Ciclo di vita}\label{subsec:ciclovita}
      Ogni documento prodotto segue un preciso iter che scandisce le fasi in cui si trova in ogni istante. Un documento può trovarsi in tre stati diversi:\\
      \begin{itemize}
        \item \textbf{In lavorazione: }Un documento entra in questa fase nel momento della sua crezione e vi rimane per tutto il periodo della sua stesura
          o per eventuali successive modifiche;
        \item \textbf{Da verificare: }Un documento entra in questa fase alla fine della sua stesura quando entra in possesso dei verificatori che avranno il compito
          di individuare e correggere eventuali errori sintattici o semantici;
        \item \textbf{Approvato: }Un documento entra in questa fase una volta che il \emph{Responsabile di Progetto} lo ha approvato dopo la fase di verifica.\\
          L'approvazione sancisce la fine del ciclo di vita del documento per la data versione.
      \end{itemize}
      Ogni fase del ciclo di vita può essere affrontata anche più volte da parte di un documento.
