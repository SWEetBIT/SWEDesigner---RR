\section{Riunioni}
  \subsection{Frequenza}
    Tutte le riunioni interne, salvo casi eccezionali, si svolgeranno settimanalmente mentre tutte quelle esterne saranno convocate solo quando sorgerà la necessità di contattare il Proponente o con il Committente.

    \subsubsection{Interna}
      Il \emph{Responsabile di Progetto} ha il compito di convocare le riunioni generali valutando, di volta in volta, la possibilità di anticipare o posticipare la data
      designata per la riunione del gruppo.\\
      Qualora un membro del gruppo lo ritenesse necessario potrà fare richiesta, attraverso il gruppo \textbf{Telegram}, di una riunione extra.\\
      È auspicabile, infine, che diversi membri del gruppo possano organizzarsi fra di loro per svolgere alcuni compiti che non richiedono la presenza  del gruppo di lavoro
      al completo. Ad esempio, è interessante e utile la collaborazione fra \emph{Progettista} e \emph{Analista}, senza che vengano coinvolte altre persone esterne ai compiti
      da loro svolti.\\
      Il responsabile avviserà tutti i membri del gruppo attraverso un messaggio sul gruppo \textbf{Telegram} che verrà fissato in alto e conterrà luogo, data ed ora della riunione.\\
      Ogni membro del gruppo è tenuto a confermare o meno la sua presenza nelle 24h successive. Il responsabile è tenuto ad avvertire telefonicamente tutti i membri che
      non hanno ancora risposto al messaggio.\\
      Ogni cambiamento nell'orario di convocazione deve essere comunicato per tempo dal responsabile attraverso le modalità sopra elencate.
    \subsubsection{Esterna}
      Concordando con gli altri membri del gruppo la necessità di effettuare una riuinione con il \glossaryItem{Proponente} o con il \glossaryItem{Committente}, il \emph{Responsabile di Progetto}
      si metterà in contatto con i diretti interessati per fissare una data consona a tutti.
    \subsection{Svolgimento riunione}
      All’apertura della riunione, verificata la presenza dei membri previsti, viene scelto un segretario che avrà il compito di annotare ogni argomento trattato e di redigere il verbale
      dell’assemblea, che dovrà poi essere inviato ai restanti componenti del gruppo.\\
      Tutti i partecipanti devono osservare un comportamento consono al miglior svolgimento della riunione e al raggiungimento degli obbiettivi della stessa. Il segretario deve inoltre
      controllare che venga seguito l’ordine del giorno in modo da non tralasciare alcun punto.
    \subsection{Verbale}
      \subsubsection{Riunione interna}
        Il verbale di una riunione interna è un documento informale che traccia semplicemente tutti gli argomenti trattati all'interno della riunione.\\
        Verrà redatto dal segretario della riuione, ruolo scelto di volta in volta e a rotazione fra i presenti, e dovrà essere condiviso attraverso il gruppo \textbf{Telegram}
        per essere a disposizione, in qualsiasi momento (\textbf{Telegram} offre la possibilità di tracciare istantaneamente i media condivisi) da ogni membro del gruppo.\\
        Il verbale dovrà essere inoltre inviato via e-mail ad ogni componente del gruppo il quale avrà cura di mantenerlo localmente al fine di avere sempre a disposizione
        gli argomenti trattati nel corso di una riunione.
      \subsubsection{Esterna}
        Il verbale generato da una riunione esterna con il Proponente o il Committente è un documento ufficiale che può assumere il valore di normativo, quindi deve essere redatto
        seguendo dei criteri specifici.\\
        Per agevolarne la scrittura è stato creato un template \LaTeX\ che ne definisce la struttura.
        Vi è, ovviamente, l'obbligo di seguire tale schema per la stesura del verbale di una riunione esterna che dovrà poi essere inviato a tutti i membri del gruppo seguendo
        le stesse regole del verbale per una riunione interna.
