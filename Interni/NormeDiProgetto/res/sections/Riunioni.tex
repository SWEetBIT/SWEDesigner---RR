\section{Riunioni}
  \subsection{Frequenza}
    Tutte le riunioni interne, salvo casi eccezionali, si svolgeranno settimanalmente mentre tutte quelle esterne saranno convocate solo qualora se ne sentisse il bisogno.
  \subsection{Convocazione riunione}
    \subsubsection{Interna}
      Il \emph{Responsabile di Progetto} ha il compito di convocare le riunioni generali valutando, di volta in volta, la possibilità di anticipare o posticipare la data
      designata per la riunione del gruppo.\\
      Qualora un membro del gruppo lo ritenesse necessario potrà fare richiesta, attraverso il gruppo \textbf{Telegram}, di una riunione extra.\\
      È auspicabile, infine, che diversi membri del gruppo possano organizzarsi fra di loro per svolgere alcuni coompiti che non richiedono la presenza  del gruppo di lavoro
      al completo. Ad esempio è interessante e utile la collaborazione fra \emph{Progettista} e \emph{Analista} senza che vengano coinvolte altre persone esterne ai compiti
      da loro svolti.\\
      Il responsabile avviserà tutti i membri del gruppo attraverso un messaggio sul gruppo \textbf{Telegram} che verrà fissato in alto e conterrà luogo, data ed ora della riunione.\\
      Ogni membro del gruppo è tenuto a confermare o meno la sua presenza nelle 24h successive. Il responsabile è tenuto ad avvertire telefonicamente tutti i membri che
      non hanno ancora risposto al messaggio.\\
      Ogni cambiamento nell'orario di convocazione deve essere comunicato per tempo dal responsabile attraverso le modalità sopra elencate.
    \subsubsection{Esterna}
      Concordando con gli altri membri del gruppo la necessità di effettuare una riuinione con il \glossaryItem{proponente} o con il \glossaryItem{committente}, il \emph{Responsabile di Progetto}
      si metterà in contatto con i diretti interessati per fissare una data che metta tutti quanti d'accordo.
    \subsection{Svolgimento riunione}
      All’apertura della riunione, verificata la presenza dei membri previsti, viene scelto un segretario che avrà il compito di annotare ogni argomento trattato e di redigere il verbale
      dell’assemblea, che dovrà poi essere inviato ai restanti componenti del gruppo.\\
      Tutti i partecipanti devono osservare un comportamento consono al miglior svolgimento della riunione e al raggiungimento degli obbiettivi della stessa. Il segretario deve inoltre
      controllare che venga seguito l’ordine del giorno in modo da non tralasciare alcun punto.
    \subsection{Verbale}
      \subsubsection{Riunione interna}
      \subsubsection{Esterna}
