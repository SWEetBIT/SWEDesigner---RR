\section{Introduzione}
  \subsection{Scopo del documento}
          Lo scopo di questo documento è quello di descrivere le motivazioni dietro la scelta del capitolato C6, SWEDesigner, da parte del gruppo SWEt BIT.
  \subsection{Scopo del Prodotto}
          Lo scopo del progetto è la realizzazone di una \glossaryItem{Web App} che fornisca all'utente un \glossaryItem{UML} \glossaryItem{Designer}, con il quale riuscire a disegnare correttamente diagrammi delle classi
          e descrivere il comportamento dei metodi interni alle stesse attraverso l'utilizzo di diagrammi delle attività.
          La \glossaryItem{Web App} permetterà all'utente di generare codice Java dal diagramma disegnato ed eventualmente andare a ritoccarne il risultato al fine di ottenere un codice
          eseguibile, funzionante e funzionale.
  \subsection{Glossario}
          Con lo scopo di evitare ambiguità di linguaggio e di massimizzare la comprensione dei documenti, il
          gruppo ha steso un documento interno: il \emph{Glossario v1.0.0}. In esso saranno definiti, in modo
          chiaro e conciso, i termini che possono causare ambiguità o incomprensione del testo.
  \subsection{Riferimenti}
    \subsubsection{Informativi}
      \begin{itemize}
        \item \textbf{Capitolato d'appalto C1:} APIM: An API Market Platform \\
        \url{http://www.math.unipd.it/~tullio/IS-1/2016/Progetto/C1.pdf}
        \item \textbf{Capitolato d'appalto C2:} AtAVi: Accoglienza tramite Assistente Virtuale \\
        \url{http://www.math.unipd.it/~tullio/IS-1/2016/Progetto/C2p.pdf}
        \item \textbf{Capitolato d'appalto C3:} DeGeOP: A Designer and Geo-localizer \glossaryItem{Web App} for
        Organizational Plants \\
        \url{http://www.math.unipd.it/~tullio/IS-1/2016/Progetto/C3p.pdf}
        \item \textbf{Capitolato d'appalto C4:} eBread: applicazione di lettura per dislessici \\
        \url{http://www.math.unipd.it/~tullio/IS-1/2016/Progetto/C4p.pdf}
        \item \textbf{Capitolato d'appalto C5:} Monolith: an interactive bubble provider \\
        \url{http://www.math.unipd.it/~tullio/IS-1/201/Progetto/C5p.pdf}
        \item \textbf{Capitolato d'appalto C6:} SWEDesigner: editor di diagrammi \glossaryItem{UML} con generazione di codice \\
        \url{http://www.math.unipd.it/~tullio/IS-1/2016/Progetto/C6p.pdf}
      \end{itemize}
    \subsubsection{Normativi}
      \begin{itemize}
        \item \textbf{Norme di progetto:} \emph{Norme di progetto v1.0.0}
      \end{itemize}
