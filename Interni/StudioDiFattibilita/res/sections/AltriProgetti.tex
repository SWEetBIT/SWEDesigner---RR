\section{Altri capitolati}
  \subsection{Capitolato C1 - An API Market Platform (APIM)}
    \subsubsection{Valutazione Generale}
    Il capitolato propone la creazione di una \glossaryItem{Web App}, che consiste in un \glossaryItem{API} market in grado di registrare, consultare ed effettuare operazioni di compravendita di \glossaryItem{microservizi}.
    Il gruppo ha ritenuto il capitolato C1 fattibile e stimolante per la possibilità di interagire con una tecnologia di recente espansione, ovvero l'architettura a \glossaryItem{microservizi}. Tuttavia non ha suscitato molto interesse la limitazione delle tecnologie necessarie, si è deciso di orientare la propria scelta su altri capitolati.
    \subsubsection{Potenziali Criticità}
     \begin{itemize}
      \item Difficoltà nel garantire la correttezza delle \glossaryItem{API} registrate dagli utenti.
     \end{itemize}
  \subsection{Capitolato C2 - Accoglienza tramite Assistente Virtuale (AtAVi)}
    \subsubsection{Valutazione Generale}
    Il capitolato propone la creazione di una \glossaryItem{Web App} che permetta, ad un ospite dell'ufficio dei proponenti, di interrogare un assistente virtuale per annunciare la propria presenza, in modo che l'applicativo lo accolga e comunichi l'arrivo a chi di dovere.
    Il gruppo ha ritenuto il capitolato C2 molto affascinante ed eccitante, sopratutto perché si affronta la tematica dell'\glossaryItem{Intelligenza Artificiale}, una tematica fortemente attuale e che è destinata a diventare sempre più fondamentale in innumerevoli settori. Nonostante ciò, il gruppo ha optato per altri capitolati poichè l'inesperienza dei componenti su un argomento così complesso avrebbe potuto aumentare notevolmente la difficoltà del capitolato fuoriuscendo obbiettivi iniziali di quest'ultimo.
     \subsubsection{Potenziali Criticità}
      \begin{itemize}
       \item Difficoltà nel creare un programma di \glossaryItem{IA} efficiente;
       \item Scarse conoscenze riguardo gli \glossaryItem{SDK} per assistenti virtuali, quindi difficoltà nell'effettuare paragoni ed analisi;
       \item Conoscenze basilari solamente da parte di alcuni membri del gruppo di NodeJS, con conseguente incremento del tempo per l'apprendimento dello stesso.
      \end{itemize}
  \subsection{Capitolato C3 - A Designer and Geo-localizer Web App for Organizational Plants (DeGeOP)}
    \subsubsection{Valutazione Generale}
    Il capitolato richiede la creazione di un'interfaccia \glossaryItem{Web App}, erogabile anche su dispositivi mobili, per inserire i processi produttivi delle aziende (macchinari, magazzini, fornitori, distributori) su mappa geografica e per disegnare i vari scenari di danno che possono interessare l'azienda. 
    Questo capitolato non è stato ritenuto interessante dal gruppo sia dal punto di vista del dominio applicativo, sia delle tecnologie da utilizzare. Di conseguenza si è preferito scegliere altro.
    \subsubsection{Potenziali Criticità}
    \begin{itemize}
     \item Difficoltà di definizione di tutti gli scenari di danno possibili.
    \end{itemize}
\subsection{Capitolato C4 - Applicazione di lettura per dislessici (eBread)}
    \subsubsection{Valutazione Generale}
    L'obiettivo di questo capitolato è quello di realizzare un'applicazione in ambiente \glossaryItem{Android} che agevoli la lettura alle persone affette da dislessia, grazie all'aiuto di tecnologie appropriate, fra cui la sintesi vocale.
Il gruppo ha deciso di non approfondire questo capitolato perché, considerando l'alto numero di applicazioni appartenenti allo stesso dominio, con il tempo a disposizione sarebbe stato complicato ottenere innovazioni degne di nota. Si è quindi preferito puntare su capitolati più originali.
\subsection{Potenziali Criticità}
 \begin{itemize}
 \item Difficoltà di implementazione di un motore di sintesi vocale che sia sincronizzato con il testo;
 \item Difficoltà di implementazione di supporto multilingua;
 \item Scarsa conoscenza da parte del gruppo delle tecnologie da utilizzare.
 \end{itemize}
    
\subsection{Capitolato C5 - An interactive bubble provider (Monolith)}
    \subsubsection{Valutazione Generale}
    Il capitolato prevede la creazione di un \glossaryItem{framework} che permetta l'istanziazione delle cosiddette bolle per la piattaforma di \glossaryItem{Web Chat} denominata Rocket.chat, dove per bolle si intendono delle funzionalità che possono venire aggiunte alla piattaforma senza nessuna nuova installazione.
    Il gruppo ha reputato questo capitolato poco interessante dato che ormai esistono numerose piattaforme di Web Chat affermate e note a milioni di utenti, quindi si sarebbe difficilmente arrivati ad una vera innovazione.
    \subsubsection{Potenziali Criticità}
    \begin{itemize}
     \item Difficoltà di contatto con i proponenti, vista la locazione della loro sede.    
    \end{itemize}
 