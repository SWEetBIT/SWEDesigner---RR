\subsubsection{UC1 - Registrazione} 
\label{sssec:UC1} 
\begin{itemize} 
\item \textbf{Attori}: Utente non autenticato.
\item \textbf{Descrizione}: L'attore desidera effettuare l'operazione di registrazione. Vengono richiesti dal sistema un username, univoco e conforme alle richieste, una password conforme alle richieste e una mail che rispetti il pattern predefinito.;
\item \textbf{Precondizione}: Il sistema richiede all'attore le informazioni necessarie per effettuare
la registrazione.;
\item \textbf{Postcondizione}: Il sistema ha elaborato i dati inseriti dall'attore e, nel caso la registrazione
sia avvenuta con successo, gli ha concesso la possibilità di accedere al sistema.
In caso contrario il sistema ha visualizzato un messaggio che illustra il tipo di errore che è stato commesso.;
\item \textbf{Scenario principale}: \begin{enumerate}\item Inserisci Username (UC1.1);\item Inserisci Password (UC1.2);\item Inserisci Email (UC1.3);\item Conferma registrazione (UC1.4);\item Messaggio errore: "Username non conforme" (UC1.5);\item Messaggio errore: "Password non conforme" (UC1.6);\item Messaggio errore: "Email non conforme" (UC1.7). 
 \end{enumerate}
\end{itemize} 
\subsubsection{UC1.1 - Inserisci Username} 
\label{sssec:UC1.1} 
\begin{itemize} 
\item \textbf{Attori}: Utente non autenticato.
\item \textbf{Descrizione}: L’attore inserisce l'username: deve essere univoco all'interno del sistema e deve essere alfanumerico.;
\item \textbf{Precondizione}: L'attore ha selezionato l'opzione di registrazione e non ha ancora inserito un username.;
\item \textbf{Postcondizione}: L'attore ha inserito un username conforme alle richieste del sistema.;
\end{itemize} 
\subsubsection{UC1.2 - Inserisci Password} 
\label{sssec:UC1.2} 
\begin{itemize} 
\item \textbf{Attori}: Utente non autenticato.
\item \textbf{Descrizione}: L’attore inserisce la password: deve essere di tipo alfanumerico e può contenere caratteri di punteggiatura.;
\item \textbf{Precondizione}: L'attore ha selezionato l'opzione di registrazione e non ha ancora inserito una password.;
\item \textbf{Postcondizione}: L'attore ha inserito una password conforme alle richieste del sistema.;
\end{itemize} 
\subsubsection{UC1.3 - Inserisci Email} 
\label{sssec:UC1.3} 
\begin{itemize} 
\item \textbf{Attori}: Utente non autenticato.
\item \textbf{Descrizione}: L’attore inserisce l'email, che deve essere scritta in maniera corretta.;
\item \textbf{Precondizione}: L'attore ha selezionato l'opzione di registrazione e non ha ancora inserito un email.;
\item \textbf{Postcondizione}: L'attore ha inserito un'email nel modo corretto.;
\end{itemize} 
\subsubsection{UC1.4 - Conferma registrazione} 
\label{sssec:UC1.4} 
\begin{itemize} 
\item \textbf{Attori}: Utente non autenticato.
\item \textbf{Descrizione}: Dopo che un attore ha effettuato correttamente una registrazione, gli viene inviato un messaggio di conferma.;
\item \textbf{Precondizione}: L'attore ha scelto di effettuare l'operazione di registrazione.;
\item \textbf{Postcondizione}: L'utente ha ricevuto un messaggio di conferma dell'avvenuta registrazione.;
\end{itemize} 
\subsubsection{UC1.5 - Messaggio errore: "Username non conforme"} 
\label{sssec:UC1.5} 
\begin{itemize} 
\item \textbf{Attori}: Utente non autenticato.
\item \textbf{Descrizione}: L'attore ha inserito un username non conforme e gli viene quindi comunicato l'errore.;
\item \textbf{Precondizione}: L'attore ha selezionato l'opzione di registrazione e ha inserito un username.;
\item \textbf{Postcondizione}: L'attore ha inserito un username non conforme e ha ricevuto la comunicazione dell'errore.;
\end{itemize} 
\subsubsection{UC1.6 - Messaggio errore: "Password non conforme"} 
\label{sssec:UC1.6} 
\begin{itemize} 
\item \textbf{Attori}: Utente non autenticato.
\item \textbf{Descrizione}: L'attore ha inserito una password non conforme e gli viene quindi comunicato l'errore.;
\item \textbf{Precondizione}: L'attore ha selezionato l'opzione di registrazione e ha inserito una password.;
\item \textbf{Postcondizione}: L'attore ha inserito una password non conforme e ha ricevuto la comunicazione dell'errore.;
\end{itemize} 
\subsubsection{UC1.7 - Messaggio errore: "Email non conforme"} 
\label{sssec:UC1.7} 
\begin{itemize} 
\item \textbf{Attori}: Utente non autenticato.
\item \textbf{Descrizione}: L'attore ha inserito un'email con una sintassi non valida e gli viene quindi comunicato l'errore.;
\item \textbf{Precondizione}: L'attore ha selezionato l'opzione di registrazione e ha inserito un'email.;
\item \textbf{Postcondizione}: L'attore ha inserito un'email non conforme e ha ricevuto la comunicazione dell'errore.;
\end{itemize} 
\subsubsection{UC2 - Autenticazione} 
\label{sssec:UC2} 
\begin{itemize} 
\item \textbf{Attori}: Utente non autenticato.
\item \textbf{Descrizione}: L'attore che già in possesso delle credenziali per accedere al sistema, potrà effettuare l'operazione di autenticazione inserendo l'username o l'email e la password utilizzati per l'autenticazione. Nel caso l’attore abbia perso la password o se la sia dimenticata, il sistema fornisce la possibilità di resettarla.;
\item \textbf{Precondizione}: L'attore decide di autenticarsi e il sistema richiede l'inserimento dei dati necessari per l'autenticazione.;
\item \textbf{Postcondizione}: L'attore ha avuto accesso alle funzionalità del sistema in caso l'autenticazione sia avvenuta con successo. In caso contrario il sistema ha visualizzato un messaggio d'errore.;
\item \textbf{Scenario principale}: \begin{enumerate}\item Inserisci Username/Email (UC2.1);\item Inserisci Password (UC2.2);\item Password dimenticata (UC2.3);\item Messaggio errore: "Username/Email o password errato" (UC2.4);\item Invio password per email (UC2.5). 
 \end{enumerate}
\end{itemize} 
\subsubsection{UC2.1 - Inserisci Username/Email} 
\label{sssec:UC2.1} 
\begin{itemize} 
\item \textbf{Attori}: Utente non autenticato.
\item \textbf{Descrizione}: Durante la fase di autenticazione viene richiesto all'attore il proprio username o la propria email.;
\item \textbf{Precondizione}: L'attore ha selezionato l'opzione di autenticazione e non ha ancora inserito il proprio username o la propria email.;
\item \textbf{Postcondizione}: L'attore ha inserito il proprio username o la propria email.;
\end{itemize} 
\subsubsection{UC2.2 - Inserisci Password} 
\label{sssec:UC2.2} 
\begin{itemize} 
\item \textbf{Attori}: Utente non autenticato.
\item \textbf{Descrizione}: Durante la fase di autenticazione viene richiesta all'attore la propria password.;
\item \textbf{Precondizione}: L'utente ha selezionato l'opzione di autenticazione e non ha ancora inserito la propria password.;
\item \textbf{Postcondizione}: L'attore ha inserito la propria email.;
\end{itemize} 
\subsubsection{UC2.3 - Password dimenticata} 
\label{sssec:UC2.3} 
\begin{itemize} 
\item \textbf{Attori}: Utente non autenticato.
\item \textbf{Descrizione}: L'attore ha smarrito oppure ha dimenticato la propria password e il sistema fornisce uno strumento per fornirgliene una temporanea.;
\item \textbf{Precondizione}: L'attore possiede un account all'interno del sistema ma non è più in possesso della password.;
\item \textbf{Postcondizione}: L'utente ha ricevuto una email contenente la nuova password temporanea.;
\end{itemize} 
\subsubsection{UC2.4 - Messaggio errore: "Username/Email o password errato"} 
\label{sssec:UC2.4} 
\begin{itemize} 
\item \textbf{Attori}: Utente non autenticato.
\item \textbf{Descrizione}: L'attore, già in possesso delle credenziali d'accesso, tenta di autenticarsi, ma l'operazione non va a buon fine e viene visualizzato un messaggio d'errore.;
\item \textbf{Precondizione}: L'attore ha intenzione di effettuare l'operazione di autenticazione ed ha inserito il proprio username o la propria email.;
\item \textbf{Postcondizione}: Il sistema ha ricevuto informazioni errate e quindi ha mostrato un messaggio d'errore.;
\end{itemize} 
\subsubsection{UC2.5 - Invio password per email} 
\label{sssec:UC2.5} 
\begin{itemize} 
\item \textbf{Attori}: Utente non autenticato.
\item \textbf{Descrizione}: Viene inviata una password temporanea all'attore per permettergli di accedere al sistema.;
\item \textbf{Precondizione}: L'attore ha inserito il proprio username o la propria password.;
\item \textbf{Postcondizione}: Il sistema ha inviato all'attore una mail contenente la password temporanea.;
\end{itemize} 
\subsubsection{UC3 - Gestione Profilo} 
\label{sssec:UC3} 
\begin{itemize} 
\item \textbf{Attori}: Utente Autenticato.
\item \textbf{Descrizione}: L' attore vuole gestire il suo username, la sua password o il suo indirizzo e-mail.;
\item \textbf{Precondizione}: L' attore ha già effettuato l'accesso e l'applicazione rende disponibile la voce Gestione Profilo.;
\item \textbf{Postcondizione}: L' applicazione visualizza i dati dell'attore con gli opportuni pulsanti di modifica.;
\item \textbf{Scenario principale}: \begin{enumerate}\item Modifica Username (UC3.1);\item Modifica Password (UC3.2);\item Modifica Email (UC3.3). 
 \end{enumerate}
\end{itemize} 
\subsubsection{UC3.1 - Modifica Username} 
\label{sssec:UC3.1} 
\begin{itemize} 
\item \textbf{Attori}: Utente Autenticato.
\item \textbf{Descrizione}: L' attore può modificare il suo nome utente.;
\item \textbf{Precondizione}: L' applicazione deve rendere disponibile il pulsante di modifica username.;
\item \textbf{Postcondizione}: L' applicazione modifica l'username dell'attore.;
\end{itemize} 
\subsubsection{UC3.2 - Modifica Password} 
\label{sssec:UC3.2} 
\begin{itemize} 
\item \textbf{Attori}: Utente Autenticato.
\item \textbf{Descrizione}: L' attore ha la possibilità di modificare la sua password.;
\item \textbf{Precondizione}: L' applicazione rende disponibile il pulsante di modifica.;
\item \textbf{Postcondizione}: L' applicazione modifica la password o restituisce il messaggio d'errore.;
\end{itemize} 
\subsubsection{UC3.3 - Modifica Email} 
\label{sssec:UC3.3} 
\begin{itemize} 
\item \textbf{Attori}: Utente Autenticato.
\item \textbf{Descrizione}: L' attore ha la possibilità di modificare la propria email.;
\item \textbf{Precondizione}: L' applicazione rende disponibile il pulsante per la modifica della email.;
\item \textbf{Postcondizione}: L' applicazione aggiorna la nuova email.;
\end{itemize} 
\subsubsection{UC4 - Logout} 
\label{sssec:UC4} 
\begin{itemize} 
\item \textbf{Attori}: Utente Autenticato.
\item \textbf{Descrizione}: L' attore può effettuare il logout dal suo profilo.;
\item \textbf{Precondizione}: L' applicazione offre il pulsante di logout visibile all'utente che ha effettuato l'accesso.;
\item \textbf{Postcondizione}: L' applicazione effettua il logout dell'attore.;
\end{itemize} 
\subsubsection{UC5 - Gestione Progetti} 
\label{sssec:UC5} 
\begin{itemize} 
\item \textbf{Attori}: Utente Autenticato.
\item \textbf{Descrizione}: L’ attore ha la possibilità di aggiungere un nuovo progetto e di aprire o modificare un progetto già esistente.;
\item \textbf{Precondizione}: L’ applicazione visualizza i pulsanti predisposti per l’esecuzione delle azioni sopra indicate.;
\item \textbf{Postcondizione}: L’ applicazione, a seconda dell’azione scelta dall’utente, svolgerà le sue funzioni;
\item \textbf{Scenario principale}: \begin{enumerate}\item Aggiunta Progetto (UC5.1);\item Apri progetto (UC5.2);\item Elimina Progetto (UC5.3). 
 \end{enumerate}
\end{itemize} 
\subsubsection{UC5.1 - Aggiunta Progetto} 
\label{sssec:UC5.1} 
\begin{itemize} 
\item \textbf{Attori}: Utente Autenticato.
\item \textbf{Descrizione}: L’ attore ha la possibilità di creare un nuovo progetto.;
\item \textbf{Precondizione}: L’applicazione rende disponibile il pulsante aggiungi progetti.;
\item \textbf{Postcondizione}: L’applicazione apre un nuovo foglio per la realizzazione del nuovo progetto.;
\end{itemize} 
\subsubsection{UC5.2 - Apri progetto} 
\label{sssec:UC5.2} 
\begin{itemize} 
\item \textbf{Attori}: Utente Autenticato.
\item \textbf{Descrizione}: L’ attore ha la possibilità di aprire un progetto precedentemente salvato.;
\item \textbf{Precondizione}: L’ applicazione rende disponibile il pulsante apri progetto, se precedentemente ne era stato salvato almeno uno.;
\item \textbf{Postcondizione}: L’applicazione apre il progetto selezionato.;
\end{itemize} 
\subsubsection{UC5.3 - Elimina Progetto} 
\label{sssec:UC5.3} 
\begin{itemize} 
\item \textbf{Attori}: Utente Autenticato.
\item \textbf{Descrizione}: L' attore ha la possibilità di eliminare un progetto precedentemente salvato.;
\item \textbf{Precondizione}: L’applicazione rende disponibile il pulsante elimina progetto, se precedentemente ne era stato salvato almeno uno.;
\item \textbf{Postcondizione}: L’applicazione elimina il progetto selezionato.;
\end{itemize} 
\subsubsection{UC6 - Tool Designer} 
\label{sssec:UC6} 
\begin{itemize} 
\item \textbf{Attori}: .
\item \textbf{Descrizione}: Tool Designer;
\item \textbf{Precondizione}: Tool Designer;
\item \textbf{Postcondizione}: Tool Designer;
\item \textbf{Scenario principale}: \begin{enumerate}\item Menù (UC6.1);\item Toolbar (UC6.2);\item Disegnatore diagrammi (UC6.3);\item Pannello laterale (UC6.4). 
 \end{enumerate}
\end{itemize} 
\subsubsection{UC6.1 - Menù} 
\label{sssec:UC6.1} 
\begin{itemize} 
\item \textbf{Attori}: Utente Autenticato.
\item \textbf{Descrizione}: L’ attore può accedere alle voci file, edit, tool,
layers e window appartenenti al menù del tool designer.;
\item \textbf{Precondizione}: L’applicazione offre all’utente una barra dei menù.;
\item \textbf{Postcondizione}: L’applicazione, a seconda dell’operazione richiesta dall’utente,
svolge le sue funzioni.;
\item \textbf{Scenario principale}: \begin{enumerate}\item File (UC6.1.1);\item Edit (UC6.1.2);\item Template (UC6.1.3);\item Layers (UC6.1.4). 
 \end{enumerate}
\end{itemize} 
\subsubsection{UC6.1.1 - File} 
\label{sssec:UC6.1.1} 
\begin{itemize} 
\item \textbf{Attori}: Utente Autenticato.
\item \textbf{Descrizione}: L’attore può accedere alle voci salva, chiudi, esporta, genera codice e salva template appartenenti alla voce file del menù.;
\item \textbf{Precondizione}: L’applicazione offre all’utente la voce file nella barra dei menù.;
\item \textbf{Postcondizione}: L’applicazione, a seconda dell’operazione richiesta dall’utente,
svolge le sue funzioni.;
\item \textbf{Scenario principale}: \begin{enumerate}\item Salva (UC6.1.1.1);\item Chiudi (UC6.1.1.2);\item Esporta (UC6.1.1.3);\item Genera codice (UC6.1.1.4);\item Salva template (UC6.1.1.5). 
 \end{enumerate}
\end{itemize} 
\subsubsection{UC6.1.1.1 - Salva} 
\label{sssec:UC6.1.1.1} 
\begin{itemize} 
\item \textbf{Attori}: Utente Autenticato.
\item \textbf{Descrizione}: L’attore può salvare il suo progetto nello stato
corrente.;
\item \textbf{Precondizione}: L’applicazione offre all’utente il salvataggio, disponibile alla voce salva del file appartenente alla barra del menù.;
\item \textbf{Postcondizione}: L’applicazione salva il progetto nello stato corrente.;
\end{itemize} 
\subsubsection{UC6.1.1.2 - Chiudi} 
\label{sssec:UC6.1.1.2} 
\begin{itemize} 
\item \textbf{Attori}: Utente Autenticato.
\item \textbf{Descrizione}: L’attore può chiudere il progetto corrente.;
\item \textbf{Precondizione}: L’applicazione offre all’attore la chiusura del progetto, disponibile
alla voce chiudi del menù.;
\item \textbf{Postcondizione}: L’applicazione chiude il progetto corrente.;
\end{itemize} 
\subsubsection{UC6.1.1.3 - Esporta} 
\label{sssec:UC6.1.1.3} 
\begin{itemize} 
\item \textbf{Attori}: Utente Autenticato.
\item \textbf{Descrizione}: L’attore può esportare il progetto corrente in
un altro formato.;
\item \textbf{Precondizione}: L’applicazione offre all’attore la possibilità di esportare il progetto,
selezionando la voce esporta nel menù.;
\item \textbf{Postcondizione}: L’applicazione esporta il progetto.;
\end{itemize} 
\subsubsection{UC6.1.1.4 - Genera codice} 
\label{sssec:UC6.1.1.4} 
\begin{itemize} 
\item \textbf{Attori}: Utente Autenticato.
\item \textbf{Descrizione}: L’attore può generare il codice relativo all’UML
prodotto.;
\item \textbf{Precondizione}: L’applicazione offre all’utente la possibilità di generare il codice
relativo all’UML, alla voce genera codice del menù.;
\item \textbf{Postcondizione}: L’applicazione genera il codice relativo al disegno UML.;
\end{itemize} 
\subsubsection{UC6.1.1.5 - Salva template} 
\label{sssec:UC6.1.1.5} 
\begin{itemize} 
\item \textbf{Attori}: Utente Autenticato.
\item \textbf{Descrizione}: L’attore ha la possibilità di salvare determinate classi o gerarchie, aggiungendole così alla lista di template già salvati.;
\item \textbf{Precondizione}: L’applicazione offre all’utente la voce salva template, sottovoce di file nel menù, solamente se sono state create una o più classi, ed ognuna di esse ha un commento che le identifica.;
\item \textbf{Postcondizione}: Viene aggiunto alla lista dei template il template desiderato.;
\end{itemize} 
\subsubsection{UC6.1.2 - Edit} 
\label{sssec:UC6.1.2} 
\begin{itemize} 
\item \textbf{Attori}: Utente Autenticato.
\item \textbf{Descrizione}: L’attore può accedere alle voci annulla, ripristina, taglia, copia, incolla, zoom in e zoom out appartenenti alla voce edit del menù.;
\item \textbf{Precondizione}: L’applicazione offre all’utente la voce edit appartenente barra
dei menù.;
\item \textbf{Postcondizione}: L’applicazione, a seconda dell’operazione richiesta dall’utente,
effettua la modifica.;
\item \textbf{Scenario principale}: \begin{enumerate}\item Annulla (UC6.1.2.1);\item Ripristina (UC6.1.2.2);\item Taglia (UC6.1.2.3);\item Copia (UC6.1.2.4);\item Incolla (UC6.1.2.5);\item Zoom in (UC6.1.2.6);\item Zoom out (UC6.1.2.7). 
 \end{enumerate}
\end{itemize} 
\subsubsection{UC6.1.2.1 - Annulla} 
\label{sssec:UC6.1.2.1} 
\begin{itemize} 
\item \textbf{Attori}: Utente Autenticato.
\item \textbf{Descrizione}: L’attore può tornare allo stato precedente l’ultima
modifica.;
\item \textbf{Precondizione}: L’applicazione offre all’utente la possibilità di selezionare la voce
annulla, sottovoce di edit nella barra del menù.;
\item \textbf{Postcondizione}: L’applicazione torna allo stato precedente l’ultima modifica.;
\end{itemize} 
\subsubsection{UC6.1.2.2 - Ripristina} 
\label{sssec:UC6.1.2.2} 
\begin{itemize} 
\item \textbf{Attori}: Utente Autenticato.
\item \textbf{Descrizione}: L’attore può ripristinare le modifiche effettuate
successivamente allo stato attuale.;
\item \textbf{Precondizione}: L’applicazione offre all’utente la possibilità di selezionare la voce
ripristina, sottovoce di edit nella barra del menù, solamente se si era tornati ad uno
stato precedente all’ultima modifica.;
\item \textbf{Postcondizione}: L’applicazione torna allo stato precedente l’ultima modifica.;
\end{itemize} 
\subsubsection{UC6.1.2.3 - Taglia} 
\label{sssec:UC6.1.2.3} 
\begin{itemize} 
\item \textbf{Attori}: Utente Autenticato.
\item \textbf{Descrizione}: L’attore può tagliare un determinato elemento
del progetto.;
\item \textbf{Precondizione}: L’applicazione offre all’utente la voce taglia, sottovoce di edit appartenente
al menù.;
\item \textbf{Postcondizione}: L’applicazione taglia l’elemento selezionato.;
\end{itemize} 
\subsubsection{UC6.1.2.4 - Copia} 
\label{sssec:UC6.1.2.4} 
\begin{itemize} 
\item \textbf{Attori}: Utente Autenticato.
\item \textbf{Descrizione}: L’attore può copiare un determinato elemento
del progetto.;
\item \textbf{Precondizione}: L’applicazione offre all’utente la voce copia, sottovoce di edit appartenente
al menù.;
\item \textbf{Postcondizione}: L’applicazione copia l’elemento selezionato.;
\end{itemize} 
\subsubsection{UC6.1.2.5 - Incolla} 
\label{sssec:UC6.1.2.5} 
\begin{itemize} 
\item \textbf{Attori}: Utente Autenticato.
\item \textbf{Descrizione}: L’attore può incollare un elemento precedentemente
tagliato o copiato.;
\item \textbf{Precondizione}: L’applicazione offre all’utente la possibilità di selezionare la voce
incolla, sottovoce di edit appartenente a menù, se un elemento era stato copiato o
tagliato precedentemente .;
\item \textbf{Postcondizione}: L’applicazione inserisce l’elemento designato.;
\end{itemize} 
\subsubsection{UC6.1.2.6 - Zoom in} 
\label{sssec:UC6.1.2.6} 
\begin{itemize} 
\item \textbf{Attori}: Utente Autenticato.
\item \textbf{Descrizione}: L’attore può ingrandire la visualizzazione di
una determinata sezione.;
\item \textbf{Precondizione}: L’applicazione offre all’utente la possibilità di selezionare la voce
zoom in, sottovoce di edit appartenente al menù.;
\item \textbf{Postcondizione}: L’applicazione ingrandisce la sezione prescelta.;
