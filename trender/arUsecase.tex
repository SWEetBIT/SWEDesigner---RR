\subsubsection{UC1 - Registrazione} 
\label{sssec:UC1} 
\begin{itemize} 
\item \textbf{Attori}: Utente non autenticato.
\item \textbf{Descrizione}: L'attore desidera effettuare l'operazione di registrazione. Vengono richiesti dal sistema un username, univoco e conforme alle richieste, una password conforme alle richieste e una mail che rispetti il pattern predefinito.;
\item \textbf{Precondizione}: Il sistema richiede all'attore le informazioni necessarie per effettuare
la registrazione.;
\item \textbf{Postcondizione}: Il sistema ha elaborato i dati inseriti dall'attore e, nel caso la registrazione
sia avvenuta con successo, gli ha concesso la possibilità di accedere al sistema.
In caso contrario il sistema ha visualizzato un messaggio che illustra il tipo di errore che è stato commesso.;
\item \textbf{Scenario principale}: \begin{enumerate}\item Inserisci Username (UC1.1);\item Inserisci Password (UC1.2);\item Inserisci Email (UC1.3);\item Conferma registrazione (UC1.4);\item Messaggio errore: "Username non conforme" (UC1.5);\item Messaggio errore: "Password non conforme" (UC1.6);\item Messaggio errore: "Email non conforme" (UC1.7). 
 \end{enumerate}
\end{itemize} 
\subsubsection{UC1.1 - Inserisci Username} 
\label{sssec:UC1.1} 
\begin{itemize} 
\item \textbf{Attori}: Utente non autenticato.
\item \textbf{Descrizione}: L’attore inserisce l'username: deve essere univoco all'interno del sistema e deve essere alfanumerico.;
\item \textbf{Precondizione}: L'attore ha selezionato l'opzione di registrazione e non ha ancora inserito un username.;
\item \textbf{Postcondizione}: L'attore ha inserito un username conforme alle richieste del sistema.;
\end{itemize} 
\subsubsection{UC1.2 - Inserisci Password} 
\label{sssec:UC1.2} 
\begin{itemize} 
\item \textbf{Attori}: Utente non autenticato.
\item \textbf{Descrizione}: L’attore inserisce la password: deve essere di tipo alfanumerico e può contenere caratteri di punteggiatura.;
\item \textbf{Precondizione}: L'attore ha selezionato l'opzione di registrazione e non ha ancora inserito una password.;
\item \textbf{Postcondizione}: L'attore ha inserito una password conforme alle richieste del sistema.;
\end{itemize} 
\subsubsection{UC1.3 - Inserisci Email} 
\label{sssec:UC1.3} 
\begin{itemize} 
\item \textbf{Attori}: Utente non autenticato.
\item \textbf{Descrizione}: L’attore inserisce l'email, che deve essere scritta in maniera corretta.;
\item \textbf{Precondizione}: L'attore ha selezionato l'opzione di registrazione e non ha ancora inserito un email.;
\item \textbf{Postcondizione}: L'attore ha inserito un'email nel modo corretto.;
\end{itemize} 
\subsubsection{UC1.4 - Conferma registrazione} 
\label{sssec:UC1.4} 
\begin{itemize} 
\item \textbf{Attori}: Utente non autenticato.
\item \textbf{Descrizione}: Dopo che un attore ha effettuato correttamente una registrazione, gli viene inviato un messaggio di conferma.;
\item \textbf{Precondizione}: L'attore ha scelto di effettuare l'operazione di registrazione.;
\item \textbf{Postcondizione}: L'utente ha ricevuto un messaggio di conferma dell'avvenuta registrazione.;
\end{itemize} 
\subsubsection{UC1.5 - Messaggio errore: "Username non conforme"} 
\label{sssec:UC1.5} 
\begin{itemize} 
\item \textbf{Attori}: Utente non autenticato.
\item \textbf{Descrizione}: L'attore ha inserito un username non conforme e gli viene quindi comunicato l'errore.;
\item \textbf{Precondizione}: L'attore ha selezionato l'opzione di registrazione e ha inserito un username.;
\item \textbf{Postcondizione}: L'attore ha inserito un username non conforme e ha ricevuto la comunicazione dell'errore.;
\end{itemize} 
\subsubsection{UC1.6 - Messaggio errore: "Password non conforme"} 
\label{sssec:UC1.6} 
\begin{itemize} 
\item \textbf{Attori}: Utente non autenticato.
\item \textbf{Descrizione}: L'attore ha inserito una password non conforme e gli viene quindi comunicato l'errore.;
\item \textbf{Precondizione}: L'attore ha selezionato l'opzione di registrazione e ha inserito una password.;
\item \textbf{Postcondizione}: L'attore ha inserito una password non conforme e ha ricevuto la comunicazione dell'errore.;
\end{itemize} 
\subsubsection{UC1.7 - Messaggio errore: "Email non conforme"} 
\label{sssec:UC1.7} 
\begin{itemize} 
\item \textbf{Attori}: Utente non autenticato.
\item \textbf{Descrizione}: L'attore ha inserito un'email con una sintassi non valida e gli viene quindi comunicato l'errore.;
\item \textbf{Precondizione}: L'attore ha selezionato l'opzione di registrazione e ha inserito un'email.;
\item \textbf{Postcondizione}: L'attore ha inserito un'email non conforme e ha ricevuto la comunicazione dell'errore.;
\end{itemize} 
\subsubsection{UC2 - Autenticazione} 
\label{sssec:UC2} 
\begin{itemize} 
\item \textbf{Attori}: Utente non autenticato.
\item \textbf{Descrizione}: L'attore che già in possesso delle credenziali per accedere al sistema, potrà effettuare l'operazione di autenticazione inserendo l'username o l'email e la password utilizzati per l'autenticazione. Nel caso l’attore abbia perso la password o se la sia dimenticata, il sistema fornisce la possibilità di resettarla.;
\item \textbf{Precondizione}: L'attore decide di autenticarsi e il sistema richiede l'inserimento dei dati necessari per l'autenticazione.;
\item \textbf{Postcondizione}: L'attore ha avuto accesso alle funzionalità del sistema in caso l'autenticazione sia avvenuta con successo. In caso contrario il sistema ha visualizzato un messaggio d'errore.;
\item \textbf{Scenario principale}: \begin{enumerate}\item Inserisci Username/Email (UC2.1);\item Inserisci Password (UC2.2);\item Messaggio errore: "Username/Email o password errato" (UC2.4);\item Invio password per email (UC2.5). 
 \end{enumerate}
\end{itemize} 
\subsubsection{UC2.1 - Inserisci Username/Email} 
\label{sssec:UC2.1} 
\begin{itemize} 
\item \textbf{Attori}: Utente non autenticato.
\item \textbf{Descrizione}: Durante la fase di autenticazione viene richiesto all'attore il proprio username o la propria email.;
\item \textbf{Precondizione}: L'attore ha selezionato l'opzione di autenticazione e non ha ancora inserito il proprio username o la propria email.;
\item \textbf{Postcondizione}: L'attore ha inserito il proprio username o la propria email.;
\end{itemize} 
\subsubsection{UC2.2 - Inserisci Password} 
\label{sssec:UC2.2} 
\begin{itemize} 
\item \textbf{Attori}: Utente non autenticato.
\item \textbf{Descrizione}: Durante la fase di autenticazione viene richiesta all'attore la propria password.;
\item \textbf{Precondizione}: L'utente ha selezionato l'opzione di autenticazione e non ha ancora inserito la propria password.;
\item \textbf{Postcondizione}: L'attore ha inserito la propria email.;
\end{itemize} 
\subsubsection{UC2.4 - Messaggio errore: "Username/Email o password errato"} 
\label{sssec:UC2.4} 
\begin{itemize} 
\item \textbf{Attori}: Utente non autenticato.
\item \textbf{Descrizione}: L'attore, già in possesso delle credenziali d'accesso, tenta di autenticarsi, ma l'operazione non va a buon fine e viene visualizzato un messaggio d'errore.;
\item \textbf{Precondizione}: L'attore ha intenzione di effettuare l'operazione di autenticazione ed ha inserito il proprio username o la propria email.;
\item \textbf{Postcondizione}: Il sistema ha ricavato informazioni errate e quindi ha mostrato un messaggio d'errore.;
\end{itemize} 
\subsubsection{UC2.5 - Invio password per email} 
\label{sssec:UC2.5} 
\begin{itemize} 
\item \textbf{Attori}: Utente non autenticato.
\item \textbf{Descrizione}: Se l'attore ha perso oppure ha dimenticato la propria password, il sistema fornisce uno strumento per riceverne una nuova.;
\item \textbf{Precondizione}: L'attore è in possesso di un account all'interno del sistema, ma non ricorda più la password.;
\item \textbf{Postcondizione}: L'attore ha ricevuto una nuova email contenente una nuova password temporanea.;
\end{itemize} 
\subsubsection{UC3 - Gestione Profilo} 
\label{sssec:UC3} 
\begin{itemize} 
\item \textbf{Attori}: Utente Autenticato.
\item \textbf{Descrizione}: L' attore vuole gestire il suo username, la sua password o il suo indirizzo e-mail.;
\item \textbf{Precondizione}: L' attore ha già effettuato l'accesso e l'applicazione rende disponibile la voce Gestione Profilo.;
\item \textbf{Postcondizione}: L' applicazione visualizza i dati dell'attore con gli opportuni pulsanti di modifica.;
\item \textbf{Scenario principale}: \begin{enumerate}\item Modifica Username (UC3.1);\item Modifica Password (UC3.2);\item Modifica Email (UC3.3). 
 \end{enumerate}
\end{itemize} 
\subsubsection{UC3.1 - Modifica Username} 
\label{sssec:UC3.1} 
\begin{itemize} 
\item \textbf{Attori}: Utente Autenticato.
\item \textbf{Descrizione}: L' attore può modificare il suo nome utente.;
\item \textbf{Precondizione}: L' applicazione deve rendere disponibile il pulsante di modifica username.;
\item \textbf{Postcondizione}: L' applicazione modifica l'username dell'attore.;
\end{itemize} 
\subsubsection{UC3.2 - Modifica Password} 
\label{sssec:UC3.2} 
\begin{itemize} 
\item \textbf{Attori}: Utente Autenticato.
\item \textbf{Descrizione}: L' attore ha la possibilità di modificare la sua password.;
\item \textbf{Precondizione}: L' applicazione rende disponibile il pulsante di modifica.;
\item \textbf{Postcondizione}: L' applicazione modifica la password o restituisce il messaggio d'errore.;
\end{itemize} 
\subsubsection{UC3.3 - Modifica Email} 
\label{sssec:UC3.3} 
\begin{itemize} 
\item \textbf{Attori}: Utente Autenticato.
\item \textbf{Descrizione}: L' attore ha la possibilità di modificare la propria email.;
\item \textbf{Precondizione}: L' applicazione rende disponibile il pulsante per la modifica della email.;
\item \textbf{Postcondizione}: L' applicazione aggiorna la nuova email.;
\end{itemize} 
\subsubsection{UC4 - Logout} 
\label{sssec:UC4} 
\begin{itemize} 
\item \textbf{Attori}: Utente Autenticato.
\item \textbf{Descrizione}: L' attore può effettuare il logout dal suo profilo.;
\item \textbf{Precondizione}: L' applicazione offre il pulsante di logout visibile all'utente che ha effettuato l'accesso.;
\item \textbf{Postcondizione}: L' applicazione effettua il logout dell'attore.;
\end{itemize} 
\subsubsection{UC5 - Gestione Progetti} 
\label{sssec:UC5} 
\begin{itemize} 
\item \textbf{Attori}: Utente Autenticato.
\item \textbf{Descrizione}: L’ attore ha la possibilità di aggiungere un nuovo progetto e di aprire o modificare un progetto già esistente.;
\item \textbf{Precondizione}: L’ applicazione visualizza i pulsanti predisposti per l’esecuzione delle azioni sopra indicate.;
\item \textbf{Postcondizione}: L’ applicazione, a seconda dell’azione scelta dall’utente, svolgerà le sue funzioni;
\item \textbf{Scenario principale}: \begin{enumerate}\item Aggiunta Progetto (UC5.1);\item Apri progetto (UC5.2);\item Elimina Progetto (UC5.3). 
 \end{enumerate}
\end{itemize} 
\subsubsection{UC5.1 - Aggiunta Progetto} 
\label{sssec:UC5.1} 
\begin{itemize} 
\item \textbf{Attori}: Utente Autenticato.
\item \textbf{Descrizione}: L’ attore ha la possibilità di creare un nuovo progetto.;
\item \textbf{Precondizione}: L’applicazione rende disponibile il pulsante aggiungi progetti.;
\item \textbf{Postcondizione}: L’applicazione apre un nuovo foglio per la realizzazione del nuovo progetto.;
\end{itemize} 
\subsubsection{UC5.2 - Apri progetto} 
\label{sssec:UC5.2} 
\begin{itemize} 
\item \textbf{Attori}: Utente Autenticato.
\item \textbf{Descrizione}: L’ attore ha la possibilità di aprire un progetto precedentemente salvato.;
\item \textbf{Precondizione}: L’ applicazione rende disponibile il pulsante apri progetto, se precedentemente ne era stato salvato almeno uno.;
\item \textbf{Postcondizione}: L’applicazione apre il progetto selezionato.;
\end{itemize} 
\subsubsection{UC5.3 - Elimina Progetto} 
\label{sssec:UC5.3} 
\begin{itemize} 
\item \textbf{Attori}: Utente Autenticato.
\item \textbf{Descrizione}: L' attore ha la possibilità di eliminare un progetto precedentemente salvato.;
\item \textbf{Precondizione}: L’applicazione rende disponibile il pulsante elimina progetto, se precedentemente ne era stato salvato almeno uno.;
\item \textbf{Postcondizione}: L’applicazione elimina il progetto selezionato.;
\end{itemize} 
\subsubsection{UC6 - Tool Designer} 
\label{sssec:UC6} 
\begin{itemize} 
\item \textbf{Attori}: .
\item \textbf{Descrizione}: Tool Designer;
\item \textbf{Precondizione}: Tool Designer;
\item \textbf{Postcondizione}: Tool Designer;
\item \textbf{Scenario principale}: \begin{enumerate}\item Menù (UC6.1);\item Toolbar (UC6.2);\item Disegnatore diagrammi (UC6.3);\item Pannello laterale (UC6.4). 
 \end{enumerate}
\end{itemize} 
\subsubsection{UC6.1 - Menù} 
\label{sssec:UC6.1} 
\begin{itemize} 
\item \textbf{Attori}: Utente Autenticato.
\item \textbf{Descrizione}: L’ attore può accedere alle voci file, edit, tool,
layers e window appartenenti al menù del tool designer.;
\item \textbf{Precondizione}: L’applicazione offre all’utente una barra dei menù.;
\item \textbf{Postcondizione}: L’applicazione, a seconda dell’operazione richiesta dall’utente,
svolge le sue funzioni.;
\item \textbf{Scenario principale}: \begin{enumerate}\item File (UC6.1.1);\item Edit (UC6.1.2);\item Template (UC6.1.3);\item Layers (UC6.1.4). 
 \end{enumerate}
\end{itemize} 
\subsubsection{UC6.1.1 - File} 
\label{sssec:UC6.1.1} 
\begin{itemize} 
\item \textbf{Attori}: Utente Autenticato.
\item \textbf{Descrizione}: L’attore può accedere alle voci salva, chiudi, esporta, genera codice e salva template appartenenti alla voce file del menù.;
\item \textbf{Precondizione}: L’applicazione offre all’utente la voce file nella barra dei menù.;
\item \textbf{Postcondizione}: L’applicazione, a seconda dell’operazione richiesta dall’utente,
svolge le sue funzioni.;
\item \textbf{Scenario principale}: \begin{enumerate}\item Salva (UC6.1.1.1);\item Chiudi (UC6.1.1.2);\item Esporta (UC6.1.1.3);\item Genera codice (UC6.1.1.4);\item Salva template (UC6.1.1.5). 
 \end{enumerate}
\end{itemize} 
\subsubsection{UC6.1.1.1 - Salva} 
\label{sssec:UC6.1.1.1} 
\begin{itemize} 
\item \textbf{Attori}: Utente Autenticato.
\item \textbf{Descrizione}: L’attore può salvare il suo progetto nello stato
corrente.;
\item \textbf{Precondizione}: L’applicazione offre all’utente il salvataggio, disponibile alla voce salva del file appartenente alla barra del menù.;
\item \textbf{Postcondizione}: L’applicazione salva il progetto nello stato corrente.;
\end{itemize} 
\subsubsection{UC6.1.1.2 - Chiudi} 
\label{sssec:UC6.1.1.2} 
\begin{itemize} 
\item \textbf{Attori}: Utente Autenticato.
\item \textbf{Descrizione}: L’attore può chiudere il progetto corrente.;
\item \textbf{Precondizione}: L’applicazione offre all’attore la chiusura del progetto, disponibile
alla voce chiudi del menù.;
\item \textbf{Postcondizione}: L’applicazione chiude il progetto corrente.;
\end{itemize} 
\subsubsection{UC6.1.1.3 - Esporta} 
\label{sssec:UC6.1.1.3} 
\begin{itemize} 
\item \textbf{Attori}: Utente Autenticato.
\item \textbf{Descrizione}: L’attore può esportare il progetto corrente in
un altro formato.;
\item \textbf{Precondizione}: L’applicazione offre all’attore la possibilità di esportare il progetto,
selezionando la voce esporta nel menù.;
\item \textbf{Postcondizione}: L’applicazione esporta il progetto.;
\end{itemize} 
\subsubsection{UC6.1.1.4 - Genera codice} 
\label{sssec:UC6.1.1.4} 
\begin{itemize} 
\item \textbf{Attori}: Utente Autenticato.
\item \textbf{Descrizione}: L’attore può generare il codice relativo all’UML
prodotto.;
\item \textbf{Precondizione}: L’applicazione offre all’utente la possibilità di generare il codice
relativo all’UML, alla voce genera codice del menù.;
\item \textbf{Postcondizione}: L’applicazione genera il codice relativo al disegno UML.;
\end{itemize} 
\subsubsection{UC6.1.1.5 - Salva template} 
\label{sssec:UC6.1.1.5} 
\begin{itemize} 
\item \textbf{Attori}: Utente Autenticato.
\item \textbf{Descrizione}: L’attore ha la possibilità di salvare determinate classi o gerarchie, aggiungendole così alla lista di template già salvati.;
\item \textbf{Precondizione}: L’applicazione offre all’utente la voce salva template, sottovoce di file nel menù, solamente se sono state create una o più classi, ed ognuna di esse ha un commento che le identifica.;
\item \textbf{Postcondizione}: Viene aggiunto alla lista dei template il template desiderato.;
\end{itemize} 
\subsubsection{UC6.1.2 - Edit} 
\label{sssec:UC6.1.2} 
\begin{itemize} 
\item \textbf{Attori}: Utente Autenticato.
\item \textbf{Descrizione}: L’attore può accedere alle voci annulla, ripristina, taglia, copia, incolla, zoom in e zoom out appartenenti alla voce edit del menù.;
\item \textbf{Precondizione}: L’applicazione offre all’utente la voce edit appartenente barra
dei menù.;
\item \textbf{Postcondizione}: L’applicazione, a seconda dell’operazione richiesta dall’utente,
effettua la modifica.;
\item \textbf{Scenario principale}: \begin{enumerate}\item Annulla (UC6.1.2.1);\item Ripristina (UC6.1.2.2);\item Taglia (UC6.1.2.3);\item Copia (UC6.1.2.4);\item Incolla (UC6.1.2.5);\item Zoom in (UC6.1.2.6);\item Zoom out (UC6.1.2.7). 
 \end{enumerate}
\end{itemize} 
\subsubsection{UC6.1.2.1 - Annulla} 
\label{sssec:UC6.1.2.1} 
\begin{itemize} 
\item \textbf{Attori}: Utente Autenticato.
\item \textbf{Descrizione}: L’attore può tornare allo stato precedente l’ultima
modifica.;
\item \textbf{Precondizione}: L’applicazione offre all’utente la possibilità di selezionare la voce
annulla, sottovoce di edit nella barra del menù.;
\item \textbf{Postcondizione}: L’applicazione torna allo stato precedente l’ultima modifica.;
\end{itemize} 
\subsubsection{UC6.1.2.2 - Ripristina} 
\label{sssec:UC6.1.2.2} 
\begin{itemize} 
\item \textbf{Attori}: Utente Autenticato.
\item \textbf{Descrizione}: L’attore può ripristinare le modifiche effettuate
successivamente allo stato attuale.;
\item \textbf{Precondizione}: L’applicazione offre all’utente la possibilità di selezionare la voce
ripristina, sottovoce di edit nella barra del menù, solamente se si era tornati ad uno
stato precedente all’ultima modifica.;
\item \textbf{Postcondizione}: L’applicazione torna allo stato precedente l’ultima modifica.;
\end{itemize} 
\subsubsection{UC6.1.2.3 - Taglia} 
\label{sssec:UC6.1.2.3} 
\begin{itemize} 
\item \textbf{Attori}: Utente Autenticato.
\item \textbf{Descrizione}: L’attore può tagliare un determinato elemento
del progetto.;
\item \textbf{Precondizione}: L’applicazione offre all’utente la voce taglia, sottovoce di edit appartenente
al menù.;
\item \textbf{Postcondizione}: L’applicazione taglia l’elemento selezionato.;
\end{itemize} 
\subsubsection{UC6.1.2.4 - Copia} 
\label{sssec:UC6.1.2.4} 
\begin{itemize} 
\item \textbf{Attori}: Utente Autenticato.
\item \textbf{Descrizione}: L’attore può copiare un determinato elemento
del progetto.;
\item \textbf{Precondizione}: L’applicazione offre all’utente la voce copia, sottovoce di edit appartenente
al menù.;
\item \textbf{Postcondizione}: L’applicazione copia l’elemento selezionato.;
\end{itemize} 
\subsubsection{UC6.1.2.5 - Incolla} 
\label{sssec:UC6.1.2.5} 
\begin{itemize} 
\item \textbf{Attori}: Utente Autenticato.
\item \textbf{Descrizione}: L’attore può incollare un elemento precedentemente
tagliato o copiato.;
\item \textbf{Precondizione}: L’applicazione offre all’utente la possibilità di selezionare la voce
incolla, sottovoce di edit appartenente a menù, se un elemento era stato copiato o
tagliato precedentemente .;
\item \textbf{Postcondizione}: L’applicazione inserisce l’elemento designato.;
\end{itemize} 
\subsubsection{UC6.1.2.6 - Zoom in} 
\label{sssec:UC6.1.2.6} 
\begin{itemize} 
\item \textbf{Attori}: Utente Autenticato.
\item \textbf{Descrizione}: L’attore può ingrandire la visualizzazione di
una determinata sezione.;
\item \textbf{Precondizione}: L’applicazione offre all’utente la possibilità di selezionare la voce
zoom in, sottovoce di edit appartenente al menù.;
\item \textbf{Postcondizione}: L’applicazione ingrandisce la sezione prescelta.;
\end{itemize} 
\subsubsection{UC6.1.2.7 - Zoom out} 
\label{sssec:UC6.1.2.7} 
\begin{itemize} 
\item \textbf{Attori}: Utente Autenticato.
\item \textbf{Descrizione}: L’attore può rimpicciolire la visualizzazione di
una determinata sezione.;
\item \textbf{Precondizione}: L’applicazione offre all’utente la possibilità di selezionare la voce
zoom out, sottovoce di edit appartenente al menù.;
\item \textbf{Postcondizione}: L’applicazione rimpicciolisce la sezione prescelta.;
\end{itemize} 
\subsubsection{UC6.1.3 - Template} 
\label{sssec:UC6.1.3} 
\begin{itemize} 
\item \textbf{Attori}: .
\item \textbf{Descrizione}: L’utente autenticato può accedere alla visualizzazione della lista dei template. I template visualizzati comprendono sia quelli già dati dagli sviluppatore, sia i propri template.;
\item \textbf{Precondizione}: L’applicazione offre all’utente una barra dei menù.;
\item \textbf{Postcondizione}: L'applicazione ha aperto una finestra dove viene visualizzata la lista dei template e fornisce la possibilità di inserimento od eliminazione.;
\item \textbf{Scenario principale}: \begin{enumerate}\item Inserisci template (UC6.1.3.1);\item Elimina template (UC6.1.3.2). 
 \end{enumerate}
\end{itemize} 
\subsubsection{UC6.1.3.1 - Inserisci template} 
\label{sssec:UC6.1.3.1} 
\begin{itemize} 
\item \textbf{Attori}: .
\item \textbf{Descrizione}: L'utente inserisce nella finestra del disegnatore dei diagrammi ( UC 6.3 ) il template selezionato.;
\item \textbf{Precondizione}: L'utente ha aperto la finestra di gestione dei template.;
\item \textbf{Postcondizione}: L'utente aggiunge il template selezionato nella finestra del disegnatore dei diagrammi( UC 6.3 ).;
\item \textbf{Scenario principale}: L'utente seleziona il template desiderato da inserire.
L'utente clicca nella finestra del disegnatore dei diagrammi( UC 6.3 ) nel punto dove inserire il template.;\item \textbf{Scenari alternativi}: L'utente chiude la finestra di gestione template ( UC 6.1.3 ) e non viene effettuata nessuna operazione..
\end{itemize} 
\subsubsection{UC6.1.3.2 - Elimina template} 
\label{sssec:UC6.1.3.2} 
\begin{itemize} 
\item \textbf{Attori}: .
\item \textbf{Descrizione}: L'utente elimina dalla lista il template selezionato.;
\item \textbf{Precondizione}: L'utente ha aperto la finestra di gestione dei template ( UC 6.1.3 ).;
\item \textbf{Postcondizione}: Il template selezionato viene rimosso dalla lista.;
\item \textbf{Scenario principale}: L'utente seleziona il template desiderato da eliminare.
L'utente clicca il bottone nella stessa linea rappresentato da una "X".
Viene fatta una richiesta di conferma di eliminazione.
Tale template viene eliminato dalla lista se la conferma ha esito positivo.;\item \textbf{Scenari alternativi}: L'utente chiude la finestra di gestione template ( UC 6.1.3 ) e non viene effettuata nessuna operazione..
\end{itemize} 
\subsubsection{UC6.1.4 - Layers} 
\label{sssec:UC6.1.4} 
\begin{itemize} 
\item \textbf{Attori}: Utente Autenticato.
\item \textbf{Descrizione}: L’utente autenticato può accedere alla visualizzazione della lista dei layers. Può inoltre creare nuovi layer oppure rinominarne o eliminarne di già esistenti.;
\item \textbf{Precondizione}: L’applicazione offre all’utente una barra dei menù.;
\item \textbf{Postcondizione}: L'applicazione ha eseguito l'operazione richiesta dallutente.;
\item \textbf{Scenario principale}: \begin{enumerate}\item Nuovo (UC6.1.4.1);\item Lista (UC6.1.4.2);\item Modifica (UC6.1.4.3). 
 \end{enumerate}
\end{itemize} 
\subsubsection{UC6.1.4.1 - Nuovo} 
\label{sssec:UC6.1.4.1} 
\begin{itemize} 
\item \textbf{Attori}: Utente Autenticato.
\item \textbf{Descrizione}: L'utente crea un nuovo layer a cui è possibile assegnare delle classi.;
\item \textbf{Precondizione}: L'utente ha aperto la finestra di gestione dei layer.;
\item \textbf{Postcondizione}: L'utente ha aggiunto un nuovo layer assegnandogli il nome desiderato.;
\end{itemize} 
\subsubsection{UC6.1.4.2 - Lista} 
\label{sssec:UC6.1.4.2} 
\begin{itemize} 
\item \textbf{Attori}: Utente Autenticato.
\item \textbf{Descrizione}: L'utente seleziona il layer di cui desidera visualizzare le classi che gli sono state assegnate.;
\item \textbf{Precondizione}: L'utente ha aperto la finestra di gestione dei layer.;
\item \textbf{Postcondizione}: L'utente ha visualizzato le classi assegnate al layer desiderato.;
\end{itemize} 
\subsubsection{UC6.1.4.3 - Modifica} 
\label{sssec:UC6.1.4.3} 
\begin{itemize} 
\item \textbf{Attori}: Utente Autenticato.
\item \textbf{Descrizione}: L'utente desidera rinominare oppure eliminare un layer tra quelli esistenti.;
\item \textbf{Precondizione}: L'utente ha aperto la finestra di gestione dei layer.;
\item \textbf{Postcondizione}: L'utente ha eseguito l'operazione desiderata.;
\item \textbf{Scenario principale}: \begin{enumerate}\item Rinomina (UC6.1.4.3.1);\item Elimina (UC6.1.4.3.2). 
 \end{enumerate}
\end{itemize} 
\subsubsection{UC6.1.4.3.1 - Rinomina} 
\label{sssec:UC6.1.4.3.1} 
\begin{itemize} 
\item \textbf{Attori}: Utente Autenticato.
\item \textbf{Descrizione}: L'utente desidera rinominare un layer tra quelli esistenti.;
\item \textbf{Precondizione}: L'utente ha aperto la finestra di gestione dei layer e ha selezionato l'opzione di modifica.;
\item \textbf{Postcondizione}: L'utente ha modificato il nome del layer che intendeva modificare, mantenendo tutte le classi associate.;
\end{itemize} 
\subsubsection{UC6.1.4.3.2 - Elimina} 
\label{sssec:UC6.1.4.3.2} 
\begin{itemize} 
\item \textbf{Attori}: Utente Autenticato.
\item \textbf{Descrizione}: L'utente desidera eliminare un layer tra quelli esistenti.;
\item \textbf{Precondizione}: L'utente ha aperto la finestra di gestione dei layer e ha selezionato l'opzione di modifica.;
\item \textbf{Postcondizione}: L'utente ha eliminato il layer che intendeva modificare e tutte le classi assegnate a tale layer, se ce n'erano, non risultano più assegnate a nessun layer.;
\end{itemize} 
\subsubsection{UC6.2 - Toolbar} 
\label{sssec:UC6.2} 
\begin{itemize} 
\item \textbf{Attori}: .
\item \textbf{Descrizione}: Toolbar;
\item \textbf{Precondizione}: Toolbar;
\item \textbf{Postcondizione}: Toolbar;
\item \textbf{Scenario principale}: \begin{enumerate}\item Class Tool (UC6.2.1);\item Strumenti Diagramma delle Attività (UC6.2.2). 
 \end{enumerate}
\end{itemize} 
\subsubsection{UC6.2.1 - Class Tool} 
\label{sssec:UC6.2.1} 
\begin{itemize} 
\item \textbf{Attori}: .
\item \textbf{Descrizione}: L'attore, per la creazione di un diagramma della classi ha la possibilità di selezionare una nuova classe, inserire un package per raggruppare delle classi o inserire delle relazioni tra le classi.;
\item \textbf{Precondizione}: è disponibile un foglio di lavoro dove inserire gli oggetti desiderati;
\item \textbf{Postcondizione}: Il sistema inserisce all'interno della mainwindows gli elementi selezionati dall'utente;
\item \textbf{Scenario principale}: \begin{enumerate}\item Nuova Classe (UC6.2.1.1);\item Nuovo Pacchetto (UC6.2.1.2);\item Nuova Relazione (UC6.2.1.3);\item Nuovo commento (UC6.2.1.4). 
 \end{enumerate}
\end{itemize} 
\subsubsection{UC6.2.1.1 - Nuova Classe} 
\label{sssec:UC6.2.1.1} 
\begin{itemize} 
\item \textbf{Attori}: Utente Autenticato.
\item \textbf{Descrizione}: L'attore ha la possibilità di inserire il disegno rappresentante una classe nello standard UML con i relativi campi compilati;
\item \textbf{Precondizione}: La main windows è in modalità modifica diagrammi delle classi;
\item \textbf{Postcondizione}: Viene rappresentato il disegno della classe e vengono salvati i dati contenuti in essa;
\end{itemize} 
\subsubsection{UC6.2.1.2 - Nuovo Pacchetto} 
\label{sssec:UC6.2.1.2} 
\begin{itemize} 
\item \textbf{Attori}: .
\item \textbf{Descrizione}: L'attore ha la possibilità di inserire il disegno rappresentante un package nello standard UML;
\item \textbf{Precondizione}: La main windows è in modalità modifica diagrammi delle classi;
\item \textbf{Postcondizione}: Viene rappresentato il disegno del package e vengono salvati i relativi dati;
\end{itemize} 
\subsubsection{UC6.2.1.3 - Nuova Relazione} 
\label{sssec:UC6.2.1.3} 
\begin{itemize} 
\item \textbf{Attori}: Utente Autenticato.
\item \textbf{Descrizione}: Nuova Relazione;
\item \textbf{Precondizione}: Nuova Relazione;
\item \textbf{Postcondizione}: Nuova Relazione;
\item \textbf{Scenario principale}: \begin{enumerate}\item Definizione classe di partenza (UC6.2.1.3.1);\item Definizione classe di arrivo (UC6.2.1.3.2). 
 \end{enumerate}
\end{itemize} 
\subsubsection{UC6.2.1.3.1 - Definizione classe di partenza} 
\label{sssec:UC6.2.1.3.1} 
\begin{itemize} 
\item \textbf{Attori}: .
\item \textbf{Descrizione}: Viene definito il punto di partenza dell'associazione;
\item \textbf{Precondizione}: esiste almeno una classe;
\item \textbf{Postcondizione}: il sistema associa il disegno dell'assegnazione alla relativa classe;
\end{itemize} 
\subsubsection{UC6.2.1.3.2 - Definizione classe di arrivo} 
\label{sssec:UC6.2.1.3.2} 
\begin{itemize} 
\item \textbf{Attori}: .
\item \textbf{Descrizione}: Viene definito il punto di arrivo dell'associazione;
\item \textbf{Precondizione}: esiste almeno una classe;
\item \textbf{Postcondizione}: il sistema associa il disegno dell'assegnazione alla relativa classe di destinazione;
\end{itemize} 
\subsubsection{UC6.2.1.4 - Nuovo commento} 
\label{sssec:UC6.2.1.4} 
\begin{itemize} 
\item \textbf{Attori}: .
\item \textbf{Descrizione}: L'utente ha la possibilità di inserire un commento associato ad un elemento del diagramma delle classi;
\item \textbf{Precondizione}: La main windows è in modalità modifica diagrammi delle classi;
\item \textbf{Postcondizione}: Viene rappresentato il disegno del commento;
\item \textbf{Scenario principale}: L'utente dopo l'inserimento del commento lo associa ad un elemento del diagramma delle classi;\end{itemize} 
\subsubsection{UC6.2.2 - Strumenti Diagramma delle Attività} 
\label{sssec:UC6.2.2} 
\begin{itemize} 
\item \textbf{Attori}: Utente Autenticato.
\item \textbf{Descrizione}: l'attore può selezionare da questa lista uno degli strumenti per la creazione di nuovi elementi nel \glossaryItem{Activity Frame};
\item \textbf{Precondizione}: nel sistema deve essere caricato correttamente un progetto;
\item \textbf{Postcondizione}: nel progetto, caricato nel sistema, vengono visualizzati gli eventuali inserimenti di elementi operati dall'attore;
\item \textbf{Scenario principale}: \begin{enumerate}\item Nuova operazione (UC6.2.2.1);\item Nuova chiamata a metodo (UC6.2.2.2);\item Nuova variabile (UC6.2.2.3);\item Nuovo connettore (UC6.2.2.4);\item Nuovo nodo decisione (UC6.2.2.5);\item Nuovo nodo merge (UC6.2.2.6);\item Nuovo commento (UC6.2.2.7);\item Nuovo output pin (UC6.2.2.8). 
 \end{enumerate}
\end{itemize} 
\subsubsection{UC6.2.2.1 - Nuova operazione} 
\label{sssec:UC6.2.2.1} 
\begin{itemize} 
\item \textbf{Attori}: Utente Autenticato.
\item \textbf{Descrizione}: l'attore può aggiungere un nuovo elemento operazione al diagramma delle attività di un metodo;
\item \textbf{Precondizione}: è aperta, nel frame del diagramma delle  classi, il diagramma delle attività del un metodo di una classe;
\item \textbf{Postcondizione}: viene aggiunto al diagramma delle attività di un metodo un nuovo elemento operazione;
\item \textbf{Scenari alternativi}: l'elemento non è posizionato all'interno della finestra del diagramma delle attività di un metodo.
\end{itemize} 
\subsubsection{UC6.2.2.2 - Nuova chiamata a metodo} 
\label{sssec:UC6.2.2.2} 
\begin{itemize} 
\item \textbf{Attori}: Utente Autenticato.
\item \textbf{Descrizione}: l'attore può aggiungere un elemento chiamata a metodo al diagramma delle attività di un metodo;
\item \textbf{Precondizione}: è selezionato il metodo di una classe ed è visualizzata correttamente la finestra del diagramma delle attività di esso;
\item \textbf{Postcondizione}: un elemento chiamata a metodo è aggiunto al diagramma delle attività del metodo di una classe;
\item \textbf{Scenari alternativi}: l'elemento non è posizionato all'interno della finestra del diagramma delle attività di un metodo.
\end{itemize} 
\subsubsection{UC6.2.2.3 - Nuova variabile} 
\label{sssec:UC6.2.2.3} 
\begin{itemize} 
\item \textbf{Attori}: Utente Autenticato.
\item \textbf{Descrizione}: l'attore può aggiungere un nuovo elemento variabile al diagramma delle attività di un metodo;
\item \textbf{Precondizione}: è aperta, nel frame delle diagramma delle  classi, il diagramma delle attività del un metodo di una classe;
\item \textbf{Postcondizione}: viene aggiunto al diagramma delle attività di un metodo un nuovo elemento variabile;
\item \textbf{Scenario principale}: l'attore ha selezionato l'elemento nuova variabile dalla lista degli strumenti del diagramma delle attività;\end{itemize} 
\subsubsection{UC6.2.2.4 - Nuovo connettore} 
\label{sssec:UC6.2.2.4} 
\begin{itemize} 
\item \textbf{Attori}: .
\item \textbf{Descrizione}: l'attore può aggiungere un nuovo elemento connettore al diagramma delle attività di un metodo;
\item \textbf{Precondizione}: è aperta, nel frame delle diagramma delle  classi, il diagramma delle attività del un metodo di una classe; devono esserci nel diagramma delle attività del metodo almeno due elementi che possono essere connessi;
\item \textbf{Postcondizione}: viene aggiunto al diagramma delle attività di un metodo un nuovo elemento connettore;
\item \textbf{Scenari alternativi}: l'elemento non è posizionato all'interno della finestra del diagramma delle attività di un metodo.
\end{itemize} 
\subsubsection{UC6.2.2.5 - Nuovo nodo decisione} 
\label{sssec:UC6.2.2.5} 
\begin{itemize} 
\item \textbf{Attori}: .
\item \textbf{Descrizione}: l'attore può aggiungere un nuovo elemento nodo decisione al diagramma delle attività di un metodo;
\item \textbf{Precondizione}: è aperta, nel frame delle diagramma delle  classi, il diagramma delle attività del un metodo di una classe;
\item \textbf{Postcondizione}: viene aggiunto al diagramma delle attività di un metodo un nuovo elemento nodo decisione;
\item \textbf{Scenario principale}: l'attore ha selezionato l'elemento nuovo nodo decisione dalla lista degli strumenti del diagramma delle attività;\end{itemize} 
\subsubsection{UC6.2.2.6 - Nuovo nodo merge} 
\label{sssec:UC6.2.2.6} 
\begin{itemize} 
\item \textbf{Attori}: .
\item \textbf{Descrizione}: l'attore può aggiungere un nuovo elemento nodo merge al diagramma delle attività di un metodo;
\item \textbf{Precondizione}: è aperta, nel frame delle diagramma delle  classi, il diagramma delle attività del un metodo di una classe;
\item \textbf{Postcondizione}: viene aggiunto al diagramma delle attività di un metodo un nuovo elemento nodo merge;
\item \textbf{Scenario principale}: l'attore ha selezionato l'elemento nuovo nodo merge dalla lista degli strumenti del diagramma delle attività;\end{itemize} 
\subsubsection{UC6.2.2.7 - Nuovo commento} 
\label{sssec:UC6.2.2.7} 
\begin{itemize} 
\item \textbf{Attori}: .
\item \textbf{Descrizione}: l'attore può aggiungere un nuovo elemento commento al diagramma delle attività di un metodo;
\item \textbf{Precondizione}: è aperta, nel frame delle diagramma delle  classi, il diagramma delle attività del un metodo di una classe;
\item \textbf{Postcondizione}: viene aggiunto al diagramma delle attività di un metodo un nuovo elemento commento;
\item \textbf{Scenario principale}: l'attore ha selezionato l'elemento nuovo commento dalla lista degli strumenti del diagramma delle attività;\end{itemize} 
\subsubsection{UC6.2.2.8 - Nuovo output pin} 
\label{sssec:UC6.2.2.8} 
\begin{itemize} 
\item \textbf{Attori}: .
\item \textbf{Descrizione}: l'attore può annettere un nuovo output pin ad un elemento chiamata a metodo, presente nel diagramma delle attività di un metodo;
\item \textbf{Precondizione}: è aperta, nel frame delle diagramma delle  classi, il diagramma delle attività del un metodo di una classe; è presente un elemento chiamata a metodo nel diagramma;
\item \textbf{Postcondizione}: viene annesso ad un elemento chiamata a metodo, presente nel diagramma delle attività di un metodo, un nuovo elemento output pin;
\item \textbf{Scenario principale}: l'attore ha selezionato l'elemento nuovo input pin dalla lista degli strumenti del diagramma delle attività;\end{itemize} 
\subsubsection{UC6.3 - Disegnatore diagrammi} 
\label{sssec:UC6.3} 
\begin{itemize} 
\item \textbf{Attori}: .
\item \textbf{Descrizione}: Disegnatore diagrammi;
\item \textbf{Precondizione}: Disegnatore diagrammi;
\item \textbf{Postcondizione}: Disegnatore diagrammi;
\item \textbf{Scenario principale}: \begin{enumerate}\item Editor Diagrammi delle Classi (UC6.3.1);\item Zoom-in (UC6.3.2);\item Zoom-out (UC6.3.3);\item Drag di un oggetto (UC6.3.4);\item Editor Diagrammi dei Metodi (UC6.3.5). 
 \end{enumerate}
\end{itemize} 
\subsubsection{UC6.3.1 - Editor Diagrammi delle Classi} 
\label{sssec:UC6.3.1} 
\begin{itemize} 
\item \textbf{Attori}: Utente Autenticato.
\item \textbf{Descrizione}: Interazione con la schermata principale;
\item \textbf{Precondizione}: L'utente crea o apre un progetto.;
\item \textbf{Postcondizione}: L'utente chiude il progetto.;
\item \textbf{Scenario principale}: \begin{enumerate}\item Elimina Classe (UC6.3.1.1);\item Elimina Relazione (UC6.3.1.2);\item Elimina pacchetto (UC6.3.1.3);\item Apri dettagli elemento (UC6.3.1.4);\item Modifica Classe (UC6.3.1.5);\item Modifica Relazione (UC6.3.1.6);\item Modifica pacchetto (UC6.3.1.7);\item Modifica commento (UC6.3.1.8);\item Elimina commento (UC6.3.1.9). 
 \end{enumerate}
\end{itemize} 
\subsubsection{UC6.3.1.1 - Elimina Classe} 
\label{sssec:UC6.3.1.1} 
\begin{itemize} 
\item \textbf{Attori}: Utente Autenticato.
\item \textbf{Descrizione}: Una classe viene eliminata dal progetto corrente;
\item \textbf{Precondizione}: Deve essere stata disegnata una classe;
\item \textbf{Postcondizione}: Viene eliminata la classe selezionata e, a cascata, tutti i metodi (quindi i flowchart) ad essa, e solo ad essa, associati, tutti gli attributi e tutte le relazioni che la coinvolgono.;
\item \textbf{Scenario principale}: \begin{enumerate}\item Elimina Nome (UC6.3.1.1.1);\item Elimina Attributo (UC6.3.1.1.2);\item Elimina Metodo (UC6.3.1.1.3). 
 \end{enumerate}
\end{itemize} 
\subsubsection{UC6.3.1.1.1 - Elimina Nome} 
\label{sssec:UC6.3.1.1.1} 
\begin{itemize} 
\item \textbf{Attori}: Utente Autenticato.
\item \textbf{Descrizione}: Eliminazione Nome di una classe;
\item \textbf{Precondizione}: È stata creata una classe con un nome.;
\item \textbf{Postcondizione}: Viene eliminato il nome della classe.;
\item \textbf{Scenario principale}: L'utente ha disegnato una classe fornendole un nome e desidera cancellarlo.;\end{itemize} 
\subsubsection{UC6.3.1.1.2 - Elimina Attributo} 
\label{sssec:UC6.3.1.1.2} 
\begin{itemize} 
\item \textbf{Attori}: Utente Autenticato.
\item \textbf{Descrizione}: Viene eliminato un attributo della classe.;
\item \textbf{Precondizione}: È stata creata una classe con almeno un attributo.;
\item \textbf{Postcondizione}: L'attributo selezionato viene eliminato.;
\item \textbf{Scenario principale}: L'utente ha disegnato una classe con almeno un attributo al suo interno e desidera eliminarlo.;\item \textbf{Scenari alternativi}: L'utente ha aggiunto un nuovo attributo e desidera eliminarlo..
\end{itemize} 
\subsubsection{UC6.3.1.1.3 - Elimina Metodo} 
\label{sssec:UC6.3.1.1.3} 
\begin{itemize} 
\item \textbf{Attori}: Utente Autenticato.
\item \textbf{Descrizione}: Viene eliminato un metodo della classe.;
\item \textbf{Precondizione}: È stata creata una classe con almeno un metodo.;
\item \textbf{Postcondizione}: Viene eliminato il metodo selezionato e, a cascata, il flowchart associato se esistente.;
\item \textbf{Scenario principale}: L'utente ha disegnato una classe con almeno un metodo ed  eventualmente anche il flowchart corrispondente e desidera eliminarlo.;\item \textbf{Scenari alternativi}: L'utente ha aggiunto un nuovo metodo e desidera eliminarlo..
\end{itemize} 
\subsubsection{UC6.3.1.2 - Elimina Relazione} 
\label{sssec:UC6.3.1.2} 
\begin{itemize} 
\item \textbf{Attori}: Utente Autenticato.
\item \textbf{Descrizione}: Viene eliminata una relazione fra oggetti all'interno del designer.;
\item \textbf{Precondizione}: È stata disegnata una Relazione.;
\item \textbf{Postcondizione}: Viene eliminata la relazione selezionata ed eventuali etichette associate.;
\item \textbf{Scenario principale}: L'utente ha disegnato una relazione, la seleziona e decide di cancellarla.;\end{itemize} 
\subsubsection{UC6.3.1.3 - Elimina pacchetto} 
\label{sssec:UC6.3.1.3} 
\begin{itemize} 
\item \textbf{Attori}: Utente Autenticato.
\item \textbf{Descrizione}: Eliminazione di un package;
\item \textbf{Precondizione}: È stato disegnato un pacchetto.;
\item \textbf{Postcondizione}: Viene eliminato il pacchetto dal designer assieme a tutti i suoi eventuali elementi contenuti.;
\item \textbf{Scenario principale}: L'utente ha disegnato un pacchetto e lo ha riempito con delle classi prima di cancellarlo.;\end{itemize} 
\subsubsection{UC6.3.1.4 - Apri dettagli elemento} 
\label{sssec:UC6.3.1.4} 
\begin{itemize} 
\item \textbf{Attori}: Utente Autenticato.
\item \textbf{Descrizione}: Visualizza i dettagli implementativi dell'elemento desiderato.;
\item \textbf{Precondizione}: È stato disegnato almeno un elemento all'interno del designer.;
\item \textbf{Postcondizione}: Vengono visualizzati tutti i dettagli dell'elemento.;
\item \textbf{Scenario principale}: L'utente ha disegnato un elemento e desidera visualizzare ogni sua parte.;\end{itemize} 
\subsubsection{UC6.3.1.5 - Modifica Classe} 
\label{sssec:UC6.3.1.5} 
\begin{itemize} 
\item \textbf{Attori}: Utente Autenticato.
\item \textbf{Descrizione}: Modifica degli elementi interni ad una classe;
\item \textbf{Precondizione}: È stata disegnata una classe sul progetto.;
\item \textbf{Postcondizione}: La classe risulta modificata.;
\item \textbf{Scenario principale}: \begin{enumerate}\item Modifica Nome (UC6.3.1.5.1);\item Modifica Attributo (UC6.3.1.5.2);\item Aggiungi attributo (UC6.3.1.5.3);\item Modifica Metodo (UC6.3.1.5.4);\item Aggiungi Metodo (UC6.3.1.5.5);\item Assegnazione abstract (UC6.3.1.5.6);\item Assegnazione interface (UC6.3.1.5.7). 
 \end{enumerate}
\end{itemize} 
\subsubsection{UC6.3.1.5.1 - Modifica Nome} 
\label{sssec:UC6.3.1.5.1} 
\begin{itemize} 
\item \textbf{Attori}: Utente Autenticato.
\item \textbf{Descrizione}: Viene modificato il nome della classe.;
\item \textbf{Precondizione}: È stata creata una classe con un nome.;
\item \textbf{Postcondizione}: Viene modificato il nome della classe.;
\item \textbf{Scenario principale}: L'utente ha disegnato una classe con un nome e desidera modificarlo.;\end{itemize} 
\subsubsection{UC6.3.1.5.2 - Modifica Attributo} 
\label{sssec:UC6.3.1.5.2} 
\begin{itemize} 
\item \textbf{Attori}: Utente Autenticato.
\item \textbf{Descrizione}: Viene modificato l'attributo della classe.;
\item \textbf{Precondizione}: È stata disegnata una classe con almeno un attributo.;
\item \textbf{Postcondizione}: Viene modificato l'attributo selezionato.;
\item \textbf{Scenario principale}: \begin{enumerate}\item Aggiunta visibilità attributo (UC6.3.1.5.2.1);\item Aggiunta nome attributo (UC6.3.1.5.2.2);\item Aggiunta tipo attributo (UC6.3.1.5.2.3);\item Aggiunta valore default attributo (UC6.3.1.5.2.4). 
 \end{enumerate}
\end{itemize} 
\subsubsection{UC6.3.1.5.2.1 - Aggiunta visibilità attributo} 
\label{sssec:UC6.3.1.5.2.1} 
\begin{itemize} 
\item \textbf{Attori}: .
\item \textbf{Descrizione}: viene scelta dall'attore la visibilità per l'attributo;
\item \textbf{Precondizione}: viene visualizzato un form che permette la selezione della visibilità;
\item \textbf{Postcondizione}: il sistema registra il dato inserito;
\item \textbf{Scenario principale}: Viene scelta la visibilità nell'insieme di valori disponibili;\end{itemize} 
\subsubsection{UC6.3.1.5.2.2 - Aggiunta nome attributo} 
\label{sssec:UC6.3.1.5.2.2} 
\begin{itemize} 
\item \textbf{Attori}: .
\item \textbf{Descrizione}: viene inserito dall'attore il nome per l'attributo;
\item \textbf{Precondizione}: viene visualizzato un form che permette l'inserimento del nome dell'attributo;
\item \textbf{Postcondizione}: il sistema registra il dato inserito;
\end{itemize} 
\subsubsection{UC6.3.1.5.2.3 - Aggiunta tipo attributo} 
\label{sssec:UC6.3.1.5.2.3} 
\begin{itemize} 
\item \textbf{Attori}: .
\item \textbf{Descrizione}: viene inserito il tipo dell'attributo che si sta inserendo;
\item \textbf{Precondizione}: viene visualizzato un form che permette l'inserimento tipo dell'attributo;
\item \textbf{Postcondizione}: il sistema registra il dato inserito;
\end{itemize} 
\subsubsection{UC6.3.1.5.2.4 - Aggiunta valore default attributo} 
\label{sssec:UC6.3.1.5.2.4} 
\begin{itemize} 
\item \textbf{Attori}: .
\item \textbf{Descrizione}: viene inserito il valore di default che può avere l'attributo;
\item \textbf{Precondizione}: viene visualizzato un form che permette l'inserimento del valore di default dell'attributo;
\item \textbf{Postcondizione}: il sistema registra il dato inserito;
\end{itemize} 
\subsubsection{UC6.3.1.5.3 - Aggiungi attributo} 
\label{sssec:UC6.3.1.5.3} 
\begin{itemize} 
\item \textbf{Attori}: .
\item \textbf{Descrizione}: l'attore inserisce un attributo per la classe selezionata;
\item \textbf{Precondizione}: viene visualizzato un form che permette l'inserimento di attributi della classe;
\item \textbf{Postcondizione}: il sistema registra il dato inserito;
\end{itemize} 
\subsubsection{UC6.3.1.5.4 - Modifica Metodo} 
\label{sssec:UC6.3.1.5.4} 
\begin{itemize} 
\item \textbf{Attori}: Utente Autenticato.
\item \textbf{Descrizione}: Viene modificato il metodo della classe.;
\item \textbf{Precondizione}: È stata creata una classe con almeno un metodo al suo interno.;
\item \textbf{Postcondizione}: Il metodo selezionato viene modificato.;
\item \textbf{Scenario principale}: \begin{enumerate}\item Aggiunta visibilità metodo (UC6.3.1.5.4.1);\item Aggiunta nome metodo (UC6.3.1.5.4.2);\item Aggiunta tipo di ritorno del metodo (UC6.3.1.5.4.3);\item Inserimento lista parametri metodo (UC6.3.1.5.4.4). 
 \end{enumerate}
\end{itemize} 
\subsubsection{UC6.3.1.5.4.1 - Aggiunta visibilità metodo} 
\label{sssec:UC6.3.1.5.4.1} 
\begin{itemize} 
\item \textbf{Attori}: .
\item \textbf{Descrizione}: viene scelta dall'attore la visibilità per il metodo;
\item \textbf{Precondizione}: viene visualizzato un form che permette la selezione della visibilità;
\item \textbf{Postcondizione}: il sistema registra il dato inserito;
\end{itemize} 
\subsubsection{UC6.3.1.5.4.2 - Aggiunta nome metodo} 
\label{sssec:UC6.3.1.5.4.2} 
\begin{itemize} 
\item \textbf{Attori}: .
\item \textbf{Descrizione}: viene inserito dall'attore il nome per il metodo;
\item \textbf{Precondizione}: viene visualizzato un form che permette l'inserimento del nome del metodo;
\item \textbf{Postcondizione}: il sistema registra il dato inserito;
\end{itemize} 
\subsubsection{UC6.3.1.5.4.3 - Aggiunta tipo di ritorno del metodo} 
\label{sssec:UC6.3.1.5.4.3} 
\begin{itemize} 
\item \textbf{Attori}: .
\item \textbf{Descrizione}: viene inserito il tipo di ritorno del metodo che si sta inserendo;
\item \textbf{Precondizione}: viene visualizzato un form che permette l'inserimento tipo di ritorno del metodo;
\item \textbf{Postcondizione}: il sistema registra il dato inserito;
\end{itemize} 
\subsubsection{UC6.3.1.5.4.4 - Inserimento lista parametri metodo} 
\label{sssec:UC6.3.1.5.4.4} 
\begin{itemize} 
\item \textbf{Attori}: .
\item \textbf{Descrizione}: viene inserita la lista dei parametri che verranno utilizzati nella realizzazione del metodo;
\item \textbf{Precondizione}: viene visualizzato un form che permette l'inserimento della lista dei parametri;
\item \textbf{Postcondizione}: il sistema registra il dato inserito;
\end{itemize} 
\subsubsection{UC6.3.1.5.5 - Aggiungi Metodo} 
\label{sssec:UC6.3.1.5.5} 
\begin{itemize} 
\item \textbf{Attori}: .
\item \textbf{Descrizione}: Aggiungi Metodo;
\item \textbf{Precondizione}: Aggiungi Metodo;
\item \textbf{Postcondizione}: Aggiungi Metodo;
\end{itemize} 
\subsubsection{UC6.3.1.5.6 - Assegnazione abstract} 
\label{sssec:UC6.3.1.5.6} 
\begin{itemize} 
\item \textbf{Attori}: .
\item \textbf{Descrizione}: l'attore definisce se la classe è astratta;
\item \textbf{Precondizione}: viene visualizzato un form che permette l'assegnazione della keyword <<abstract>>;
\item \textbf{Postcondizione}: il sistema registra il dato inserito;
\end{itemize} 
\subsubsection{UC6.3.1.5.7 - Assegnazione interface} 
\label{sssec:UC6.3.1.5.7} 
\begin{itemize} 
\item \textbf{Attori}: .
\item \textbf{Descrizione}: l'attore definisce se la classe un'interfaccia;
\item \textbf{Precondizione}: viene visualizzato un form che permette l'assegnazione della keyword <<interface>>;
\item \textbf{Postcondizione}: il sistema registra il dato inserito;
\end{itemize} 
\subsubsection{UC6.3.1.6 - Modifica Relazione} 
\label{sssec:UC6.3.1.6} 
\begin{itemize} 
\item \textbf{Attori}: Utente Autenticato.
\item \textbf{Descrizione}: Viene modificata l'etichetta della relazione.;
\item \textbf{Precondizione}: È stata creata una relazione fra due o più classi.;
\item \textbf{Postcondizione}: L'etichetta della relazione corrente è stata modificata.;
\item \textbf{Scenario principale}: L'utente ha creato una nuova relazione fra classi e desidera modificarne l'etichetta.;\end{itemize} 
\subsubsection{UC6.3.1.7 - Modifica pacchetto} 
\label{sssec:UC6.3.1.7} 
\begin{itemize} 
\item \textbf{Attori}: .
\item \textbf{Descrizione}: Modifica un pacchetto.;
\item \textbf{Precondizione}: È stato creato un pacchetto.;
\item \textbf{Postcondizione}: Il pacchetto è stato modificato.;
\item \textbf{Scenario principale}: \begin{enumerate}\item Modifica Nome (UC6.3.1.7.1);\item Eliminare classe (UC6.3.1.7.2);\item Estrarre classe (UC6.3.1.7.3). 
 \end{enumerate}
\end{itemize} 
\subsubsection{UC6.3.1.7.1 - Modifica Nome} 
\label{sssec:UC6.3.1.7.1} 
\begin{itemize} 
\item \textbf{Attori}: Utente Autenticato.
\item \textbf{Descrizione}: Modifica il nome di un pacchetto che deve essere univoco.;
\item \textbf{Precondizione}: È stato creato un pacchetto e viene visualizzato un form che permette l'inserimento del nome.;
\item \textbf{Postcondizione}: Il nome del pacchetto selezionato viene salvato.;
\item \textbf{Scenario principale}: L'utente ha creato un nuovo pacchetto di cui desidera cambiare il nome.;\end{itemize} 
\subsubsection{UC6.3.1.7.2 - Eliminare classe} 
\label{sssec:UC6.3.1.7.2} 
\begin{itemize} 
\item \textbf{Attori}: Utente Autenticato.
\item \textbf{Descrizione}: Viene eliminata una classe all'interno del pacchetto.;
\item \textbf{Precondizione}: Esiste un pacchetto con almeno una classe al suo interno.;
\item \textbf{Postcondizione}: La classe selezionata viene eliminata dal pacchetto.;
\item \textbf{Scenario principale}: L'utente ha inserito almeno una classe all'interno del pacchetto.;\end{itemize} 
\subsubsection{UC6.3.1.7.3 - Estrarre classe} 
\label{sssec:UC6.3.1.7.3} 
\begin{itemize} 
\item \textbf{Attori}: Utente Autenticato.
\item \textbf{Descrizione}: Una classe viene estratta dal pacchetto.;
\item \textbf{Precondizione}: Esiste un pacchetto con almeno una classe al suo interno.;
\item \textbf{Postcondizione}: La classe selezionata si trova fuori dal pacchetto.;
\item \textbf{Scenario principale}: L'utente trascina la classe dal pacchetto al suo esterno.;\end{itemize} 
\subsubsection{UC6.3.1.8 - Modifica commento} 
\label{sssec:UC6.3.1.8} 
\begin{itemize} 
\item \textbf{Attori}: .
\item \textbf{Descrizione}: Modifica del testo inserito nel commento;
\item \textbf{Precondizione}: è stato inserito un commento;
\item \textbf{Postcondizione}: il commento risulta modificato;
\end{itemize} 
\subsubsection{UC6.3.1.9 - Elimina commento} 
\label{sssec:UC6.3.1.9} 
\begin{itemize} 
\item \textbf{Attori}: .
\item \textbf{Descrizione}: Eliminazione di un commento del diagramma delle classi;
\item \textbf{Precondizione}: È stato creato un diagramma delle classi con almeno un commento al suo interno.
;
\item \textbf{Postcondizione}: Il commento selezionato è stato eliminato.;
\end{itemize} 
\subsubsection{UC6.3.2 - Zoom-in} 
\label{sssec:UC6.3.2} 
\begin{itemize} 
\item \textbf{Attori}: Utente Autenticato.
\item \textbf{Descrizione}: Zoom-in su una particolare zona del designer.;
\item \textbf{Precondizione}: È stato aperto o avviato un nuovo progetto e lo zoom non è ancora al massimo.;
\item \textbf{Postcondizione}: SI è effettuato uno zoom nel punto indicato dal mouse.
;
\item \textbf{Scenario principale}: L'utente, utilizzando lo scroll del mouse, vuole effettuare uno zoom nel punto desiderato.;\end{itemize} 
\subsubsection{UC6.3.3 - Zoom-out} 
\label{sssec:UC6.3.3} 
\begin{itemize} 
\item \textbf{Attori}: Utente Autenticato.
\item \textbf{Descrizione}: Zoom-out su una porzione del designer.;
\item \textbf{Precondizione}: È stato aperto o avviato un nuovo progetto ed è già stato operato almeno uno zoom-in.;
\item \textbf{Postcondizione}: Viene effettuato uno zoom-out sulla porzione di designer desiderata.;
\item \textbf{Scenario principale}: L'utente, dopo aver effettuato uno zoom-in, desidera effettuare l'operazione contraria.;\end{itemize} 
\subsubsection{UC6.3.4 - Drag di un oggetto} 
\label{sssec:UC6.3.4} 
\begin{itemize} 
\item \textbf{Attori}: Utente Autenticato.
\item \textbf{Descrizione}: Un oggetto viene spostato all'interno del designer.;
\item \textbf{Precondizione}: È stato disegnato almeno un oggetto, di qualsiasi natura, sul designer.;
\item \textbf{Postcondizione}: L'oggetto selezionato viene spostato nella posizione desiderata.;
\item \textbf{Scenario principale}: L'utente, tramite un click del mouse, desidera trascinare un oggetto da un punto all'altro del designer.;\end{itemize} 
\subsubsection{UC6.3.5 - Editor Diagrammi dei Metodi} 
\label{sssec:UC6.3.5} 
\begin{itemize} 
\item \textbf{Attori}: Utente Autenticato.
\item \textbf{Descrizione}: L'attore può modificare il diagramma del metodo di una classe;
\item \textbf{Precondizione}: Nel sistema è caricato correttamente un progetto che  contiene almeno una classe con un metodo, è visualizzata la finestra di modifica del metodo;
\item \textbf{Postcondizione}: Il diagramma del metodo di una classe viene cambiato dalle eventuali modifiche dell'attore;
\item \textbf{Scenario principale}: \begin{enumerate}\item Eliminazione operazione (UC6.3.5.1);\item Modifica operazione (UC6.3.5.10);\item Modifica chiamata a metodo (UC6.3.5.11);\item Modifica variabile (UC6.3.5.12);\item Modifica connettore (UC6.3.5.13);\item Modifica commento (UC6.3.5.14);\item Eliminazione chiamata a metodo (UC6.3.5.2);\item Eliminazione variabile (UC6.3.5.3);\item Eliminazione connettore (UC6.3.5.4);\item Eliminazione nodo decisione (UC6.3.5.5);\item Eliminazione nodo merge (UC6.3.5.6);\item Eliminazione commento (UC6.3.5.7);\item Eliminazione output pin (UC6.3.5.8). 
 \end{enumerate}
\end{itemize} 
\subsubsection{UC6.3.5.1 - Eliminazione operazione} 
\label{sssec:UC6.3.5.1} 
\begin{itemize} 
\item \textbf{Attori}: Utente Autenticato.
\item \textbf{Descrizione}: L'attore può eliminare un elemento operazione fra gli elementi del digramma del metodo di una classe.;
\item \textbf{Precondizione}: È visualizzato il diagramma di un metodo che contiene almeno un elemento operazione;
\item \textbf{Postcondizione}: Viene eliminato, dal diagramma del metodo,  l'elemento operazione selezionato dall'attore;
\end{itemize} 
\subsubsection{UC6.3.5.10 - Modifica operazione} 
\label{sssec:UC6.3.5.10} 
\begin{itemize} 
\item \textbf{Attori}: Utente Autenticato.
\item \textbf{Descrizione}: L'attore ha la possibilità di modificare un elemento operazione presente nel diagramma del metodo di una classe;
\item \textbf{Precondizione}: È visualizzato il diagramma di un metodo che contiene almeno un elemento operazione;
\item \textbf{Postcondizione}: Viene modificato l'elemento operazione con i cambiamenti apportati dall'utente;
\item \textbf{Scenario principale}: \begin{enumerate}\item Definizione operazione (UC6.3.5.10.1). 
 \end{enumerate}
\end{itemize} 
\subsubsection{UC6.3.5.10.1 - Definizione operazione} 
\label{sssec:UC6.3.5.10.1} 
\begin{itemize} 
\item \textbf{Attori}: .
\item \textbf{Descrizione}: L'attore ha la possibilità di definire un'operazione tramite input testuale;
\item \textbf{Precondizione}: È aperta la finestra di modifica di un elemento operazione;
\item \textbf{Postcondizione}: È applicata l'operazione definita dall'utente;
\end{itemize} 
\subsubsection{UC6.3.5.11 - Modifica chiamata a metodo} 
\label{sssec:UC6.3.5.11} 
\begin{itemize} 
\item \textbf{Attori}: Utente Autenticato.
\item \textbf{Descrizione}: L'attore può modificare l'elemento chiamata a metodo di un diagramma del metodo di una classe;
\item \textbf{Precondizione}: È visualizzato il diagramma di un metodo che contiene almeno un elemento chiamata a metodo;
\item \textbf{Postcondizione}: I'elemento chiamata a metodo viene modificato con le modifiche apportate dall'attore;
\item \textbf{Scenario principale}: \begin{enumerate}\item Selezione metodo (UC6.3.5.11.1). 
 \end{enumerate}
\end{itemize} 
\subsubsection{UC6.3.5.11.1 - Selezione metodo} 
\label{sssec:UC6.3.5.11.1} 
\begin{itemize} 
\item \textbf{Attori}: .
\item \textbf{Descrizione}: L'attore può selezionare un metodo dalla lista dei metodi, e parametri in ingresso da esso richiesti;
\item \textbf{Precondizione}: La finestra di modifica di un elemento chiamata a metodo è visualizzata correttamente;
\item \textbf{Postcondizione}: Viene assegnato il metodo, selezionato dall'attore, all'elemento chiamata a metodo e eventuali parametri richiesti;
\item \textbf{Scenario principale}: L'attore intende assegnare un metodo all'elemento chiamata a metodo;\end{itemize} 
\subsubsection{UC6.3.5.12 - Modifica variabile} 
\label{sssec:UC6.3.5.12} 
\begin{itemize} 
\item \textbf{Attori}: Utente Autenticato.
\item \textbf{Descrizione}: L'attore può modificare l'elemento variabile di un diagramma del metodo di una classe;
\item \textbf{Precondizione}: È visualizzato il diagramma di un metodo che contiene almeno un elemento variabile;
\item \textbf{Postcondizione}: L'elemento variabile viene modificato con le modifiche apportate dall'attore;
\item \textbf{Scenario principale}: \begin{enumerate}\item Istanziazione nuova variabile (UC6.3.5.12.1). 
 \end{enumerate}
\end{itemize} 
\subsubsection{UC6.3.5.12.1 - Istanziazione nuova variabile} 
\label{sssec:UC6.3.5.12.1} 
\begin{itemize} 
\item \textbf{Attori}: .
\item \textbf{Descrizione}: L'attore può definire una nuova variabile per l'elemento variabile corrente;
\item \textbf{Precondizione}: Viene visualizzata correttamente la finestra di modifica dell'elemento variabile;
\item \textbf{Postcondizione}: È applicata, all'elemento variabile, la variabile definita dall'attore;
\item \textbf{Scenario principale}: L'attore intende definire una nuova variabile;\end{itemize} 
\subsubsection{UC6.3.5.13 - Modifica connettore} 
\label{sssec:UC6.3.5.13} 
\begin{itemize} 
\item \textbf{Attori}: .
\item \textbf{Descrizione}: L'attore può operare delle modifiche su un elemento connettore presente nel diagramma del metodo di una classe;
\item \textbf{Precondizione}: È visualizzato il diagramma di un metodo che contiene almeno un elemento connettore;
\item \textbf{Postcondizione}: Vengono applicate le modifiche eventualmente apportate dall'attore;
\item \textbf{Scenario principale}: \begin{enumerate}\item Definizione condizione di guardia (UC6.3.5.13.1). 
 \end{enumerate}
\end{itemize} 
\subsubsection{UC6.3.5.13.1 - Definizione condizione di guardia} 
\label{sssec:UC6.3.5.13.1} 
\begin{itemize} 
\item \textbf{Attori}: .
\item \textbf{Descrizione}: L'attore può definire una guardia per l'elemento connettore;
\item \textbf{Precondizione}: Viene visualizzata la finestra di modifica dell'elemento connettore;
\item \textbf{Postcondizione}: Viene ridefinita la guardia come specificato dall'attore;
\end{itemize} 
\subsubsection{UC6.3.5.14 - Modifica commento} 
\label{sssec:UC6.3.5.14} 
\begin{itemize} 
\item \textbf{Attori}: .
\item \textbf{Descrizione}: L'attore può modificare un elemento commento presente nel diagramma del metodo di una classe;
\item \textbf{Precondizione}: È visualizzato il diagramma di un metodo che contiene almeno un elemento commento;
\item \textbf{Postcondizione}: Vengono apportate le modifiche, all'elemento commento, effettuate dall'attore;
\end{itemize} 
\subsubsection{UC6.3.5.2 - Eliminazione chiamata a metodo} 
\label{sssec:UC6.3.5.2} 
\begin{itemize} 
\item \textbf{Attori}: Utente Autenticato.
\item \textbf{Descrizione}: L'attore può eliminare, dal diagramma del metodo di una classe, un elemento chiamata a metodo;
\item \textbf{Precondizione}: È visualizzato il diagramma di un metodo che contiene almeno un elemento chiamata a metodo;
\item \textbf{Postcondizione}: Viene eliminato, dal diagramma del metodo,  l'elemento chiamata a metodo selezionato dall'attore;
\end{itemize} 
\subsubsection{UC6.3.5.3 - Eliminazione variabile} 
\label{sssec:UC6.3.5.3} 
\begin{itemize} 
\item \textbf{Attori}: Utente Autenticato.
\item \textbf{Descrizione}: L'attore può eliminare un elemento variabile presente nel diagramma del metodo di una classe;
\item \textbf{Precondizione}: È visualizzato il diagramma di un metodo che contiene almeno un elemento variabile;
\item \textbf{Postcondizione}: Viene eliminato, dal diagramma del metodo,  l'elemento variabile selezionato dall'attore;
\end{itemize} 
\subsubsection{UC6.3.5.4 - Eliminazione connettore} 
\label{sssec:UC6.3.5.4} 
\begin{itemize} 
\item \textbf{Attori}: Utente Autenticato.
\item \textbf{Descrizione}: L'attore può eliminare, dal diagramma del metodo di una classe, un elemento connettore;
\item \textbf{Precondizione}: È visualizzato il diagramma di un metodo che contiene almeno un elemento connettore;
\item \textbf{Postcondizione}: Viene eliminato, dal diagramma del metodo,  l'elemento chiamata a metodo selezionato dall'attore;
\end{itemize} 
\subsubsection{UC6.3.5.5 - Eliminazione nodo decisione} 
\label{sssec:UC6.3.5.5} 
\begin{itemize} 
\item \textbf{Attori}: Utente Autenticato.
\item \textbf{Descrizione}: L'attore ha la possibilità di eliminare un elemento nodo decisione dal diagramma del metodo di una classe;
\item \textbf{Precondizione}: È visualizzato il diagramma di un metodo che contiene almeno un elemento nodo decisione;
\item \textbf{Postcondizione}: Viene eliminato, dal diagramma del metodo,  l'elemento nodo decisione selezionato dall'attore;
\end{itemize} 
\subsubsection{UC6.3.5.6 - Eliminazione nodo merge} 
\label{sssec:UC6.3.5.6} 
\begin{itemize} 
\item \textbf{Attori}: Utente Autenticato.
\item \textbf{Descrizione}: L'attore può eliminare, dal diagramma del metodo di una classe, un elemento nodo merge;
\item \textbf{Precondizione}: È visualizzato il diagramma di un metodo che contiene almeno un elemento nodo merge;
\item \textbf{Postcondizione}: Viene eliminato, dal diagramma del metodo,  l'elemento nodo merge selezionato dall'attore;
\end{itemize} 
\subsubsection{UC6.3.5.7 - Eliminazione commento} 
\label{sssec:UC6.3.5.7} 
\begin{itemize} 
\item \textbf{Attori}: Utente Autenticato.
\item \textbf{Descrizione}: L'attore può eliminare, dal diagramma del metodo di una classe, un elemento commento;
\item \textbf{Precondizione}: È visualizzato il diagramma di un metodo che contiene almeno un elemento commento;
\item \textbf{Postcondizione}: Viene eliminato, dal diagramma del metodo,  l'elemento commento selezionato dall'attore;
\end{itemize} 
\subsubsection{UC6.3.5.8 - Eliminazione output pin} 
\label{sssec:UC6.3.5.8} 
\begin{itemize} 
\item \textbf{Attori}: Utente Autenticato.
\item \textbf{Descrizione}: L'attore può eliminare, dal diagramma del metodo di una classe, un elemento output pin;
\item \textbf{Precondizione}: È visualizzato il diagramma di un metodo che contiene almeno un elemento output pin;
\item \textbf{Postcondizione}: L'elemento output pin, selezionato dall'utente, viene rimosso dal diagramma del metodo;
\end{itemize} 
\subsubsection{UC6.4 - Pannello laterale} 
\label{sssec:UC6.4} 
\begin{itemize} 
\item \textbf{Attori}: .
\item \textbf{Descrizione}: Sezione di schermo nella quale è possibile visionare il flusso del programma a partire dal metodo main;
\item \textbf{Precondizione}: È aperto un progetto;
\item \textbf{Postcondizione}: Visualizza un diagramma delle attività;
\item \textbf{Scenario principale}: \begin{enumerate}\item Esplorazione metodi (UC6.4.1);\item Visione Breadcrumb (UC6.4.2). 
 \end{enumerate}
\end{itemize} 
\subsubsection{UC6.4.1 - Esplorazione metodi} 
\label{sssec:UC6.4.1} 
\begin{itemize} 
\item \textbf{Attori}: .
\item \textbf{Descrizione}: Selezionando una chiamata a metodo è possibile visionare il diagramma delle attività di tale metodo;
\item \textbf{Precondizione}: È visualizzato un diagramma delle attività;
\item \textbf{Postcondizione}: Viene mostrato il diagramma delle attività del metodo selezionato;
\end{itemize} 
\subsubsection{UC6.4.2 - Visione Breadcrumb} 
\label{sssec:UC6.4.2} 
\begin{itemize} 
\item \textbf{Attori}: .
\item \textbf{Descrizione}: Viene visualizzata una Breadcrumb che rappresenta il percorso effettuato dall'utente durante la navigazione dei vari metodo selezionati;
\item \textbf{Precondizione}: è visualizzato un diagramma delle attività;
\item \textbf{Postcondizione}: vengono visualizzati in ordine di accesso tutti i metodi selezionati dall'utente nella visione principale del programma;
\end{itemize} 
