\def\arraystretch{1.5}
\rowcolors{2}{D}{P}
\begin{longtable}{p{2cm}!{\VRule[1pt]}p{2cm}!{\VRule[1pt]}p{5cm}!{\VRule[1pt]}p{1.5cm}}
\rowcolor{I}
\color{white} \textbf{Requisito} & \color{white} \textbf{Tipologia} & \color{white} \textbf{Descrizione} & \color{white} \textbf{Fonti} \\ 
\endfirsthead 
\rowcolor{I} 
\color{white} \textbf{Requisito} & \color{white} \textbf{Tipologia} & \color{white} \textbf{Descrizione} & \color{white} \textbf{Fonti} \\ 
\endhead 
R0F1&Funzionale\newline Obbligatorio & L'attore deve registrarsi per accedere ai servizi forniti dal  programma & Interno \newline UC1
 \\
R0F1.1&Funzionale\newline Obbligatorio & L'attore deve inserire un username univoco & Interno \newline UC1
 \newline UC1.1
 \\
R0F1.2&Funzionale\newline Obbligatorio & L'attore deve inserire una password & Interno \\
R0F1.3&Funzionale\newline Obbligatorio & L'attore deve inserire una email & Interno \newline UC1
 \newline UC1.3
 \\
R0F1.4&Funzionale\newline Obbligatorio & L'attore può confermare la registrazione & Interno \newline UC1
 \newline UC1.4
 \\
R0F1.5&Funzionale\newline Obbligatorio & L'applicazione deve visualizzare un messaggio d'errore se l'username non è conforme alle richieste & Interno \newline UC1
 \newline UC1.5
 \\
R0F1.6&Funzionale\newline Obbligatorio & L'applicazione deve visualizzare un messaggio d'errore se la password non è conforme alle richieste & Interno \newline UC1
 \newline UC1.6
 \\
R0F1.7&Funzionale\newline Obbligatorio & L'applicazione deve visualizzare un messaggio d'errore se l'email non è conforme alle richieste & Interno \newline UC1
 \newline UC1.7
 \\
R0F2&Funzionale\newline Obbligatorio & L'attore per accedere ai servizi forniti deve aver effettuato l'autenticazione e diventare utente autenticato & Interno \newline UC2
 \\
R0F2.1&Funzionale\newline Obbligatorio & L'attore deve inserire il proprio Username o Email utilizzati in fase di registrazione & Interno \newline UC2
 \newline UC2.1
 \\
R0F2.2&Funzionale\newline Obbligatorio & L'attore deve inserire il propria password utilizzata in fase di registrazione & Interno \newline UC2
 \newline UC2.2
 \\
R1F2.3&Funzionale\newline Desiderabile & L'utente può accedere alla pagina per il recupero della password & Interno \newline UC2
 \newline UC2.3
 \\
R1F2.3.1&Funzionale\newline Desiderabile & Il sistema invia la password all'indirizzo email inserito durante registrazione & Interno \newline UC2
 \newline UC2.3
 \newline UC2.5
 \\
R0F4&Funzionale\newline Obbligatorio & L'utente esce dal suo profilo & Interno \newline UC4
 \\
R0F5&Funzionale\newline Obbligatorio & L'utente visiona l'elenco dei suoi progetti & Interno \newline UC5
 \\
R0F5.1&Funzionale\newline Obbligatorio & L'utente aggiunge un nuovo progetto & Interno \newline UC5
 \newline UC5.1
 \\
R0F5.1.1&Funzionale\newline Obbligatorio & L'utente crea un nuovo progetto vuoto & Interno \newline UC5.1
 \newline UC5.1.1
 \\
R0F5.1.2&Funzionale\newline Obbligatorio & L'utente importa un progetto che è stato precedentemente espostato & Interno \newline UC5.1
 \newline UC5.1.2
 \\
R0F5.2&Funzionale\newline Obbligatorio & L'utente  apre un progetto precedentemente salvato & Interno \newline UC5
 \newline UC5.2
 \\
R0F5.3&Funzionale\newline Obbligatorio & L'utente elimina un progetto precedentemente salvato & Interno \newline UC5
 \newline UC5.3
 \\
R0F6&Funzionale\newline Obbligatorio & Il sistema visualizza l'editor dei diagrammi & Interno \newline UC6
 \\
R0F6.1&Funzionale\newline Obbligatorio & Il sistema visualizza la barra del menu & Interno \newline UC6
 \newline UC6.1
 \\
R0F6.1.1&Funzionale\newline Obbligatorio & Il sistema visualizza le voci presenti nel menu File & Interno \newline UC6.1
 \newline UC6.1.1
 \\
R0F6.1.1.1&Funzionale\newline Obbligatorio & L'utente può salvare il progetto in uso & Interno \newline UC6.1.1
 \newline UC6.1.1.1
 \\
R0F6.1.1.2&Funzionale\newline Obbligatorio & L'utente può chiudere il progetto in uso & Interno \newline UC6.1
 \newline UC6.1.1.2
 \\
R0F6.1.1.3&Funzionale\newline Obbligatorio & L'utente può esportare il progetto in uso & Interno \newline UC6.1
 \newline UC6.1.1.3
 \\
R0F6.1.1.4&Funzionale\newline Obbligatorio & L'utente può selezionare la generazione del codice  & Capitolato \newline UC6.1.1
 \newline UC6.1.1.4
 \\
R1F6.1.1.5&Funzionale\newline Desiderabile & L'utente può salvare il progetto corrente come template & Riunione esterna \\
R1F6.1.2&Funzionale\newline Desiderabile & Il sistema visualizza le voci presenti nel menu Edit & Interno \newline UC6.1
 \newline UC6.1.2
 \\
R1F6.1.2.1&Funzionale\newline Desiderabile & L'utente può annullare l'ultima operazione effettuata & Interno \newline UC6.1.2
 \newline UC6.1.2.1
 \\
R1F6.1.2.2&Funzionale\newline Desiderabile & L'utente può ripristinare l'ultima operazione effettuata & Interno \newline UC6.1.2
 \newline UC6.1.2.2
 \\
R1F6.1.2.3&Funzionale\newline Desiderabile & L'utente può effettuare l'operazione "taglia" di un oggetto selezionato & Interno \newline UC6.1.2
 \newline UC6.1.2.3
 \\
R1F6.1.2.4&Funzionale\newline Desiderabile & L'utente può effettuare l'operazione "copia" di un oggetto selezionato & Interno \newline UC6.1.2
 \newline UC6.1.2.4
 \\
R1F6.1.2.5&Funzionale\newline Desiderabile & L'utente può effettuare l'operazione "incolla" di un oggetto precedentemente copiato & Interno \newline UC6.1.2
 \newline UC6.1.2.5
 \\
R1F6.1.2.6&Funzionale\newline Desiderabile & L'utente può effettuare l'operazione "zoom-in" della schermata & Interno \newline UC6.1.2
 \newline UC6.1.2.6
 \\
R1F6.1.2.7&Funzionale\newline Desiderabile & L'utente può effettuare l'operazione "zoom-out" della schermata & Interno \newline UC6.1.2
 \newline UC6.1.2.7
 \\
R1F6.1.3&Funzionale\newline Desiderabile & Il sistema visualizza l'elenco dei template disponibili & Interno \newline UC6.1
 \newline UC6.1.3
 \\
R1F6.1.3.1&Funzionale\newline Desiderabile & L'utente può aggiungere al proprio progetto un template & Riunione esterna \newline UC6.1.3
 \newline UC6.1.3.1
 \\
R1F6.1.3.2&Funzionale\newline Desiderabile & L'utente può eliminare un proprio template precedentemente salvato & Interno \newline UC6.1.3
 \newline UC6.1.3.2
 \\
R2F6.1.4&Funzionale\newline Opzionale & Il sistema visualizza le voci presenti nel menu Layers & Riunione esterna \newline UC6.1
 \newline UC6.1.4
 \\
R2F6.1.4.1&Funzionale\newline Opzionale & L'utente può creare un nuovo livello & Interno \newline UC6.1.4
 \newline UC6.1.4.1
 \\
R2F6.1.4.2&Funzionale\newline Opzionale & L'utente seleziona i livelli che vuole visionare nella lista visualizzata dal sistema & Interno \newline UC6.1.4
 \newline UC6.1.4.2
 \\
R2F6.1.4.3&Funzionale\newline Opzionale & L'utente può modificare i livelli presenti & Interno \newline UC6.1.4
 \newline UC6.1.4.3
 \\
R2F6.1.4.3.1&Funzionale\newline Opzionale & L'utente può modificare il nome del livello & Interno \newline UC6.1.4.3
 \newline UC6.1.4.3.1
 \\
R2F6.1.4.3.2&Funzionale\newline Opzionale & L'utente può eliminare un livello & Interno \newline UC6.1.4.3
 \newline UC6.1.4.3.2
 \\
R0F6.2&Funzionale\newline Obbligatorio & Il sistema visualizza la barra degli strumenti grafici & Interno \newline UC6
 \newline UC6.2
 \\
R0F6.2.1&Funzionale\newline Obbligatorio & Il sistema visualizza il menu per il disegno dei diagrammi delle classi & Capitolato \newline UC6.2
 \newline UC6.2.1
 \\
R0F6.2.1.1&Funzionale\newline Obbligatorio & L'utente può aggiungere un nuovo elemento di tipo "classe" & Interno \newline UC6.2.1
 \newline UC6.2.1.1
 \\
R0F6.2.1.2&Funzionale\newline Obbligatorio & L'utente può aggiungere un nuovo elemento di tipo "pacchetto" & Interno \newline UC6.2.1
 \newline UC6.2.1.2
 \\
R0F6.2.1.3&Funzionale\newline Obbligatorio & L'utente può aggiungere un nuovo elemento di tipo "relazione" & Interno \newline UC6.2.1
 \newline UC6.2.1.3
 \\
R0F6.2.1.3.1&Funzionale\newline Obbligatorio & L'utente seleziona la classe di partenza della relazione & Interno \newline UC6.2.1.3
 \newline UC6.2.1.3.1
 \\
R0F6.2.1.3.2&Funzionale\newline Obbligatorio & L'utente seleziona la classe di destinazione della relazione & Interno \newline UC6.2.1.3
 \newline UC6.2.1.3.2
 \\
R0F6.2.1.3.3&Funzionale\newline Obbligatorio & L'utente seleziona il tipo di relazione da inserire & Interno \newline UC6.2.1.3
 \newline UC6.2.1.3.3
 \\
R0F6.2.1.4&Funzionale\newline Obbligatorio & L'utente può aggiungere un nuovo elemento di tipo "commento" & Interno \newline UC6.2.1
 \newline UC6.2.1.4
 \\
R0F6.2.1.4.1&Funzionale\newline Obbligatorio & L'utente seleziona la classe a cui è associato il commento & Interno \newline UC6.2.1.4
 \newline UC6.2.1.4.1
 \\
R0F6.2.2&Funzionale\newline Obbligatorio & Il sistema visualizza il menu per il disegno dei diagrammi delle attività & Capitolato \newline UC6.2
 \newline UC6.2.2
 \\
R0F6.2.2.1&Funzionale\newline Obbligatorio & L'utente può aggiungere un nuovo elemento di tipo "operazione" & Interno \newline UC6.2.2
 \newline UC6.2.2.1
 \\
R0F6.2.2.2&Funzionale\newline Obbligatorio & L'utente può aggiungere un nuovo elemento di tipo "chiamata a metodo" & Interno \newline UC6.2.2
 \newline UC6.2.2.2
 \\
R0F6.2.2.3&Funzionale\newline Obbligatorio & L'utente può aggiungere un nuovo elemento di tipo "variabile" & Interno \newline UC6.2.2
 \newline UC6.2.2.3
 \\
R0F6.2.2.4&Funzionale\newline Obbligatorio & L'utente può aggiungere un nuovo elemento di tipo "connettore" & Interno \newline UC6.2.2
 \newline UC6.2.2.4
 \\
R0F6.2.2.5&Funzionale\newline Obbligatorio & L'utente può aggiungere un nuovo elemento di tipo "nodo decisione" & Interno \newline UC6.2.2
 \newline UC6.2.2.5
 \\
R0F6.2.2.6&Funzionale\newline Obbligatorio & L'utente può aggiungere un nuovo elemento di tipo "nodo merge" & Interno \newline UC6.2.2
 \newline UC6.2.2.6
 \\
R0F6.2.2.7&Funzionale\newline Obbligatorio & L'utente può aggiungere un nuovo elemento di tipo "commento" & Interno \newline UC6.2.2
 \newline UC6.2.2.7
 \\
R0F6.2.2.8&Funzionale\newline Obbligatorio & L'utente può aggiungere un nuovo elemento di tipo "output pin" & Interno \newline UC6.2.2
 \newline UC6.2.2.8
 \\
R1F3&Funzionale\newline Desiderabile & L'utente ha la possibilità di gestire i propri dati personali & Interno \newline UC3
 \\
R1F3.1&Funzionale\newline Desiderabile & L'utente inserisce un nuovo username & Interno \newline UC3
 \newline UC3.1
 \\
R1F3.2&Funzionale\newline Desiderabile & L'utente inserisce una nuova password & Interno \newline UC3
 \newline UC3.2
 \\
R1F3.3&Funzionale\newline Desiderabile & L'utente inserisce una nuova email & Interno \newline UC3
 \newline UC3.3
 \\
R1F3.4&Funzionale\newline Desiderabile & L'attore elimina il suo profilo & Interno \newline UC3
 \newline UC3.4
 \\
R1F3.5&Funzionale\newline Desiderabile & L'applicazione deve visualizzare un messaggio d'errore se l'username non è conforme alle richieste & Interno \newline UC3
 \newline UC3.5
 \\
R1F3.6&Funzionale\newline Desiderabile & L'applicazione deve visualizzare un messaggio d'errore se la password non è conforme alle richieste & Interno \newline UC3
 \newline UC3.6
 \\
R1F3.7&Funzionale\newline Desiderabile & L'applicazione deve visualizzare un messaggio d'errore se l'email non è conforme alle richieste & Interno \newline UC3
 \newline UC3.7
 \\
\rowcolor{white}
\caption{Tracciamento requisiti funzionali}
\end{longtable}