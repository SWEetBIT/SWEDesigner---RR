\def\arraystretch{1.5}

Ogni requisito è rappresentato da un codice univoco e gerarchico, nella forma:
\begin{center}
        R[importanza][tipo][codice]
      \end{center}
      \begin{itemize}
        \item \textbf{Importanza} può assumere i seguenti valori:
          \bgroup
            \begin{itemize}
              \item [0]: Requisito obbligatorio;
              \item [1]: Requisito desiderabile;
              \item [2]: Requisito opzionale.
            \end{itemize}
          \egroup
        \item \textbf{Tipo} può assumere i seguenti valori:
          \bgroup
            \begin{itemize}
              \item [F]: Funzionale;
              \item [Q]: Di Qualità;
              \item [P]: Prestazionale;
              \item [V]: Vincolo.
            \end{itemize}
          \egroup
        \item \textbf{Codice} è il codice univoco di ogni requisito espresso in modo gerarchico.
      \end{itemize}

\rowcolors{2}{D}{P}
\begin{longtable}{p{2.5cm}!{\VRule[1pt]}p{2cm}!{\VRule[1pt]}p{5cm}!{\VRule[1pt]}p{2.5cm}}
\rowcolor{I}
\color{white} \textbf{Requisito} & \color{white} \textbf{Tipologia} & \color{white} \textbf{Descrizione} & \color{white} \textbf{Fonti} \\ 
\endfirsthead 
\rowcolor{I} 
\color{white} \textbf{Requisito} & \color{white} \textbf{Tipologia} & \color{white} \textbf{Descrizione} & \color{white} \textbf{Fonti} \\ 
\endhead 
R0F1&Funzionale\newline Obbligatorio & L'attore deve registrarsi per accedere ai servizi forniti dal  programma & Interno \newline UC1
 \\
R0F1.1&Funzionale\newline Obbligatorio & L'attore deve inserire un \glossaryItem{Username} univoco & Interno \newline UC1
 \newline UC1.1
 \\
R0F1.2&Funzionale\newline Obbligatorio & L'attore deve inserire una password & Interno \newline UC1.2
 \\
R0F1.3&Funzionale\newline Obbligatorio & L'attore deve inserire una email & Interno \newline UC1
 \newline UC1.3
 \\
R0F1.4&Funzionale\newline Obbligatorio & L'attore può confermare la registrazione & Interno \newline UC1
 \newline UC1.4
 \\
R0F1.5&Funzionale\newline Obbligatorio & L'\glossaryItem{Applicazione} deve visualizzare un messaggio d'errore se l'\glossaryItem{Username} non è conforme alle richieste & Interno \newline UC1
 \newline UC1.5
 \\
R0F1.6&Funzionale\newline Obbligatorio & L'\glossaryItem{Applicazione} deve visualizzare un messaggio d'errore se la password non è conforme alle richieste & Interno \newline UC1
 \newline UC1.6
 \\
R0F1.7&Funzionale\newline Obbligatorio & L'\glossaryItem{Applicazione} deve visualizzare un messaggio d'errore se l'email non è conforme alle richieste & Interno \newline UC1
 \newline UC1.7
 \\
R0F2&Funzionale\newline Obbligatorio & L'attore per accedere ai servizi forniti deve aver effettuato l'autenticazione e diventare \glossaryItem{Utente} autenticato & Interno \newline UC2
 \\
R0F2.1&Funzionale\newline Obbligatorio & L'attore deve inserire il proprio \glossaryItem{Username} o Email utilizzati in fase di registrazione & Interno \newline UC2
 \newline UC2.1
 \\
R0F2.2&Funzionale\newline Obbligatorio & L'attore deve inserire la propria password utilizzata in fase di registrazione & Interno \newline UC2
 \newline UC2.2
 \\
R0F2.3&Funzionale\newline Obbligatorio & L'\glossaryItem{Applicazione} visualizza un messaggio di errore dato dall'inserimento di credenziali errate & Interno \newline UC2
 \newline UC2.3
 \\

R0F4&Funzionale\newline Obbligatorio & L'attore esce dal suo profilo & Interno \newline UC4
 \\
R0F5&Funzionale\newline Obbligatorio & L'attore visualizza l'elenco dei suoi progetti & Interno \newline UC5
 \\
R0F5.1&Funzionale\newline Obbligatorio & L'attore aggiunge un nuovo progetto & Interno \newline UC5
 \newline UC5.1
 \\
R0F5.1.1&Funzionale\newline Obbligatorio & L'attore crea un nuovo progetto vuoto & Interno \newline UC5.1
 \newline UC5.1.1
 \\
R0F5.1.2&Funzionale\newline Obbligatorio & L'attore importa un progetto & Interno \newline UC5.1
 \newline UC5.1.2
 \\
R0F5.2&Funzionale\newline Obbligatorio & L'attore  apre un progetto precedentemente salvato & Interno \newline UC5
 \newline UC5.2
 \\
R0F5.3&Funzionale\newline Obbligatorio & L'attore elimina un progetto precedentemente salvato & Interno \newline UC5
 \newline UC5.3
 \\
R0F6&Funzionale\newline Obbligatorio & L'\glossaryItem{Applicazione} visualizza l'editor dei \glossaryItem{Diagrammi} & Interno \newline UC6
 \\
R0F6.1&Funzionale\newline Obbligatorio & L'\glossaryItem{Applicazione} visualizza la barra del menu & Interno \newline UC6
 \newline UC6.1
 \\
R0F6.1.1&Funzionale\newline Obbligatorio & L'\glossaryItem{Applicazione} visualizza le voci presenti nel menu \glossaryItem{File} & Interno \newline UC6.1
 \newline UC6.1.1
 \\
R0F6.1.1.1&Funzionale\newline Obbligatorio & L'attore può salvare il progetto in uso & Interno \newline UC6.1.1
 \newline UC6.1.1.1
 \\
R0F6.1.1.2&Funzionale\newline Obbligatorio & L'attore può chiudere il progetto in uso & Interno \newline UC6.1
 \newline UC6.1.1.2
 \\
R0F6.1.1.3&Funzionale\newline Obbligatorio & L'attore può esportare il progetto in uso & Interno \newline UC6.1
 \newline UC6.1.1.3
 \\
R0F6.1.1.4&Funzionale\newline Obbligatorio & L'attore può selezionare la generazione del \glossaryItem{Codice}  & \glossaryItem{Capitolato} \newline UC6.1.1
 \newline UC6.1.1.4
 \\
R1F6.1.1.5&Funzionale\newline Desiderabile & L'attore può salvare il progetto corrente come \glossaryItem{Template} & Riunione esterna \\
R1F6.1.2&Funzionale\newline Desiderabile & L'\glossaryItem{Applicazione} visualizza le voci presenti nel menu Edit & Interno \newline UC6.1
 \newline UC6.1.2
 \\
R1F6.1.2.1&Funzionale\newline Desiderabile & L'attore può annullare l'ultima operazione effettuata & Interno \newline UC6.1.2
 \newline UC6.1.2.1
 \\
R1F6.1.2.2&Funzionale\newline Desiderabile & L'attore può ripristinare l'ultima operazione effettuata & Interno \newline UC6.1.2
 \newline UC6.1.2.2
 \\
R1F6.1.2.3&Funzionale\newline Desiderabile & L'attore può effettuare l'operazione "taglia" di un oggetto selezionato & Interno \newline UC6.1.2
 \newline UC6.1.2.3
 \\
R1F6.1.2.4&Funzionale\newline Desiderabile & L'attore può effettuare l'operazione "copia" di un oggetto selezionato & Interno \newline UC6.1.2
 \newline UC6.1.2.4
 \\
R1F6.1.2.5&Funzionale\newline Desiderabile & L'attore può effettuare l'operazione "incolla" di un oggetto precedentemente copiato & Interno \newline UC6.1.2
 \newline UC6.1.2.5
 \\
R1F6.1.2.6&Funzionale\newline Desiderabile & L'attore può effettuare l'operazione "\glossaryItem{Zoom}-in" della schermata & Interno \newline UC6.1.2
 \newline UC6.1.2.6
 \\
R1F6.1.2.7&Funzionale\newline Desiderabile & L'attore può effettuare l'operazione "\glossaryItem{Zoom}-out" della schermata & Interno \newline UC6.1.2
 \newline UC6.1.2.7
 \\
R1F6.1.3&Funzionale\newline Desiderabile & L'\glossaryItem{Applicazione} visualizza l'elenco dei \glossaryItem{Template} disponibili & Interno \newline UC6.1
 \newline UC6.1.3
 \\
R1F6.1.3.1&Funzionale\newline Desiderabile & L'attore può aggiungere al proprio progetto un \glossaryItem{Template} & Riunione esterna \newline UC6.1.3
 \newline UC6.1.3.1
 \\
R1F6.1.3.2&Funzionale\newline Desiderabile & L'attore può eliminare un proprio \glossaryItem{Template} precedentemente salvato & Interno \newline UC6.1.3
 \newline UC6.1.3.2
 \\
R2F6.1.4&Funzionale\newline Opzionale & L'\glossaryItem{Applicazione} visualizza le voci presenti nel menu \glossaryItem{Layer}s & Riunione esterna \newline UC6.1
 \newline UC6.1.4
 \\
R2F6.1.4.1&Funzionale\newline Opzionale & L'attore può creare un nuovo \glossaryItem{Layer} & Interno \newline UC6.1.4
 \newline UC6.1.4.1
 \\
R2F6.1.4.2&Funzionale\newline Opzionale & L'attore seleziona i \glossaryItem{Layer}s che vuole visionare nella lista visualizzata dal sistema & Interno \newline UC6.1.4
 \newline UC6.1.4.2
 \\
R2F6.1.4.3&Funzionale\newline Opzionale & L'attore può modificare i \glossaryItem{Layer}s presenti & Interno \newline UC6.1.4
 \newline UC6.1.4.3
 \\
R2F6.1.4.3.1&Funzionale\newline Opzionale & L'attore può modificare il nome del \glossaryItem{Layer} & Interno \newline UC6.1.4.3
 \newline UC6.1.4.3.1
 \\
R2F6.1.4.3.2&Funzionale\newline Opzionale & L'attore può eliminare un \glossaryItem{Layer} & Interno \newline UC6.1.4.3
 \newline UC6.1.4.3.2
 \\
R0F6.2&Funzionale\newline Obbligatorio & L'\glossaryItem{Applicazione} visualizza la barra degli strumenti grafici & Interno \newline UC6
 \newline UC6.2
 \\
R0F6.2.1&Funzionale\newline Obbligatorio & L'\glossaryItem{Applicazione} visualizza il menu per il disegno dei \glossaryItem{Diagrammi} delle \glossaryItem{Classi} & \glossaryItem{Capitolato} \newline UC6.2
 \newline UC6.2.1
 \\
R0F6.2.1.1&Funzionale\newline Obbligatorio & L'attore può aggiungere un nuovo elemento di tipo "\glossaryItem{Classe}" & Interno \newline UC6.2.1
 \newline UC6.2.1.1
 \\
R0F6.2.1.2&Funzionale\newline Obbligatorio & L'attore può aggiungere un nuovo elemento di tipo "pacchetto" & Interno \newline UC6.2.1
 \newline UC6.2.1.2
 \\
R0F6.2.1.3&Funzionale\newline Obbligatorio & L'attore può aggiungere un nuovo elemento di tipo "relazione" & Interno \newline UC6.2.1
 \newline UC6.2.1.3
 \\
R0F6.2.1.3.1&Funzionale\newline Obbligatorio & L'attore seleziona la \glossaryItem{Classe} di partenza della relazione & Interno \newline UC6.2.1.3
 \newline UC6.2.1.3.1
 \\
R0F6.2.1.3.2&Funzionale\newline Obbligatorio & L'attore seleziona la \glossaryItem{Classe} di destinazione della relazione & Interno \newline UC6.2.1.3
 \newline UC6.2.1.3.2
 \\
R0F6.2.1.3.3&Funzionale\newline Obbligatorio & L'attore seleziona il tipo di relazione da inserire & Interno \newline UC6.2.1.3
 \newline UC6.2.1.3.3
 \\
R0F6.2.1.4&Funzionale\newline Obbligatorio & L'attore può aggiungere un nuovo elemento di tipo "commento" & Interno \newline UC6.2.1
 \newline UC6.2.1.4
 \\
R0F6.2.1.4.1&Funzionale\newline Obbligatorio & L'attore seleziona la \glossaryItem{Classe} a cui è associato il commento & Interno \newline UC6.2.1.4
 \newline UC6.2.1.4.1
 \\
R0F6.2.2&Funzionale\newline Obbligatorio & L'\glossaryItem{Applicazione} visualizza il menu per il disegno dei \glossaryItem{Diagrammi} delle attività & \glossaryItem{Capitolato} \newline UC6.2
 \newline UC6.2.2
 \\
R0F6.2.2.1&Funzionale\newline Obbligatorio & L'attore può aggiungere un nuovo elemento di tipo "operazione" & Interno \newline UC6.2.2
 \newline UC6.2.2.1
 \\
R0F6.2.2.2&Funzionale\newline Obbligatorio & L'attore può aggiungere un nuovo elemento di tipo "chiamata a \glossaryItem{Metodo}" & Interno \newline UC6.2.2
 \newline UC6.2.2.2
 \\
R0F6.2.2.3&Funzionale\newline Obbligatorio & L'attore può aggiungere un nuovo elemento di tipo "variabile" & Interno \newline UC6.2.2
 \newline UC6.2.2.3
 \\
R0F6.2.2.4&Funzionale\newline Obbligatorio & L'attore può aggiungere un nuovo elemento di tipo "connettore" & Interno \newline UC6.2.2
 \newline UC6.2.2.4
 \\
R0F6.2.2.5&Funzionale\newline Obbligatorio & L'attore può aggiungere un nuovo elemento di tipo "nodo decisione" & Interno \newline UC6.2.2
 \newline UC6.2.2.5
 \\
R0F6.2.2.6&Funzionale\newline Obbligatorio & L'attore può aggiungere un nuovo elemento di tipo "nodo \glossaryItem{Merge}" & Interno \newline UC6.2.2
 \newline UC6.2.2.6
 \\
R0F6.2.2.7&Funzionale\newline Obbligatorio & L'attore può aggiungere un nuovo elemento di tipo "commento" & Interno \newline UC6.2.2
 \newline UC6.2.2.7
 \\
R0F6.2.2.8&Funzionale\newline Obbligatorio & L'attore può aggiungere un nuovo elemento di tipo "output pin" & Interno \newline UC6.2.2
 \newline UC6.2.2.8
 \\
R0F6.3&Funzionale\newline Obbligatorio & L'\glossaryItem{Applicazione} visualizza il disegnatore dei \glossaryItem{Diagrammi} & \glossaryItem{Capitolato} \newline UC6
 \newline UC6.3
 \\
R0F6.3.1&Funzionale\newline Obbligatorio & L'\glossaryItem{Applicazione} visualizza i \glossaryItem{Diagrammi} delle \glossaryItem{Classi} disegnati & \glossaryItem{Capitolato} \newline UC6.3
 \newline UC6.3.1
 \\
R0F6.3.1.1&Funzionale\newline Obbligatorio & L'attore può eliminare un elemento "\glossaryItem{Classe}" presente nel disegnatore & Interno \newline UC6.3.1
 \newline UC6.3.1.1
 \\
R0F6.3.1.2&Funzionale\newline Obbligatorio & L'attore può eliminare un elemento "Relazione" presente nel disegnatore & Interno \newline UC6.3.1
 \newline UC6.3.1.2
 \\
R0F6.3.1.3&Funzionale\newline Obbligatorio & L'attore può eliminare un elemento "pacchetto" presente nel disegnatore & Interno \newline UC6.3.1
 \newline UC6.3.1.3
 \\
R0F6.3.1.4&Funzionale\newline Obbligatorio & L'attore può visionare più o meno dettagli dell'elemento & Interno \newline UC6.3.1
 \newline UC6.3.1.4
 \\
R0F6.3.1.5&Funzionale\newline Obbligatorio & L'attore può modificare un elemento "\glossaryItem{Classe}" presente nel disegnatore & Interno \newline UC6.3.1
 \newline UC6.3.1.5
 \\
R0F6.3.1.5.1&Funzionale\newline Obbligatorio & L'attore può modificare il campo nome di una \glossaryItem{Classe} & Interno \newline UC6.3.1.5
 \newline UC6.3.1.5.1
 \\
R0F6.3.1.5.2&Funzionale\newline Obbligatorio & L'attore può modificare un campo attributo di una \glossaryItem{Classe} & Interno \newline UC6.3.1.5
 \newline UC6.3.1.5.2
 \\
R0F6.3.1.5.2.1&Funzionale\newline Obbligatorio & L'attore può modificare la visibilità di un attributo & Interno \newline UC6.3.1.5.2
 \newline UC6.3.1.5.2.1
 \\
R0F6.3.1.5.2.2&Funzionale\newline Obbligatorio & L'attore può modificare il campo nome di un attributo & Interno \newline UC6.3.1.5.2
 \newline UC6.3.1.5.2.2
 \\
R0F6.3.1.5.2.3&Funzionale\newline Obbligatorio & L'attore può modificare il campo tipo di un attributo & Interno \newline UC6.3.1.5.2
 \newline UC6.3.1.5.2.3
 \\
R0F6.3.1.5.2.4&Funzionale\newline Obbligatorio & L'attore può modificare il campo "valore di default" di un attributo & Interno \newline UC6.3.1.5.2
 \newline UC6.3.1.5.2.4
 \\
R0F6.3.1.5.3&Funzionale\newline Obbligatorio & L'attore può aggiungere un nuovo campo "attributo" in una \glossaryItem{Classe} & Interno \newline UC6.3.1.5
 \newline UC6.3.1.5.3
 \\
R0F6.3.1.5.4&Funzionale\newline Obbligatorio & L'attore può modificare un campo "\glossaryItem{Metodo}" di una \glossaryItem{Classe} & Interno \newline UC6.3.1.5
 \newline UC6.3.1.5.4
 \\
R0F6.3.1.5.4.1&Funzionale\newline Obbligatorio & L'attore può modificare la visibilità di un \glossaryItem{Metodo} & Interno \newline UC6.3.1.5.4
 \newline UC6.3.1.5.4.1
 \\
R0F6.3.1.5.4.2&Funzionale\newline Obbligatorio & L'attore può modificare il nome di un \glossaryItem{Metodo} & Interno \newline UC6.3.1.5.4
 \newline UC6.3.1.5.4.2
 \\
R0F6.3.1.5.4.3&Funzionale\newline Obbligatorio & L'attore può modificare il tipo di ritorno di un \glossaryItem{Metodo} & Interno \newline UC6.3.1.5.4
 \newline UC6.3.1.5.4.3
 \\
R0F6.3.1.5.4.4&Funzionale\newline Obbligatorio & L'attore può modificare la lista di parametri di un \glossaryItem{Metodo} & Interno \newline UC6.3.1.5.4
 \newline UC6.3.1.5.4.4
 \\
R0F6.3.1.5.5&Funzionale\newline Obbligatorio & L'attore può aggiungere un nuovo campo "\glossaryItem{Metodo}" in una \glossaryItem{Classe} & Interno \newline UC6.3.1.5
 \newline UC6.3.1.5.5
 \\
R0F6.3.1.5.6&Funzionale\newline Obbligatorio & L'attore può definire se la \glossaryItem{Classe} è astratta & Interno \newline UC6.3.1.5
 \newline UC6.3.1.5.6
 \\
R0F6.3.1.5.7&Funzionale\newline Obbligatorio & L'attore può definire se la \glossaryItem{Classe} è un'interfaccia & Interno \newline UC6.3.1.5
 \newline UC6.3.1.5.7
 \\
R0F6.3.1.5.8&Funzionale\newline Obbligatorio & L'attore può modificare il campo "assegnazione \glossaryItem{Layer}" di una \glossaryItem{Classe} & Interno \newline UC6.3.1.5
 \newline UC6.3.1.5.8
 \\
R0F6.3.1.6&Funzionale\newline Obbligatorio & L'attore può modificare un elemento "Relazione" presente nel disegnatore & Interno \newline UC6.3.1
 \newline UC6.3.1.6
 \\
R0F6.3.1.7&Funzionale\newline Obbligatorio & L'attore può modificare un elemento "pacchetto" presente nel disegnatore & Interno \newline UC6.3.1
 \newline UC6.3.1.7
 \\
R0F6.3.1.7.1&Funzionale\newline Obbligatorio & L'attore può modificare un campo "nome" di un pacchetto & Interno \newline UC6.3.1.7
 \newline UC6.3.1.7.1
 \\
R0F6.3.1.7.2&Funzionale\newline Obbligatorio & L'attore può eliminare una \glossaryItem{Classe} in un pacchetto & Interno \newline UC6.3.1.7
 \newline UC6.3.1.7.2
 \\
R0F6.3.1.7.3&Funzionale\newline Obbligatorio & L'attore può estrarre una \glossaryItem{Classe} all'interno di un pacchetto e portarla all'esterno & Interno \newline UC6.3.1.7
 \newline UC6.3.1.7.3
 \\
R0F6.3.1.8&Funzionale\newline Obbligatorio & L'attore può modificare un elemento "Commento" presente nel disegnatore & Interno \newline UC6.3.1
 \newline UC6.3.1.8
 \\
R0F6.3.1.8.1&Funzionale\newline Obbligatorio & L'attore può modificare il campo testo all'interno di un commento & Interno \newline UC6.3.1.8
 \newline UC6.3.1.8.1
 \\
R0F6.3.1.9&Funzionale\newline Obbligatorio & L'attore può eliminare un elemento "Commento" presente nel disegnatore & Interno \newline UC6.3.1
 \newline UC6.3.1.9
 \\
 R0F6.3.1.10&Funzionale\newline Obbligatorio & L'attore può eliminare un elemento "\glossaryItem{Metodo}" presente nella \glossaryItem{Classe} & Interno \newline UC6.3.1
 \newline UC6.3.1.10
 \\
R0F6.3.1.11&Funzionale\newline Obbligatorio & L'attore può eliminare un elemento "Attributo" presente nella \glossaryItem{Classe} & Interno \newline UC6.3.1
 \newline UC6.3.1.11
 \\
R0F6.3.2&Funzionale\newline Obbligatorio & L'attore può effettuare l'operazione "\glossaryItem{Zoom}-in" della schermata dal disegnatore & Interno \newline UC6.3
 \newline UC6.3.2
 \\
R0F6.3.3&Funzionale\newline Obbligatorio & L'attore può effettuare l'operazione "\glossaryItem{Zoom}-out" della schermata dal disegnatore & Interno \newline UC6.3
 \newline UC6.3.3
 \\
R0F6.3.4&Funzionale\newline Obbligatorio & L'attore può effettuare l'operazione di "drag" degli oggetti nella schermata dal disegnatore & Interno \newline UC6.3
 \newline UC6.3.4
 \\
R0F6.3.5&Funzionale\newline Obbligatorio & L'\glossaryItem{Applicazione} visualizza il \glossaryItem{Diagramma} del \glossaryItem{Metodo} selezionato & \glossaryItem{Capitolato} \newline UC6.3
 \newline UC6.3.5
 \\
R0F6.3.5.1&Funzionale\newline Obbligatorio & L'attore può eliminare un elemento "Operazione" presente nel disegnatore & Interno \newline UC6.3.5
 \newline UC6.3.5.1
 \\
R0F6.3.5.2&Funzionale\newline Obbligatorio & L'attore può eliminare un elemento "Chiamata a \glossaryItem{Metodo}" presente nel disegnatore & Interno \newline UC6.3.5
 \newline UC6.3.5.2
 \\
R0F6.3.5.3&Funzionale\newline Obbligatorio & L'attore può eliminare un elemento "Variabile" presente nel disegnatore & Interno \newline UC6.3.5
 \newline UC6.3.5.3
 \\
R0F6.3.5.4&Funzionale\newline Obbligatorio & L'attore può eliminare un elemento "Connettore" presente nel disegnatore & Interno \newline UC6.3.5
 \newline UC6.3.5.4
 \\
R0F6.3.5.5&Funzionale\newline Obbligatorio & L'attore può eliminare un elemento "Nodo decisione" presente nel disegnatore & Interno \newline UC6.3.5
 \newline UC6.3.5.5
 \\
R0F6.3.5.6&Funzionale\newline Obbligatorio & L'attore può eliminare un elemento "Nodo \glossaryItem{Merge}" presente nel disegnatore & Interno \newline UC6.3.5
 \newline UC6.3.5.6
 \\
R0F6.3.5.7&Funzionale\newline Obbligatorio & L'attore può eliminare un elemento "Commento" presente nel disegnatore & Interno \newline UC6.3.5
 \newline UC6.3.5.7
 \\
R0F6.3.5.8&Funzionale\newline Obbligatorio & L'attore può eliminare un elemento "Output pin" presente nel disegnatore & Interno \newline UC6.3.5
 \newline UC6.3.5.8
 \\
R0F6.3.5.9&Funzionale\newline Obbligatorio & L'attore può modificare un elemento "Commento" presente nel disegnatore & Interno \newline UC6.3.5
 \newline UC6.3.5.9
 \\

R0F6.3.5.10&Funzionale\newline Obbligatorio & L'attore può modificare un elemento "Operazione" presente nel disegnatore & Interno \newline UC6.3.5
 \newline UC6.3.5.10
 \\
R0F6.3.5.10.1&Funzionale\newline Obbligatorio & L'attore può definire un'operazione tra variabili & Interno \newline UC6.3.5.10
 \newline UC6.3.5.10.1
 \\
R0F6.3.5.11&Funzionale\newline Obbligatorio & L'attore può modificare un elemento "Chiamata a \glossaryItem{Metodo}" presente nel disegnatore & Interno \newline UC6.3.5
 \newline UC6.3.5.11
 \\
R0F6.3.5.11.1&Funzionale\newline Obbligatorio & L'attore può selezionare uno dei \glossaryItem{Metodi} precedentemente dichiarati & Interno \newline UC6.3.5.11
 \newline UC6.3.5.11.1
 \\
R0F6.3.5.12&Funzionale\newline Obbligatorio & L'attore può modificare un elemento "Variabile" presente nel disegnatore & Interno \newline UC6.3.5
 \newline UC6.3.5.12
 \\
R0F6.3.5.12.1&Funzionale\newline Obbligatorio & L'attore può modificare l'istanziazione di una variabile in un \glossaryItem{Metodo} & Interno \newline UC6.3.5.12
 \newline UC6.3.5.12.1
 \\
R0F6.3.5.13&Funzionale\newline Obbligatorio & L'attore può modificare un elemento "Connettore" presente nel disegnatore & Interno \newline UC6.3.5
 \newline UC6.3.5.13
 \\
R0F6.3.5.13.1&Funzionale\newline Obbligatorio & L'attore può modificare una condizione di guardia di un connettore & Interno \newline UC6.3.5.13
 \newline UC6.3.5.13.1
 \\ 
 
R0F6.4&Funzionale\newline Obbligatorio & L'\glossaryItem{Applicazione} visualizza il pannello laterale & Interno \newline UC6
 \newline UC6.4
 \\
R0F6.4.1&Funzionale\newline Obbligatorio & L'attore può navigare tra i vari \glossaryItem{Metodi} presenti selezionando l'elemento "chiamata a \glossaryItem{Metodo}" & Interno \newline UC6.4
 \newline UC6.4.1
 \\
R0F6.4.2&Funzionale\newline Obbligatorio & L'\glossaryItem{Applicazione} visualizza una breadcrumb che rappresenta il percorso effettuato dall'\glossaryItem{Utente} tra i vari \glossaryItem{Metodi} & Interno \newline UC6.4
 \newline UC6.4.2
 \\
R1F3&Funzionale\newline Desiderabile & L'attore ha la possibilità di gestire i propri dati personali & Interno \newline UC3
 \\
R1F3.1&Funzionale\newline Desiderabile & L'attore inserisce una nuova password & Interno \newline UC3
 \newline UC3.1
 \\
R1F3.2&Funzionale\newline Desiderabile & L'attore inserisce una nuova email & Interno \newline UC3
 \newline UC3.2
 \\
R1F3.3&Funzionale\newline Desiderabile & L'attore elimina il suo profilo & Interno \newline UC3
 \newline UC3.3
 \\
R1F3.4&Funzionale\newline Desiderabile & L'\glossaryItem{Applicazione} deve visualizzare un messaggio d'errore se la password non è conforme alle richieste & Interno \newline UC3
 \newline UC3.4
 \\
R1F3.5&Funzionale\newline Desiderabile & L'\glossaryItem{Applicazione} deve visualizzare un messaggio d'errore se l'email non è conforme alle richieste & Interno \newline UC3
 \newline UC3.5
 \\
 R1F13&Funzionale\newline Desiderabile & L'attore può accedere alla pagina per il recupero della password & Interno \newline UC7
 \\
R1F14&Funzionale\newline Desiderabile & L'\glossaryItem{Applicazione} invia la password all'indirizzo email inserito durante registrazione & Interno \newline UC7
 \newline UC7.1
 \\
\rowcolor{white}
\caption{Tracciamento requisiti funzionali}
\end{longtable}